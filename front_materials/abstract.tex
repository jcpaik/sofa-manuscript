The \emph{moving sofa problem}, first published by Leo Moser in 1966,
asks for the connected shape with the largest area $\mu_{\text{max}}$ that can move around the
right-angled corner of a hallway $L$ with unit width.
The best bounds currently known on $\mu_{\max}$ are summarized as
$\mu_G := 2.2195\ldots \leq \mu_{\max} \leq 2.37$.
The lower bound $2.2195\ldots \leq \mu_{\max}$ comes from Gerver's sofa $S_G$ of area $\mu_G :=
2.2195\ldots$ constructed in 1994, with 18 analytic curves and segments constituting the boundary.
The upper bound $\mu_{\max} \leq 2.37$ was proved by Kallus and Romik in 2018
using extensive computer assistance.

We prove that any moving sofa satisfying a certain property, named as the
\emph{injectivity condition},
has an area at most $1 + \pi^2/8 =
2.2337\ldots$. This upper bound, while conditional, is much closer to the lower bound $2.2195\ldots$
of Gerver than the upper bound $2.37$ of Kallus and Romik. Since Gerver's sofa $S_G$ satisfies the
injectivity condition, our conditional upper bound is effective on the domain containing the
conjectured optimum $S_G$. The proof does not rely on any computer assistance, and introduces a
calculus of variation based on the Brunn-Minkowski theory on convex bodies as a new approach to the
moving sofa problem. We also conjecture the \emph{injectivity hypothesis}
that there exists a maximum-area moving sofa satisfying the injectivity
condition. With our result, proving the injectivity hypothesis would imply the upper bound
to $\mu_{\max} \leq 1 + \pi^2/8 = 2.2337\ldots$.