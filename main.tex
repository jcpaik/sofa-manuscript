\documentclass[11pt, letterpaper]{book}

\usepackage[dvips, hcentering,
includehead,width=14.2cm,
top=0.5cm,
height=21.6cm,
paperheight=23.8cm,paperwidth=17cm
]{geometry}

\usepackage{textcomp}
\usepackage{amsmath}
\usepackage{amsthm}
\usepackage{amsfonts}
\usepackage{amssymb}
\usepackage{hyperref}
\usepackage{cleveref}
\usepackage{enumitem}
\usepackage{svg}
\svgsetup{inkscapearea=page,inkscapelatex=false}
\usepackage{float}
\usepackage{gensymb}

\usepackage{xcolor}
\hypersetup{
    colorlinks,
    linkcolor={red!50!black},
    citecolor={blue!50!black},
    urlcolor={blue!80!black}
}

\usepackage[backend=bibtex]{biblatex}
\addbibresource{refs.bib}

\theoremstyle{plain}
\newtheorem{theorem}{Theorem}[section]
\newtheorem{lemma}[theorem]{Lemma}
\newtheorem{corollary}[theorem]{Corollary}
\newtheorem{conjecture}[theorem]{Conjecture}
\newtheorem{proposition}[theorem]{Proposition}
\theoremstyle{definition}
\newtheorem{definition}{Definition}[section]
\theoremstyle{remark}
\newtheorem{remark}{Remark}[section]

\floatplacement{figure}{h}

\providecommand{\tightlist}{%
  \setlength{\itemsep}{0pt}\setlength{\parskip}{0pt}}
\makeatletter
\def\maxwidth{\ifdim\Gin@nat@width>\linewidth\linewidth\else\Gin@nat@width\fi}
\def\maxheight{\ifdim\Gin@nat@height>\textheight\textheight\else\Gin@nat@height\fi}
\makeatother
% Scale images if necessary, so that they will not overflow the page
% margins by default, and it is still possible to overwrite the defaults
% using explicit options in \includegraphics[width, height, ...]{}
\setkeys{Gin}{width=\maxwidth,height=\maxheight,keepaspectratio}

\title{Rotation Angle of a Moving Sofa}

\author{Jineon Baek}
\address{Yonsei University, Department of Mathematics}
\email{jineon@umich.edu}

\begin{document}

\maketitle

\begin{abstract}
The \emph{moving sofa problem} asks for the connected shape with the largest area $\mu_{\text{max}}$
that can move around the right-angled corner of a hallway $L$ with unit width.
It is conjectured that Gerver's sofa $S_G$ attains the maximum area,
but all known derivations of $S_G$ assumes without proof that
a maximum-area sofa should rotate clockwise by $\pi/2$ in its movement.
We prove this assumption. This is a part of a bigger project.
\textcolor{red}{DO NOT DISTRIBUTE YET}
\emph{The full text width in this document is \the\textwidth, while the column width is \the\columnwidth.  We will use these values to prepare the figures in Mathematica.}
\end{abstract}

\chapter{Introduction}
\label{sec:introduction}
Fix an arbitrary planar convex body \(K\), which by \Cref{def:convex-body} is a nonempty, compact and convex subset of \(\mathbb{R}^2\). This appendix defines and proves numerous properties of \(K\). For the ease of understanding, it is recommended to read only the parts of this appendix when needed. However, an interested reader can read this appendix from beginning to end to verify the correctness of all proofs and theorems. Note that we allow \(K\) to have empty interior. That is, \(K\) can be a point or a closed line segment. If a theorem requires \(K\) to have nonempty interior, we state it explicitly.
\subsection{Moving sofa problem}
\label{sec:moving-sofa-problem}
Moving a large couch through a narrow hallway requires a well-planned pivoting. The \emph{moving sofa problem}, first published by Leo Moser in 1966 \autocite{moser1966problem}, is asked in a two-dimensional idealization of such a situation:

\begin{quote}
What is the largest area \(\mu_{\text{max}}\) of a connected shape that can move around the right-angled corner of a hallway with unit width?
\end{quote}

More precisely, define the hallway \(L\) as the union \(L = L_H \cup L_V\) of sets \(L_H = (-\infty, 1] \times [0, 1]\) and \(L_V = [0, 1] \times (-\infty, 1]\) representing the horizontal and vertical side of \(L\) respectively. A \emph{moving sofa} \(S\) may be defined as a connected subset of \(L_H\) that can be moved inside \(L\) by a continuous rigid motion to a subset of \(L_V\). It is known that there exists a moving sofa attaining the maximum area \(\mu_{\text{max}}\) \autocite{gerverMovingSofaCorner1992,croft2012unsolved}, but the precise value of \(\mu_{\text{max}}\) remains unknown despite decades of partial progress \autocite{hammersley1968enfeeblement,gerverMovingSofaCorner1992,romikDifferentialEquationsExact2018,kallusImprovedUpperBounds2018}.

The best bounds currently known on \(\mu_{\max}\) are summarized as
\begin{equation}
\label{eqn:best-bounds}
\mu_G := 2.2195\dots \leq \mu_{\max} \leq 2.37.
\end{equation}
The lower bound \(2.2195\dots \leq \mu_{\max}\) comes from Gerver’s sofa \(S_G\) of area \(\mu_G := 2.2195\dots\) constructed in 1994 \autocite{gerverMovingSofaCorner1992} (see \Cref{fig:gerver}). Gerver derived his sofa from local optimality considerations\footnote{Gerver assumed five stages of the movement of a sofa to derive his sofa \(S_G\) \autocite{gerverMovingSofaCorner1992}. While his sofa \(S_G\) is locally optimal (Theorem 2 of \autocite{gerverMovingSofaCorner1992}), this does not eliminate the possibility of a maximum-area sofa with a different kind of movement. Romik’s simplified derivation of \(S_G\) in \autocite{romikDifferentialEquationsExact2018} also relies on the same assumption (Equation 24, p324 of \autocite{romikDifferentialEquationsExact2018}). So their derivations do not constitute a full proof of Gerver’s conjecture \(\mu_{\max} = \mu_G\).} and conjectured \(\mu_{\max} = \mu_G\) that his sofa attains the maximum area. Approximate solutions found by computer experiments are consistent with Gerver’s conjecture.\footnote{Wagner used Monte Carlo simulation to find an approximate solution (Figure 2 of \autocite{wagner1976sofa}) that resembles Gerver’s sofa in shape. More recent approximate solutions, as found by Gibbs \autocite{gibbsComputationalStudySofas2014} in 2014 and Batsch \autocite{batschNumericalApproachAnalysing2022} in 2022, deviate in area from Gerver’s sofa by small margins of \(1.7 \times 10^{-7}\) and \(5.7 \times 10^{-9}\) respectively.}

\begin{figure}
\centering
\includesvg[width=1\textwidth,height=\textheight]{images/gerver-full.svg}
\caption{Gerver’s sofa \(S_G\). The ticks denote the endpoints of 18 analytic curves and segments constituting the boundary of \(S_G\) (see \autocite{romikDifferentialEquationsExact2018} for details). The lower portion of \(S_G\) is made of two small ‘tails’ (depicted red) and one large ‘core’ (depicted blue).}
\label{fig:gerver}
\end{figure}

On the other hand, the upper bound \(\mu_{\max} \leq 2.37\) was proved by Kallus and Romik \autocite{kallusImprovedUpperBounds2018}. If Gerver’s conjecture \(\mu_{\max} = \mu_G\) is true, then the remaining task would be to bring the upper bound of \(\mu_{\max}\) down to the lower bound \(\mu_G\). However, not many methods are known for bounding \(\mu_{\max}\) from above. All known upper bounds of \(\mu_{\max}\), including that of Kallus and Romik, approximate the moving sofa by a polygonal intersection \(S_\Theta\) of the horizontal strip \(H = \mathbb{R} \times [0, 1]\) and copies of hallway \(L\) rotated counterclockwise by each angle in a finite set \(\Theta \subset (0, \pi/2)\) (see \Cref{fig:polygon-sofa}).\footnote{The polygonal intersection \(S_\Theta = H \cap \bigcap_{t \in \Theta} L_t\) is the overestimation of the monotone sofa \(\mathcal{M}(S) = H \cap V_\omega \cap \bigcap_{0 \leq t \leq \omega} L_t\) in \Cref{eqn:monotone}.} The bound \(\mu_{\max} \leq 2 \sqrt{2} = 2.828\dots\) by Hammersley \autocite{hammersley1968enfeeblement} is obtained by taking a single angle \(\Theta = \left\{ \pi/4 \right\}\), and the bound \(\mu_{\max} \leq 2.37\) by Kallus and Romik \autocite{kallusImprovedUpperBounds2018} is achieved by taking five specific angles \(\Theta\) and estimating the area of \(S_\Theta\) from above with extensive computer assistance.

\begin{figure}
\centering
\includesvg[width=0.7\textwidth,height=\textheight]{images/polygon-sofa.svg}
\caption{Polygonal intersection \(S_\Theta\) used by the upper bound \(2 \sqrt{2} = 2.828\dots\) of Hammersley \autocite{hammersley1968enfeeblement}, and \(2.37\) of Kallus and Romik \autocite{kallusImprovedUpperBounds2018}.}
\label{fig:polygon-sofa}
\end{figure}

We develop a new approach for bounding \(\mu_{\max}\) from above by interpreting the moving sofa problem as an infinite-dimensional convex quadratic programming. Consequently, we prove that any moving sofa satisfying a certain property, named as the \emph{injectivity condition} (\Cref{def:injectivity}), has an area at most \(1 + \pi^2/8 = 2.2337\dots\) (\Cref{thm:main}). This conditional upper bound does not rely on any computer assistance, while being much closer to the lower bound \(2.2195\dots\) of Gerver than the computer-assisted upper bound \(2.37\) of Kallus and Romik. Gerver’s sofa \(S_G\), the conjectured optimum, satisfies the injectivity condition in particular. We also conjecture the \emph{injectivity hypothesis} (\Cref{con:injectivity}) that there exists a maximum-area moving sofa satisfying the injectivity condition. With our result, proving the injectivity hypothesis would imply the unconditional upper bound \(\mu_{\max} \leq 1 + \pi^2/8\).

The idea for proving the main \Cref{thm:main} is to overestimate the area of a moving sofa \(S\) by ignoring the effect of the inner walls to the moving sofa \(S\) (compare the left side of \Cref{fig:presofa} to \Cref{fig:gerver}). The overestimated area \(\mathcal{A}_1(K)\) (\Cref{fig:a1-upper-bound}) turns out to be a quadratic functional on a convex body \(K\) that we call the \emph{cap} of \(S\). We establish the concavity of \(\mathcal{A}_1\) using Mamikon’s theorem \autocite{mnatsakanianAnnularRingsEqual1997}, a theorem in classical geometry (\Cref{fig:mamikon-sofa}). Then we introduce a calculus of variation based on the Brunn-Minkowski theory to find a global optimum \(K_1\) of \(\mathcal{A}_1\). The optimum of \(\mathcal{A}_1\) gives an unmovable sofa \(S_1\) of area \(1 + \pi^2/8 = 2.2337\dots\) and width \(\pi\) very close to the area of Gerver’s sofa \(S_G\) (see the right side of \Cref{fig:presofa}).

\begin{figure}
\centering
\includesvg[width=1\textwidth,height=\textheight]{images/presofa-combined.svg}
\caption{The maximizing (unmovable) sofa \(S_1\) of upper bound \(\mathcal{A}_1\) of area \(1 + \pi^2/8 = 2.2337\dots\). The regions below two tails (red curves) stick out of the hallway \(L\) during the movement of \(S_1\) in \(L\) (left). The shape \(S_1\) is very similar to Gerver’s sofa \(S_G\) whose boundary is drawn in dotted lines (right).}
\label{fig:presofa}
\end{figure}

\subsection{Moving hallway problem}
\label{sec:moving-hallway-problem}
The \emph{rotation angle} \(\omega\) of a moving sofa \(S\) is defined as the clockwise angle that \(S\) rotates as it moves from \(L_H\) to \(L_V\) inside \(L\). Gerver’s sofa has the rotation angle \(\omega = \pi/2\). On the other hand, the unit square \([0, 1]^2\) can move inside \(L\) with only translation, so it has the rotation angle \(\omega = 0\). Let \(\omega\) be the rotation angle of a maximum-area moving sofa. Gerver proved that we can assume \(\pi/ 3 \leq \omega \leq \pi/2\) \autocite{gerverMovingSofaCorner1992}. Kallus and Romik improved the lower bound of \(\omega\) by showing that \(\omega \geq \arcsin(84/85) = 81.203\dots^\circ\) \autocite{kallusImprovedUpperBounds2018}. With this, we will only consider moving sofas with rotation angle \(\omega \in (0, \pi/2]\), and it seems reasonable to conjecture that a maximum-area moving sofa has \(\omega = \pi/2\).

\begin{conjecture}

(Angle Hypothesis) There exists a maximum-area moving sofa with a movement of rotation angle \(\omega = \pi/2\).

\label{con:angle}
\end{conjecture}

\begin{remark}

A single moving sofa \(S\) may admit different movements with varying rotation angles \(\omega\). For any moving sofa \(S\) mentioned in this paper, we will always assume a fixed rotation angle \(\omega\) attached to it. So any moving sofa in this paper is technically a tuple of a shape and its fixed rotation angle. In this way, we can talk about \emph{the} rotation angle of a moving sofa.

\label{rem:angle}
\end{remark}

Once we fix the rotation angle \(\omega \in (0, \pi/2]\) of a moving sofa \(S\), we can change the moving sofa problem as the moving \emph{hallway} problem. In \autocite{gerverMovingSofaCorner1992}, Gerver looked at the hallway \(L\) in the perspective of the moving sofa \(S\), so that \(S\) is fixed and the hallway \(L\) moves around the sofa (see \Cref{fig:monotone-sofa}). In this perspective, we claim that the sofa \(S\) is a common subset of the following sets.

\begin{enumerate}
\def\labelenumi{\arabic{enumi}.}
\tightlist
\item
  For every angle \(t \in [0, \omega]\), the rotated hallway \(L_t\), which is the hallway \(L\) rotated counterclockwise by \(t\) and translated so that the outer walls of \(L_t\) are in contact with \(S\).
\item
  The horizontal strip \(H = \mathbb{R} \times [0, 1]\).
\item
  A translation of \(V_\omega\), where \(V_\omega\) is the vertical strip \(V = [0, 1] \times \mathbb{R}\) rotated counterclockwise by \(\omega\) around the origin.
\end{enumerate}

To observe (1) that \(S \subseteq L_t\) for every angle \(t \in [0, \omega]\), note that the sofa \(S\) is rotated initially by a clockwise angle of \(0\) and finally by \(\omega\) during its movement. By the intermediate value theorem, there is a moment where a copy of \(S\) is rotated clockwise by \(t\) inside \(L\). Push the rotated copy of \(S\) in \(L\) towards the positive \(x\) and \(y\) directions, until it makes contact with the outer walls \(x=1\) and \(y=1\) of \(L\). See this configuration in the perspective of \(S\) to conclude \(S \subseteq L_t\). To observe (2) and (3), note that the sofa \(S\) is initialy in \(L_H\) and finally in \(L_H\) rotated clockwise by \(\omega\); see the configurations in perspective of \(S\) ot conclude that \(S\) is in \(H\) and a translation of \(V_\omega\).

\begin{figure}
\centering
\includesvg[width=1\textwidth,height=\textheight]{images/monotone-sofa-combined.svg}
\caption{The movement of a moving sofa \(S\) with rotation angle \(\omega = \pi/2\) in perspective of the hallway (left) and the sofa (right). The monotone sofa \(S\) is equal to the cap \(K\) minus its niche \(N\). The cap \(K\) is a convex body with the outer walls of \(L_t = L_S(t)\) as the tangent lines of cap \(K\) (and sofa \(S\)). The niche \(N\) is the union of all triangular regions bounded from above by the inner walls of \(L_t = L_S(t)\).}
\label{fig:monotone-sofa}
\end{figure}

Without loss of generality, we can translate the moving sofa \(S\) horizontally and assume that \(S\) is contained in \(V_\omega\), not its arbitrary translation.\footnote{Translating the moving sofa \(S\) may invalidate the initial condition \(S \subseteq L_H\). To resolve this, we relax the \Cref{def:sofa} of a moving sofa \(S\) in so that only some translation of \(S\) is required to be movable from \(L_H\) to \(L_V\) inside \(L\).} Then from (1), (2) and (3) above, we can always assume that \(S\) is contained in the intersection
\begin{equation}
\label{eqn:monotone}
\mathcal{M}(S) := H \cap V_\omega \cap \bigcap_{0 \leq t \leq \omega} L_t
\end{equation}
that we will define as the \emph{monotonization} of a moving sofa \(S\). The set \(\mathcal{M}(S)\) is also movable inside \(L\); observe that \(\mathcal{M}(S)\) is in each rotating hallway \(L_t\) of angle \(t \in [0, \omega]\), and see this in the perspective of the fixed hallway \(L\) instead of the moving hallway \(L_t\) and vary \(t\) from \(0\) to \(\omega\). So the monotonization \(\mathcal{M}(S)\) is a moving sofa containing \(S\) (\Cref{thm:monotonization-is-sofa})\footnote{This uses an implicit assumption that the intersection \(\mathcal{M}(S)\) should be connected. Indeed, \autocite{gerverMovingSofaCorner1992} assumes the connectedness of \(\mathcal{M}(S)\) in its proof of Theorem 1 of \autocite{gerverMovingSofaCorner1992} without further justification. We prove this assumption rigorously in \Cref{thm:monotonization-is-connected}.}, and we call such a sofa \(\mathcal{M}(S)\) from monotonization a \emph{monotone sofa} (\Cref{def:monotone-sofa}). Since any moving sofa \(S\) is contained in a larger monotone sofa \(\mathcal{M}(S)\), we only need to consider monotone sofas for the moving sofa problem.

\subsection{Main result}
\label{sec:main-result}
Assume the maximum-area monotone sofa \(S\) with rotation angle \(\omega \in (0, \pi/2]\). Then we can assume \(S = \mathcal{M}(S)\) in \Cref{eqn:monotone} as taking further monotonization does not increase the area.\footnote{In fact, that \(S\) attains the maximum-area only guarantees that the gain \(\mathcal{M}(S) \setminus S\) of monotonization is of measure zero. We show in \Cref{thm:monotonization-fixpoint} that for any monotone sofa, we have \(\mathcal{M}(S) = S\).} For each angle \(t \in [0, \omega]\), let \(R_t : \mathbb{R}^2 \to \mathbb{R}^2\) be the rotation of \(\mathbb{R}^2\) around the origin, and let \(\mathbf{x}_S(t)\) be the coordinate of the inner corner of the rotated hallway \(L_t\). Then the hallway \(L_t = R_t(L) + \mathbf{x}_S(t)\) rotated counterclockwise by angle \(t \in [0, \pi/2]\) is determined by its inner corner \(\mathbf{x}_S(t)\). So the parametrization \(\mathbf{x}_S : [0, \omega] \to \mathbb{R}^2\) of the inner corner with respect to angle \(t\), defined as the \emph{rotation path} of \(S\) in \autocite{romikDifferentialEquationsExact2018}, determines the monotone sofa
\[
S = H \cap V_\omega \cap \bigcap_{0 \leq t \leq \omega} \left( R_t(L) + \mathbf{x}_S(t) \right) 
\]
completely. In particular, Romik in \autocite{romikDifferentialEquationsExact2018} derived Gerver’s sofa \(S = S_G\) by solving for the local optimality of area of \(S\) as a set of ordinary differential equations on \(\mathbf{x}_S\).

The \emph{injectivity condition} of \(S\) essentially states that the inner corner \(\mathbf{x}_S(t)\) is injective with respect to angle \(t\).

\begin{definition}

(Injectivity condition) A monotone sofa \(S\) with rotation angle \(\omega \in (0, \pi/2]\) satisfies the \emph{injectivity condition}, if its rotation path \(\mathbf{x}_S : [0, \omega] \to \mathbb{R}^2\) is injective and never below the bottom line \(y = 0\) of \(H\) nor the bottom line \(x \cos \omega + y \sin \omega = 0\) of \(V_\omega\).

\label{def:injectivity}
\end{definition}

In particular, Gerver’s sofa \(S_G\) is a monotone sofa satisfying the injectivity condition. Note that the blue trajectory of \(\mathbf{x}_{S_G}\) in \Cref{fig:gerver} is injective and never below the bottom line \(y=0\). For any monotone sofa satisfying the injectivity condition, the following upper bound is established.

\begin{theorem}

(Main theorem) The area of any monotone sofa \(S\) with rotation angle \(\omega \in (0, \pi/2]\) satisfying the injectivity condition is at most \(1 + \omega^2/2\).

\label{thm:main}
\end{theorem}

The upper bound \(1 + \omega^2/2\) of \Cref{thm:main} maximizes at \(\omega = \pi/2\) with the value \(1 + \pi^2/8 = 2.2337\dots\), which is much closer to the lower bound \(2.2195\dots\) of Gerver than the currently best upper bound of \(2.37\) of Kallus and Romik (\Cref{eqn:best-bounds}). We conjecture that a monotone sofa of maximum area should satisfy the premise of \Cref{thm:main}.

\begin{conjecture}

(Injectivity hypothesis) There exists a monotone sofa \(S\) of maximum area with rotation angle \(\omega \in (0, \pi/2]\), satisfying the injectivity condition.

\label{con:injectivity}
\end{conjecture}

With \Cref{thm:main}, proving \Cref{con:injectivity} would imply the unconditional upper bound \(\mu_{\max} \leq 1 + \pi^2/2 = 2.2337\dots\). Since Gerver’s sofa \(S_G\) satisfies the injectivity condition, \Cref{con:injectivity} is a weakening of Gerver’s conjecture \(\mu_{\max} = \mu_G\).

The main idea for proving \Cref{thm:main} is to overestimate the area of a monotone sofa \(S\) (see \Cref{fig:a1-upper-bound}). For the sake of explanation, fix the rotation angle \(\omega = \pi/2\). Observe that the lower boundary of Gerver’s sofa \(S_G\) consists of two red ‘tails’ and one blue ‘core’ in \Cref{fig:gerver}. The core is parametrized by the inner corner \(\mathbf{x}_{S_G}(t)\) for the interval \(t \in [\varphi, \pi/2 - \varphi]\) with constant \(\varphi = 0.0392\dots\) \autocite{romikDifferentialEquationsExact2018}, and forms the majority of the lower boundary. The region below the two tails, trimmed out by the inner left and right walls of \(L_t\) respectively, constitutes only about \(0.28 \%\) of the area \(2.2195\dots\) of the whole sofa. Motivated by this, we define the overestimation \(\mathcal{A}_1\) of the area of a monotone sofa \(S\) with rotation angle \(\pi/2\) as: the area of the convex hull \(K\) of \(S\), that we call the \emph{cap} of \(S\), subtracted by the region enclosed by \(\mathbf{x}_S : [0, \omega] \to \mathbb{R}^2\) and the line \(y=0\).

\begin{figure}
\centering
\includesvg[width=0.7\textwidth,height=\textheight]{images/a1-upper-bound.svg}
\caption{Overestimation \(\mathcal{A}_1(K)\) of the area of a monotone sofa with cap \(K\), including the two red ‘tails’ but excluding the blue ‘core’.}
\label{fig:a1-upper-bound}
\end{figure}

Another key idea for proving \Cref{thm:main} is to consider the area of \(S\) as a function of the cap \(K\). The overestimated area \(\mathcal{A}_1(K)\) in \Cref{fig:a1-upper-bound}, as a function of the cap \(K\), turns out to be \emph{concave} and \emph{quadratic} with respect to \(K\). The space \(\mathcal{K}_\omega\) of all caps \(K\) with a fixed rotation angle \(\omega\) forms a convex space equipped with the convex combination \(c_\lambda(K_1, K_2) = (1 - \lambda) K_1 + \lambda K_2\) defined by the Minkowski sum of convex bodies. Then the Brunn-Minkowski theory on convex bodies \cite{schneider_2013} is used to establish that \(\mathcal{A}_1\) is a quadratic functional on \(\mathcal{K}_\omega\). We use Mamikon’s theorem \autocite{mnatsakanianAnnularRingsEqual1997}, a theorem in classical geometry, to prove that \(\mathcal{A}_1\) concave on \(\mathcal{K}_\omega\) (\Cref{fig:mamikon-sofa}).\footnote{For our application, we rigorously state and prove a generalization (\Cref{thm:mamikon}) of Mamikon’s theorem that is effective on any planar convex body with potentially non-differentiable boundary, which could be of independent interest.} Finally, we introduce a calculus of variation based on the Brunn-Minkowski theory to find a global optimum \(K_1\) of \(\mathcal{A}_1\). \Cref{thm:main} is established by computing the maximum value \(\mathcal{A}_1(K_1) = 1 + \omega^2/2\) of \(\mathcal{A}_1\).

For the rotation angle \(\omega = \pi/2\), the maximizer of \(\mathcal{A}_1\) gives an unmovable sofa \(S_1\) of area \(1 + \pi^2/8 = 2.2337\dots\) and width \(\pi\) very close to the area of Gerver’s sofa \(S_G\) (see the right side of \Cref{fig:presofa}). The shape of \(S_1\) is very close to \(S_G\), and cutting away the region under the red curves from \(S_1\) gives a valid moving sofa of area approximately \(2.2009\dots\), which is again very close to \(S_G\) in its shape and area. The boundary of \(S_1\) consists of three line segments and three parametric curves. The right side of \(S_1\) is parametrized by the curve \(\gamma : [0, \pi/2] \to \mathbb{R}^2\) with \(\gamma(0) = (1, 1)\) and \(\gamma'(t) = t(\cos t, -\sin t)\), which ends at \(\gamma(\pi/2) = (\pi/2, 0)\). The left side of \(S_1\) is parametrized by the curve symmetric to \(\gamma\) along the \(y\)-axis. The bottom middle part of \(S_1\) is parametrized by the curve \(\mathbf{x}_{S_1} : [0, \pi/2] \to \mathbb{R}^2\) with \(\mathbf{x}_{S_1}(0) = (\pi/2-1, 0)\) and
\[
\mathbf{x}_{S_1}'(t) = -t (\cos t, \sin t) + (\pi/2- t) (-\sin t, \cos t)
\]
which is symmetric along the \(y\)-axis. The endpoints of the three curves are connected by three horizontal line segments of length 1 or 2 to form the boundary of \(S_1\).

\subsection{Overview of Sections}
\label{sec:overview-of-sections}
\Cref{sec:notations-and-conventions} contains basic definitions that will be used thoroughout this paper. Since \Cref{sec:notations-and-conventions} is comprehensive, the reader is not expected to read everything in \Cref{sec:notations-and-conventions} at one setting. Instead, the reader can start off by reading the definitions on moving sofas and convex bodies, and later refer to the parts of this section as needed.

\Cref{sec:monotone-sofas} proves \Cref{thm:monotonization-is-sofa} that any moving sofa can be enlarged to a monotone sofa, and proves structural \Cref{thm:monotone-sofa-structure} and \Cref{thm:niche-in-cap} on monotone sofas. Using this, \Cref{sec:sofa-area-functional-a} reduces the moving sofa problem to the maximization of the \emph{sofa area functional} \(\mathcal{A} : \mathcal{K}_\omega \to \mathbb{R}\) on the space of caps \(\mathcal{K}_\omega\). \Cref{sec:conditional-upper-bound-a1} proves the main \Cref{thm:main} by establishing the upper bound \(1 + \omega^2/2\) of the sofa area \(\mathcal{A}\) using the upper bound \(\mathcal{A}_1\). Each section starts with an overview of its own.

\Cref{sec:convex-bodies} proves numerous properties of an arbitrary planar convex body \(K\) that we will use thoroughly in this paper. So the logical ordering of this paper is \Cref{sec:notations-and-conventions}, followed by \Cref{sec:convex-bodies}, then the sections starting \Cref{sec:monotone-sofas}. A logically inclined reader may read in this ordering to verify the correctness of all arguments. On the other hand, readers who are interested in the overall idea may either start by reading the sections in order and refer to the appendix when needed, or skim the introduction of each section of \Cref{sec:convex-bodies} and then read the rest of the paper.



\chapter{Notations and Conventions}
\label{sec:notations-and-conventions}
In this section, we set up basic notations, definitions and conventions that will be assumed thoroughout the rest of the document. We also gather the notions that will be defined later for easier reference. Definitions not listed in this section will be always referenced before its first use in a proof.

This section is comprehensive and not meant to be read in one setting. Instead, the reader may start by reading only the definitions related to moving sofas, shapes, and convex bodies and move on, referring to parts of this section later as needed.
\subsection{Notations}
\label{sec:notations}
Denote the area (Borel measure) of a measurable \(X \subseteq \mathbb{R}^2\) as \(|X|\). For any subset \(X\) of \(\mathbb{R}^2\), denote the topological closure, boundary, and interior as \(\overline{X}\), \(\partial X\), and \(X^\circ\) respectively.

For a subset \(X\) of \(\mathbb{R}^2\) and a vector \(v\) in \(\mathbb{R}^2\), define the set \(X + v = \left\{ x + v : x \in X \right\}\). For any two subsets \(X, Y\) of \(\mathbb{R}^2\), the set \(X + Y = \left\{ x + y : x \in X, y \in Y \right\}\) denotes the Minkowski sum of \(X\) and \(Y\). For any subset \(X\) of \(\mathbb{R}^2\) and a real value \(a\), define the set \(aX = \left\{ ax : a \in X \right\}\).

We use the convention \(S^1 = \mathbb{R} / 2 \pi \mathbb{Z}\). For any function \(f\) on \(S^1\) and any \(t \in \mathbb{R}\), the notation \(f(t)\) denotes the value \(f(t + 2 \pi \mathbb{Z})\). That is, a real value coerces to a value in \(S^1\) when used as an argument of a function that takes a value in \(S^1\). We will often denote an interval of \(S^1\) by its lift under the canonical map \(\mathbb{R} \to \mathbb{R} / 2 \pi \mathbb{Z} = S^1\). Specifically, for any \(t_1 \in \mathbb{R}\) and \(t_2 \in (t_1, t_1 + 2\pi]\), the intervals \((t_1, t_2]\) and \([t_1, t_2)\) of \(\mathbb{R}\) are used to denote the corresponding intervals of \(S^1\) mapped under \(\mathbb{R} \to S^1\). Likewise, for any \(t_1 \in \mathbb{R}\) and \(t_2 \in [t_1, t_1 + 2\pi)\), the interval \([t_1, t_2]\) of \(\mathbb{R}\) is used to denote the corresponding interval of \(S^1\) mapped under \(\mathbb{R} \to S^1\).

For any function \(f : \mathbb{R} \to \mathbb{R}\) or \(f : S^1 \to \mathbb{R}\), \(f(t-)\) denotes the left limit of \(f\) at \(t\) and \(f(t+)\) denotes the right limit of \(f\) at \(t\). For any function \(f : X \to \mathbb{R}\) defined on some open subset \(X\) of either \(\mathbb{R}\) and \(S^1\), and \(t \in X\), define \(\partial^+f(t)\) and \(\partial^-f(t)\) as the right and left differentiation of \(f\) at \(t\) if they exists.

We denote the integral of a measurable function \(f\) with respect to a measure \(\mu\) on a set \(X\) as either \(\int_{x \in X} f(x) \, \mu(dx)\) or \(\left< f, \mu \right>_X\). The latter notation is used especially when we want to emphasize that the integral is bi-linear with respect to both \(f\) and \(\mu\).

\subsection{Moving sofa}
\label{sec:moving-sofa}
\begin{definition}

The \emph{hallway} \(L = L_H \cup L_V\) is the union of sets \(L_H = (-\infty, 1] \times [0, 1]\) and \(L_V = [0, 1] \times (-\infty, 1]\), each representing the horizontal and vertical side of \(L\) respectively.

\label{def:hallway}
\end{definition}

\begin{definition}

Define the unit-width horizontal and vertical strips \(H = \mathbb{R} \times [0, 1]\) and \(V = [0, 1] \times \mathbb{R}\) respectively.

\label{def:strip}
\end{definition}

In the introduction, we gave a definition of a moving sofa \(S\) as a subset of \(L_H\). However, the condition that \(S\) should be confined in \(L_H\) is a bit restrictive for our future use. So we will also call any translation of such \(S \subseteq L_H\) a \emph{moving sofa} as well without loss of generality.

\begin{definition}

A \emph{moving sofa} \(S\) is a connected, nonempty and compact subset of \(\mathbb{R}^2\), such that a translation of \(S\) is a subset of \(L_H\) that admits a continuous rigid motion inside \(L\) from \(L_H\) to \(L_V\).

\label{def:sofa}
\end{definition}

It is safe assume that a moving sofa is always closed, since for any subset of \(L\) its closure is also contained in \(L\). We also define the rotation angle \(\omega\) of a moving sofa \(S\).

\begin{definition}

Say that a moving sofa \(S\) have the \emph{rotation angle} \(\omega \in (0, \pi/2]\) if the continuous rigid motion of a translate of \(S\) from \(L_H\) to \(L_V\) inside \(L\) rotates the body clockwise by \(\omega\) in its full movement.

\label{def:rotation-angle}
\end{definition}

With the result of \autocite{kallusImprovedUpperBounds2018} that \(\omega \in [81.203\dots^\circ, 90^\circ]\) for a maximum-area moving sofa, we will always assume that a moving sofa have rotation angle \(\omega \in (0, \pi/2]\). For each rotation angle \(\omega\), we define the following notions for future use.

\begin{definition}

Define \(R_\theta : \mathbb{R}^2 \to \mathbb{R}^2\) as the rotation map of \(\mathbb{R}^2\) around the origin by a counterclockwise angle of \(\theta \in S^1\).

\label{def:rotation-map}
\end{definition}

\begin{definition}

For any \(\omega \in (0, \pi/2]\), define the \emph{parallelogram} \(P_\omega = H \cap R_\omega(V)\) with \emph{rotation angle} \(\omega\).\footnote{If \(\omega = \pi/2\), then the set \(P_{\pi/2} = H\) is technically not a parallelogram. We will however call it as the parallelogram with rotation angle \(\pi/2\).}

\label{def:parallelogram}
\end{definition}

\begin{definition}

For any \(\omega \in (0, \pi/2]\), define the set \(J_\omega = [0, \omega] \cup [\pi/2, \omega + \pi/2]\).

\label{def:j-cap}
\end{definition}

For reference, the notion of \emph{standard position} of a moving sofa will be defined in \Cref{def:standard-position}. The \emph{monotonization} \(\mathcal{M}(S)\), \emph{cap} \(\mathcal{C}(S)\), and \emph{niche} \(\mathcal{N}(K)\) of a moving sofa \(S\) will be defined in \Cref{def:monotonization}, \Cref{def:cap-sofa}, and \Cref{def:niche} respectively.

\subsection{Convex body}
\label{sec:convex-body}
\begin{definition}

In this paper, a \emph{shape} \(S\) is a nonempty and compact subset of \(\mathbb{R}^2\).

\label{def:shape}
\end{definition}

\begin{definition}

For any angle \(t\) in \(S^1\) or \(\mathbb{R}\), define the unit vectors \(u_t = \left( \cos t, \sin t \right)\) and \(v_t = \left( -\sin t,\cos t \right)\).

\label{def:frame}
\end{definition}

Any line on \(\mathbb{R}^2\) can be described by the angle \(t\) of its normal vector \(u_t\) and its (signed) distance from the origin.

\begin{definition}

For any angle \(t\) in \(S^1\) and a value \(h \in \mathbb{R}\), define the line \(l(t, h)\) with the \emph{normal angle} \(t\) and the signed distance \(h\) from the origin as
\[
l(t, h) = \left\{ p \in \mathbb{R}^2 : p \cdot u_t = h \right\}.
\]

\label{def:line}
\end{definition}

A line on \(\mathbb{R}^2\) divids the plane into two half-planes. Following \Cref{def:line}, we also give a name to one of the half-planes in the direction of \(-u_t\).

\begin{definition}

For any angle \(t\) in \(S^1\) and a value \(h \in \mathbb{R}\), define the closed \emph{half-plane} \(H(t, h)\) with the boundary \(l(t, h)\) as
\[
H(t, h) = \left\{ p \in \mathbb{R}^2 : p \cdot u_t \leq h \right\}.
\]
We say that the closed half-plane \(H(t, h)\) has the \emph{normal angle} \(t\).

\label{def:half-plane}
\end{definition}

Fix a shape \(S\) and angle \(t \in S^1\). Take a sufficiently large \(h \in \mathbb{R}\) so that \(H(t, h) \supseteq S\). As we decrease \(h\) continuously, the line \(l(t, h)\) will get close to \(S\) until it makes contact with \(S\) for the first time. We define the value of \(h\), tangent line \(l(t, h)\), tangent half-plane \(H(t, h)\) as the following \Cref{def:support-function}, \Cref{def:tangent-line} and \Cref{def:tangent-half-plane}.

\begin{definition}

For any shape \(S\), define its \emph{support function} \(p_S : S^1 \to \mathbb{R}\) as the value \(p_S(t) = \sup \left\{ p \cdot u_t : p \in S \right\}\).

\label{def:support-function}
\end{definition}

\begin{definition}

For any shape \(S\) and angle \(t \in S^1\), define the \emph{tangent line} \(l_S(t)\) of \(S\) with \emph{normal angle} \(t\) as the line \(l_S(t) := l(t, p_S(t))\).

\label{def:tangent-line}
\end{definition}

\begin{definition}

For any shape \(S\) and angle \(t \in S^1\), define the \emph{tangent half-plane} \(H_S(t)\) of \(S\) with \emph{normal angle} \(t\) as the line \(H_S(t) := H(t, p_S(t))\).

\label{def:tangent-half-plane}
\end{definition}

Observe that the support function \(p_S(t)\) measures the signed distance from the origin \((0, 0)\) to the tangent line \(l_S(t)\) of \(S\) with the normal vector \(u_t\) directing outwards from \(S\). Support function and tangent lines of \(S\) are usually studied when \(S\) is a convex body (e.g.~p45 of \cite{schneider_2013}), but in this paper we generalize the notion to arbitrary shape \(S\).

The following notion of \emph{width} along a direction is also studied for convex bodies (e.g.~p49 of \cite{schneider_2013}).

\begin{definition}

For any shape \(S\) and angle \(t\) in \(S^1\) or \(\mathbb{R}\), the \emph{width} of \(S\) along the direction of unit vector \(u_t\) is defined as \(p_S(t) + p_S(t + \pi)\).

\label{def:width}
\end{definition}

Geometrically, the width of \(S\) along \(u_t\) measures the distance between the parallel tangent lines \(l_S(t)\) and \(l_S(t + \pi)\) of \(S\).

We adopt the following definition of a convex body (p8 of \cite{schneider_2013}).

\begin{definition}

A \emph{convex body} \(K\) is a nonempty, compact, and convex subset of \(\mathbb{R}^2\).

\label{def:convex-body}
\end{definition}

Many authors often also include the condition that \(K^\circ\) is nonempty, but we allow \(K^\circ\) to be empty (that is, \(K\) can be a closed line segment or a point).

In this paper only, we use the following notions of \emph{vertices} and \emph{edges} of a planar convex body \(K\).

\begin{definition}

For any convex body \(K\) and \(t \in S^1\), define the \emph{edge} \(e_K(t)\) of \(K\) as the intersection of \(K\) with the tangent line \(l_K(t)\).

\label{def:convex-body-edge}
\end{definition}

\begin{definition}

For any convex body \(K\) and \(t \in S^1\), let \(v_K^+(t)\) and \(v_K^-(t)\) be the endpoints of the edge \(e_K(t)\) such that \(v_K^+(t)\) is positioned farthest in the direction of \(v_t\) and \(v_K^-(t)\) is positioned farthest in the opposite direction of \(v_t\). We call \(v_K^{\pm}(t)\) the \emph{vertices} of \(K\).

\label{def:convex-body-vertex}
\end{definition}

It is possible that the edge \(e_K(t)\) can be a single point. In such case, the tangent line \(l_K(t)\) touches \(K\) at the single point \(v_K^+(t) = v_K^-(t)\). In fact, it turns out that this holds for every \(t \in S^1\) except for a countable number of values of \(t\) (\Cref{pro:surface-area-singleton-almost-everywhere}).

\begin{figure}
\centering
\includesvg[width=0.5\textwidth,height=\textheight]{images/convex-body.svg}
\caption{A convex body \(K\) with its edge, vertices, tangent line, and half-plane.}
\label{fig:convex-body}
\end{figure}

For reference, the notion of cap \(K\) as a kind of convex bodies is defined in \Cref{def:cap}. The space of all caps \(\mathcal{K}_\omega\) with rotation angle \(\omega\) is defined in \Cref{def:cap-space}. The vertices \(A_K^{\pm}(t)\), \(C_K^{\pm}(t)\) of a cap \(K\) is defined in \Cref{def:cap-vertices}. The upper boundary \(\delta K\) of a cap \(K\) is defined in \Cref{def:upper-boundary-of-cap}.

\subsection{Area functionals}
\label{sec:area-functionals}
Refer to \autocite{steinRealAnalysisMeasure2005} for the standard notion of \emph{Lipschitz}, \emph{bounded variation} and \emph{absolute continuity} of a single-variable, real-valued function. Recall that a continuous curve with parametrization \(\mathbf{x} : [a, b] \to \mathbb{R}^2\) is \emph{rectifiable} if and only if its \(x\) and \(y\) coordinates are both of bounded variation (Theorem 3.1 of \autocite{steinRealAnalysisMeasure2005}). We now define the \emph{curve area functional} \(\mathcal{I}(\mathbf{x})\) of \(\mathbf{x}\).

\begin{definition}

For two points \((a, b), (c, d) \in \mathbb{R}^2\), denote their cross product as \((a, b) \times (c, d) = ad - bc \in \mathbb{R}\).

\label{def:planar-cross-product}
\end{definition}

\begin{definition}

Let \(\Gamma\) be any curve equipped with a rectifiable parametrization \(\mathbf{x} : [a, b] \to \mathbb{R}^2\). With \(\mathbf{x}(t) = (x(t), y(t))\), define the \emph{curve area functional}
\[
\mathcal{I}(\mathbf{x}) := \frac{1}{2} \int_a^b \mathbf{x}(t) \times d\mathbf{x}(t) := \frac{1}{2} \int_a^b x(t) dy(t) - y(t) dx(t)
\]
of curve \(\Gamma\).

\label{def:curve-area-functional}
\end{definition}

The integral in \Cref{def:curve-area-functional} is the Lebesgue-Stieltjes integral, for which we again refer to \autocite{steinRealAnalysisMeasure2005}. By change of variables (e.g.~Equation 2 of \autocite{falknerSubstitutionRuleLebesgue2012}), the value of \(\mathcal{I}(\mathbf{x}) = \mathcal{I}(\mathbf{x} \circ \alpha)\) is the same even if we replace \(\mathbf{x}\) with a reparametrization \(\mathbf{x} \circ \alpha : [a', b'] \to \mathbb{R}^2\) of curve \(\Gamma\), where \(\alpha : [a', b'] \to [a, b]\) is any monotonically increasing, continuous, and surjective function. In particular, for any parametrization \(\mathbf{x}\) of the line segment from point \(p\) to \(q\), its curve area functional \(\mathcal{I}(\mathbf{x})\) is equal to \(1/2 \cdot (p \times q)\).

\begin{definition}

Write \(\mathcal{I}(p, q)\) for the area functional of the line segment connecting the point \(p\) to \(q\), so that we have \(\mathcal{I}(p, q) = 1/2 \cdot (p \times q)\).

\label{def:curve-area-functional-segment}
\end{definition}

For reference, the function \(\mathcal{A}(K)\) is the \emph{sofa area functional} on cap \(K\) defined in \Cref{def:sofa-area-functional}. \(\mathcal{A}_1(K)\) is a conditional upper bound of \(\mathcal{A}(K)\) defined in \Cref{def:a1}.

\subsection{Parts of hallway}
\label{sec:parts-of-hallway}
We give names to the different parts of the hallway \(L\) for future reference.

\begin{definition}

Let \(\mathbf{x} = (0, 0)\) and \(\mathbf{y} = (1, 1)\) be the inner and outer corner of \(L\) respectively.

\label{def:hallway-corners}
\end{definition}

\begin{definition}

Let \(a\) and \(c\) be the lines \(x=1\) and \(y=1\) representing the \emph{outer walls} of \(L\) passing through \(\mathbf{y}\). Let \(b\) and \(d\) be the half-lines \((-\infty, 0] \times \left\{ 0 \right\}\) and \(\left\{ 0 \right\} \times (-\infty, 0]\) representing the \emph{inner walls} of \(L\) emanating from the inner corner \(\mathbf{x}\).

\label{def:hallway-walls}
\end{definition}

\begin{definition}

Let \(Q^+ = (-\infty, 1]^2\) be the closed quarter-plane bounded by outer walls \(a\) and \(c\). Let \(Q^- = (-\infty, 0)^2\) be the open quarter-plane bounded by inner walls \(b\) and \(d\), so that \(L = Q^+ \setminus Q^-\).

\label{def:hallway-regions}
\end{definition}

\begin{figure}
\centering
\includesvg[width=0.4\textwidth,height=\textheight]{images/hallway-detailed.svg}
\caption{The standard hallway \(L\) and its parts.}
\label{fig:hallway-detailed}
\end{figure}

For reference, the \emph{tangent hallway} \(L_S(t)\) of a shape \(S\) will be defined in \Cref{def:tangent-hallway}. The corresponding parts \(\mathbf{x}_S(t)\), \(\mathbf{y}_S(t)\), \(a_S(t)\), \(b_S(t)\), \(c_S(t)\), \(d_S(t)\), \(Q^+_S(t)\), \(Q^-_S(t)\) of the tangent hallway will be defined in \Cref{def:rotating-hallway-parts}.



\chapter{Monotone Sofas}
\label{sec:monotone-sofas}
In this section, we rigorously define what is a \emph{monotone sofa}. \Cref{thm:monotonization-is-sofa} shows that the process named \emph{monotonization} enlarges any moving sofa \(S\) to a larger moving sofa \(\mathcal{M}(S)\) as described in \Cref{eqn:monotone} of \Cref{sec:moving-hallway-problem}. A monotone sofa is then simply defined as the monotonization \(\mathcal{M}(S)\) of some moving sofa \(S\) (\Cref{def:monotone-sofa}). Proving the connectedness of \(\mathcal{M}(S)\) (\Cref{thm:monotonization-is-connected}) will be the key step in establishing \Cref{thm:monotonization-is-sofa}.

\Cref{thm:monotonization-structure} shows that any monotone sofa \(S\) is equal to \(K \setminus \mathcal{N}(K)\), where \(K = \mathcal{C}(S)\) is a convex set called the \emph{cap of} \(S\) (\Cref{thm:cap-hallway-intersection}), and \(\mathcal{N}(K)\) is a subset of \(K\) called the \emph{niche} determined by the cap \(K\) (\Cref{def:niche}). Then we show that the niche \(\mathcal{N}(K)\) is always contained in the cap \(K\) of sofa \(S\) (\Cref{thm:niche-in-cap}). With this, the area \(|S| = |K| - |\mathcal{N}(K)|\) of a monotone sofa \(S\) can be understood in terms of cap \(K\) and niche \(\mathcal{N}(K)\) separately.
\section{Tangent Hallway}
\label{sec:tangent-hallway}
\subsection{Tangent Hallway}

Define the \emph{tangent hallways} for a shape \(S\) (that is, any nonempty compact subset \(S\) of \(\mathbb{R}^2\) by \Cref{def:shape}).

\begin{definition}

For any shape \(S\) and angle \(t \in S^1\), define the \emph{tangent hallway} \(L_S(t)\) of \(S\) with angle \(t\) as
\[
L_S(t) = R_t(L) + (p_S(t) - 1)  u_t + (p_S(t + \pi/2) - 1) v_t.
\]

\label{def:tangent-hallway}
\end{definition}

Note that \(R_t\) is the rotation of \(\mathbb{R}^2\) along the origin by a counterclockwise angle of \(t\) (\Cref{def:rotation-map}). The equation of \(L_S(t)\) in \Cref{def:tangent-hallway} is determined by the following defining property of \(L_S(t)\).

\begin{definition}

Here, a \emph{rigid transformation} \(f : \mathbb{R}^2 \to \mathbb{R}^2\) on \(\mathbb{R}^2\) is the composition \(z \mapsto R_t(z) + q\) of translation by a vector \(q \in \mathbb{R}^2\) and rotation by an angle \(t \in S^1\) along the origin. We also say that a shape \(S'\) is a rigid transformation \emph{of} another shape \(S\) if there exists a rigid transformation \(f : \mathbb{R}^2 \to \mathbb{R}^2\) such that \(S' = f(S)\).

\label{def:rigid-transformation}
\end{definition}

\begin{proposition}

For any shape \(S\) and angle \(t \in S^1\), the tangent hallway \(L_S(t)\) is the unique rigid transformation of \(L\) rotated counterclockwise by \(t\), such that the outer walls of \(L_S(t)\) corresponding to the outer walls \(a\) and \(c\) of \(L\) are the tangent lines \(l_S(t)\) and \(l_S(t + \pi/2)\) of \(S\) respectively.

\label{pro:tangent-hallway}
\end{proposition}

\begin{proof}
Let \(c_1\) and \(c_2\) be arbitrary real values. Then \(L' = R_t(L) + c_1 u_t + c_2 v_t\) is an arbitrary rigid transformation of \(L\) rotated counterclockwise by \(t\). The outer walls of \(L'\) corresponding to the outer walls \(a\) and \(c\) of \(L\) (\Cref{def:hallway-walls}) are \(l(t, c_1 + 1)\) and \(l(t + \pi/2, c_2 + 1)\) respectively. They match with the tangent lines \(l_S(t) = l(t, p_S(t))\) and \(l_S(t + \pi/2) = l(t + \pi/2, p_S(t + \pi/2))\) of \(S\) if and only if \(c_1 = p_S(t) - 1\) and \(c_2 = p_S(t + \pi/2) - 1\). That is, if and only if \(L' = L_S(t)\).
\end{proof}

Name the parts of tangent hallway \(L_S(t)\) according to the parts of \(L\) (\Cref{def:hallway-corners}, \Cref{def:hallway-walls}, and \Cref{def:hallway-regions}) for future use.

\begin{definition}

For any shape \(S\) and angle \(t \in S^1\), define the rigid transformation \(f_{S, t} : \mathbb{R}^2 \to \mathbb{R}^2\) as
\[
f_{S, t}(z) = R_t(z) + (p_S(t) - 1)  u_t + (p_S(t + \pi/2) - 1) v_t
\]
so that \(f_{S, t}\) maps \(L\) to \(L_S(t)\).

\label{def:tangent-hallway-map}
\end{definition}

\begin{definition}

For any shape \(S\) and angle \(t \in S^1\), let \(\mathbf{x}_S(t), \mathbf{y}_S(t), a_S(t), b_S(t), c_S(t), d_S(t), Q^+_S(t), Q^-_S(t)\) be the parts of \(L_S(t)\) corresponding to the parts \(\mathbf{x}, \mathbf{y}, a, b, c, d, Q^+, Q^-\) of \(L\) respectively. That is, for any \(? = \mathbf{x}, \mathbf{y}, a, b, c, d, Q^+, Q^-\), let \(?_S(t) := f_{S, t}(?)\).

\label{def:rotating-hallway-parts}
\end{definition}

\begin{proposition}

We have \(L_S(t) = Q_S^+(t) \setminus Q_S^-(t)\) and \(Q^+_S(t) = H_S(t) \cap H_S(t + \pi/2)\). Also we have the following representations of the parts purely in terms of the supporting function \(p_S\) of \(S\).

\begin{gather*}
\mathbf{x}_S(t) = (p_S(t) - 1) u_t + (p_S(t + \pi/2) - 1) v_t \\
\mathbf{y}_S(t) = p_S(t) u_t + p_S(t + \pi/2) v_t \\
a_S(t) = l_S(t) = l(t, p_S(t)) \\
b_S(t) \subseteq l(t, p_S(t) - 1) \\
c_S(t) = l_S(t + \pi/2) = l(t + \pi/2, p_S(t + \pi/2)) \\
d_S(t) \subseteq l(t + \pi/2, p_S(t + \pi/2) - 1) \\
Q_S^+(t) = H(t, p_S(t)) \cap H(t + \pi/2, p_S(t + \pi/2)) \\
Q_S^-(t) = H(t, p_S(t) - 1)^{\circ} \cap H(t + \pi/2, p_S(t + \pi/2))^{\circ}
\end{gather*}

\label{pro:rotating-hallway-parts}
\end{proposition}

\begin{proof}
The formulas for \(\mathbf{x}_S(t)\) and \(\mathbf{y}_S(t)\) are obtained by letting \(z\) equal to \(\mathbf{x} = (0, 0)\) or \(\mathbf{y} = (1, 1)\) in the equation of \Cref{def:tangent-hallway-map}. The formulas for \(a_S(t), b_S(t), c_S(t)\), and \(d_S(t)\) follows from the proof of \Cref{pro:tangent-hallway}. The equality \(L_S(t) = Q_S^+(t) \setminus Q_S^-(t)\) follows from mapping \(L = Q^+ \setminus Q^-\) under the transformation \(f_{S, t}\). The equality \(Q^+_S(t) = H_S(t) \cap H_S(t + \pi/2)\) follows from that \(Q^+_S(t)\) is a cone bounded by tangent lines \(a_S(t) = l_S(t)\) and \(c_S(t) = l_S(t + \pi/2)\) as in the proof of \Cref{pro:tangent-hallway}. The formulas for \(Q_S^-(t)\) and \(Q_S^+(t)\) in terms of \(p_S\) now follow from \Cref{def:tangent-half-plane} and that \(Q_S^-(t)\) is bounded by \(b_S(t)\) and \(d_S(t)\).
\end{proof}

Assume that a rigid transformation \(L'\) of \(L\) rotated counterclockwise by an angle of \(t \in S^1\) contains a shape \(S\). By translating the outer walls of \(L'\) towards \(S\) until they make contact with \(S\), we can see that the tangent hallway \(L_S(t)\) also contains \(S\).

\begin{proposition}

Let \(S\) be any shape contained in a translation of \(R_t(L)\) with angle \(t \in S^1\). Then the tangent hallway \(L_S(t)\) with angle \(t\) also contains \(S\).

\label{pro:tangent-hallway-contains}
\end{proposition}

\begin{proof}
Assume that the translation \(L'\) of \(R_t(L)\) contains \(S\). Then while keeping \(S\) inside \(L'\), we can push \(L'\) towards \(S\) in the directions \(-u_t\) and \(-v_t\) until the outer walls of the final \(L' = L_S(t)\) make contact with \(S\). The pushed hallway \(L_S(t)\) still contains \(S\) because the directions \(-u_t\) and \(-v_t\) of the movement only push the inner walls of \(L'\) away from \(S\).
\end{proof}

\subsection{Moving Hallway Problem}

By our \Cref{def:sofa} of a moving sofa \(S\), any translation of \(S\) is also a valid moving sofa. Without loss of generality, we will always assume that a moving sofa \(S\) is in \emph{standard position} by translating it.

\begin{definition}

A moving sofa \(S\) with rotation angle \(\omega \in (0, \pi/2]\) is in \emph{standard position} if \(p_S(\omega) = p_S(\pi/2) = 1\).

\label{def:standard-position}
\end{definition}

\begin{proposition}

For any angle \(\omega \in (0, \pi/2]\) and shape \(S\), there is a translation \(S'\) of \(S\) such that \(p_{S'}(\omega) = p_{S'}(\pi/2) = 1\) which is (i) unique if \(\omega < \pi/2\), or (ii) unique up to horizontal translations if \(\omega = \pi/2\).

\label{pro:standard-position-shape}
\end{proposition}

\begin{proof}
Since the support function \(p_{S'}(t)\) measures the signed distance from origin to tangent line \(l_{S'}(t)\) (see the remark above \Cref{def:support-function}), the translation \(S'\) of \(S\) satisfies the condition \(p_{S'}(\omega) = p_{S'}(\pi/2) = 1\) if and only if the lines \(l(\omega, 1)\) and \(l(\pi/2, 1)\) are tangent to \(S'\) and \(S'\) is below the lines. Translate \(S\) below the lines \(l(\omega, 1)\) and \(l(\pi/2, 1)\) so that it makes contact with the two lines. If \(\omega < \pi/2\), then the constraints determine the unique location of \(S'\). If \(\omega = \pi/2\), then the two lines are equal to the horizontal line \(y=1\), and \(S'\) can move freely horizontally as long as the line \(y=1\) makes contact with \(S'\) from above.
\end{proof}

Assume any moving sofa \(S\) with rotation angle \(\omega \in (0, \pi/2]\). By \Cref{pro:standard-position-shape} any moving sofa can be put in standard position by translating it. Gerver also observed in \autocite{gerverMovingSofaCorner1992} that \(S\) should be contained in the tangent hallways \(L_S(t)\) for all \(t \in [0, \omega]\) (\Cref{pro:tangent-hallway-contains}). We summarize the full details of Gerver’s observation (line 18-22, p269; line 24-31, p270 of \autocite{gerverMovingSofaCorner1992}) in the following theorem.

\begin{theorem}

Let \(\omega \in (0, \pi/2]\) be an arbitrary angle. For a connected shape \(S\), the following conditions are equivalent.

\begin{enumerate}
\def\labelenumi{\arabic{enumi}.}
\tightlist
\item
  \(S\) is a moving sofa with rotation angle \(\omega\).
\item
  \(S\) is contained in a translation of \(H\) and \(R_\omega(V)\). Also, for every \(t \in [0, \omega]\), \(S\) is contained in a translation of \(R_t(L)\), the hallway rotated counterclockwise by an angle of \(t\).
\item
  Let \(S'\) be any translation of \(S\) such that \(p_{S'}(\omega) = p_{S'}(\pi/2) = 1\). Then (i) \(S' \subseteq P_\omega\) (\Cref{def:parallelogram}), (ii) \(S' \subseteq L_{S'}(t)\) for all \(t \in [0, \omega]\), and (iii) \(S'\) is a moving sofa with rotation angle \(\omega\) in standard position.
\end{enumerate}

\label{thm:moving-sofa-iff-hallway-intersection}
\end{theorem}

\begin{proof}
(1 \(\Rightarrow\) 2) Consider the movement of \(S\) inside the hallway \(L\). For any angle \(t \in [0, \omega]\), there is a moment where the sofa \(S\) is rotated clockwise by an angle of \(t\) inside \(L\), by the intermediate value theorem on the angle of rotation of \(S\) inside \(L\). Viewing this from the perspective of the sofa \(S\), \(S\) is contained in some translation of \(L\) rotated \emph{counterclockwise} by an arbitrary \(t \in [0, \omega]\). Likewise, by looking at the initial (resp. final) position of \(S\) inside \(L_H\) (resp. \(L_V\)) from the perspective of \(S\), the set \(S\) should be contained in a translation of \(H\) and \(R_\omega(V)\) respectively.

(2 \(\Rightarrow\) 3) Take any \(S\) satisfying (2) and its arbitrary translation \(S'\) satisfying \(p_{S'}(\omega) = p_{S'}(\pi/2) = 1\) which is the premise of (3). Then the translate \(S'\) of \(S\) also satisfies (2). So without loss of generality, we can simply assume \(S' = S\) and show (i), (ii) and (iii). Since \(S\) is contained in a translation of \(H\) and \(R_\omega(V)\), the width of \(S\) along the direction of \(u_\omega\) and \(v_0\) (\Cref{def:width}) are at most 1. So \(p_S(\omega) = p_S(\pi/2) = 1\) implies (i) \(S \subseteq P_\omega\). \Cref{pro:tangent-hallway-contains} implies (ii) \(S \subseteq L_S(t)\). It remains to show (iii) that \(S\) is a moving sofa.

Because the support function \(p_S(t)\) of \(S\) is continuous, the tangent hallway \(L_S(t)\) moves continuously with respect to \(t\) by \Cref{def:tangent-hallway}. For every \(t \in [0, \omega]\), let \(g_t := f_{S, t}^{-1}\) be the unique rigid transformation that maps \(L_S(t)\) to \(L\). Then the rigid transformation \(S_t := g_t(S)\) of \(S\) also changes continuously with respect to \(t\). Because \(L_S(0)\) is a translation of \(L\) by letting \(t=0\) in \Cref{def:tangent-hallway}, \(g_0\) is a translation and so \(S_0\) is a translation of \(S\). Mapping \(S \subseteq L_S(t)\) under \(g_t\) we have \(S_t \subseteq L\). So \(S_t\) over the angle \(t \in [0, \omega]\) as time is a continuous movement of a translation \(S_0\) of \(S\) inside \(L\).

It remains to show that \(S_0 \subseteq H\) and \(S_\omega \subseteq V\). Because \(p_S(\pi/2) = 1\), \(L_S(0)\) is a translation of \(L\) along the direction \(u_0\), and the map \(g_0\) is also a translation along the direction \(u_0\). Because \(S \subseteq H\), we also have \(S_0 = g_0(S) \subseteq H\). Likewise, since \(p_S(\omega) = 1\) the hallway \(L_S(\omega)\) is a translation of \(L\) along the direction \(v_\omega\). So the map \(g_\omega\) is the composition of a translation along the direction \(v_\omega\) and \(R_{-\omega}\). Because \(S \subseteq R_\omega(V)\), we also have \(S_\omega = g_\omega(S) \subseteq g_\omega(R_\omega(V)) = V\).

(3 \(\Rightarrow\) 1) By \Cref{pro:standard-position-shape}, any connected shape \(S\) have a translation \(S'\) that satisfies the premise \(p_{S'}(\omega) = p_{S'}(\pi/2) = 1\) of (3). So \(S'\) is a moving sofa by (3), and its translation \(S\) is a moving sofa as well.
\end{proof}

\section{Monotonization}
\label{sec:monotonization}
We now define the notion of monotone sofas and establish \Cref{thm:monotonization-is-sofa} that a moving sofa is contained in a larger monotone sofa. Define the \emph{monotonization} of any moving sofa \(S\) in standard position as the following set.

\begin{definition}

Let \(S\) be any moving sofa with rotation angle \(\omega \in (0, \pi/2]\) in standard position. The \emph{monotonization} of \(S\) is the intersection
\[
\mathcal{M}(S) = P_\omega \cap \bigcap_{0 \leq t \leq \omega} L_S(t).
\]

\label{def:monotonization}
\end{definition}

Compare the equation in \Cref{def:monotonization} to \Cref{eqn:monotone} in \Cref{sec:moving-hallway-problem}. The paralleogram \(P_\omega\) is the intersection of \(H\) and \(V_\omega\) (\Cref{def:parallelogram}), and the tangent hallways \(L_S(t)\) are the rotating hallways \(L_t\) making contact with \(S\) in the outer walls as described in \Cref{sec:moving-hallway-problem}. Condition 3 of \Cref{thm:moving-sofa-iff-hallway-intersection} implies that the set \(\mathcal{M}(S)\) contains \(S\).

\begin{corollary}

\(\mathcal{M}(S) \supseteq S\) for any moving sofa \(S\) in standard position.

\label{cor:monotonization-is-larger}
\end{corollary}

We will establish the connectedness of \(\mathcal{M}(S)\).

\begin{theorem}

Let \(S\) be a moving sofa with rotation angle \(\omega \in (0, \pi/2]\) in standard position. Then the monotonization \(\mathcal{M}(S)\) is connected.

\label{thm:monotonization-is-connected}
\end{theorem}

Once the connectedness of \(\mathcal{M}(S)\) is established, we can immediately show that the monotonization \(\mathcal{M}(S)\) is a moving sofa containing the initial moving sofa \(S\).

\begin{theorem}

Let \(S\) be any moving sofa with rotation angle \(\omega \in (0, \pi/2]\) in standard position. The monotonization \(\mathcal{M}(S)\) of \(S\) is a moving sofa containing \(S\) with the same rotation angle \(\omega\) in standard position.

\label{thm:monotonization-is-sofa}
\end{theorem}

\begin{proof}
By \Cref{thm:monotonization-is-connected}, the shape \(\mathcal{M}(S)\) is connected. By \Cref{def:monotonization}, the set \(\mathcal{M}(S)\) is contained \(P_\omega\) and \(L_S(t)\) for all \(t \in [0, \omega]\), so it satisfies the second condition of \Cref{thm:moving-sofa-iff-hallway-intersection}. So the set \(\mathcal{M}(S)\) is a moving sofa with rotation angle \(\omega\). \(\mathcal{M}(S)\) contains \(S\) by \Cref{cor:monotonization-is-larger}. From \(S \subseteq \mathcal{M}(S) \subseteq P_\omega\) and
\[
p_S(\omega) = p_{P_\omega}(\omega) = p_S(\pi/2) = p_{P_\omega}(\pi/2) = 1
\]
we have \(p_{\mathcal{M}(S)}(\omega) = p_{\mathcal{M}(S)}(\pi/2) = 1\). So \(\mathcal{M}(S)\) is in standard position.
\end{proof}

Now any monotonization of a moving sofa is also a moving sofa. Call the resulting monotonization a \emph{monotone sofa}.

\begin{definition}

A \emph{monotone sofa} is the monotonization of some moving sofa with rotation angle \(\omega \in (0, \pi/2]\) in standard position.

\label{def:monotone-sofa}
\end{definition}

The moving sofa problem asks for the largest-area moving sofas. So \Cref{thm:monotonization-is-sofa} tells us that we only need to consider monotone sofas for the problem. The rest of this \Cref{sec:monotonization} proves \Cref{thm:monotonization-is-connected} as promised.

\subsubsection{\texorpdfstring{Proof of \Cref{thm:monotonization-is-connected}}{Proof of }}

We prepare the following terminologies.

\begin{definition}

Let \(S\) be any moving sofa with rotation angle \(\omega \in (0, \pi/2]\) in standard position. Define the set
\[
\mathcal{C}(S) = P_\omega \cap \bigcap_{0 \leq t \leq \omega} Q^+_S(t).
\]

\label{def:cap-sofa}
\end{definition}

The set \(\mathcal{C}(S)\) will later be called as the \emph{cap} of \(S\) (\Cref{thm:cap-hallway-intersection}) after defining the notion of cap in \Cref{def:cap}. We don’t need this notion of cap as of now.

\begin{definition}

Say that a set \(X \subseteq \mathbb{R}^2\) is \emph{closed in the direction of} vector \(v \in \mathbb{R}^2\) if, for any \(x \in X\) and \(\lambda \geq 0\), we have \(x + \lambda v \in X\).

\label{def:closed-in-direction}
\end{definition}

\begin{definition}

Any line \(l\) of \(\mathbb{R}^2\) divides the plane into two half-planes. If \(l\) is not parallel to the \(y\)-axis, call the \emph{left side} (resp. \emph{right side}) of \(l\) as the closed half-plane with boundary \(l\) containing the point \(- Nu_0\) (resp. \(Nu_0\)) for a sufficiently large \(N\). If a point \(p\) is on the left (resp. right) side of \(l\) and not on the boundary \(l\), we say that \(p\) is \emph{strictly on the left} (resp. \emph{right}) \emph{side} of \(l\).

\label{def:line-half-plane-directions}
\end{definition}

We also prepare a lemma.

\begin{lemma}

Let \(S\) be any moving sofa with rotation angle \(\omega \in [0, \pi/2]\) in standard position. Then the support functions \(p_S\), \(p_{\mathcal{M}(S)}\), and \(p_{\mathcal{C}(S)}\) of \(S\), \(\mathcal{M}(S)\) and \(\mathcal{C}(S)\) agree on the set \(J_\omega\).

\label{lem:cap-same-support-function}
\end{lemma}

\begin{proof}
We have \(S \subseteq \mathcal{M}(S) \subseteq \mathcal{C}(S)\) by \Cref{cor:monotonization-is-larger} and \(L_S(t) \subset Q_S^+(t)\). So it remains to show \(p_{\mathcal{C}(S)}(t) \leq p_S(t)\) for every \(t\) in \(J_\omega\). By the definition of \(\mathcal{C}(S)\) we have \(S \subseteq \mathcal{C}(S) \subseteq H_S(t)\). So we have \(p_{\mathcal{C}(S)}(t) \leq p_S(t)\) indeed.
\end{proof}

A moving sofa \(S\) and its cap \(\mathcal{C}(S)\) shares the same tangent hallways \(L_S(t) = L_{\mathcal{C}(S)}(t)\).

\begin{proposition}

For any moving sofa \(S\) with rotation angle \(\omega \in [0, \pi/2]\) in standard position, the tangent hallway \(L_S(t)\) of \(S\) and the tangent hallway \(L_K(t)\) of set \(K = \mathcal{C}(S)\) are equal for every \(t \in [0, \omega]\).

\label{pro:cap-same-tangent-hallway}
\end{proposition}

\begin{proof}
The tangent hallways \(L_X(t)\) of \(X = S, K\) depend solely on the values of the support function \(p_X\) of \(X\) on \(J_\omega\), by the equation of \(L_X(t)\) in \Cref{def:tangent-hallway}. The support functions of \(S\) and \(K\) match on the set \(J_\omega\) by \Cref{lem:cap-same-support-function}, so the result follows.
\end{proof}

We are now ready to show that \(\mathcal{M}(S)\) is connected.

\begin{proof}[Proof of \Cref{thm:monotonization-is-connected}]
Define the set \(X := \bigcup_{0 \leq t \leq \omega} Q^-_S(t)\). By plugging the equation \(L_S(t) = Q_S^+(t) \setminus Q_S^-(t)\) to \Cref{def:monotonization}, we have \(\mathcal{M}(S) = \mathcal{C}(S) \setminus X\). Observe that \(\mathcal{C}(S)\) is a convex body containing \(S\) (say, by \Cref{cor:monotonization-is-larger} and \(\mathcal{M}(S) \subseteq \mathcal{C}(S)\)).

Fix an arbitrary point \(p\) in \(\mathcal{M}(S)\). Take an arbitrary angle \(\theta \in [\omega, \pi/2]\). Observe that the set \(X = \bigcup_{t \in [0, \omega]} Q^-_S(t)\) is closed in the direction of \(-u_\theta\) (\Cref{def:closed-in-direction}) for all angle \(\theta \in [\omega, \pi/2]\), since each \(Q_S^-(t)\) is closed in the direction of \(-u_\theta\). Take the line \(l_\theta\) passing the point \(p\) in the direction of \(u_\theta\). The set \(s_\theta = l_\theta \cap \mathcal{M}(S)\) contains \(p\), and \(s_\theta\) is a nonempty line segment because \(s_\theta\) is the line segment \(l_\theta \cap \mathcal{C}(S)\) minus the half-line \(l_\theta \setminus X\). If the line \(l_\theta\) meets \(S\) for any \(\theta \in [\omega, \pi/2]\), then \(p\) is connected to \(S\) along the line segment \(s_\theta\) inside \(\mathcal{M}(S)\) and the proof is done. Our goal now is to prove that there is some \(\theta \in [\omega, \pi/2]\) such that \(l_\theta\) meets \(S\).

Assume by contradiction that for every \(\theta \in [\omega, \pi/2]\) the line \(l_\theta\) is disjoint from \(S\). By \Cref{lem:cap-same-support-function}, we have \(l_{\mathcal{M}(S)}(t) = l_S(t)\) for every \(t \in J_\omega = [0, \omega] \cup [\pi/2, \pi/2 + \omega]\). Because \(p \in \mathcal{M}(S)\), the line \(l_{\pi/2}\) passing through \(p\) is either equal to \(l_{\mathcal{M}(S)}(0) = l_S(0)\) or strictly on the left side of \(l_{S}(0)\). If \(l_{\pi/2} = l_S(0)\) then \(l_{\pi/2}\) contains some point of \(S\) contradicting our assumption. So the line \(l_{\pi/2}\) is strictly on the left side of \(l_{S}(0)\), and there is a point of \(S\) strictly on the right side of \(l_{\pi/2}\). Likewise, as \(p \in \mathcal{M}(S)\), the line \(l_{\omega}\) that passes through \(p\) is either equal to \(l_{\mathcal{M}(S)}(\omega + \pi/2) = l_S(\omega + \pi/2)\) or strictly on the right side of \(l_S(\omega + \pi/2)\). The line \(l_\omega\) cannot be equal to \(l_S(\omega + \pi/2)\) because we assumed that \(l_\omega\) is disjoint from \(S\). So the line \(l_{\omega}\) is strictly on the right side of \(l_S(\omega + \pi/2)\), and there is a point of \(S\) strictly on the left side of \(l_{\omega}\).

Because the line \(l_\theta\) is disjoint from \(S\) for any \(\theta \in [\omega, \pi/2]\), the set \(S\) is inside the set \(Y = \mathbb{R}^2 \setminus \bigcup_{\theta \in [\omega, \pi/2]} l_\theta\). Note that \(Y\) has exactly two connected components \(Y_L\) and \(Y_R\) on the left and the right side of the lines \(l_\theta\) respectively. As there is a point of \(S\) strictly on the right side of \(l_{\pi/2}\), the set \(S \cap Y_R\) is nonempty. As there is also a point of \(S\) strictly on the left side of \(l_\omega\), the set \(S \cap Y_L\) is also nonempty. We get contradiction as \(S\) should be a connected subset of \(Y\).
\end{proof}

\section{Structure of a Monotone Sofa}
\label{sec:structure-of-a-monotone-sofa}
Here, we show that any monotone sofa \(S\) is always equal to a \emph{cap} \(K\) minus its \emph{niche} \(\mathcal{N}(K)\) (\Cref{thm:monotone-sofa-structure}; see \Cref{fig:monotone-sofa} in \Cref{sec:moving-hallway-problem}).

Define a \emph{cap} as a convex body satisfying certain properties.

\begin{definition}

A \emph{cap} \(K\) with \emph{rotation angle} \(\omega \in (0, \pi/2]\) is a convex body such that the followings hold.

\begin{enumerate}
\def\labelenumi{\arabic{enumi}.}
\tightlist
\item
  \(p_K(\omega) = p_K(\pi/2) = 1\) and \(p_K(\pi + \omega) = p_K(3\pi/2) = 0\).
\item
  \(K\) is an intersection of closed half-planes with normal angles (\Cref{def:half-plane}) in \(J_\omega \cup \{\pi + \omega, 3\pi/2\}\).
\end{enumerate}

\label{def:cap}
\end{definition}

Geometrically, the first condition of \Cref{def:cap} states that \(K\) is contained in the parallelogram \(P_\omega\) making contact with all sides of \(P_\omega\). By \Cref{thm:convex-set-support}, the second condition of \Cref{def:cap} is equivalent to saying that the \emph{normal angles} \(\mathbf{n}(K)\) of \(K\) (\Cref{def:convex-set-support}) is contained in the set \(J_\omega \cup \{\pi + \omega, 3\pi/2\}\). See \Cref{sec:normal-angles} for a quick overview of \(\mathbf{n}(K)\).

We will show that the set \(\mathcal{C}(S)\) in \Cref{def:cap-sofa} is a cap with rotation angle \(\omega\). This justifies calling \(\mathcal{C}(S)\) \emph{the cap of} \(S\) associated to \(S\).

\begin{theorem}

The set \(\mathcal{C}(S)\) in \Cref{def:cap-sofa} is a cap with rotation angle \(\omega\) as in \Cref{def:cap}. With this, call \(\mathcal{C}(S)\) the \emph{cap of the moving sofa} \(S\).

\label{thm:cap-hallway-intersection}
\end{theorem}

We postpone the proof of \Cref{thm:cap-hallway-intersection} at the end of this \Cref{sec:structure-of-a-monotone-sofa}. Define the \emph{niche} \(\mathcal{N}(K)\) associated to any cap \(K\).

\begin{definition}

For any angle \(\omega \in [0, \pi/2]\), define the \emph{fan} \(F_\omega = H(\pi+\omega, 0) \cap H(3\pi/2, 0)\) with angle \(\omega\) as the convex cone pointed at the origin, bounded from below by the bottom edges \(l(\omega, 0)\) and \(l(3\pi/2, 0)\) of the parallelogram \(P_\omega\).

\label{def:fan}
\end{definition}

\begin{definition}

Let \(K\) be any cap with rotation angle \(\omega \in [0, \pi/2]\). Define the \emph{niche} of \(K\) as
\[
\mathcal{N}(K) := F_{\omega} \cap \bigcup_{0 \leq t \leq \omega} Q^-_K(t).
\]

\label{def:niche}
\end{definition}

Now we establish the structure of any monotonization of a sofa.

\begin{theorem}

Let \(S\) be a moving sofa with rotation angle \(\omega \in (0, \pi/2]\) in standard position. The monotonization \(\mathcal{M}(S)\) of \(S\) is equal to \(K \setminus \mathcal{N}(K)\), where \(K = \mathcal{C}(S)\) is the cap of sofa \(S\) and \(\mathcal{N}(K)\) is the niche of the cap \(K\).

\label{thm:monotonization-structure}
\end{theorem}

\begin{proof}
Let \(K = \mathcal{C}(S)\) be the cap of \(S\). By breaking down each \(L_S(t)\) into \(Q_S^+(t) \setminus Q_S^-(t)\), the monotonization \(\mathcal{M}(S)\) of \(S\) can be represented as the following subtraction of two sets.
\begin{equation}
\label{eqn:monotonization}
\begin{split}
\mathcal{M}(S) & = P_\omega \cap \bigcap_{0 \leq t \leq \omega} L_S(t) \\
& = \left( P_\omega \cap \bigcap_{0 \leq t \leq \omega} Q^+_S(t) \right) \setminus \left( F_\omega \cap \bigcup_{0 \leq t \leq \omega} Q^-_S(t) \right)
\end{split}
\end{equation}
By \Cref{pro:cap-same-tangent-hallway} we have \(Q_S^-(t) = Q_K^-(t)\). So we have \(\mathcal{M}(S) = K \setminus \mathcal{N}(K)\) by the definitions of \(K\) and \(\mathcal{N}(K)\).
\end{proof}

\begin{remark}

\Cref{eqn:monotonization} can understood intuitively as the following (see \Cref{fig:monotone-sofa}). The cap \(K\) is a convex body bounded from below by the edges of fan \(F_\omega\), and from above by the outer walls \(a_S(t)\) and \(c_S(t)\) of \(L_S(t)\). Imagine the set \(K\) as a block of clay that rotates inside the hallway \(L\) in the clockwise angle of \(t \in [0, \omega]\) while always touching the outer walls \(a\) and \(c\) of \(L\). As \(K\) rotates inside \(L\), the inner corner of \(L\) carves out the niche \(\mathcal{N}(K)\) which is the regions bounded by inner walls \(b_S(t)\) and \(d_S(t)\) of \(L_S(t)\) from \(K\). After the full movement of \(K\), the final clay \(K \setminus \mathcal{N}(K)\) is a moving sofa \(\mathcal{M}(S)\).

\label{rem:cap-niche-intuition}
\end{remark}

A moving sofa \(S\) and its monotonization \(\mathcal{M}(S)\) shares the same cap.

\begin{proposition}

For any moving sofa \(S\) with rotation angle \(\omega \in [0, \pi/2]\) in standard position, we have \(\mathcal{C}(\mathcal{M}(S)) = \mathcal{C}(S)\).

\label{pro:monotonization-cap}
\end{proposition}

\begin{proof}
By \Cref{def:cap-sofa} and \Cref{pro:rotating-hallway-parts}, the set \(\mathcal{C}(X)\) of \(X = S\) or \(\mathcal{M}(S)\) depend only on the values of the support function \(p_X\) on \(J_\omega\). The support functions of \(S\) and \(\mathcal{M}(S)\) match on \(J_\omega\) by \Cref{lem:cap-same-support-function}, completing the proof.
\end{proof}

We will use the following intrinsic variant of \Cref{thm:monotonization-structure} to represent any monotone sofa \(S\) as its cap minus niche.

\begin{theorem}

Let \(S\) be any monotone sofa with rotation angle \(\omega \in (0, \pi/2]\). Then \(S\) is in standard position and \(S = K \setminus \mathcal{N}(K)\), where \(K := \mathcal{C}(S)\) is the cap of \(S\) with rotation angle \(\omega\), and \(\mathcal{N}(K)\) is the niche of the cap \(K\).

\label{thm:monotone-sofa-structure}
\end{theorem}

\begin{proof}
Let \(S = \mathcal{M}(S')\) be any monotone sofa, so that it is the monotonization of a moving sofa \(S'\) in standard position. Then \(K := \mathcal{C}(S) = \mathcal{C}(S')\) by \Cref{pro:monotonization-cap}. Now apply \Cref{thm:monotonization-structure} to \(\mathcal{M}(S')\) to conclude that \(S = \mathcal{M}(S') = \mathcal{C}(S') \setminus \mathcal{N}(\mathcal{C}(S')) = K \setminus \mathcal{N}(K)\).
\end{proof}

Note that this variant does not mention anything about monotonization. In particular, by \Cref{thm:monotone-sofa-structure} any monotone sofa \(S\) can be recovered from its cap \(K = \mathcal{C}(S)\).

Monotization \(S \mapsto \mathcal{M}(S)\) is a process that enlarges any moving sofa \(S\) by \Cref{thm:monotonization-is-sofa}. Moreover, if \(S\) is already monotone (so that \(S = \mathcal{M}(S')\) for some \(S'\)), then the monotonization fixes \(S\).

\begin{theorem}

For any monotone sofa \(S\), we have \(\mathcal{M}(S) = S\).

\label{thm:monotonization-fixpoint}
\end{theorem}

\begin{proof}
Since \(S\) is monotone, \(S = \mathcal{M}(S')\) for some other moving sofa \(S'\). Now check
\[
\mathcal{M}(S) = \mathcal{C}(S) \setminus \mathcal{N}(\mathcal{C}(S)) = \mathcal{C}(S') \setminus \mathcal{N}(\mathcal{C}(S')) = \mathcal{M}(S') = S
\]
which holds from \Cref{thm:monotonization-structure} and \Cref{pro:monotonization-cap}.
\end{proof}

Thus, the monotonization \(S \mapsto \mathcal{M}(S)\) can be said as a ‘projection’ from all moving sofas to monotone sofas, in the sense that \(\mathcal{M}\) is a surjective map that fixes monotone sofas.

\subsection{\texorpdfstring{Proof of \Cref{thm:cap-hallway-intersection}}{Proof of }}

If \(\omega = \pi / 2\), then the set \(P_\omega\) is the horizontal strip \(H\). If \(\omega < \pi/2\), \(P_\omega\) is a proper parallelogram with the following points as vertices.

\begin{definition}

Let \(O = (0, 0)\) be the origin. For any angle \(\omega \in (0, \pi/2]\), define the point \(o_\omega = (\tan(\omega/2), 1)\).

\label{def:parallelogram-vertices}
\end{definition}

Note that if \(\omega < \pi/2\), then \(O\) is the lower-left corner of \(P_\omega\) and \(o_{\omega} = l(\omega, 1) \cap l(\pi/2, 1)\) is the upper-right corner of \(P_\omega\). Define the following subset of \(P_\omega\).

\begin{definition}

Let \(\omega \in (0, \pi/2]\) be arbitrary. Define \(M_\omega\) as the convex hull of the points \(O, o_\omega, o_\omega-u_\omega, o_\omega-v_0\).

\label{def:middle-set}
\end{definition}

Geometrically, \(M_\omega\) is a subset of \(P_\omega\) enclosed by the perpendicular legs from \(o_\omega\) to the bottom sides \(l(\omega, 0)\) and \(l(\pi/2, 0)\) of \(P_\omega\). We also introduce the following terminology.

\begin{definition}

Say that a point \(p_1\) is \emph{further than} (resp. \emph{strictly further than}) the point \(p_2\) \emph{in the direction} of nonzero vector \(v \in \mathbb{R}^2\) if \(p_1 \cdot v \geq p_2 \cdot v\) (resp. \(p_1 \cdot v > p_2 \cdot v\)).

\label{def:further-in-direction}
\end{definition}

We show the following lemma.

\begin{lemma}

If \(\omega < \pi/2\), then the set \(\mathcal{C}(S)\) in \Cref{def:cap-sofa} contains \(M_\omega\).

\label{lem:cap-contains-middle-set}
\end{lemma}

\begin{proof}
Since \(p_S(\omega) = p_S(\pi/2) = 1\), we can take points \(q\) and \(r\) of \(S\) so that \(q\) is on the line \(l(\pi/2, 1)\) further than \(o_\omega\) in the direction of \(-u_0\), and \(r\) is on the line \(l(\omega, 1)\) further than \(o_\omega\) in the direction of \(-v_\omega\). Take an arbitrary \(t \in [0, \omega]\). Because \(Q^+_S(t)\) is a right-angled convex cone with normal vectors \(u_t\) and \(v_t\) containing \(q\) and \(r\), \(Q_S^+(t)\) also contains \(o_\omega\). Because \(Q_S^+(t)\) contains \(o_\omega\) and is closed in the direction of \(-u_t\) and \(-v_t\) (\Cref{def:closed-in-direction}), \(Q_S^+(t)\) contains \(M_\omega\) as a subset. So the intersection \(\mathcal{C}(S)\) of \(P_\omega\) and \(Q_S^+(t)\) contains \(M_\omega\).
\end{proof}

We finish the proof of \Cref{thm:cap-hallway-intersection}.

\begin{proof}[Proof of \Cref{thm:cap-hallway-intersection}]
Let \(S\) be any moving sofa with rotation angle \(\omega \in (0, \pi/2]\) in standard position. Let \(K = \mathcal{C}(S)\). That \(S \subseteq K\) is an immediate consequence of the third condition of \Cref{thm:moving-sofa-iff-hallway-intersection}. We now show that \(K\) is a cap with rotation angle \(\omega\).

Assume the case \(\omega < \pi/2\). Then by \Cref{def:cap-sofa} and \Cref{lem:cap-contains-middle-set} we have \(M_\omega \subseteq K \subseteq P_\omega\), and the support function of \(M_\omega\) and \(P_\omega\) agree on the angles \(\omega, \pi/2, \omega + \pi, 3\pi/2\). So the first condition of \Cref{def:cap} is satisfied. Now assume \(\omega = \pi/2\). Since \(S \subseteq K \subseteq H\) and \(S\) is in standard position we have \(p_S(\pi/2) = p_K(\pi/2) = 1\). With \(p_K(\pi/2) = 1\), take the point \(z \in K\) on the line \(y=1\). Let \(X := \bigcap_{t \in [0, \pi/2]} Q_S^+(t)\), then by the definition of \(K\) we have \(K = H \cap X\). Since \(X\) is closed in the direction of \(-v_0\) (\Cref{def:closed-in-direction}), the point \(z' := z - (0, 1)\) is also in \(X\). So \(z' \in H \cap X = K\) and \(z'\) is on the line \(y=0\). This implies that \(p_K(3\pi/2) = 0\). So the first condition of \Cref{def:cap} is true.

The set \(P_\omega\) is the intersection of four half-planes with normal angles \(\omega, \pi/2, \pi + \omega, 3\pi/2\). The set \(Q_S^+(t)\) is an intersection of two half-planes with normal angles \(t\) and \(t + \pi/2\). Now the second condition of \Cref{def:cap} follows.
\end{proof}

\section{Cap Contains Niche}
\label{sec:cap-contains-niche}
We will now establish the following theorem.

\begin{theorem}

For any monotone sofa \(S\) with cap \(K = \mathcal{C}(S)\), the cap \(K\) contains the niche \(\mathcal{N}(K)\).

\label{thm:niche-in-cap}
\end{theorem}

Note that \(S = K \setminus \mathcal{N}(K)\) by \Cref{thm:monotone-sofa-structure}. With \Cref{thm:niche-in-cap}, the area \(|S| = |K| - |\mathcal{N}(K)|\) of a monotone sofa can be understood separately in terms of its cap and niche.

\begin{remark}

In spite of \Cref{thm:niche-in-cap}, a general cap \(K\) following \Cref{def:cap} may not always contain its niche \(\mathcal{N}(K)\) in \Cref{def:niche}. For an example, take \(K = [0, 100] \times [0, 1]\) with rotation angle \(\omega = \pi/2\). Then \(K\) is too wide and the inner quadrant \(Q_K^-(\pi/4)\) of \(L_K(\pi/4)\) pushes out of \(K\), so we have \(\mathcal{N}(K) \not\subseteq K\). In this case, the cap \(K\) is never the cap \(\mathcal{C}(S)\) associated to a particular moving sofa \(S\) as in \Cref{thm:cap-hallway-intersection}. \Cref{thm:monotonization-connected-iff} identifies the exact condition of \(K\) where \(\mathcal{N}(K) \subseteq K\).

\label{rem:niche-not-in-cap}
\end{remark}

\subsection{Geometric definitions on cap and niche}

We need a handful of geometric definitions on a cap \(K\) to prove \Cref{thm:niche-in-cap}. We will also use them throughout the rest of the document as well. Define the \emph{vertices} of a cap \(K\).

\begin{definition}

Let \(K\) be a cap with rotation angle \(\omega\). For any \(t \in [0, \omega]\), define the vertices \(A^+_K(t) = v^+_K(t)\), \(A^-_K(t) = v^-_K(t)\), \(C^+_K(t) = v^+_K(t + \pi/2)\), and \(C^-_K(t) = v^-_K(t + \pi/2)\) of \(K\).

\label{def:cap-vertices}
\end{definition}

Note that the outer wall \(a_K(t)\) (resp. \(c_K(t)\)) of \(L_K(t)\) is in contact with the cap \(K\) at the vertices \(A_K^+(t)\) and \(A_K^-(t)\) (resp. \(C_K^+(t)\) and \(C_K^-(t)\)) respectively. We also define the \emph{upper boundary} of a cap \(K\).

\begin{definition}

For any cap \(K\) with rotation angle \(\omega\), define the \emph{upper boundary} \(\delta K\) of \(K\) as the set \(\delta K = \bigcup_{t \in [0, \omega + \pi/2]} e_K(t)\).

\label{def:upper-boundary-of-cap}
\end{definition}

For any cap \(K\) with rotation angle \(\omega\), the upper boundary \(\delta K\) is exactly the points of \(K\) making contact with the outer walls \(a_K(t)\) and \(c_K(t)\) of tangent hallways \(L_K(t)\) for every \(t \in [0, \omega]\). We collect some observations on \(\delta K\).

\begin{proposition}

Let \(K\) be a cap with rotation angle \(\omega\). The set \(K \setminus \delta K\) is the interior of \(K\) in the subset topology of \(F_\omega\).

\label{pro:upper-boundary-interior}
\end{proposition}

\begin{proof}
Since \(K\) and \(F_\omega\) are closed in \(\mathbb{R}^2\), the set \(K\) is closed in the subset topology of \(F_\omega\). Let \(X\) be the boundary of \(K\) in the subset topology of \(F_\omega\), then we have \(X \subseteq K\) because \(K\) is closed in the subset topology of \(F_\omega\). We will show that \(\delta K\) is equal to \(X\), then it follows that the set \(K \setminus \delta K\) is the interior of \(K\) in the subset topology of \(F_\omega\).

We show \(\delta K \subseteq X\) and \(X \subseteq \delta K\) respectively. Take any point \(z\) of \(\delta K\). Then \(z \in e_K(t)\) for some \(t \in [0, \omega + \pi/2]\). Since \(K\) is a planar convex body, for any \(\epsilon > 0\) the point \(z' = z + \epsilon u_t\) is not in \(K\). Since the set \(F_\omega\) is closed in the direction of \(u_t\) (\Cref{def:closed-in-direction}), the point \(z'\) is also in \(F_\omega\). Thus we have a point \(z'\) in the neighborhood of \(z\) which is outside \(K\), and \(\delta K\) is a subset of \(X\).

On the other hand, take any point \(z\) of \(X\) and assume by contradiction that \(z \in K \setminus \delta K\). Then for every \(t \in [0, \omega + \pi/2]\) we have \(z \not \in e_K(t)\) so that \(z \cdot u_t < p_K(t)\). Since \(p_K\) is continuous, the value \(p_K(t) - z \cdot u_t\) has a global lower bound \(\epsilon > 0\) on the compact interval \([0, \omega + \pi/2]\). So an open ball \(U\) of radius \(\epsilon\) centered at \(z\) is contained in the half-space \(H_K(t)\) for all \(t \in [0, \omega + \pi/2]\). Now \(U \cap F_\omega \subseteq K\) and so \(z \not\in X\), leading to contradiction.
\end{proof}

Geometrically, the upper boundary \(\delta K\) is an arc from \(A_K^-(0)\) to \(C_K^+(\omega)\) in the counterclockwise direction along the boundary \(\partial K\) of \(K\). This is rigorously justified by the following consequence of \Cref{cor:closed-param-segment}. For full details, read the introduction of \Cref{sec:parametrization-of-boundary}.

\begin{corollary}

Let \(K\) be a cap with rotation angle \(\omega\). The upper boundary \(\delta K\) admits an absolutely-continuous, arc-length parametrization \(\mathbf{b}_K^{0-, \pi/2 + \omega}\) (\Cref{def:closed-param}) from \(A_K^-(0)\) to \(C_K^+(\omega)\) in the counterclockwise direction along \(\partial K\).

\label{cor:upper-boundary-param}
\end{corollary}

We also give name to the convex polygons \(F_\omega \cap Q^-_K(t)\) whose union over all \(t \in [0, \omega]\) constitutes the niche \(\mathcal{N}(K)\).

\begin{definition}

For any cap \(K\) with rotation angle \(\omega\), define \(T_K(t) = F_\omega \cap Q^-_K(t)\) as the \emph{wedge} of \(K\) with angle \(t \in [0, \omega]\).

\label{def:wedge}
\end{definition}

\begin{proposition}

For any cap \(K\) with rotation angle \(\omega\), we have \(\mathcal{N}(K) = \cup_{t \in [0, \omega]} T_K(t)\).

\label{pro:wedge}
\end{proposition}

\begin{proof}
Immediate from \Cref{def:niche}.
\end{proof}

We give names to the parts of the wedge \(T_K(t)\).

\begin{definition}

For any cap \(K\) with rotation angle \(\omega\) and \(t \in (0, \omega)\), define \(W_K(t)\) as the intersection of lines \(b_K(t)\) and \(l(\pi, 0)\). Define \(w_K(t) = (A_K^-(0) - W_K(t)) \cdot u_0\) as the signed distance from point \(W_K(t)\) and the vertex \(A_K^-(0)\) along the line \(l(\pi, 0)\) in the direction of \(u_0\).

Likewise, define \(Z_K(t)\) as the intersection of lines \(d_K(t)\) and \(l(\omega, 0)\). Define \(z_K(t) = (C_K^+(\omega) - Z_K(t)) \cdot v_\omega\) as the signed length between \(Z_K(t)\) and the vertex \(C_K^+(\omega)\) along the line \(l(\omega, 0)\) in the direction of \(v_\omega\).

\label{def:wedge-side-lengths}
\end{definition}

Note that if the wedge \(T_K(t)\) contains the origin \(O\), then \(T_K(t)\) is a quadrilateral with vertices \(O, W_K(t), Z_K(t)\), and \(\mathbf{x}_K(t)\), and the points \(W_K(t)\) and \(Z_K(t)\) are the leftmost and rightmost point of \(\overline{T_K(t)}\) respectively.

\subsection{Controlling the wedge inside cap}

To show \(\mathcal{N}(K) \subseteq K\) we need to control each wedge \(T_K(t)\) inside \(K\). First we show \(w_K(t), z_K(t) \geq 0\). This controls the endpoints \(W_K(t)\) and \(Z_K(t)\) of \(T_K(t)\) inside \(K\).

\begin{lemma}

Let \(K\) be any cap with rotation angle \(\omega\). For any angle \(t \in (0, \omega)\), we have \(w_K(t), z_K(t) \geq 0\).

\label{lem:wedge-ends-in-cap}
\end{lemma}

\begin{proof}
To show that \(w_K(t) \geq 0\), we need to show that the point \(A_K^-(0)\) is further than the point \(W_K(t)\) in the direction of \(u_0\) (see \Cref{def:further-in-direction} for the terminology). The point \(q := a_K(t) \cap l(\pi/2, 1)\) is further than \(W_K(t) = b_K(t) \cap l(\pi/2, 0)\) in the direction of \(u_0\), because the lines \(a_K(t)\) and \(b_K(t)\) form the boundary of a unit-width vertical strip rotated counterclockwise by \(t\). The point \(A^-_K(t)\) is further than \(q = l_K(t) \cap l_K(\pi/2)\) in the direction of \(u_0\) because \(K\) is a convex body. Finally, the point \(A^-_K(0)\) is further than \(A_K^-(t)\) in the direction of \(u_0\) because \(K\) is a convex body. Summing up, the points \(W_K(t), q, A_K^-(t), A_K^-(0)\) are aligned in the direction of \(u_0\), completing the proof. A symmetric argument will show that the points \(Z_K(t)\), \(r := c_K(t) \cap l(\omega, 1)\), \(C_K^+(t)\), \(C_K^+(\omega)\) are aligned in the direction of \(v_\omega\), proving \(z_K(t) \geq 0\).
\end{proof}

\begin{corollary}

Let \(K\) be any cap with rotation angle \(\omega\). Then \(A^-_K(0), C^+_K(\omega) \in K \setminus \mathcal{N}(K)\).

\label{cor:cap-ends-not-in-niche}
\end{corollary}

\begin{proof}
We only need to show that \(A^-_K(0), C^+_K(\omega)\) are not in \(\mathcal{N}(K)\). That is, for any \(t \in (0, \omega)\), neither points are in \(T_K(t)\). Since \(w_K(t) \geq 0\) by \Cref{lem:wedge-ends-in-cap}, the point \(A_K^-(0)\) is on the right side of the boundary \(b_K(t)\) of \(T_K(t)\). So \(A_K^-(0) \not\in T_K(t)\). Similarly, \(z_K(t) \geq 0\) implies \(C_K^+(\omega) \not\in T_K(t)\).
\end{proof}

We then show that if the corner \(\mathbf{x}_K(t)\) is inside \(K\), then the whole wedge \(T_K(t)\) is always inside \(K\).

\begin{lemma}

Fix any cap \(K\) with rotation angle \(\omega \in [0, \pi/2]\) and an angle \(t \in (0, \omega)\). If the inner corner \(\mathbf{x}_K(t)\) is in \(K\), then the wedge \(T_K(t)\) is a subset of \(K\).

\label{lem:niche-in-cap}
\end{lemma}

\begin{proof}
Assume \(\mathbf{x}_K(t) \in K\). If \(\omega = \pi/2\), then by \(\mathbf{x}_K(t) \in K\), the wedge \(T_K(t)\) is the triangle with vertices \(W_K(t)\), \(\mathbf{x}_K(t)\), and \(Z_K(t)\) in counterclockwise order. Note also that \(W_K(t)\) is further than \(Z_K(t)\) in the direction of \(u_0\) (\Cref{def:further-in-direction}). As \(w_K(t), z_K(t) \geq 0\), this implies that all the three vertices of \(T_K(t)\) are in \(K\).

If \(\omega < \pi/2\), we divide the proof into four cases on whether the origin \(O\) lies strictly below the lines \(b_K(t)\) and \(d_K(t)\) or not respectively.

\begin{itemize}
\tightlist
\item
  If \((0, 0)\) lies on or above both \(b_K(t)\) and \(d_K(t)\), then we get contradiction as the corner \(\mathbf{x}_K(t)\) should be outside the interior \(F_\omega^\circ\) of fan \(F_\omega\), but \(\mathbf{x}_K(t) \in K\).
\item
  If \((0, 0)\) lies on or above \(b_K(t)\) but lies strictly below \(d_K(t)\), then \(T_K(t)\) is a triangle with vertices \(\mathbf{x}_K(t)\), \(Z_K(t)\) and the intersection \(p := l(\omega, 0) \cap b_K(t)\). In this case, the point \(p\) is in the line segment connecting \(Z_K(t)\) and \((0, 0)\). Also, as \(z_K(t) \geq 0\) (\Cref{lem:wedge-ends-in-cap}) the point \(Z_K(t)\) lies in the segment connecting \(C^+_K(\omega)\) and the origin \((0, 0)\). So the points \(\mathbf{x}_K(t), Z_K(t), p\) are in \(K\) and by convexity of \(K\) we have \(T \subseteq K\).
\item
  The case where \((0, 0)\) lies strictly below \(b_K(t)\) but lies on or above \(d_K(t)\) can be handed by an argument symmetric to the previous case.
\item
  If \((0, 0)\) lies strictly below both \(b_K(t)\) and \(d_K(t)\), then \(T_K(t)\) is a quadrilateral with vertices \(\mathbf{x}_K(t)\), \(Z_K(t)\), \(W_K(t)\) and \((0, 0)\). As \(w_K(t) \geq 0\) (resp. \(z_K(t) \geq 0\)) by \Cref{lem:wedge-ends-in-cap}, the point \(W_K(t)\) (resp. \(Z_K(t)\)) is in the line segment connecting \((0, 0)\) and \(A^-_K(0)\) (resp. \(C^+_K(\omega)\)). So all the vertices of \(T_K(t)\) are in \(K\), and \(T_K(t)\) is in \(K\) by convexity.
\end{itemize}

\end{proof}

\subsection{\texorpdfstring{Equivalent conditions for \(\mathcal{N}(K) \subseteq K\)}{Equivalent conditions for \textbackslash mathcal\{N\}(K) \textbackslash subseteq K}}

Now we prove \Cref{thm:niche-in-cap}. In fact, we identify the exact condition where \(\mathcal{N}(K) \subseteq K\) for a general cap \(K\) following \Cref{def:cap}.

\begin{theorem}

Let \(K\) be any cap with rotation angle \(\omega\). Then the followings are all equivalent.

\begin{enumerate}
\def\labelenumi{\arabic{enumi}.}
\tightlist
\item
  \(\mathcal{N}(K) \subseteq K\)
\item
  \(\mathcal{N}(K) \subseteq K \setminus \delta K\)
\item
  For every \(t \in [0, \omega]\), either \(\mathbf{x}_K(t) \not\in F_\omega^\circ\) or \(\mathbf{x}_K(t) \in K\).
\item
  The set \(S = K \setminus \mathcal{N}(K)\) is connected.
\end{enumerate}

\label{thm:monotonization-connected-iff}
\end{theorem}

\begin{proof}
The conditions (1) and (2) are equivalent because the niche \(\mathcal{N}(K)\) is open in the subset topology of \(F_\omega\) by \Cref{def:niche}, and the set \(K \setminus \delta K\) is the interior of \(K\) in the subset topology of \(F_\omega\) by \Cref{pro:upper-boundary-interior}.

(1 \(\Rightarrow\) 3) We will prove the contraposition and assume \(\mathbf{x}_K(t) \in F_\omega^\circ \setminus K\). Then a neighborhood of \(\mathbf{x}_K(t)\) is inside \(F_\omega\) and disjoint from \(K\), so a subset of \(T_K(t)\) is outside \(K\), showing \(\mathcal{N}(K) \not\subseteq K \setminus \delta K\).

(3 \(\Rightarrow\) 1) If \(\mathbf{x}_K(t) \not \in F_\omega^\circ\), then \(T_K(t)\) is an empty set. If \(\mathbf{x}_K(t) \in K\), then by \Cref{lem:niche-in-cap} we have \(T_K(t) \subseteq K\).

(2 \(\Rightarrow\) 4) As \(\delta K\) is disjoint from \(\mathcal{N}(K)\), we have \(\delta K \subseteq S\). We show that \(S\) is connected. First, note that \(\delta K\) is connected by \Cref{cor:upper-boundary-param}. Next, take any point \(p \in S\). Take the half-line \(r\) starting from \(p\) in the upward direction \(v_0\). Then \(r\) touches a point in \(\delta K\) as \(p \in K\). Moreover, \(r\) is disjoint from \(\mathcal{N}(K)\) as the set \(\mathcal{N}(K) \cup (\mathbb{R}^2 \setminus F_\omega)\) is closed in the direction \(-v_0\) (\Cref{def:closed-in-direction}). Now \(r \cap K\) is a line segment inside \(S\) that connects the arbitrary point \(p \in S\) to a point in \(\delta K\). So \(S\) is connected.

(4 \(\Rightarrow\) 3) Assume by contradiction that \(\mathbf{x}_K(t) \in F_\omega^\circ \setminus K\) for some \(t \in [0, \omega]\). Then it should be that \(t \neq 0\) or \(\omega\). We first show that the vertical line \(l\) passing through \(\mathbf{x}_K(t)\) in the direction of \(v_0\) is disjoint from \(S\). The ray with initial point \(\mathbf{x}_K(t)\) and direction \(v_0\) is disjoint from \(K\) as the set \(F_\omega^\circ \setminus K\) is closed in the direction \(v_0\). The ray with initial point \(\mathbf{x}_K(t)\) and direction \(-v_0\) is not in \(S\) because \(\mathbf{x}_K(t)\) is the corner of \(Q_K^-(t)\), and \(Q_K^-(t)\) is closed in the direction of \(-v_0\). So the vertical line \(l\) passing through \(\mathbf{x}_K(t)\) does not overlap with \(S\).

Now separate the horizontal strip \(H\) into two chunks by the vertical line \(l\) passing through \(\mathbf{x}_K(t)\). As \(S\) is connected, \(S\) should lie either strictly on left or strictly on right of \(l\). As \(\mathbf{x}(t)\) lies strictly inside \(F_\omega\), the point \(W_K(t)\) is strictly further than \(\mathbf{x}(t)\) in the direction of \(u_0\), and by \Cref{lem:wedge-ends-in-cap} the point \(A_K^-(0)\) is further than \(W_K(t)\) in the direction of \(u_0\). So the endpoint \(A_K^-(0)\) of \(K\) lies strictly on the right side of \(l\). Similarly, the point \(Z_K(t)\) is strictly further than \(\mathbf{x}_K(t)\) in the direction of \(-u_0\), and by \Cref{lem:wedge-ends-in-cap} the point \(C_K^+(\omega)\) is further than \(W_K(t)\) in the direction of \(-u_0\). So the endpoint \(C^+_K(\omega)\) of \(K\) lies strictly on the left side of \(l\). As the endpoints \(A^-_K(0)\) and \(C^+_K(\omega)\) are in \(K \setminus \mathcal{N}(K)\) by \Cref{cor:cap-ends-not-in-niche}, and the line \(l\) separates the two points, the set \(K \setminus \mathcal{N}(K)\) is disconnected.
\end{proof}

\Cref{thm:niche-in-cap} is an immediate consequence of \Cref{thm:monotonization-connected-iff}.

\begin{proof}[Proof of \Cref{thm:niche-in-cap}]
We have \(S = K \setminus \mathcal{N}(K)\) by \Cref{thm:monotone-sofa-structure}. In particular, \(K \setminus \mathcal{N}(K)\) is a moving sofa so it is connected. Use that condition 4 implies condition 1 in \Cref{thm:monotonization-connected-iff} to complete the proof.
\end{proof}



\chapter{Sofa Area Functional $\mathcal{A}$}
\label{sec:sofa-area-functional-a}
We use the findings in \Cref{sec:monotone-sofas} to reduce the moving sofa problem to the maximization of \emph{sofa area functional} \(\mathcal{A} : \mathcal{K}_\omega \to \mathbb{R}\) defined on the \emph{space of caps} \(\mathcal{K}_\omega\) with rotation angle \(\omega\). We first define the domain of \(\mathcal{A}\).

\begin{definition}

Define the \emph{space of caps} \(\mathcal{K}_\omega\) with the \emph{rotation angle} \(\omega \in (0, \pi/2]\) as the collection of all caps \(K\) with rotation angle \(\omega\) as in \Cref{def:cap}.

\label{def:cap-space}
\end{definition}

Now define the sofa area functional \(\mathcal{A} : \mathcal{K}_\omega \to \mathbb{R}\).

\begin{definition}

For any angle \(\omega \in (0, \pi/2]\), define the \emph{sofa area functional} \(\mathcal{A}_\omega : \mathcal{K}_\omega \to \mathbb{R}\) on the space of caps \(\mathcal{K}_\omega\) as \(\mathcal{A}_\omega(K) = |K| - |\mathcal{N}(K)|\).

\label{def:sofa-area-functional}
\end{definition}

We can restate \Cref{thm:niche-in-cap} using \Cref{def:sofa-area-functional} as:

\begin{corollary}

If \(K \in \mathcal{K}_\omega\) is the cap \(\mathcal{C}(S)\) for a monotone sofa \(S\) with rotation angle \(\omega\), we have \(\mathcal{A}(K) = |S|\).

\label{cor:sofa-area-functional}
\end{corollary}

As we observed in \Cref{rem:niche-not-in-cap}, not all cap \(K \in \mathcal{K}_\omega\) is the cap \(\mathcal{C}(S)\) of a monotone sofa \(S\).

\begin{definition}

For any angle \(\omega \in (0, \pi/2]\), define \(\mathcal{M}_\omega\) as the subset of \(\mathcal{K}_\omega\) of the caps \(\mathcal{C}(S)\) (\Cref{def:cap-sofa}) of an arbitrary monotone sofa \(S\) with rotation angle \(\omega\).

\label{def:monotone-sofa-embedding}
\end{definition}

\(\mathcal{M}_\omega\) is a proper subset of \(\mathcal{K}_\omega\) by \Cref{rem:niche-not-in-cap}. By \Cref{thm:monotone-sofa-structure}, the set of all monotone sofas \(S\) with rotation angle \(\omega\) embeds to the subset \(\mathcal{M}_\omega\) of \(\mathcal{K}_\omega\) by taking the cap \(S \mapsto \mathcal{C}(S)\). By \Cref{thm:monotonization-is-sofa} and \Cref{cor:sofa-area-functional}, the moving sofa problem for a fixed rotation angle \(\omega \in (0, \pi/2]\) is now equivalent to the maximization of the sofa area functional \(\mathcal{A} : \mathcal{K}_\omega \to \mathbb{R}\) on the subspace \(\mathcal{M}_\omega\) of \(\mathcal{K}_\omega\).

We will, however, try to optimize the sofa area functional \(\mathcal{A} : \mathcal{K}_\omega \to \mathbb{R}\) over the whole \(\mathcal{K}_\omega\), not the subspace \(\mathcal{M}_\omega\) of \(\mathcal{K}_\omega\). This is because the space \(\mathcal{K}_\omega\) of all caps is turns out to be a convex space (\Cref{def:convex-space}) which is easier to understand than the subspace \(\mathcal{M}_\omega\). By extending the domain of optimization of \(\mathcal{A}\) from \(\mathcal{M}_\omega\) to \(\mathcal{K}_\omega\), we get a counterpart of every statement on maximum-area monotone sofas. To start, we have the following strenghtening of Gerver’s conjecture \(\mu_{\max} = \mu_G\).

\begin{conjecture}

The cap \(K = K_G\) of Gerver’s sofa \(S_G\) attains the maximum value \(\mathcal{A}(K)\) of the sofa area functional \(\mathcal{A} : \mathcal{K}_\omega \to \mathbb{R}\) over all rotation angle \(\omega \in (0, \pi/2]\).

\label{con:gerver-cap}
\end{conjecture}

While we cannot prove \Cref{con:gerver-cap}, we expect it to be true. The rest of the paper proves the following strenghtening of main \Cref{thm:main} that extends the domain from \(\mathcal{M}_\omega\) to \(\mathcal{K}_\omega\).

\begin{theorem}

(Generalized main theorem) Let \(K \in \mathcal{K}_\omega\) be any cap with rotation angle \(\omega \in (0, \pi/2]\). If the rotation path \(\mathbf{x}_K : [0, \omega] \to \mathbb{R}^2\) of \(K\) is injective and always on the fan \(F_\omega\), then we have \(\mathcal{A}(K) \leq 1 + \omega^2/2\).

\label{thm:main-cap}
\end{theorem}

\begin{proposition}

\Cref{thm:main-cap} implies \Cref{thm:main}.

\label{pro:main-cap}
\end{proposition}

\begin{proof}
Let \(S\) be any monotone sofa of rotation angle \(\omega \in (0 ,\pi/2]\) with cap \(K = \mathcal{C}(S)\), so that \(S\) satisfies the injectivity condition (\Cref{def:injectivity}). By \Cref{pro:cap-same-tangent-hallway} we have \(\mathbf{x}_S = \mathbf{x}_K\). So the injectivity condition of \(S\) implies that \(\mathbf{x}_K : [0, \omega] \to \mathbb{R}^2\) is injective and always on the fan \(F_\omega\). By \Cref{thm:main-cap} we have \(\mathcal{A}(K) \leq 1 + \omega^2/2\). By \Cref{cor:sofa-area-functional} we have \(|S| = \mathcal{A}(K)\), completing the proof.
\end{proof}

We finish this section by mentioning the counterparts of angle and injectivity hypotheses (\Cref{con:angle} and ) that extend the domain from \(\mathcal{M}_\omega\) to \(\mathcal{K}_\omega\).

\begin{conjecture}

The supremum of \(\mathcal{A}_{\omega} : \mathcal{K}_\omega \to \mathbb{R}\) maximizes at \(\omega = \pi/2\).

\label{con:angle-cap}
\end{conjecture}

\begin{conjecture}

There exists a maximizer \(K \in \mathcal{K}_\omega\) of the sofa area functional \(\mathcal{A}_{\omega}\) over all rotation angle \(\omega \in (0, \pi/2]\), such that the rotation path \(\mathbf{x}_K : [0, \omega] \to \mathbb{R}^2\) is injective and always on the fan \(F_\omega\).

\label{con:injectivity-cap}
\end{conjecture}

Perhaps surprisingly, \Cref{con:angle-cap} does not necessarily imply \Cref{con:angle}. Similarly, \Cref{con:injectivity-cap} does not necessarily imply \Cref{con:injectivity}. This is because a maximizer of \(\mathcal{A}_\omega\) is not necessarily a cap of a monotone sofa in \(\mathcal{M}_\omega\). However, observe that proving \Cref{con:angle-cap} would allow us to assume \(\omega = \pi/2\) in a potential proof of \Cref{con:injectivity-cap}, and \Cref{thm:main-cap} with a proof of \Cref{con:injectivity-cap} would imply the upper bound \(1 + \pi^2/8\) of sofa area unconditionally. Proving \Cref{con:angle-cap} and \Cref{con:injectivity-cap} will be the main goal of subsequent works.


\chapter{Conditional Upper Bound $\mathcal{A}_1$}
\label{sec:conditional-upper-bound-a1}
In this \Cref{sec:conditional-upper-bound-a1}, we prove the generalized \Cref{thm:main-cap} of the main \Cref{thm:main} by establishing a conditional upper bound \(\mathcal{A}_1 : \mathcal{K}_\omega \to \mathbb{R}\) of the sofa area functional \(\mathcal{A} : \mathcal{K}_\omega \to \mathbb{R}\). For any monotone sofa \(S\) satisfying the injectivity condition, we prove \(\mathcal{A}_1(K) \geq \mathcal{A}(K)\). Then we calculate the maximum value \(1 + \omega^2/2\) of \(\mathcal{A}_1\).

\Cref{sec:definition-of-a1} defines the value \(\mathcal{A}_1(K)\) of a cap \(K\) as the area of \(K\) subtracted by the area \(\mathcal{I}(\mathbf{x}_K)\) enclosed by the inner corner \(\mathbf{x}_K\) (\Cref{def:a1}). Then \Cref{thm:a1-upper-bound} proves that \(\mathcal{A}_1\) is indeed an upper bound of \(\mathcal{A}\) for monotone sofas satisfying the injectivity condition.

\Cref{sec:calculus-of-variation} shows that the domain \(\mathcal{K}_\omega\) of \(\mathcal{A}_1\), the space of all caps \(K\) with rotation angle \(\omega\), is convex (\Cref{pro:cap-space-convex-space}). Then \Cref{sec:calculus-of-variation} sets up the calculus of variation on a general concave quadratic functional \(f\) on a convex domain \(\mathcal{K}\). In particular, we introduce the notion of \emph{directional derivative} \(Df(K; -)\) of \(f\) (\Cref{def:convex-space-directional-derivative}), and show that any critical point \(K \in \mathcal{K}\) satisfying \(D f(K; -) \leq 0\) is a global maximum (\Cref{thm:quadratic-variation}). This will be used to do the calculus of variation on \(f = \mathcal{A}_1\) with domain \(\mathcal{K} = \mathcal{K}_\omega\).

\Cref{sec:boundary-measure} establishes that \(\mathcal{A}_1\) is a quadratic functional on its domain \(\mathcal{K}_\omega\). To do so, we introduce the \emph{boundary measure} \(\beta_K\) of a cap \(K\) (\Cref{def:boundary-measure}) that measures the side lengths of the upper boundary \(\delta K\); read the description following \Cref{def:boundary-measure} for an example. The measure \(\beta_K\) will be useful in computing the derivative of \(\mathcal{A}_1\) (\Cref{thm:variation-a1}). We will establish the bijective correspondence between \(K\) and \(\beta_K\) (\Cref{thm:boundary-measure-cap} and \Cref{thm:cap-from-boundary-measure}).

\Cref{sec:concavity-of-a1} establishes the concavity of \(\mathcal{A}_1\) (\Cref{thm:a1-negative-semidefinite}), so that a local optimum of \(\mathcal{A}_1\) is also a global optimum of \(\mathcal{A}_1\). We will define the area \(\mathcal{S}(K)\) of a region swept by segments tangent to cap \(K\) (see \Cref{fig:mamikon-sofa}). We will apply \emph{Mamikon’s theorem}, a theorem in classical geometry, to show that \(\mathcal{S}(K)\) is convex with respect to \(K\). We then show that \(\mathcal{S}(K) + \mathcal{A}_1(K)\) is linear with respect to \(K\), showing the concavity of \(\mathcal{A}_1\).

\Cref{sec:directional-derivative-of-a1} calculates the directional derivative \(D\mathcal{A}_1(K; -)\) of \(\mathcal{A}_1\) (\Cref{thm:variation-a1}) using the boundary measure \(\beta_K\) and the facts on convex bodies in \Cref{sec:convex-bodies}. Finally, \Cref{sec:maximizer-of-a1} computes the maximizer \(K = K_{\omega, 1}\) of \(\mathcal{A}_1(K)\) with maximum value \(1 + \omega^2/2\) by solving for the condition where directional derivative \(D\mathcal{A}_1(K; -)\) is always zero.
\section{Definition of $\mathcal{A}_1$}
\label{sec:definition-of-a1}
We first define the upper bound \(\mathcal{A}_1 : \mathcal{K}_\omega \to \mathbb{R}\) of the sofa area functional \(\mathcal{A}\). Recall the \Cref{def:curve-area-functional} of the \emph{curve area functional}
\[
\mathcal{I}(\mathbf{x}) := \frac{1}{2} \int_a^b \mathbf{x}(t) \times d\mathbf{x}(t) := \frac{1}{2} \int_a^b x(t) dy(t) - y(t) dx(t)
\]
for an arbitrary curve \(\Gamma\) with rectifiable parametrization \(\mathbf{x} : [a, b] \to \mathbb{R}^2\).. By Green’s theorem, we have the following.

\begin{proposition}

(Theorem 10.43, p289 of \autocite{apostol2000visual}) If \(\mathbf{x}\) is a Jordan curve oriented counterclockwise (resp. clockwise), \(\mathcal{I}(\mathbf{x})\) is the exact area of the region enclosed by \(\mathbf{x}\) (resp. the area with a negative sign).

\label{pro:curve-area-functional-area}
\end{proposition}

If \(\mathbf{x}\) is not closed (that is, \(\mathbf{x}(a) \neq \mathbf{x}(b)\)), the sofa area functional \(\mathcal{I}(\mathbf{x})\) measures the signed area of the region bounded by the curve \(\mathbf{x}\), and two line segments connecting the origin to \(\mathbf{x}(a)\) and \(\mathbf{x}(b)\) respectively. We also have the following additivity of \(\mathcal{I}\).

\begin{proposition}

If \(\gamma\) is the concatenation of two curves \(\alpha\) and \(\beta\) then \(\mathcal{I}(\gamma) = \mathcal{I}(\alpha) + \mathcal{I}(\beta)\).

\label{pro:curve-area-functional-additive}
\end{proposition}

For any \(\omega \in (0, \pi/2]\) and cap \(K \in \mathcal{K}_\omega\), we will define \(\mathcal{A}_1(K)\) as the area of \(K\) minus the area of the region enclosed by \(\mathbf{x}_K : [0, \omega]\). We will express the area enclosed by \(\mathbf{x}_K\) as \(\mathcal{I}(\mathbf{x}_K)\).

\begin{proposition}

For any \(\omega \in (0, \pi/2]\) and cap \(K \in \mathcal{K}_\omega\), the inner corner \(\mathbf{x}_K : [0, \omega] \to \mathbb{R}\) is Lipschitz.

\label{pro:inner-corner-lipschitz}
\end{proposition}

\begin{proof}
The support function \(p_K\) of \(K\) is Lipschitz (\Cref{thm:support-function-lipschitz}), so
\[
\mathbf{x}_K(t) = (p_K(t) - 1) u_t + (p_K(t + \pi/2) - 1) v_t
\]
is also Lipschitz.
\end{proof}

Thus \(\mathbf{x}_K\) is rectifiable and the value \(\mathcal{I}(\mathbf{x}_K)\) is well-defined. With this, define the functional \(\mathcal{A}_1 : \mathcal{K}_\omega \to \mathbb{R}\) as the following.

\begin{definition}

For any angle \(\omega \in (0, \pi/2]\) and cap \(K\) in \(\mathcal{K}_\omega\), define \(\mathcal{A}_{1, \omega}(K) = |K| - \mathcal{I}(\mathbf{x}_K)\). If the angle \(\omega\) is clear from the context, denote \(\mathcal{A}_{1, \omega}\) as simply \(\mathcal{A}_1\).

\label{def:a1}
\end{definition}

We now show that \(\mathcal{A}_1(K)\) is an upper bound of the area functional \(\mathcal{A}(K)\) if \(\mathbf{x}_K\) is injective and in the fan \(F_\omega\). Our key observation is the following.

\begin{lemma}

Let \(\omega \in (0, \pi/2]\) and \(K \in \mathcal{K}_{\omega}\) be arbitrary. Let \(\mathbf{z} : [t_0, t_1] \to \mathbb{R}^2\) be any open simple curve (that is, a curve with \(t_0 < t_1\) and injective parametrization \(\mathbf{z}\)) inside the set \(F_{\omega} \cap \bigcup_{0 \leq t \leq \omega} \overline{Q^-_K(t)}\), such that the starting point \(\mathbf{z}(t_0)\) is on the boundary \(l(\pi/2, 0) \cap F_\omega\) of \(F_\omega\), and the endpoint \(\mathbf{z}(t_1)\) is on the boundary \(l(\omega, 0) \cap F_\omega\) of \(F_\omega\). Then we have \(\mathcal{I}(\mathbf{z}) \leq |\mathcal{N}(K)|\).

\label{lem:curve-area-functional-lower-bound}
\end{lemma}

\begin{proof}
Define \(\mathbf{b}\) as the curve from \(\mathbf{z}(t_1)\) to \(\mathbf{z}(t_0)\) along the boundary \(\partial F_\omega\) of fan \(F_\omega\) (so \(\mathbf{b}\) is either a segment or the concatenation of two segments). Since \(\mathbf{z}\) is injective, we have \(\mathbf{z}(t_0) \neq \mathbf{z}(t_1)\) so \(\mathbf{b}\) is also an open simple curve. For every \(\epsilon > 0\), define the closed curve \(\Gamma_\epsilon\) as the concatenation of the following curves in order: the curve \(\mathbf{z}(t)\), the vertical segment from \(\mathbf{z}(t_1)\) to \(\mathbf{z}(t_1) - (0, \epsilon)\), the curve \(\mathbf{b} - (0, \epsilon)\) shifted downwards by \(\epsilon\), and then the vertical segment from \(\mathbf{z}(t_0) - (0, \epsilon)\) to \(\mathbf{z}(t_0)\). The curve \(\Gamma_{\epsilon}\) is a Jordan curve because \(\mathbf{z}\) is an open simple curve inside \(F_\omega\). By Jordan curve theorem, the curve \(\Gamma_\epsilon\) encloses an open set \(\mathcal{N}_\epsilon\). Define \(\mathcal{N}_0\) as the intersection \(F_{\omega} \cap \mathcal{N}_{\epsilon}\), then \(\mathcal{N}_0\) is independent of the choice of \(\epsilon > 0\); for any \(\epsilon_1 > \epsilon_2 > 0\), there is a continuous deformation of \(\mathbb{R}^2\) that fixes \(F_\omega\) and shrinks \(\mathbb{R}^2 \setminus F_\omega\) vertically so that it shrinks \(\Gamma_{\epsilon_1}\) to \(\Gamma_{\epsilon_2}\). Moreover, \(\mathcal{N}_{\epsilon}\) is the disjoint union of \(\mathcal{N}_0\) and the fixed region below \(\partial F_\omega\) of area \(\left| \mathbf{z}(t_1) - \mathbf{z}(t_0) \right| \epsilon\).

We have \(\left| \mathcal{N}_\epsilon \right| = \left| \mathcal{I}(\Gamma_\epsilon) \right|\) by Green’s theorem on \(\Gamma_\epsilon\) regardless of the orientation of \(\Gamma_\epsilon\). By sending \(\epsilon \to 0\), we have \(\left| \mathcal{N}_0 \right| = \left| \mathcal{I}(\mathbf{z}) \right|\). We now show \(\mathcal{N}_0 \subseteq \mathcal{N}(K)\) which finishes the proof. Take any \(p \in \mathcal{N}_0\). Take the ray \(r\) emanating from \(p\) in the direction \(v_0\), then it should cross some point \(q \neq p\) in the curve \(\mathbf{z}\). As \(\mathbf{z}\) is inside the set \(F_{\omega} \cap \bigcup_{0 \leq t \leq \omega} \overline{Q^-_K(t)}\), the point \(q\) is contained in \(F_{\omega} \cap \overline{Q_K^-(t)}\) for some \(0 \leq t \leq \omega\). We have \(t \neq 0, \omega\) because \(q\) is strictly above the boundary of \(F_\omega\), and for \(t=0, \pi/2\) the set \(Q^-_K(t)\) is either on or below \(\partial F_\omega\). Because the point \(p\) is in \(F_{\omega}\) and strictly below the point \(q\), it should be that \(p\) is contained in \(F_{\pi/2} \cap Q_K^-(t)\). So the point \(p\) is in the niche \(\mathcal{N}(K)\), and we have \(\mathcal{N}_0 \subseteq \mathcal{N}(K)\).
\end{proof}

We can freely choose the curve \(\mathbf{z}\) inside the set \(F_{\pi/2} \cap \bigcup_{0 \leq t \leq \pi/2} \overline{Q^-_K(t)}\). In this paper, we simply choose \(\mathbf{z} = \mathbf{x}_K\) and get the following.

\begin{theorem}

For any \(\omega \in (0, \pi/2]\) and \(K \in \mathcal{K}_{\omega}\), if the curve \(\mathbf{x}_K : [0, \omega] \to \mathbb{R}^2\) is injective and in \(F_\omega\), we have \(\mathcal{A}(K) \leq \mathcal{A}_1(K)\).

\label{thm:a1-upper-bound}
\end{theorem}

\begin{proof}
Since \(\mathbf{x}_K(t) \in \overline{Q_K^-(t)}\) for all \(t \in [0, \omega]\) and we assumed that \(\mathbf{x}_K(t) \in F_\omega\), the curve \(z := \mathbf{x}_K\) is an open simple curve inside \(F_{\omega} \cap \bigcup_{0 \leq t \leq \omega} \overline{Q^-_K(t)}\). Also, by \(p_K(\omega) = p_K(\pi/2) = 1\) we have \(\mathbf{x}_K(0) \in l(\pi/2, 0)\) and \(\mathbf{x}_K(\omega) \in l(\omega, 0)\). So the curve \(\mathbf{z} := \mathbf{x}_K\) satisfies the condition of \Cref{lem:curve-area-functional-lower-bound}, and we have \(\mathcal{I}(\mathbf{x}_K) \leq |\mathcal{N}(K)|\). So we have
\[
\mathcal{A}(K) = |K| - |\mathcal{N}(K)| \leq |K| - \mathcal{I}(\mathbf{x}_K) = \mathcal{A}_1(K)
\]
which finishes the proof.
\end{proof}

\subsection{Derivative of the inner corner}

The formula (\Cref{def:curve-area-functional}) of \(\mathcal{I}(\mathbf{x}_K)\) requires us to take derivative of the inner corner \(\mathbf{x}_K\). We end this subsection by calculating the derivative of \(\mathbf{x}_K\) explicitly, which will be used later. Define the \emph{arm lengths} \(g_K^{\pm}(t)\) and \(h_K^{\pm}(t)\) of tangent hallways of a cap \(K\) as the following.

\begin{definition}

Let \(K \in \mathcal{K}_\omega\) be arbitrary. For any \(t \in [0, \omega]\), let \(g_K^+(t)\) (resp. \(g_K^-(t)\)) be the unique real value such that \(\mathbf{y}_K(t) = A^+_K(t) + g_K^+(t) v_t\) (resp. \(\mathbf{y}(t) = A^-_K(t) + g_K^-(t) v_t\)). Similarly, let \(h_K^+(t)\) (resp. \(h_K^-(t)\)) be the unique real value such that \(\mathbf{y}_K(t) = C^+_K(t) + h_K^+(t) u_t\) (resp. \(\mathbf{y}(t) = C^-_K(t) + h_K^-(t) u_t\)).

\label{def:cap-tangent-arm-length}
\end{definition}

Observe that the point \(\mathbf{y}_K(t)\) and the vertices \(C_K^{\pm}(t)\) are on the tangent line \(c_K(t)\) in the direction of \(u_t\). Likewise \(y_K(t)\) and \(A_K^{\pm}(t)\) are on the tangent line \(a_K(t)\) in the direction of \(v_t\). So \Cref{def:cap-tangent-arm-length} is well-defined. For the same reason, we also have the following equations.

\begin{proposition}

Let \(K \in \mathcal{K}_\omega\) and \(t \in [0, \omega]\) be arbitrary. Then \(g_K^{\pm}(t) = \left( C_K^{\pm}(t) - A_K^{\pm}(t) \right) \cdot v_t\) and \(h_K^{\pm}(t) = (A_K^{\pm}(t) - C_K^{\pm}(t)) \cdot u_t\).

\label{pro:cap-tangent-arm-length}
\end{proposition}

The derivative of \(\mathbf{x}_K\) can be expressed in terms of the arm lengths of \(K\).

\begin{theorem}

For any cap \(K \in \mathcal{K}_\omega\), the right derivative of the outer and inner corner \(\mathbf{y}_K(t)\) exists for any \(0 \leq t < \omega\) and is equal to the following.

\begin{align*}
\partial^+ \mathbf{y}_K(t) = -g_K^+(t) u_t + h_K^+(t) v_t \qquad \partial^+ \mathbf{x}_K(t) = -(g_K^+(t) - 1) u_t + (h_K^+(t) - 1) v_t
\end{align*}
Likewise, the left derivative of \(\mathbf{y}_K\) and \(\mathbf{x}_K\) exists for all \(0 < t \leq \omega\) and is equal to the following.

\begin{align*}
\partial^- \mathbf{y}_K(t) = -g_K^-(t) u_t + h_K^-(t) v_t \qquad \partial^+ \mathbf{x}_K(t) = -(g_K^-(t) - 1) u_t + (h_K^-(t) - 1) v_t
\end{align*}

\label{thm:inner-corner-deriv}
\end{theorem}

\begin{proof}
Fix an arbitrary cap \(K\) and omit the subscript \(K\) in vertices \(\mathbf{y}_K(t)\), \(\mathbf{x}_K(t)\) and tangent lines \(a_K(t)\). Take any \(0 \leq t < \omega\) and set \(s = t + \delta\) for sufficiently small and arbitrary \(\delta > 0\). We evaluate \(\partial^+ \mathbf{y}(t) = \lim_{\delta \rightarrow 0^+}(\mathbf{y}(s) - \mathbf{y}(t)) / \delta\). Define \(A_{t, s} = a(t) \cap a(s)\). Since \(A_{t, s}\) is on the lines \(a(t)\) and \(a(s)\), it satisfies both \(A_{t, s} \cdot u_t = \mathbf{y}(t) \cdot u_t\) and \(A_{t, s} \cdot u_s = \mathbf{y}(s) \cdot u_s\). Rewrite \(u_s = \cos \delta \cdot u_t + \sin \delta \cdot v_t\) on the second equation and we have

\begin{align*}
	A_{t, s} \cos \delta \cdot u_t + A_{t, s} \sin \delta \cdot v_t =  	\cos \delta (\mathbf{y}(s) \cdot u_t) + \sin \delta (\mathbf{y}(s) \cdot v_t).
\end{align*}
Group by \(\cos \delta\) and \(\sin \delta\) and substitute \(A_{t, s} \cdot u_t\) with \(\mathbf{y}(t) \cdot u_t\), then
\[ \cos \delta (\mathbf{y}(s) \cdot u_t - \mathbf{y}(t) \cdot u_t)
	= \sin \delta (A_{t, s}  (s) \cdot v_t - \mathbf{y}(s) \cdot v_t) .
	\]
Divide by \(\delta\) and send \(\delta \to 0^+\). We get the following limit as \(A_{t, s} \to A^+(t)\) (\Cref{thm:limits-converging-to-vertex}).
\[ \partial^+ (\mathbf{y}(t) \cdot u_t)  = (A^+(t) - \mathbf{y}(t)) \cdot v_t = - g^+(t)\]
A similar argument can be applied to show \(\partial^+ (\mathbf{y}(t) \cdot v_t) = h^+(t)\) and thus the first equation of the theorem. The right derivative of \(\mathbf{x}_K(t)\) comes from \(\mathbf{x}_K(t) = \mathbf{y}_K(t) - u_t - v_t\). A symmetric argument calculates the left derivative of \(\mathbf{y}_K\) and \(\mathbf{x}_K\).
\end{proof}

We prepare some observations on arm lengths that will be used later.

\begin{remark}

The resulting equation \(\partial^+ \mathbf{x}_K(t) = -(g_K^+(t) - 1) u_t + (h_K^+(t) - 1) v_t\) in \Cref{thm:inner-corner-deriv} can be interpreted intuitively as the following. Imagine the hallway \(L_K(t)\) where we increment \(t\) slightly by \(\epsilon > 0\). If \(\epsilon\) is very small, the wall \(c_K(t)\) rotates around the pivot \(C_K^+(t)\). As \(\mathbf{x}_K(t)\) rotates differentially with the pivot \(C_K^+(t)\) as center, the \(v_t\) component \(h_K^+(t) - 1\) of the derivative \(\partial^+ \mathbf{x}_K(t)\) is the distance from the pivot \(C_K^+(t)\) to \(\mathbf{x}_K(t)\) measured in the direction of \(u_t\). The \(u_t\) component \(-(g_K^+(t) - 1)\) of \(\partial^+ \mathbf{x}_K(t)\) can be interpreted similarly as the distance from pivot \(A^+_K(t)\) to \(\mathbf{x}_K(t)\) along the direction \(v_t\).

\label{rem:inner-corner-deriv}
\end{remark}

Except for a countable number of \(t\), we do not need to differentiate \(g_K^+(t)\) and \(g_K^-(t)\) (and likewise for \(h_K^{\pm}(t)\)).

\begin{definition}

For any cap \(K\) of rotation angle \(\omega \in (0, \pi/2]\) and any angle \(t \in [0, \omega]\), if \(g_K^+(t) = g_K^-(t)\) (resp. \(h_K^+(t) = h_K^-(t)\)) then simply denote the matching value as \(g_K(t)\) (resp. \(h_K(t)\)).

\label{def:arm-length-unsinged}
\end{definition}

\begin{proposition}

Let \(K\) be any cap of rotation angle \(\omega \in (0, \pi/2]\) and take any angle \(t \in [0, \omega]\). The condition \(g_K^+(t) = g_K^-(t)\) (resp. \(h_K^+(t) = h_K^-(t)\)) holds if and only if \(\beta_K(\left\{ t \right\}) = 0\) (resp. \(\beta_K(\left\{ t + \pi/2 \right\}) = 0\)) by \Cref{thm:surface-area-singleton}. So \(g_K\) and \(h_K\) are almost everywhere defined and integrable functions on \([0, \omega]\).

\label{pro:arm-length-unsigned}
\end{proposition}

By \Cref{pro:cap-tangent-arm-length} and \Cref{thm:limits-converging-to-vertex} we have the following.

\begin{corollary}

Let \(K\) be any cap of rotation angle \(\omega \in (0, \pi/2]\) and take any angle \(t \in [0, \omega]\). If \(t > 0\), then \(g_K^{\pm}(u) \to g_K^-(t)\) and \(h_K^{\pm}(u) \to h_K^-(t)\) as \(u \to t^-\). If \(t < \omega\), then \(g_K^{\pm}(u) \to g_K^+(t)\) and \(h_K^{\pm}(u) \to h_K^+(t)\) as \(u \to t^+\).

\label{cor:arm-length-continuity}
\end{corollary}

\section{Calculus of Variation}
\label{sec:calculus-of-variation}
We will prove \Cref{thm:main-cap} by optimizing \(\mathcal{A}_1 : \mathcal{K}_\omega \to \mathbb{R}\). Here, we prepare a minimal amount of language needed to execute the calculus of variation to a concave quadratic functional \(f\) on a general convex space \(\mathcal{K}\). Later, we will show that our functional \(f = \mathcal{A}_1\) is concave and quadratic on the cap space \(\mathcal{K} = \mathcal{K}_\omega\), and then deploy the calculus of variation on \(\mathcal{A}_1\).

\begin{definition}

A \emph{convex space} is a set \(\mathcal{K}\) equipped with the \emph{convex combination} operation \(c_\lambda : \mathcal{K} \times \mathcal{K} \to \mathcal{K}\) for all \(\lambda \in [0, 1]\), such that the following list of equalities hold \cite{nlab-convex-space}.

\begin{enumerate}
\def\labelenumi{\arabic{enumi}.}
\tightlist
\item
  \(c_0(x, y) = x\)
\item
  \(c_\lambda(x, x) = x\) for all \(\lambda \in [0, 1]\)
\item
  \(c_\lambda (x, y) = c_{1 - \lambda}(y, x)\) for all \(\lambda \in [0, 1]\)
\item
  \(c_\lambda(x, c_\mu(y, z)) = c_{\lambda \mu} (c_\tau(x, y), z)\) for all \(\lambda, \mu, \tau \in [0, 1]\) such that \(\lambda (1 - \mu) = (1 - \lambda \mu) \tau\).
\end{enumerate}

\label{def:convex-space}
\end{definition}

\begin{proposition}

The space of caps \(\mathcal{K}_\omega\) in \Cref{def:cap-space} is a convex space, equipped with the convex combination

\begin{align*}
c_\lambda(K_1, K_2) & := (1 - \lambda)K_1 + \lambda K_2 \\
& = \left\{ (1 - \lambda) p_1 + \lambda p_2 : p_1 \in K_1, p_2 \in K_2 \right\} 
\end{align*}
given by the Minkowski sum of convex bodies.

\label{pro:cap-space-convex-space}
\end{proposition}

\begin{proof}
It is rudimentary to verify all the axiomatic equations in \Cref{def:convex-space}.
\end{proof}

\begin{definition}

A function \(f : \mathcal{K} \to V\) from a convex space \(\mathcal{K}\) to a convex space \(V\) is \emph{convex-linear} if \(f(c_\lambda(K_1, K_2)) = c_\lambda(f(K_1), f(K_2))\) for all \(K_1, K_2 \in \mathcal{K}\) and \(\lambda \in [0, 1]\). Call a functional \(f : \mathcal{K} \to \mathbb{R}\) on \(\mathcal{K}\) a \emph{linear functional} on \(\mathcal{K}\) if it is convex-linear.

\label{def:convex-linear}
\end{definition}

\begin{definition}

For convex spaces \(\mathcal{K}\) and \(V\), call a function \(g : \mathcal{K} \times \mathcal{K} \to V\) \emph{convex-bilinear} if the maps \(K \mapsto g(K_1, K)\) and \(K \mapsto g(K, K_2)\) are convex-linear for any fixed \(K_1, K_2 \in \mathcal{K}\).

\label{def:convex-bilinear}
\end{definition}

\begin{definition}

Call \(h : \mathcal{K} \to \mathbb{R}\) a \emph{quadratic functional} on a convex space \(\mathcal{K}\), if \(h(K) = g(K, K)\) for some convex-bilinear function \(g : \mathcal{K} \times \mathcal{K} \to \mathbb{R}\).

\label{def:convex-space-quadratic}
\end{definition}

The \emph{directional derivative} of a quadratic functional \(f\) on a convex space \(\mathcal{K}\) is the key for executing the calculus of variation on \(f\).

\begin{definition}

For any quadratic functional \(f : \mathcal{K} \to \mathbb{R}\) on a convex space \(\mathcal{K}\) and \(K, K' \in \mathcal{K}\), define the \emph{directional derivative}
\[
Df(K; K') = \left. \frac{d}{d \lambda} \right|_{\lambda = 0} f(c_\lambda(K, K'))
\]
of \(f\) at \(K\) from \(K\) to \(K'\).

\label{def:convex-space-directional-derivative}
\end{definition}

\Cref{def:convex-space-directional-derivative} can be seen as a generalization of the Gateaux derivative to functionals on a general convex space. For any quadratic functional \(f\) and a fixed \(K \in \mathcal{K}\), the value \(Df(K; K')\) is well-defined and always a linear functional of \(K'\).

\begin{lemma}

Let \(f\) be a quadratic functional on a convex space \(\mathcal{K}\), so that \(f(K) = h(K, K)\) for a convex-bilinear map \(h : \mathcal{K} \times \mathcal{K} \to \mathbb{R}\). Then we have the following for any \(K, K' \in \mathcal{K}\).
\[
Df(K; K') = h(K, K') + h(K', K) - 2 h (K, K)
\]
So in particular, the map \(Df(K; -) : \mathcal{K} \to \mathbb{R}\) is always well-defined and a linear functional.

\label{lem:derivative-calculation}
\end{lemma}

\begin{proof}
We have
\begin{equation}
\label{eqn:quadratic-functional}
\begin{split}
f(c_\lambda(K, K')) & = h(c_\lambda(K, K'), c_\lambda(K, K')) \\
& = (1 - \lambda)^2 h(K, K) + \lambda (1 - \lambda) \left( h(K, K') + h (K', K) \right) + \lambda^2 h(K', K')
\end{split}
\end{equation}
by bilinearity of \(h\). Now take the derivative at \(\lambda = 0\).
\end{proof}

\begin{definition}

A functional \(f : \mathcal{K} \to \mathbb{R}\) on a convex space \(\mathcal{K}\) is \emph{concave} (resp. \emph{convex}) if \(f(c_\lambda(K_1, K_2)) \geq (1 - \lambda) f(K_1) + \lambda f(K_2)\) (resp. \(f(c_\lambda(K_1, K_2)) \leq (1 - \lambda) f(K_1) + \lambda f(K_2)\)) for all \(K_1, K_2 \in \mathcal{K}\) and \(\lambda \in [0, 1]\).

\label{def:convex-space-concavity}
\end{definition}

To prove that \(K\) maximizes a concave quadratic functional \(f(K)\) on \(\mathcal{K}\), we only need to prove that \(Df(K; -)\) is a nonpositive linear functional on \(\mathcal{K}\).

\begin{theorem}

For any concave quadratic functional \(f\) on a convex space \(\mathcal{K}\), the value \(K \in \mathcal{K}\) maximizes \(f(K)\) if and only if the linear functional \(Df(K; -)\) is nonpositive.

\label{thm:quadratic-variation}
\end{theorem}

\begin{proof}
Assume that \(K\) is the maximizer of \(f(K)\). Then for any \(K' \in \mathcal{K}\), the value \(f(c_\lambda(K, K'))\) over all \(\lambda \in [0, 1]\) is maximized at \(\lambda = 0\). So taken derivative at \(\lambda = 0\), we should have \(Df(K; K') \leq 0\).

Now assume on the other hand that \(K \in \mathcal{K}\) is chosen such that \(Df(K; -)\) is always nonpositive. Take an arbitrary \(K' \in \mathcal{K}\) and fix it. Observe that \(f(c_\lambda(K, K'))\) is a polynomial \(p(\lambda)\) of \(\lambda \in [0, 1]\) by \Cref{eqn:quadratic-functional}. Because \(f\) is concave, the polynomial \(p(\lambda)\) is also concave with respect to \(\lambda\) and the quadratic coefficient of \(p(\lambda)\) is nonpositive. The linear coefficient of \(p(\lambda)\) is \(Df(K; K')\) which is nonpositive as well. So \(p(\lambda)\) is monotonically decreasing with respect to \(\lambda\) and we have \(f(K) \geq f(K')\) as desired.
\end{proof}

\section{Boundary Measure}
\label{sec:boundary-measure}
We will show that \(\mathcal{A}_1 : \mathcal{K}_\omega \to \mathbb{R}\) is a quadratic functional.

\begin{theorem}

For any \(\omega \in (0, \pi/2]\), the functional \(\mathcal{A}_1 : \mathcal{K}_{\omega} \to \mathbb{R}\) is quadratic.

\label{thm:a1-quadratic}
\end{theorem}

To establish \Cref{thm:a1-quadratic}, we will define the \emph{boundary measure} \(\beta_K\) of \(K \in \mathcal{K}_\omega\) and utilize it. Also, we will establish a correspondence between any cap \(K \in \mathcal{K}_\omega\) and its boundary measure \(\beta_K\) (\Cref{thm:boundary-measure-cap} and \Cref{thm:cap-from-boundary-measure}).

\subsubsection{Convex-linear Values of Cap}

We observe that a lot of values defined on the cap \(K \in \mathcal{K}_\omega\) is convex-linear with respect to \(K\). A reader interested in the details of proofs can read \Cref{sec:vertex-and-support-function} for the full details.

\begin{theorem}

The following values are convex-linear with respect to \(K \in \mathcal{K}_\omega\).

\begin{itemize}
\tightlist
\item
  Support function \(p_K\)
\item
  Vertices \(A^{\pm}_K(t)\) and \(C^{\pm}_K(t)\) for a fixed \(t \in [0, \omega]\)
\item
  The inner and outer corner \(\mathbf{x}_K(t)\) and \(\mathbf{y}_K(t)\) of the tangent hallway with any angle \(t \in [0, \omega]\)
\item
  The points \(W_K(t)\), \(Z_K(t)\) and the values \(w_K(t)\), \(z_K(t)\) for a fixed \(t \in (0, \omega)\)
\item
  The perpendicular leg lengths \(g^{\pm}_K(t)\) and \(h^{\pm}_K(t)\) for all \(t \in [0, \omega]\)
\end{itemize}

\label{thm:cap-convex-linear}
\end{theorem}

\begin{proof}
Use \Cref{pro:support-function-linear} for \(p_K\), \Cref{cor:vertex-linear} for \(A^{\pm}_K(t)\) and \(C^{\pm}_K(t)\), \Cref{lem:tangent-lines-intersection-linear} for \(\mathbf{y}_K(t), W_K(t)\), and \(Z_K(t)\). Use the equality \(\mathbf{y}_K(t) = \mathbf{x}_K(t) + u_t + v_t\) for \(\mathbf{x}_K(t)\), the equalities in \Cref{def:wedge-side-lengths} for \(w_K(t)\) and \(z_K(t)\), and the equalities in \Cref{def:cap-tangent-arm-length} for \(g^{\pm}_K(t)\) and \(h^{\pm}_K(t)\).
\end{proof}

\Cref{thm:cap-convex-linear} in particular establishes that the curve area functional \(\mathcal{I}(\mathbf{x}_K)\) (\Cref{def:curve-area-functional}) is quadratic with respect to \(K\).

\begin{corollary}

The value \(\mathcal{I}(\mathbf{x}_K)\) of a cap \(K \in \mathcal{K}_\omega\) is quadratic with respect to \(K\).

\label{cor:inner-corner-quadratic}
\end{corollary}

\subsubsection{Boundary Measure}

We now define the \emph{boundary measure} \(\beta_K\) of a cap \(K\) as the restriction of the \emph{surface measure} \(\sigma_K\) of \(K\) (\Cref{def:surface-area-measure}).

\begin{definition}

For any cap \(K \in \mathcal{K}_\omega\) with rotation angle \(\omega\), define the \emph{boundary measure} \(\beta_K\) of \(K\) on the set \(J_\omega\) (\Cref{def:j-cap}) as the surface area measure \(\sigma_K\) of \(K\) restricted to \(J_\omega\).

\label{def:boundary-measure}
\end{definition}

See \Cref{sec:surface-area-measure} for a brief introduction on the surface area measure \(\sigma_K\). The boundary measure \(\beta_K\) of cap \(K\) describes the information of length of the upper boundary \(\delta K\). For example, let \(K = [0, 1]^2 \cup \left\{ (x, y) : x \leq 0, y \geq 0, x^2 + y^2 \leq 1 \right\}\) be a cap with rotation angle \(\pi/2\), which is the union of a unit square and a quarter-circle of radius one. Then the boundary measure \(\beta_K\) is a measure on \(J_{\pi/2} = [0, \pi]\) such that \(\beta_K\left( \left\{ 0 \right\} \right) = \beta_K\left( \left\{ \pi/2 \right\} \right) = 1\), and \(\beta_K\) equal to zero on the interval \((0, \pi/2)\) and the Lebesgue measure \(\beta_K(dt) = dt\) on the interval \((\pi/2, \pi)\). We now collect the properties of \(\beta_K\).

\begin{proposition}

The boundary measure \(\beta_K\) is convex-linear with respect to \(K \in \mathcal{K}_\omega\).

\label{pro:boundary-measure-linear}
\end{proposition}

\begin{proof}
Immediate from \Cref{thm:surface-area-measure-convex-linear}.
\end{proof}

\begin{proposition}

For any cap \(K \in \mathcal{K}_\omega\), we have
\[
|K| = \left< p_K, \beta_K \right>_{J_\omega}.
\]

\label{pro:boundary-measure-area}
\end{proposition}

\begin{proof}
By \Cref{thm:surface-area-measure-area} we have \(|K| = \left< p_K, \sigma_K \right>_{S^1}\). Apply \Cref{thm:convex-set-support} to the second condition of \Cref{def:cap} to obtain that \(\sigma_K\) is supported on the set \(J_{\omega} \cup \left\{ \omega + \pi, 3\pi/2 \right\}\). The first condition of \Cref{def:cap} gives \(p_K(\omega + \pi) = p_K(3\pi/2) = 0\). From these, we have \(|K| = \left< p_K, \sigma_K \right>_{S^1} = \left< p_K, \beta_K \right>_{J_\omega}\).
\end{proof}

Now the quadraticity of \(|K|\) comes from convex-linearity of \(p_K\) (\Cref{pro:support-function-linear}) and \(\beta_K\) (\Cref{pro:boundary-measure-linear}) with respect to \(K\).

\begin{corollary}

The area \(|K|\) of a cap \(K \in \mathcal{K}_{\omega}\) is a quadratic functional on \(\mathcal{K}_\omega\)

\label{cor:area-quadratic-functional}
\end{corollary}

The quadraticity of \(\mathcal{A}_1\) is now obtained.

\begin{proof}[Proof of \Cref{thm:a1-quadratic}]
Immediate consequence of \Cref{cor:inner-corner-quadratic} and \Cref{cor:area-quadratic-functional}.
\end{proof}

Gauss-Minkowski theorem (\Cref{thm:gauss-minkowski}) states that any convex set \(K\), up to translation, corresponds one-to-one to a measure \(\sigma\) on \(S^1\) such that \(\int_{S^1}u_t\,\sigma(dt) = 0\) by taking the surface area measure \(\sigma = \sigma_K\). Using this correspondence, we can always construct a bijection between a cap \(K \in \mathcal{K}_\omega\) and its boundary measure \(\beta = \beta_K\).

\begin{theorem}

For any cap \(K \in \mathcal{K}_\omega\) with rotation angle \(\omega\), its boundary measure \(\beta_K\) satisfies the following equations.
\[
\int_{t \in [0, \omega]} \cos(t) \, \beta_K(dt) = 1 \qquad \int_{t \in [\pi/2, \omega + \pi/2]} \cos\left( \omega + \pi/2 - t \right)  \, \beta_K(dt) = 1
\]

\label{thm:boundary-measure-cap}
\end{theorem}

\begin{proof}
By the second condition of \Cref{def:cap} and \Cref{thm:convex-set-support}, we have \(\mathbf{n}(K) \subseteq J_\omega \cup \left\{ \pi + \omega, 3\pi/2 \right\}\). Now by \Cref{thm:convex-set-support-disjoint}, that the interval \((-\pi/2, 0)\) of \(S^1\) is disjoint from \(\Pi\) implies that the point \(A_K^-(0)\) is on the line \(l_K(3\pi/2)\) which is \(y=0\). Likewise, that the interval \((\omega, \pi/2)\) of \(S^1\) is disjoint from \(\Pi\) implies that the point \(A_K^+(\omega)\) is on the line \(l_K(\pi/2)\) which is \(y=1\). By \Cref{cor:boundary-measure-closed} we have
\[
\int_{t \in [0, \omega]} v_t \, \beta_K(dt) = A^+_K(\omega) - A^-_K(0)
\]
and by taking the dot product with \(v_0\), we have the first equality. The second equality can be proved similarly by measuring the displacement from \(C_K^+(\omega)\) to \(C_K^-(0)\) along the direction \(u_\omega\).
\end{proof}

\begin{theorem}

Take arbitrary \(\omega \in (0, \pi/2]\). Conversely to \Cref{thm:boundary-measure-cap}, let \(\beta\) be a measure on \(J_\omega\) that satisfies the following equations.
\[
\int_{t \in [0, \omega]} \cos(t) \, \beta(dt) = 1 \qquad \int_{t \in [\pi/2, \omega + \pi/2]} \cos\left( \omega + \pi/2 - t \right)  \, \beta(dt) = 1
\]
Then there exists a cap \(K \in \mathcal{K}_\omega\) such that \(\beta_K = \beta\). Such \(K\) is unique if \(\omega < \pi/2\), and unique up to horizontal translation if \(\omega = \pi/2\).

\label{thm:cap-from-boundary-measure}
\end{theorem}

\begin{proof}
We first show that there is a unique extension \(\sigma\) of \(\beta\) on the set \(\Pi = J_\omega \cup \{\pi + \omega, 3\pi/2\}\) such that \(\int_{t \in \Pi} v_t \, \sigma(dt) = 0\). The values of \(\sigma\) are determined on \(J_\omega\), and we need to find the values of \(\sigma(\left\{ \pi + \omega \right\})\) and \(\sigma(\left\{ 3 \pi/2 \right\})\) that satisfies the equation \(\int_{t \in \Pi} v_t \, \sigma(dt) = 0\).

If \(\omega = \pi/2\), then by subtracting the two equations in \Cref{thm:cap-from-boundary-measure} we have \(\int_{t \in [0, \pi]} \cos(t)\,\beta(dt) = 0\). So the equation \(\int_{t \in \Pi} v_t \, \sigma(dt) = 0\) becomes \(\sigma(\left\{ 3\pi/2 \right\}) = \int_{t \in [0, \pi]} \sin (t) \,\beta(dt)\) which immediately gives a unique solution \(\sigma\).

Now assume \(\omega < \pi/2\). Let \(A := \int_{t \in [0, \omega]}\sin(t)\,\beta(dt) \geq 0\), then we have \(\int_{t \in [0, \omega]} v_t \,\beta(dt) = - A u_0 + v_0\) by the first equality of \Cref{thm:cap-from-boundary-measure}. Likewise, if we let \(B := \int_{t \in [\pi/2, \omega + \pi/2]} \sin(\omega + \pi/2 - t)\,\beta(dt) \geq 0\), then we have \(\int_{t \in [\pi/2, \omega + \pi/2]}v_t\,\beta(dt) = B v_\omega - u_\omega\) by the second equality of \Cref{thm:cap-from-boundary-measure}. Now the equation \(\int_{t \in \Pi} v_t \, \sigma(dt) = 0\) we are solving for becomes
\[
(-Au_0 + v_0) + (Bv_\omega - u_\omega) + \sigma\left( \left\{ 3\pi/2 \right\}  \right)  u_0 - \sigma\left( \left\{ \pi + \omega \right\}  \right)  v_\omega = 0
\]
and \(\sigma(\left\{ \pi + \omega \right\}) = B + v_\omega \cdot o_\omega \geq 0\) and \(\sigma(\left\{ 3 \pi/2 \right\}) = A + u_0 \cdot o_\omega \geq 0\) (remark that \(o_\omega\) is in \Cref{def:parallelogram-vertices}) gives the unique solution of \(\sigma\).

We now use \Cref{cor:supported-gauss-measure} on the measure \(\sigma\) extended on the set \(\Pi\). There is a unique convex body \(K\) up to translation so that \(\mathbf{n}(K) \subseteq \Pi\) (see \Cref{def:convex-set-support}) and \(\sigma_K|_{\Pi} = \sigma\). Our goal now is to translate \(K\) so that it is a cap with rotation angle \(\omega\). Since \(\mathbf{n}(K) \subseteq \Pi\), any translate \(K\) satisfy the second condition of cap in \Cref{def:cap}. It remains to prove the first condition of \Cref{def:cap}.

The width of \(K\) along the directions \(u_\omega\) and \(v_0\) are equal to 1 by applying the equations given in \Cref{thm:cap-from-boundary-measure} to \Cref{cor:boundary-measure-width} with angles \(t = \omega, \pi/2\). If \(\omega = \pi/2\), we only need \(\beta_K(\pi/2) = 1\) for \(K\) to satisfy the first condition of \Cref{def:cap}, and such a cap \(K\) is unique up to horizontal translation. If \(\omega < \pi/2\), we need both \(\beta_K(\pi/2) = \beta_K(\omega) = 1\) to satisfy the first condition of \Cref{def:cap}, so such a cap \(K\) exists uniquely among all translates.
\end{proof}

\section{Concavity of $\mathcal{A}_1$}
\label{sec:concavity-of-a1}
We now show that \(\mathcal{A}_1\) is concave.

\begin{theorem}

For every rotation angle \(\omega \in (0, \pi/2]\), the functional \(\mathcal{A}_1 : \mathcal{K}_\omega \to \mathbb{R}\) is concave.

\label{thm:a1-negative-semidefinite}
\end{theorem}

\emph{Mamikon’s theorem} \autocite{mnatsakanianAnnularRingsEqual1997} is used to prove \Cref{thm:a1-negative-semidefinite}. To explain Mamikon’s theorem, first assume an arbitrary convex body \(K\) (see \Cref{fig:mamikon}). Also, for every angle \(t\) in a fixed interval \([t_0, t_1]\), assume a tangent segment \(s(t)\) of \(K\) of length \(f(t)\). The segment \(s(t)\) has an endpoint \(v_K^+(t)\) on \(\partial K\), and the other endpoint \(\mathbf{y}(t)\) on the tangent line \(l_K(t)\) of \(K\) with angle \(t\) from the \(y\)-axis. Mamikon’s theorem states that the area swept by the segment \(s(t)\) from \(t=t_1\) to \(t=t_2\) has an area of \(\frac{1}{2}\int_{t_1}^{t_2} f(t)\,dt\).

\begin{figure}
\centering
\includesvg[width=0.7\textwidth,height=\textheight]{images/mamikon.svg}
\caption{An illustration of Mamikon’s theorem (\Cref{thm:mamikon}).}
\label{fig:mamikon}
\end{figure}

For the proof of \Cref{thm:a1-negative-semidefinite}, we will use \Cref{thm:mamikon} and \Cref{thm:mamikon-closed} which are the precise statements of Mamikon’s theorem that work for any convex body \(K\) with potentially non-differentiable boundaries (see \Cref{sec:mamikon's-theorem} for details). Using Mamikon’s theorem, we will express the area \(\mathcal{S}(K)\) of a. Then we will use the following \Cref{lem:sum-of-squares} to show that \(\mathcal{S}(K)\) is \emph{concave} with respect to \(K\). By showing that \(\mathcal{S}(K) + \mathcal{A}_1(K)\) is linear with respect to \(K\),

\begin{lemma}

Let \(h : \mathcal{K} \to V\) be a convex-linear map from a convex space \(\mathcal{K}\) to a real vector space \(V\) with inner product \(\left< -, - \right>_V\). Then the quadratic form \(f\) on \(\mathcal{K}\) defined as \(f(K) = \left< h(K), h(K) \right>_V\) is convex.

\label{lem:sum-of-squares}
\end{lemma}

\begin{proof}
Take arbitrary \(K_1, K_2 \in \mathcal{K}\). Fixing \(K_1\) and \(K_2\), observe that \(f(c_\lambda(K_1, K_2))\) is a quadratic polynomial of \(\lambda \in [0, 1]\) with the leading coefficient \(\left\lVert h(K_2) - h(K_1) \right\rVert_V^2\) of \(\lambda^2\), by expanding the term \(h(c_\lambda(K_1, K_2)) = h(K_1) + \lambda (h(K_2) - h(K_1))\) with respect to the inner product \(\left< -, - \right>_V\). This shows the convexity of \(f\) along the line segment connecting \(K_1\) and \(K_2\). Since \(K_1\) and \(K_2\) are chosen arbitrarily, \(f\) is convex.
\end{proof}

\begin{figure}
\centering
\includesvg[width=0.5\textwidth,height=\textheight]{images/mamikon-sofa.svg}
\caption{Mamikon’s theorem applied to the region with area \(\mathcal{S}(K)\), bounded from below by the upper boundary \(\delta K\) of cap \(K\), and bounded from above by outer corner \(\mathbf{y}_K : [0, \omega] \to \mathbb{R}^2\) of tangent hallways \(L_K(t)\).}
\label{fig:mamikon-sofa}
\end{figure}

\begin{proof}[Proof of \Cref{thm:a1-negative-semidefinite}]
(See \Cref{fig:mamikon-sofa}) We will define \(\mathcal{S}(K)\) as the area of the region bounded from below by the upper boundary \(\delta K\) of cap \(K\), and bounded from above by curve \(\mathbf{y}_K : [0, \omega] \to \mathbb{R}^2\). We will express \(\mathcal{S}(K)\) as an integral of squares using Mamikon’s theorem, then use \Cref{lem:sum-of-squares} to show that \(\mathcal{S}(K)\) is convex with respect to \(K\). Then we will show that \(\mathcal{S}(K) + \mathcal{A}_1(K)\) is linear with respect to \(K\). This will complete the proof of concavity of \(\mathcal{A}_1(K)\).

We first show \(|K| = \mathcal{I}(\delta K)\), where \(\delta K\) is the upper boundary of \(K\). By \Cref{cor:upper-boundary-param}, \(\delta K\) is the segment \(\mathbf{b}_K^{0-, \pi/2 + \omega}\) of the whole boundary \(\partial K\). By \Cref{thm:param-positive-jordan} and \Cref{thm:param-positive-area}, we have \(|K| = \mathcal{I}(\partial K)\). By the second condition of \Cref{def:cap}, the boundary \(\partial K\) is the concatenation of \(\delta K\) and two line segments \(e_K(3\pi/2)\), \(e_K(\pi + \omega)\) if \(\omega < \pi/2\), or one line segment \(e_K(3\pi/2)\) if \(\omega = \pi/2\). In either case, the line segments are aligned with the origin \(O\), so their curve area functionals (\Cref{def:curve-area-functional-segment}) are zero. This shows \(|K| = \mathcal{I}(\delta K)\) as we wanted.

Next, we subdivide the upper boundary \(\delta K = \mathbf{b}_K^{0-, \pi/2+\omega}\) into two curves \(\mathbf{b}_1 := \mathbf{b}_K^{0-, \omega}\) and \(\mathbf{b}_2 := \mathbf{b}_K^{\omega, \omega + \pi/2}\) using \Cref{cor:closed-param-concatenation}. The curve \(\mathbf{b}_1\) is the path from \(A_K^-(0)\) to \(A_K^+(\omega)\) along \(\delta K\), and the curve \(\mathbf{b}_2\) is the path from \(A_K^+(\omega)\) to \(C_K^+(\omega)\) along \(\delta K\). Now we have \(|K| = \mathcal{I}(\delta K) = \mathcal{I}(\mathbf{b}_1) + \mathcal{I}(\mathbf{b}_2)\).

We now show that the area \(\mathcal{S}(K)\) of the region bounded from below by \(\delta K\) and bounded from above by \(\mathbf{y}_K : [0, \omega] \to \mathbb{R}^2\) is convex with respect to \(K\). The region is enclosed two curves \(\delta K\), \(\mathbf{y}_K\) and two line segment from \(\mathbf{y}_K(\omega)\) to \(C_K^+(\omega)\) and from \(A_K^-(0)\) to \(\mathbf{y}_K(0)\) respectively. Accordingly, define \(\mathcal{S}(K)\) as the value
\[
\mathcal{S}(K) := \mathcal{I}(\mathbf{y}_K) + \mathcal{I}(\mathbf{y}_K(\omega), C_K^+(\omega)) - \mathcal{I}(\delta K) - \mathcal{I}(\mathbf{y}_K(0), A_K^-(0)).
\]

We will express \(\mathcal{S}(K)\) as a sum of integrals of squares by stitching two instances of Mamikon’s theorem on curves \(\mathbf{b}_1\) and \(\mathbf{b}_2\) respectively. First, apply \Cref{thm:mamikon-closed} to the curve \(\mathbf{b}_1\) and the outer corner \(\mathbf{y}_K(t)\) for \(t \in [0, \omega]\). Note that the value \(h_K^+(t)\) in \Cref{def:cap-tangent-arm-length} measures the distance from the point \(v_K^+(t)\) on \(\mathbf{b}_1\) to \(\mathbf{y}_K(t)\). Now we get

\begin{align*}
\mathcal{I}(\mathbf{y}_K) + \mathcal{I}(\mathbf{y}_K(\omega), A_K^+(\omega)) - \mathcal{I}(\mathbf{b}_1) - \mathcal{I}(\mathbf{y}_K(0), A_K^-(0)) & = \frac{1}{2} \int_0^\omega h^+_K(t)^2 \, dt.
\end{align*}
Second, apply \Cref{thm:mamikon-tangent-line} to the curve \(\mathbf{b}_2\) and the tangent line \(c_K(\omega)\) (which is \(l_K(\pi/2 + \omega)\) by \Cref{pro:rotating-hallway-parts}) of \(K\) with angle range \(t \in [\omega, \pi/2 + \omega]\), to get

\begin{align*}
\mathcal{I}(\mathbf{y}_K(\omega), C^+_K(\omega)) -
\mathcal{I}(\mathbf{b}_2) - 
\mathcal{I}(\mathbf{y}_K(\omega), A_K^+(\omega))
& = \frac{1}{2} \int_{\omega}^{\pi/2 + \omega} \tau_K(t, \omega + \pi/2)^2 \, dt.
\end{align*}
Note that the value \(\tau_K(t, \omega + \pi/2)\) measures the distance from the point \(v_K^+(t)\) on \(\mathbf{b}_2\) to the intersection \(l_K(t) \cap c_K(\omega)\) (\Cref{def:tangent-leg-length}).

Add the two equations from Mamikon’s theorem and use \(\mathcal{I}(\delta K) = \mathcal{I}(\mathbf{b}_1) + \mathcal{I}(\mathbf{b}_2)\) to express \(\mathcal{S}(K)\) as a sum of integrals of squares.
\begin{equation}
\label{eqn:sk}
\begin{split}
\mathcal{S}(K) & = \frac{1}{2} \int_0^\omega h^+_K(t)^2 \, dt +  \frac{1}{2} \int_{\omega}^{\pi/2 + \omega} \tau_K(t, \omega + \pi/2)^2 \, dt
\end{split}
\end{equation}
Note that the base \(h_K^+(t)\) and \(\tau_K(t, \omega + \pi/2)\) of integrands are convex-linear with respect to \(K\) by \Cref{thm:cap-convex-linear} and \Cref{cor:tangent-line-length-linear}. So the term \(\mathcal{S}(K)\) is convex with respect to \(K\) by \Cref{lem:sum-of-squares}.

Next, we show that \(\mathcal{S}(K) + \mathcal{A}_1(K)\) is convex-linear with respect to \(K\). Using \(\mathcal{A}_1(K) = |K| - \mathcal{I}(\mathbf{x}_K) = \mathcal{I}(\delta K) - \mathcal{I}(\mathbf{x}_K)\), we have
\[
\mathcal{S}(K) + \mathcal{A}_1(K) = \left( \mathcal{I}(\mathbf{y}_K) - \mathcal{I}(\mathbf{x}_K) \right) + \mathcal{I}(\mathbf{y}_K(\omega), C_K^+(\omega)) - \mathcal{I}(\mathbf{y}_K(0), A_K^-(0)).
\]
Observe that the points \(O, \mathbf{y}_K(\omega), C_K^+(\omega)\) form the vertices of a right-angled triangle of height 1 along the direction \(u_\omega\) with base \(p_K(\omega + \pi/2)\). So the term \(\mathcal{I}(\mathbf{y}_K(\omega), C_K^+(\omega)) = p_K(\pi/2 + \omega) / 2\) is linear with respect to \(K\) (\Cref{pro:support-function-linear}). Likewise, the points \(O, \mathbf{y}_K(0), A_K^-(0)\) form the vertices of a right-angled triangle of height 1 along the direction \(v_0\) with base \(p_K(0)\). So the term \(\mathcal{I}(\mathbf{y}_K(0), A_K^-(0)) = -p_K(\omega)/2\) is linear with respect to \(K\) (\Cref{pro:support-function-linear}). It remains to show that \(\mathcal{I}(\mathbf{y}_K) - \mathcal{I}(\mathbf{x}_K)\) is linear with respect to \(K\). Note that the difference \(\mathbf{y}_K(t) - \mathbf{x}_K(t) = u_t + v_t\) of \(\mathbf{y}_K\) and \(\mathbf{x}_K\) is constant with respect to \(K\). Write \(u_t + v_t\) as \(c_t\) for simplicity, then we have
\[
\begin{split}
\mathcal{I}(\mathbf{y}_K) & = \frac{1}{2} \int_{0}^\omega \mathbf{y}_K(t) \times d \mathbf{y}_K(t) \\
& = \frac{1}{2} \int_{0}^\omega (\mathbf{x}_K(t) + c_t) \times d (\mathbf{x}_K(t) + c_t)  \\
& = \mathcal{I}(\mathbf{x}_K) + \frac{1}{2} \left( \int_{0}^\omega c_t \times d \mathbf{x}_K(t) 
+ \int_{0}^\omega \mathbf{x}_K(t) \times d c_t + \int_{0}^\omega c_t \times d c_t \right) 
\end{split}
\]
so \(\mathcal{I}(\mathbf{y}_K) - \mathcal{I}(\mathbf{x}_K)\) is convex-linear with respect to \(K\), by the convex-linearity of \(\mathbf{x}_K\) (\Cref{thm:cap-convex-linear}).

Since \(\mathcal{S}(K)\) is convex and \(\mathcal{S}(K) + \mathcal{A}_1(K)\) is convex-linear, \(\mathcal{A}_1(K)\) is concave with respect to \(K\).
\end{proof}

\section{Directional Derivative of $\mathcal{A}_1$}
\label{sec:directional-derivative-of-a1}
In this section, we calculate the directional derivative \(D\mathcal{A}_1(K; -)\) of \(\mathcal{A}_1\) (\Cref{def:convex-space-directional-derivative}) at any \(K \in \mathcal{K}_{\omega}\). As \(\mathcal{A}_1(K) = \left| K \right| - \mathcal{I}(\mathbf{x}_K)\), we calculate the directional derivative of \(|K|\) and \(\mathcal{I}(\mathbf{x}_K)\) separately.

First we calculate the derivative of the area \(|K|\) with respect to \(K\).

\begin{theorem}

Let \(K\) and \(K'\) be arbitrary convex bodies. Then we have
\[
\left. \frac{d}{d \lambda} \right|_{\lambda=0} \left| (1-\lambda) K + \lambda K' \right|  = \left< p_{K'} - p_K , \beta_K \right>_{S^1}.
\]

\label{thm:convex-body-area-variation}
\end{theorem}

\begin{proof}
For any convex body \(K\) we have \(|K| = V(K, K)\) where \(V\) is the mixed volume of two planar convex bodies. So by applying \Cref{lem:derivative-calculation} to \(|K| = V(K, K)\) and using that \(V(K, K') = V(K', K)\), we have the following.
\[
\left. \frac{d}{d \lambda} \right|_{\lambda=0} \left| (1-\lambda) K + \lambda K' \right| = 2V(K', K) - 2V(K, K)  
\]
By applying \Cref{thm:surface-area-measure-area} we get the result.
\end{proof}

We now calculate the derivative of \(\mathcal{I}(\mathbf{x}_K)\) with respect to \(K\). We have the following general calculation for the curve area functional \(\mathcal{I}(\mathbf{x})\).

\begin{theorem}

Let \(\mathbf{x}_1, \mathbf{x}_2 : [a, b]\to\mathbb{R}^2\) be two rectifiable curves. Then the following holds.
\[
\left. \frac{d}{d \lambda} \right|_{\lambda = 0} \mathcal{I}((1 - \lambda)\mathbf{x}_1 + \lambda \mathbf{x}_2) = \left[ \int_a^b (\mathbf{x}_2(t) - \mathbf{x}_1(t))  \times d\mathbf{x}_1 (t) \right] +  \mathcal{I}(\mathbf{x}_1(b), \mathbf{x}_2(b)) - \mathcal{I}(\mathbf{x}_1(a), \mathbf{x}_1(a))
\]

\label{thm:variation-curve}
\end{theorem}

\begin{proof}
Consider the bilinear form \(\mathcal{J}(\mathbf{x}_1, \mathbf{x}_2) = \int_a ^b \mathbf{x}_1(t) \times d \mathbf{x}_2(t)\) on rectifiable \(\mathbf{x}_1, \mathbf{x}_2 : [a, b] \to \mathbb{R}^2\). Apply \Cref{lem:derivative-calculation} to \(2\mathcal{I}(\mathbf{x}) = \mathcal{J}(\mathbf{x}, \mathbf{x})\) to get
\begin{equation}
\label{eqn:variation-curve}
\left. \frac{d}{d \lambda} \right|_{\lambda = 0} 2\mathcal{I}((1 - \lambda)\mathbf{x}_1 + \lambda \mathbf{x}_2) = \mathcal{J}(\mathbf{x}_1, \mathbf{x}_2) + \mathcal{J}(\mathbf{x}_2, \mathbf{x}_1) - 2 \mathcal{J}(\mathbf{x}_1, \mathbf{x}_1).
\end{equation}
Using the integration by parts (\Cref{pro:lebesgue-stieltjes-product}), we have
\[
\int_a^b \mathbf{x}_1(t) \times d \mathbf{x}_2(t) = \mathbf{x}_1 (b) \times \mathbf{x}_2(b) - \mathbf{x}_1(a) \times \mathbf{x}_2(a) + \int_a^b \mathbf{x}_2(t) \times d\mathbf{x}_1 (t)
\]
or
\[
\mathcal{J}(\mathbf{x}_1, \mathbf{x}_2) = 2\mathcal{I}(\mathbf{x}_1(b), \mathbf{x}_2(b)) - 2\mathcal{I}(\mathbf{x}_1(a) - \mathbf{x}_2(a)) + \mathcal{J}(\mathbf{x}_2, \mathbf{x}_1).
\]
Plug this back in \Cref{eqn:variation-curve} and rearrange to get the claimed equality in \Cref{thm:variation-curve}.
\end{proof}

We introduce a measure \(\iota_K\) to calculate the derivative of \(\mathcal{I}(\mathbf{x}_K)\) with respect to cap \(K\).

\begin{definition}

For any cap \(K \in \mathcal{K}_{\omega}\), define the function \(i_K : J_\omega \to \mathbb{R}\) as \(i_K(t) = h_K^+(t) - 1\) and \(i_K(t + \pi / 2) = g^+_K(t) - 1\) for every \(t \in [0, \omega]\). Define \(\iota_K\) as the measure on \(J_\omega\) derived from the density function \(i_K\). That is, \(\iota_K(dt) = i_K(t) dt\).

\label{def:i-cap}
\end{definition}

\Cref{def:i-cap} is motivated by the following lemma.

\begin{lemma}

Let \(I \subseteq [0, \omega]\) be an arbitrary Borel subset. Let \(K_1, K_2 \in \mathcal{K}_{\omega}\) be arbitrary. Then the following holds.
\[
\int_{t \in I} \mathbf{x}_{K_1}(t) \times d \mathbf{x}_{K_2} (t) = \left< p_{K_1} - 1, \iota_{K_2} \right>_{I \cup (I + \pi/2)} 
\]

\label{lem:i-cap}
\end{lemma}

\begin{proof}
By \Cref{thm:inner-corner-deriv} and \Cref{pro:arm-length-unsigned}, the derivative of \(\mathbf{x}_{K_2}(t)\) with respect to \(t\) exists almost everywhere and is the following.
\[
\mathbf{x}'_{K_2}(t) = -(g_{K_2}^+(t) - 1) u_t + (h_{K_2}^+(t) - 1) v_t
\]
Meanwhile, we have the following.
\[
\mathbf{x}_{K_1}(t) = (p_{K_1} (t) - 1) u_t + 
(p_{K_1} (t + \pi / 2) - 1) v_t
\]
So the cross-product \(\mathbf{x}_{K_1}(t) \times \mathbf{x}_{K_2}'(t)\) is equal to the following almost everywhere.
\[
(h_{K_2}^+(t) - 1) (p_{K_1} (t) - 1) + (g_{K_2}^+(t) - 1) (p_{K_1} (t + \pi / 2) - 1)
\]
Now the left-hand side is equal to
\[
\int_{t \in J} (h_{K_2}^+(t) - 1) (p_{K_1} (t) - 1) + (g_{K_2}^+(t) - 1) (p_{K_1} (t + \pi / 2) - 1) \, dt
\]
and by \Cref{def:i-cap} this integral is equal to \(\left< p_{K_1} - 1, \iota_{K_2} \right>_{I \cup (I + \pi/2)}\).
\end{proof}

\begin{theorem}

Let \(K_1\) and \(K_2\) be two caps in \(\mathcal{K}_{\omega}\). Then we have the following.
\[
D\mathcal{A}_1(K_1; K_2) = \left. \frac{d}{d\lambda} \right|_{\lambda = 0} \mathcal{A}_1((1 - \lambda) K_1 + \lambda K_2)
= \left< p_{K_2} - p_{K_1}, \beta_{K_1} - \iota_{K_1} \right>_{J_\omega}
\]

\label{thm:variation-a1}
\end{theorem}

\begin{proof}
We have \(\mathcal{A}_1(K) = |K| - \mathcal{I}(\mathbf{x}_K)\). Apply \Cref{thm:convex-body-area-variation} to the term \(|K|\) to have the following.
\[
\left. \frac{d}{d \lambda} \right|_{\lambda=0} \left| (1-\lambda) K_1 + \lambda K_2 \right|  = \left< p_{K_2} - p_{K_1} , \beta_{K_1} \right>_{S^1}
\]
Note that \(\beta_{K_1}\) and \(\beta_{K_2}\) are supported on the set \(J_\omega \cup \left\{ \pi + \omega, 3\pi/2 \right\}\), and both \(p_{K_2}\) and \(p_{K_1}\) have function value equal to 1 on the set \(\left\{ \pi + \omega, 3\pi/2 \right\}\). So we have \(\left< p_{K_2} - p_{K_1} , \beta_{K_1} \right>_{S^1} = \left< p_{K_2} - p_{K_1} , \beta_{K_1} \right>_{J_\omega}\).

Apply \Cref{thm:variation-curve} to the term \(\mathcal{I}(\mathbf{x}_K)\), and use that the points \(O\), \(\mathbf{x}_{K}(0)\), and \(A^-_K(0)\) (respectively, the points \(O\), \(\mathbf{x}_K(\omega)\), and \(C_K^+(\omega)\)) are colinear to get the following.

\begin{align*}
\left. \frac{d}{d \lambda} \right|_{\lambda = 0} \mathcal{I}((1 - \lambda)\mathbf{x}_{K_1} + \lambda \mathbf{x}_{K_2}) & = \int_0^{\omega} (\mathbf{x}_{K_2}(t) - \mathbf{x}_{K_1}(t)) \times d\mathbf{x}_{K_1}(t) \\
& = \left< p_{K_2} - p_{K_1}, \iota_{K_1} \right> 
\end{align*}
The second equality comes from applying \Cref{lem:i-cap} twice. Subtract the derivates of \(|K|\) and \(\mathcal{I}(\mathbf{x}_K)\) above to conclude the proof.
\end{proof}

We explain the intuitive meaning of \Cref{thm:variation-a1} by comparing it to the local optimization argument of Theorem 2 in \autocite{romikDifferentialEquationsExact2018}. Assume for the sake of explanation that \(S\) is a monotone sofa of rotation angle \(\pi/2\) with cap \(K\), such that the niche \(\mathcal{N}(K)\) is exactly the region bounded by the curve \(\mathbf{x}_K(t)\). Assume that our cap \(K\) has vertices \(A_K(t) = A_K^{\pm}(t)\) and \(C_K(t) = C_K^{\pm}(t)\) continuously differentiable with respect to \(t\). Assume also that \(\mathbf{x}_K(t)\) is continuously differentiable.

Take an arbitrary angle \(t_0\) and fix small positive \(\delta\) and \(\epsilon\). In \autocite{romikDifferentialEquationsExact2018}, Romik pertubed the sofa \(S\) to obtain a new sofa \(S'\) as the following. Initially, the monotone sofa \(S = H \cap \bigcap_{0 \leq t \leq \pi/2} L_K(t)\) is the intersection of rotating hallways \(L_K(t)\). For every \(t \in [0, \pi/2]\), Romik pertubed each hallway \(L_K(t)\) to a new hallways \(L'(t)\) as the following.

\begin{itemize}
\tightlist
\item
  For every \(t \in [t_0, t_0 + \delta]\), let \(L'(t) = L_K(t) + \epsilon u_t\).
\item
  For every other \(t\), let \(L'(t) = L_K(t)\).
\end{itemize}

That is, we move \(L_S(t)\) in the direction of \(\epsilon u_t\) for only \(t \in [t_0, t_0 + \delta]\). Now define \(S' = H \cap \bigcap_{0 \leq u \leq \pi/2} L'(u)\) so that \(S'\) is a sofa which is a slight perturbation of \(S\). If \(S\) attains the maximum area, it should be that \(S'\) has area equal to or less than \(S\).

We now compare the area of \(S\) and \(S'\). As we perturb \(S\) to \(S'\), some area is gained near \(A_K(t_0)\) as the walls \(a_K(t)\) are pushed in the direction of \(\epsilon u_t\) for \(t \in [t_0, t_0 + \delta]\). The gain near \(A_K(t_0)\) is approximately \(\epsilon \delta \left\lVert A_K'(t_0) \right\rVert\), as the shape of the gain is approximately a rectangle of sides \(A_K(t_0 + \delta) - A_K(t_0) \simeq \delta \left\lVert A_K'(t_0) \right\rVert v_0\) and \(\epsilon u_{t_0}\). Likewise, some area is lost near \(\mathbf{x}_K(t_0)\) as we perturb \(S\) to \(S'\) as the corners \(\mathbf{x}_K(t)\) are pushed in the direction of \(\epsilon u_t\) for \(t \in [t_0, t_0 + \delta]\). The loss near \(\mathbf{x}_K(t_0)\) is \(\epsilon \delta \mathbf{x}_K'(t_0) \cdot v_{t_0}\) as the shape of the loss is approximately a parallelogram of sides \(\delta \mathbf{x}_K'(t_0)\) and \(\epsilon u_{t_0}\). So the total gain of area from \(S\) to \(S'\) is approximately \(\epsilon \delta \left( \left\lVert A_K'(t_0) \right\rVert - \mathbf{x}'(t_0) \cdot v_{t_0} \right)\). In \autocite{romikDifferentialEquationsExact2018}, Romik solved for the critical condition \(\left\lVert A_K'(t_0) \right\rVert = \mathbf{x}_K'(t_0) \cdot v_{t_0}\) (and another condition \(\left\lVert C_K'(t_0) \right\rVert = - \mathbf{x}_K'(t_0) \cdot u_{t_0}\) obtained by perturbing each \(L_K(t)\) in the orthogonal direction of \(\epsilon v_{t_0}\)) to derive an ordinary differential equation of \(\mathbf{x}_K\) (ODE3 of Theorem 2, \autocite{romikDifferentialEquationsExact2018}).

We now observe that this total gain of area \(\epsilon \delta \left\lVert A_K'(t) \right\rVert - \epsilon \delta \mathbf{x}'(t) \cdot v_t\) is captured in \Cref{thm:variation-a1}. The perturbation of hallways from \(L_K(t)\) to \(L'(t)\) in \(\epsilon u_t\) can be described in terms of their support functions as \(p_{K'} = p_K + \epsilon 1_{[t, t + \delta]}\). Correspondingly, the value \(\left< p_{K'} - p_{K}, \beta_{K} \right>_{[0, \pi]} = \epsilon \beta_{K}((t, t + \delta])\) is approximately \(\epsilon \delta \left\lVert A_K'(t) \right\rVert\) which is equal to the gain of \(|K|\) near \(A_K(t)\). The value \(\left< p_{K'} - p_{K}, \iota_{K} \right>_{[0, \pi]} = \epsilon \iota_{K}((t, t + \delta])\), by \Cref{thm:inner-corner-deriv} and \Cref{def:i-cap}, is approximately \(\epsilon \delta \mathbf{x}_{K}'(t) \cdot v_t\) which is equal to the loss of area by the gain of \(\mathcal{N}(K)\) near \(\mathbf{x}_K(t)\). Summing up, we have \(\left< p_{K'} - p_{K}, \beta_{K} - \iota_{K} \right>_{[0, \pi]}\) measuring the total difference in the area of \(S\).

To summarize, the values \((p_{K'} - p_{K})(t)\) and \((p_{K'} - p_{K})(t + \pi/2)\) measures the movement of \(\mathbf{x}_K(t)\) along the direction \(u_t\) and \(v_t\) respectively. Then the measure \(\beta_{K}\) near \(t\) and \(t + \pi/2\) respectively measures the differential side lengths of the boundary of \(K\) near \(A_K(t)\) and \(C_K(t)\) respectively. Likewise, \(i_K(t)\) and \(i_K(t + \pi/2)\) measures the component of \(\mathbf{x}'(t)\) in direction of \(v_t\) and \(-u_t\) respectively. The formula in \Cref{thm:variation-a1} multiplies the contribution of change in \(p_K\) with , to get the change in total area of \(S\) for angle \(t\). Then the formula in \Cref{thm:variation-a1} sums this change in area over all \(t\) to get the derivative of \(\mathcal{A}_1(K)\) with respect to \(K\).

\section{Maximizer of $\mathcal{A}_1$}
\label{sec:maximizer-of-a1}
We now solve for the maximizer \(K = K_{\omega, 1}\) of our concave quadratic upper bound \(\mathcal{A}_1 : \mathcal{K}_\omega \to \mathbb{R}\). We do this by solving for the cap \(K\) that satisfies \(\beta_{K} = \iota_{K}\) on the set \(J_\omega \setminus \left\{ \omega, \pi/2 \right\}\). Once we find such \(K\), then it happens that the directional derivative \(D\mathcal{A}_1(K; K') = 0\) for every other cap \(K'\) by \Cref{thm:variation-a1} because \(p_{K'}(\omega) = p_K(\omega) = p_{K'}(\pi/2) = p_K(\pi/2) = 1\). Then by \Cref{thm:quadratic-variation} the cap \(K\) attains the maximum value of \(\mathcal{A}_1\).

As we have seen in the previous subsection, the equation \(\beta_{K} = \iota_{K}\) on the set \(J_\omega \setminus \left\{ \omega, \pi/2 \right\}\) can be compared to the local optimality equation (ODE3) in Theorem 2 of \autocite{romikDifferentialEquationsExact2018}. However, unlike \autocite{romikDifferentialEquationsExact2018} which solved the equation for the inner corner \(\mathbf{x}_K\), we will solve for the arm lengths \(g_K^{\pm}(t)\) and \(h_K^{\pm}(t)\) to find \(K\). This will lead to a simpler set of differential equations to solve. From now on, let \(K \in \mathcal{K}_{\omega}\) be an arbitrary cap with rotation angle \(\omega \in (0, \pi/2]\).

The following theorem shows that under sufficient conditions, we do not need to differentiate \(g_K^+(t)\) and \(g_K^+(t)\) (resp. \(h_K^+(t)\) and \(h_K^-(t)\)) and also can calculate the derivatives of \(g_K\) and \(h_K\) in terms of \(\beta_K\).

\begin{theorem}

Assume that there is a open interval \(U\) in \((0, \pi/2)\) and a continuous function \(f : U \cup \left( U + \pi/2 \right) \to \mathbb{R}\) such that the measure \(\beta_K\) has density function \(f\) on \(U \cup (U + \pi/2)\). That is, we have \(\beta_K(X) = \int_X f(x)\,dx\) for every Borel subset \(X \subseteq U \cup (U + \pi/2)\). Then we have \(g_K^+(t) = g_K^-(t)\) and \(h_K^+(t) = h_K^-(t)\) for every \(t \in U\) so the function \(g_K(t)\) and \(h_K(t)\) are well-defined on \(t \in U\). Moreover, \(g_{K}'(t) = -f(t) + h_{K}(t)\) and \(h_K'(t) = f(t + \pi/2) - g_K(t)\) for every \(t \in U\).

\label{thm:arm-length-derivative}
\end{theorem}

\begin{proof}
We have \(g_K(t) = g_K^{\pm}(t)\) and \(h_K(t) = h_K^{\pm}(t)\) for all \(t \in U\) by \Cref{pro:arm-length-unsigned}. To calculate the derivatives, first apply \Cref{thm:boundary-measure} to \Cref{pro:cap-tangent-arm-length} to get the following equations for all \(t \in U\).
\[
g_{K}(t) = \int_{t}^{t+\pi/2} \cos \left( u - t \right) \beta_K(du)
\]
\[
h_{K}(t) = \int_{t}^{t+\pi/2} \sin \left( u - t \right) \beta_K(du)
\]
Differentiate them at \(t \in U\) using Leibniz integral rule to complete the proof.
\[
g_{K}'(t) = -f(t) + \int_{t}^{t+\pi/2} \sin (u-t)\, \beta_K(du) = -f(t) + h_{K}(t) 
\]
\[
h_{K}'(t) = f\left( t + \pi/2 \right) - \int_{t}^{t+\pi/2} \cos (u-t)\, \beta_K(du) = f(t + \pi/2) - g_{K}(t)
\]

\end{proof}

This theorem is a converse of \Cref{thm:arm-length-derivative} that calculates \(\beta_K\) from continuously differentiable \(g_K\) and \(h_K\).

\begin{theorem}

Assume there is a open interval \(U\) in \((0, \pi/2)\) so that \(g_K(t)\) and \(h_K(t)\) are well-defined and continuously differentiable on \(U\). Define the continous function \(f\) on \(U \cup (U + \pi/2)\) as \(f(t) = h_K(t) - g_K'(t)\) and \(f(t + \pi/2) = g_K(t) + h'_K(t)\) for all \(t \in U\). Then the measure \(\beta_K\) has density function \(f\) on \(U \cup (U + \pi/2)\).

\label{thm:arm-length-derivative-converse}
\end{theorem}

\begin{proof}
By \Cref{thm:inner-corner-deriv}, \(\mathbf{y}_K(t)\) has continuous differentiation \(-g_K(t) u_t + h_K(t) v_t\) on \(t \in U\). So \(A_K^{\pm}(t) = \mathbf{y}_K(t) - g_K(t) u_t\) and \(C_K^{\pm}(t) = \mathbf{y}_K(t) - h_K(t) v_t\) are continuously differentiable on \(t \in U\) too. Then by \Cref{pro:boundary-measure-differential} and \Cref{def:boundary-measure} the boundary measure \(\beta_K\) has a continuous density function \(f_0\) on \(U \cup (U + \pi/2)\), where \(f_0(t) = A_K'(t) \cdot v_t\) and \(f_0(t + \pi/2) = - C_K'(t) \cdot v_t\). Now by \Cref{thm:arm-length-derivative} we should have \(f_0 = f\), completing the proof.
\end{proof}

Now we solve the equation \(\beta_K = \iota_K\) on any open set \((t_1, t_2) \cup (t_1 + \pi/2, t_2 + \pi/2)\) in terms of \(\beta_K\).

\begin{theorem}

Let \(0 \leq t_1 < t_2 \leq \omega\) be two arbitrary angles. Let \(U = (t_1, t_2)\) and \(X = U \cup (U + \pi/2)\). Then the followings are equivalent.

\begin{enumerate}
\def\labelenumi{\arabic{enumi}.}
\tightlist
\item
  We have \(\beta_{K} = \iota_{K}\) on the set \(X\)
\item
  We have \(g_K(t) = a + t\) and \(h_K(t) = b - t\) for some constants \(a, b \in \mathbb{R}\) on \(t \in U\).
\end{enumerate}

\label{thm:balanced-ACx}
\end{theorem}

\begin{proof}
Assume (1) that \(\beta_K = \iota_K\) on \(X\). The measure of \(\beta_K = \iota_K\) is zero for every point of \(X\) by \Cref{def:i-cap}. So we have \(g_K(t) = g^\pm_K(t)\) and \(h_K(t) = h_K^{\pm}(t)\) for every \(t \in U\) by \Cref{pro:arm-length-unsigned}. Also, \(g_K(t)\), \(h_K(t)\) are continuous with respect to \(t \in U\) by \Cref{cor:arm-length-continuity}. Now since \(\beta_K = \iota_K\) has the continuous density function \(i_K\) on \(X\), we can apply \Cref{thm:arm-length-derivative} to \(K\). By applying so, we have

\begin{gather}
g_{K}'(t) = -i_K(t) + h_{K}(t) = 1 \\
h_K'(t) = i_K(t + \pi/2) - g_K(t) = -1
\end{gather}
on \(t \in U\) and this immediately proves (2).

Now assume (2). By \Cref{thm:arm-length-derivative-converse} the measure \(\beta_K\) should have density function \(f(t) = b - t - 1\) and \(f(t + \pi/2) = a + t + 1\) over all \(t \in U\). This matches with the density function of \(\iota_K\) in \Cref{def:i-cap}, completing the proof of (1).
\end{proof}

We now solve for the equation \(\beta_K = \iota_K\) on the set \(J_\omega \setminus \left\{ \omega, \pi/2 \right\}\) by solving for \(\beta_K\). Note that our derivation aims to not solve the equation completely, but to derive enough properties of such \(K\) to get a single value of \(K\). First, we should have \(\beta_K(\left\{ 0 \right\}) = \beta_K(\left\{ \omega + \pi/2 \right\}) = 0\) because \(0, \omega + \pi/2 \in J_\omega \setminus \left\{ \omega, \pi/2 \right\}\) and \(\iota_K\) have measure zero on every singleton. So the values \(g_K(0)\) and \(h_K(\omega)\) exist, and we have \(g_K(0) = h_K(\omega) = 1\) as the width of \(K\) measured in the direction of \(u_\omega\) or \(v_0\) is one. So by \Cref{thm:balanced-ACx} with \(t_1 = 0\) and \(t_2 = \omega\) we should have \(g_K(t) = t + 1\) and \(h_K(t) = \omega - t + 1\) on the set \(t \in (0, \omega)\). Now by \Cref{def:i-cap}, the measure \(\beta_K\) has density \(\beta_K(dt) = (\omega - t) dt\) and \(\beta_K(dt + \pi/2) = t dt\) on \(t \in (0, \omega)\). It remains to find the values of \(\beta_K\) on the points \(\omega\) and \(\pi/2\). The measure \(\beta_K\) has to satisfy the equations in \Cref{thm:boundary-measure-cap}. Since we have calculations
\begin{gather*}
\label{eqn:k1-conditions}
\int_{t \in [0, \omega)} (\omega - t) \cos t  \, dt = 1 - \cos \omega \\
\int_{t \in (\pi/2, \omega + \pi/2]} (t - \pi/2) \cos\left( \omega + \pi/2 - t \right)  \, dt = 1 - \cos \omega
\end{gather*}
we should have \(\beta_K(\left\{ \omega \right\}) = \beta_K(\left\{ \pi/2 \right\}) = 1\) if \(\omega < \pi/2\). Motivated by the calculations made here, we define the following candidate \(K = K_{\omega, 1}\)

\begin{definition}

Define the cap \(K = K_{\omega, 1} \in \mathcal{K}_{\omega}\) with rotation angle \(\omega \in (0, \pi/2]\) as the unique cap with following boundary measure \(\beta_{K}\).

\begin{enumerate}
\def\labelenumi{\arabic{enumi}.}
\tightlist
\item
  \(\beta_{{K}}(dt) = (\omega -t)dt\) on \(t \in [0, \omega)\) and \(\beta_{K}(dt + \pi/2) = t dt\) on \(t \in (0, \omega]\).
\item
  If \(\omega < \pi/2\), \(\beta_K(\left\{ \omega \right\}) = \beta_K(\left\{ \pi/2 \right\}) = 1\).
\item
  If \(\omega = \pi/2\), \(\beta_K(\left\{ \pi/2 \right\}) = 2\) and \(v_K^+(\pi/2) = (-1, 1)\) and \(v_K^-(\pi/2) = (1, 1)\).
\end{enumerate}

\label{def:maximum-presofa-a1}
\end{definition}

Let us justify the unique existence of such \(K_{\omega, 1}\). Observe that the two equations in \Cref{thm:cap-from-boundary-measure} are true by \Cref{eqn:k1-conditions}. So if \(\omega < \pi/2\), the unique existence of \(K_{\omega, 1}\) comes from the statement of \Cref{thm:cap-from-boundary-measure} immediately. If \(\omega = \pi/2\), then the additional constraints \(v_K^+(\pi/2) = (-1, 1)\) and \(v_K^-(\pi/2) = (1, 1)\) fixes the single \(K\) among all possible horizontal translates. So the uniqueness of \(K\) is also guaranteed.

We now check back that the solution \(K = K_{\omega, 1}\) indeed satisfies the equation \(\beta_K = \iota_K\) on \(J_\omega \setminus \left\{ \omega, \pi/2 \right\}\).

\begin{theorem}

The cap \(K_{\omega, 1}\) maximizes \(\mathcal{A}_1 : \mathcal{K}_{\omega} \to \mathbb{R}\).

\label{thm:maximum-presofa-a1}
\end{theorem}

\begin{proof}
Let \(K := K_{\omega, 1}\). First we verify that \(K\) satisfies the equation \(\beta_K = \iota_K\) on \(J_\omega \setminus \{\omega, \pi/2\}\). Because \(\beta_K(\left\{ 0 \right\}) = \beta_K(\left\{ \omega + \pi/2 \right\}) = 0\), the values \(g_K(0)\) and \(h_K(\omega)\) exist. We have \(g_K(0) = 1\) and \(h_K(\omega) = 1\) as the width of \(K\) measured in the direction of \(u_\omega\) and \(v_0\) are one. Then by \Cref{thm:arm-length-derivative} on the interval \((0, \omega)\) we have

\begin{align*}
g_{K}'(t) & = -\left( \omega - t \right)  + h_{K}(t) \\
h_K'(t) & = t - g_K(t)
\end{align*}
on \(t \in (0, \omega)\). This imply \(g''_K(t) = 1 + t - g_K(t)\) on \(t \in (0, \omega)\). So we have \(g_K(t) = 1 + t + C_1 \sin t + C_2 \cos t\) for some constants \(C_1, C_2\) on \(t \in (0, \omega)\). Because \(g_K(t) \to g_K(0) = 1\) as \(t \to 0\) by \Cref{cor:arm-length-continuity}, we should have \(C_2 = 0\). Because \(h_K(t) \to h_K(\omega) = 1\) as \(t \to \omega\) by \Cref{cor:arm-length-continuity}, we have \(g'_K(t) \to 1\) as \(t \to \omega\). This then imply \(1 + C_1 \sin \omega = 1\) and \(C_1 = 0\). Now \(g_K(t) = 1 + t\) and correspondingly \(h_K(t) = 1 + \omega - t\) on \(t \in (0, \omega)\). As \(\iota_K\) is defined from the values of \(g_K\) and \(h_K\) (\Cref{def:i-cap}), we can verify \(\beta_K = \iota_K\) on \(J_\omega \setminus \{\omega, \pi/2\}\).

Take any \(K' \in \mathcal{K}_\omega\). Apply \(\beta_K = \iota_K\) on \(J_\omega \setminus \{\omega, \pi/2\}\) and \(p_K(\omega) = p_K(\pi/2) = p_{K'}(\omega) = p_{K'}(\pi/2) = 1\) to \Cref{thm:variation-a1}, to get \(D\mathcal{A}_1(K; K') = 0\). The functional \(\mathcal{A}_1\) is concave by \Cref{thm:a1-negative-semidefinite}. So \Cref{thm:quadratic-variation} and \(D\mathcal{A}_1(K; -) = 0\) implies that \(K = K_{\omega, 1}\) is the maximizer of \(\mathcal{A}_1\).
\end{proof}

A consequence of the proof of \Cref{thm:maximum-presofa-a1} is:

\begin{corollary}

For \(K = K_{\omega, 1}\), we have \(g_K(t) = 1 + t\) and \(h_K(t) = 1 + \omega - t\) for all \(t \in (0, \omega)\).

\label{cor:maximum-presofa-a1-arm-length}
\end{corollary}

Now we compute the maximum value of \(\mathcal{A}_1\).

\begin{theorem}

The maximum value \(\mathcal{A}_1(K_{\omega, 1})\) of \(\mathcal{A}_1\) is \(1 + \omega^2/2\).

\label{thm:maximum-presofa-a1-area}
\end{theorem}

\begin{proof}
Let \(K = K_{\omega, 1}\). We will exploit the mirror symmetry of \(K_{\omega, 1}\) along the line \(l\) connecting \(O\) to \(o_\omega\) (\Cref{def:parallelogram-vertices}). The line \(l\) divides \(K_{\omega, 1}\) into two pieces which are mirror images to each other. Call the piece on the right side of \(l\) as \(K_h\). Observe that the boundary of the half-piece \(K_h\) consists of the curve from \(A_K^-(0)\) to \(o_\omega\), and two segments from \(O\) to \(A_K^-(0)\) and \(o_\omega\) respectively. So \(p_{K}(t) = p_{K_h}(t)\) for \(t \in [0, \omega]\) and \(\beta_K\) and \(\sigma_{K_h}\) agree on \([0, \omega)\). Observe that \(\sigma_{K_h}\left( \left\{ \omega \right\} \right) = 1\) no matter if either \(\omega < \pi/2\) or \(\omega = \pi/2\), unlike the value \(\beta_K(\left\{ \omega \right\})\) that may change depending on \(\omega\). We also have \(|K| = 2 |K_h|\).

Now we compute the value of \(p_{K}(t) = p_{K_h}(t)\) for \(t \in [0, \omega]\). For the second equality, we are using \Cref{cor:boundary-measure-closed} with \(t_1 = t\) and \(t_2 = \omega\).

\begin{align*}
p_{K_h}(t) - o_\omega \cdot u_t & = (A^-_{K_h}(t) - o_\omega) \cdot u_t =  \\
& = \sin(\omega - t) + \int_{u \in [t, \omega] } (\omega - u) \sin \left( u - t \right) \, du \\
& = \omega - t
\end{align*}
So \(p_{K}(t) = p_{K_h}(t) = \omega - t + o_\omega \cdot u_t\). By the symmetry of \(K\) along \(l\) we also have \(p_K(t + \pi/2) = t + o_{\omega} \cdot v_t\).

Now calculate half the area of \(K\).

\begin{align*}
|K_h| = \frac{1}{2} \int_{t \in [0, \omega]} p_{K_h}(t) \, \beta(dt) & = 
\frac{1}{2} + \frac{1}{2} \int_{t \in [0, \omega]} \left( \omega - t + o_\omega \cdot u_t \right)  (\omega-t) \, dt \\
& = \frac{1}{2} + \frac{1}{2} \left( \omega^3 / 3 + o_\omega \cdot \int_{0}^{\omega} u_t (\omega - t)\, dt \right) 
\end{align*}
Define \(R := o_\omega \cdot \int_{0}^{\omega} u_t (\omega - t)\, dt\). Multiplying by 2, we get \(|K| = 1 + \omega^3 / 3 + R\)

Next, we compute the curve area functional \(\mathcal{I}(\mathbf{x}_{K})\). We have
\[
\mathbf{x}_{K}(t) = (p_{K}(t) - 1)u_t + (p_{K}(t + \pi/2) - 1) v_t
\]
by \Cref{pro:rotating-hallway-parts}. We also have
\begin{equation}
\label{eqn:a1-maximizer-x-deriv}
\mathbf{x}'_{K}(t) = -(g_{K}^+(t) - 1) \cdot u_t + (h_{K}^+(t) - 1) \cdot v_t = 
- t \cdot u_t + (\omega - t) \cdot v_t.
\end{equation}
by \Cref{thm:inner-corner-deriv} and \Cref{cor:maximum-presofa-a1-arm-length}. Now compute \(\mathcal{I}(\mathbf{x}_K)\).

\begin{align*}
\mathcal{I}(\mathbf{x}_{K}) & = \frac{1}{2} \int_0^\omega (p_{K}(t) - 1)(\omega - t) + (p_{K}(t+\pi/2) - 1) t \, dt  \\
& = \frac{1}{2} \int_0^\omega (\omega - t + o_{\omega} \cdot u_t - 1) (\omega - t) \, dt + 
\frac{1}{2} \int_0^\omega (t + o_{\omega} \cdot v_t - 1) t \, dt \\
& = \omega^3 / 3 - \omega^2 / 2 + R
\end{align*}
Finally, we compute \(\mathcal{A}_1(K) = |K| - \mathcal{I}(\mathbf{x}_{K}) = 1 + \omega^2 / 2\).
\end{proof}

We finish the proof of the main \Cref{thm:main-cap}.

\begin{proof}[Proof of \Cref{thm:main-cap}]
Assume any cap \(K \in \mathcal{K}_\omega\) with rotation angle \(\omega \in (0, \pi/2]\) and the rotation path \(\mathbf{x}_K : [0, \omega] \to \mathbb{R}^2\) injective and on the fan \(F_\omega\). Then by \Cref{thm:a1-upper-bound} we have \(\mathcal{A}(K) \leq \mathcal{A}_1(K)\). By \Cref{thm:maximum-presofa-a1-area} we have \(\mathcal{A}_1(K) \leq 1 + \omega^2/2\), completing the proof.
\end{proof}

We justify the description of the maximizer \(S_1\) with cap \(K_{\pi/2, 1}\) in \Cref{sec:main-result}. The right side of the cap \(K_{\pi/2, 1}\) is parametrized by the curve \(\gamma : [0, \pi/2] \to \mathbb{R}^2\) with \(\gamma(0) = (1, 1)\) and \(\gamma'(t) = t(\cos t, -\sin t)\), from \((\pi/2 - t, \pi/2)\) the description of the on the interval \Cref{cor:boundary-measure-open} The description of the inner corner \(\mathbf{x}_{S_1} : [0, \pi/2] \to \mathbb{R}^2\) with \(\mathbf{x}_{S_1}(0) = (\pi/2-1, 0)\) and
\[
\mathbf{x}_{S_1}'(t) = -t (\cos t, \sin t) + (\pi/2- t) (-\sin t, \cos t)
\]
comes from \Cref{eqn:a1-maximizer-x-deriv}.



\appendix
\chapter{Convex Bodies}
\label{sec:convex-bodies}
Fix an arbitrary planar convex body \(K\), which by \Cref{def:convex-body} is a nonempty, compact and convex subset of \(\mathbb{R}^2\). This appendix defines and proves numerous properties of \(K\). For the ease of understanding, it is recommended to read only the parts of this appendix when needed. However, an interested reader can read this appendix from beginning to end to verify the correctness of all proofs and theorems. Note that we allow \(K\) to have empty interior. That is, \(K\) can be a point or a closed line segment. If a theorem requires \(K\) to have nonempty interior, we state it explicitly.
\subsection{Vertex and support function}
\label{sec:vertex-and-support-function}
Here, we prove properties of the support function \(p_K\) (\Cref{def:support-function}) and the vertices \(v_K^{\pm}(t)\) (\Cref{def:convex-body-vertex}) of a convex body \(K\).

\subsection{Continuity and Linearity}

\begin{definition}

For any function \(f\) that maps an arbitrary convex body \(K\) to a value \(f(K)\) in a vector space \(V\), say that \(f\) is \emph{linear} with respect to \(K\) if the followings hold.

\begin{enumerate}
\def\labelenumi{\arabic{enumi}.}
\tightlist
\item
  For any \(a \geq 0\) and a convex body \(K\), we have \(f(aK) = a f(K)\).
\item
  For any \(a, b \geq 0\) and convex bodies \(K_1, K_2\), we have \(f(K_1 + K_2) = f(K_1) + f(K_2)\).
\end{enumerate}

\label{def:convex-body-linear}
\end{definition}

Note that the sum used in \Cref{def:convex-body-linear} is the Minkowski sum of convex bodies. That is, \(aK = \left\{ ap : p \in K \right\}\) and \(K_1 + K_2 = \left\{ p_1 + p_2 : p_1 \in K_1, p_2 : K_2 \right\}\).

\begin{theorem}

For any convex body \(K\), its support function \(p_K\) is Lipschitz.

\label{thm:support-function-lipschitz}
\end{theorem}

\begin{proof}
If \(K\) is a single point \(z \in \mathbb{R}^{2}\), then \(p_K = p_z\) is a sinusoidal function with amplitude \(|z|\) where \(|z|\) denotes the distance of \(z\) from the origin. For a general convex body \(K\), take \(R \geq 0\) so that \(K\) is contained in a closed ball of radius \(R\) centered at zero. Then
\[
p_K(t) = \max_{z \in K} z \cdot u_t = \sup_{z \in K} p_z(t)
\]
and note that each function \(p_z : S^1 \to \mathbb{R}\) is \(R\)-Lipschitz. So the supremum \(p_K\) of \(p_z\) over all \(z\) is also \(R\)-Lipschitz.
\end{proof}

\begin{proposition}

The support function \(p_K\) is linear with respect to \(K\).

\label{pro:support-function-linear}
\end{proposition}

\begin{proof}
First condition of \Cref{def:convex-body-linear} on \(p_K\) follows from a direct argument.
\[
p_{aK}(t) = \max_{z \in aK} z \cdot u_t = \max_{z' \in K} (az') \cdot u_t = a p_K(t)
\]

For arbirary convex bodies \(K_1, K_2\) and a fixed \(t \in S^1\),

\begin{align*}
p_{K_1 + K_2} (t) & = \max_{z \in K_1 + K_2} z \cdot u_t = \max_{z_1 \in K_1, z_2 \in K_2} (z_1 + z_2) \cdot u_t \\
& = \max_{z_1 \in K_1} z_1 \cdot u_t + \max_{z_2 \in K_2} z_2 \cdot u_t = p_{K_1}(t) + p_{K_2}(t)
\end{align*}
so the second condition of \Cref{def:convex-body-linear} is true.
\end{proof}

We will soon show that the vertex \(v_K^{+}(t)\) (resp. \(v_K^-(t)\)) is right-continuous (resp. left-continuous) respect to \(t\) and is linear with respect to \(K\). To do so, we compute the limit of vertices via \Cref{thm:limits-converging-to-vertex}.

\begin{definition}

For every \(t_1, t_2 \in S^1\) such that \(t_2 \neq t_1, t_1 + \pi\), define \(v_K(t_1, t_2)\) as the intersection \(l_K(t_1) \cap l_K(t_2)\).

\label{def:convex-body-tangent-lines-intersection}
\end{definition}

\begin{lemma}

For any fixed \(t_1, t_2 \in S^1\) such that \(t_2 \neq t_1, t_1 + \pi\), the point \(v_K(t_1, t_2)\) is linear with respect to \(K\).

\label{lem:tangent-lines-intersection-linear}
\end{lemma}

\begin{proof}
The point \(p = v_K(t_1, t_2)\) is the unique point such that \(p \cdot u_{t_1} = p_K(t_1)\) and \(p \cdot u_{t_2} = p_K(t_2)\) holds. By solving the linear equations, observe that the coordinates of \(p\) are linear combinations of \(p_K(t_1)\) and \(p_K(t_2)\). By \Cref{pro:support-function-linear} the result follows.
\end{proof}

\begin{theorem}

Let \(K\) be a convex body and \(t\) be an arbitrary angle. We have the following right limits all converging to \(v_K^+(t)\). In particular, the vertex \(v_K^+(t)\) is a right-continuous function on \(t \in S^1\).
\[
\lim_{ t' \to t^+ } v_K^+(t) = \lim_{ t' \to t^+ } v_K^-(u) = \lim_{ t' \to t^+ } v_K(t, t') = v_K^+(t)
\]
Similarly, we have the following left limits.
\[
\lim_{ t' \to t^- } v_K^+(t') = \lim_{ t' \to t^- } v_K^-(t') = \lim_{ t' \to t^- } v_K(t', t) = v_K^-(t)
\]

\label{thm:limits-converging-to-vertex}
\end{theorem}

\begin{proof}
We only compute the right limits. Left limits can be shown using a symmetric argument.

Let \(\epsilon > 0\) be arbitrary. Let \(p = v_K^+(t) + \epsilon v_t\). Then by the definition of \(v_K^+(t)\) the point \(p\) is not in \(K\). As \(\mathbb{R}^2 \setminus K\) is open, any sufficiently small open neighborhood of \(p\) is disjoint from \(K\), so we can take some positive \(\epsilon' < \epsilon\) such that the closed line segment connecting \(p\) and \(q = p - \epsilon' u_t\) is disjoint from \(K\) as well. Define the closed right-angled triangle \(T\) with vertices \(v_K^+(t)\), \(p\), and \(q\). Take the line \(l\) that passes through both \(q\) and \(v_K^+(t)\). Call the two closed half-planes divided by the line \(l\) as \(H_T\) and \(H'\), where \(H_T\) is the half-plane containing \(T\) and \(H'\) is the other one. By definition of \(H'\), the half-plane \(H'\) contains \(v_K^+(t)\) and \(q\) on its boundary and does not contain the point \(q\). And consequently \(H'\) has normal angle \(t' \in (t, t + \pi/2)\) (\Cref{def:half-plane}) because \(p = v_K^+(t) + \epsilon v_t\) and \(q = p - \epsilon' u_t\).

We show that \(K \cap H_T \subseteq T\). Assume by contradiction that there is \(r \in K \cap H_T\) not in \(T\). As \(r \in K\), \(r\) should be in the tangential half-plane \(H_K(t)\). So \(r\) is in the cone \(H_T \cap H_K(t)\) sharing the vertex \(v_K^+(t)\) and two edge \(l_K(t)\), \(l\) with \(T\). Since \(r \not\in T\), the line segment connecting \(r\) and the vertex \(v_K^+(t)\) of \(T\) should cross the edge of \(T\) connecting \(p\) and \(q\) at some point \(s\). As \(r, v_K^+(t) \in K\) we also have \(s \in K\) by convexity. But the line segment connecting \(p\) and \(q\) is disjoint from \(K\) by the definition of \(q\), so we get contradiction. Thus we have \(K \cap H_T \subseteq T\).

Now take arbitrary \(t_0 \in (t, t')\). We show that the edge \(e_K(t_0)\) should lie inside \(T\). It suffices to show that any point \(z\) in \(K\) that attains the maximum value of \(z \cdot u_{t_0}\) is in \(T\). Define the fan \(F := H_K(t) \cap H'\), so that \(F\) is bounded by lines \(l_K(t)\) and \(l\) with the vertex \(v_K^+(t)\). If \(z \in F\), it should be that \(z = v_K^+(t) \in T\), because \(v_K^+(t) \in K\) and \(v_K^+(t) \cdot u_{t_0} > z \cdot u_{t_0}\) for every point \(z\) in \(F\) other than \(z = v_K^+(t)\). If \(z \in K \setminus F\) on the other hand, we have \(K \setminus F = K \setminus H' \subseteq K \cap H_T \subseteq T\) so \(z \in T\). This completes the proof of \(e_K(t_0) \subseteq T\).

Observe that the triangle \(T\) contains \(v_K^+(t)\) and has diameter \(< 2\epsilon\) because the two perpendicular sides of \(T\) containing \(p\) have length \(\leq \epsilon\). So the endpoints \(v_K^+(u)\) and \(v_K^-(u)\) of the edge \(e_K(t_0) \subseteq T\) are distance at most \(2\epsilon\) away from \(v_K^+(t)\). This completes the epsilon-delta argument for \(\lim_{ t' \to t^+ } v_K^+(t') = \lim_{ t' \to t^+ } v_K^-(t') = v_K^+(t)\).

From \(e_K(t_0) \subseteq T\) and that the vertex \(p\) of \(T\) maximizes the value of \(z \cdot u_{t_0}\) over all \(z \in T\), we get that \(p\) is either on \(l_K(t_0)\) or outside the half-plane \(H_K(t_0)\). On the other hand we have \(v_K^+(t) \in H_K(t_0)\). So the line \(l_K(t_0)\) passes through the segment connecting \(p\) and \(v_K^+(t)\), and the intersection \(v_K(t, t_0) = l_K(t) \cap l_K(t_0)\) is inside \(T\). This with that the diameter of \(T\) is less than \(2 \epsilon\) proves \(\lim_{ t' \to t^+ } v_K(t, t') = v_K^+(t)\).
\end{proof}

The vertex \(v_K^+(t)\) is right-continuous by \Cref{thm:limits-converging-to-vertex}.

\begin{corollary}

The vertex \(v_K^+(t)\) is right-continuous with respect to \(t \in S^1\). Likewise, the vertex \(v_K^-(t)\) is left-continuous with respect to \(t \in S^1\).

\label{cor:vertex-right-continuous}
\end{corollary}

From \Cref{lem:tangent-lines-intersection-linear} and \Cref{thm:limits-converging-to-vertex}, we have the linearlity of vertices \(v_K^{\pm}(t)\).

\begin{corollary}

For a fixed \(t \in S^1\), the vertices \(v_K^{\pm}(t)\) are linear with respect to \(K\).

\label{cor:vertex-linear}
\end{corollary}

\subsection{Parametrization of Tangent Line}

We calculate \(v_K(t, t')\) as the following.

\begin{lemma}

Let \(t, t' \in S^1\) be arbitrary such that \(t' \neq t, t + \pi\). The following calculation holds.
\[
v_K(t, t') = p_K(t) u_{t} + \left( \frac{p_K(t') - p_K(t) \cos (t' - t)}{\sin (t' - t)} \right)  v_{t}
\]

\label{lem:intersection-tangent-lines}
\end{lemma}

\begin{proof}
Because the point \(p = v_K(t, t') = l_{K}(t) \cap l_K(t')\) is on the line \(l_K(t)\), we have \(p = p_K(t) u_{t} + \beta v_{t}\) for some constant \(\beta \in \mathbb{R}\). We use \(p \cdot u_{t'} = p_K(t')\) to derive the unique value \(\beta\). Write \(t' - t\) as \(\theta\).

\begin{align*}
p_K(t') &= p_K(t) (u_{t} \cdot u_{t'}) + \beta (v_{t} \cdot u_{t'}) \\
&= p_K(t) \cos \theta + \beta \sin \theta
\end{align*}
This gives \(\beta = p_C(t') \csc \theta - p_C(t) \cot \theta\) as claimed and completes the calculation. The value \(\alpha\) is continuous on \((-\pi, \pi) \setminus \left\{ 0 \right\}\) by the formula.
\end{proof}

Using \Cref{lem:intersection-tangent-lines}, we can show that \(v_K(t, t')\) parametrizes the half-lines in \(l_K(t)\) emanating from \(v_K^{\pm}(t)\).

\begin{theorem}

Take any \(t \in S^1\) and assume that the width \(p_K(t + \pi) + p_K(t)\) of \(K\) measured in the direction of \(u_t\) is strictly positive (e.g.~when \(K\) has nonempty interior). Define \(\mathbf{p} : [t, t + \pi) \to \mathbb{R}^2\) as \(\mathbf{p}(t) = v_K^+(t)\) and \(\mathbf{p}(t') = v_K(t, t')\) for all \(t' \in (t, t + \pi)\). Then the followings hold.

\begin{enumerate}
\def\labelenumi{\arabic{enumi}.}
\tightlist
\item
  \(\mathbf{p}\) is absolutely continuous on any closed and bounded subinterval of \([t, t+ \pi)\).
\item
  \(\mathbf{p}(t') = v_K^+(t') + \alpha(t') v_t\) where \(\alpha(t) = 0\) and the function \(\alpha : [t, t + \pi) \to \mathbb{R}\) is monotonically increasing.
\item
  \(\mathbf{p}\) is a parametrization of the half-line emanating from \(v_K^+(t)\) in the direction of \(v_t\).
\end{enumerate}

\label{thm:tangent-line-parametrization}
\end{theorem}

\begin{proof}
The function \(\mathbf{p}\) is continuous at \(t\) because of \Cref{thm:limits-converging-to-vertex}. The function \(\mathbf{p}\) is absolutely continuous on any closed subinterval of \((t, t+ \pi)\) by \Cref{lem:intersection-tangent-lines} and \Cref{thm:support-function-lipschitz}. So the derivative \(\mathbf{p}'(u)\) of \(\mathbf{p}\) on \(u \in (t, t + \pi)\) exists almost everywhere and \(\mathbf{p}(u_2) - \mathbf{p}(u_1) = \int_{u_1}^{u_2} \mathbf{p}'(u) \, du\) for every \(t < u_1 < u_2 < t + \pi\). Take the limit \(u_1 \to t^+\) to obtain \(\mathbf{p}(u_2) - \mathbf{p}(t) = \int_{t}^{u_2} \mathbf{p}'(u)\,du\). So \(\mathbf{p}\) is absolutely continuous on any closed subinterval of \([t, t + \pi)\) including the endpoint \(t\). This verifies (1).

\begin{enumerate}
\def\labelenumi{(\arabic{enumi})}
\tightlist
\item
  comes from the geometric fact that for every \(t < t_1 < t_2 < t + \pi\), the point \(v_K(t, t_1)\) lies in the segment connecting \(v_K^+(t)\) and \(v_K(t, t_2)\).
\end{enumerate}

By taking the limit \(u \to t + \pi^-\) in \Cref{lem:intersection-tangent-lines}, we have \(\alpha(u) \to \infty\) (note that we use the fact that the width \(p_K(t + \pi) + p_K(t)\) is positive). Now (3) follows from (1), (2), and \(\alpha(u) \to \infty\).
\end{proof}

\begin{theorem}

Let \(K\) be any convex body. Let \(t_0 \in \mathbb{R}\) be any angle. On the interval \(t \in [t_0, t_0 + \pi]\), the value \(v_K^+(t) \cdot v_{t_0}\) is monotonically decreasing. On the interval \(t \in [t_0 - \pi, t_0]\), the value \(v_K^+(t) \cdot v_{t_0}\) is monotonically increasing.

\label{thm:vertex-monotone}
\end{theorem}

\begin{proof}
Take two arbitrary values \(t_1 < t_2\) in the interval \([t_0, t_0 + \pi]\). The points \(v_K^+(t_1), l_K(t_1) \cap l_K(t_2), v_K^-(t_2), v_K^+(t_2)\) goes further in the direction of \(-v_{t_0}\) (\Cref{def:further-in-direction}) in the increasing order. This shows that the value \(v_K^+(t) \cdot v_{t_0}\) is monotonically decreasing on the interval \(t \in [t_0, t_0 + \pi]\). A symmetric argument proves the other claim.
\end{proof}

\begin{theorem}

On any bounded interval \(t \in I\) of \(\mathbb{R}\), \(v_K^+(t)\) is of bounded variation.

\label{thm:vertex-bounded-variation}
\end{theorem}

\begin{proof}
The \(x\) and \(y\) coordinates of \(v_K^+(t)\) either monotonically increases or decreases on each of the intervals \([0, \pi/2]\), \([\pi/2, \pi]\), \([\pi, 3\pi/2]\), \([3\pi/2, 2\pi]\) by \Cref{thm:vertex-monotone}. So the coordinates are of bounded variation on each interval.
\end{proof}

\subsection{Lebesgue-Stieltjes measure}
\label{sec:lebesgue-stieltjes-measure}
For any right-continuous, real-valued function \(F\) of bounded variation on domain \(X = \mathbb{R}\) or \(S^1 = \mathbb{R} / 2\pi \mathbb{Z}\), we will rigorously define its \emph{Lebesgue-Stieltjes measure} \(dF\) on \(X\). The Lebesgue-Stieltjes measure \(dF\) on any half-open interval \((a, b]\) calculates the difference \(F(b) - F(a)\) of \(F\) along the boundary.

Most literature constructs \(dF\) for any right-continuous and monotonically increasing \(F : \mathbb{R} \to \mathbb{R}\) \autocite{steinRealAnalysisMeasure2005,hewittRealAbstractAnalysis2013,halmos2013measure}.

\begin{theorem}

(Theorem 3.5. of \autocite{steinRealAnalysisMeasure2005}) For any right-continuous and monotonically increasing function \(F : \mathbb{R} \to \mathbb{R}\), there exists a unique measure \(dF\) on \(\mathbb{R}\) such that \(dF((c, d]) = F(d) - F(c)\) for any half-open interval \((c, d]\) of \(\mathbb{R}\).

\label{thm:lebesgue-stieltjes}
\end{theorem}

We will often appeal to the following lemma for the uniqueness of Lebesgue-Stieltjes measure. See p288 of \autocite{steinRealAnalysisMeasure2005} for the notion of signed measure.

\begin{lemma}

Let \(X\) be the domain, which is one of \(\mathbb{R}\), a closed interval \([a, b]\), or the circle \(S^1\). Let \(\mu_1, \mu_2\) be two measures on \(X\) that may or may not be signed. If \(\mu_1\) and \(\mu_2\) agrees on any half-open interval \((c, d]\) of \(X\), and additionally \(\left\{ a \right\}\) if \(X = [a, b]\). Then we have \(\mu_1 = \mu_2\).

\label{lem:measure-interval-uniqueness}
\end{lemma}

\begin{proof}
If \(\mu_1, \mu_2\) are measures (not signed) on \(X\), then since the Borel \(\sigma\)-algebra of \(X\) is generated by half-open intervals \(I\) (and also \(I = \left\{ a \right\}\) if \(X = [a, b]\)), we can appeal to the Carathéodory extension theorem to conclude \(\mu_1 = \mu_2\). If \(\mu_1, \mu_2\) are signed measures on \(X\), for each \(i=1, 2\) first write \(\mu_i = \mu_i^+ - \mu_i^-\) as the difference of two measures on \(X\) (p288 of \autocite{steinRealAnalysisMeasure2005}). Then because \(\mu_1(I) = \mu_2(I)\) for any half-open intervals \(I\) that generate the Borel \(\sigma\)-algebra of \(X\), the measures \(\lambda_1 = \mu_1^+ + \mu_2^-\) and \(\lambda_2 = \mu_2^+ + \mu_1^-\) agree on \(I\). So \(\lambda_1 = \lambda_2\) and in turn we have \(\mu_1 = \mu_2\).
\end{proof}

Now let \(F : [a, b] \to \mathbb{R}\) be any right-continuous function of bounded variation, not necessarily increasing. We can still define \(dF\) by allowing \(dF\) to be a \emph{signed} measure.

\begin{theorem}

For any right-continuous function \(F : [a, b] \to \mathbb{R}\) of bounded variation, there exists a unique signed measure \(dF\) on \(\mathbb{R}\) such that \(dF(\left\{ a \right\}) = 0\) and \(dF((c, d]) = F(d) - F(c)\) for any half-open subinterval \((c, d]\) of \([a, b]\).

\label{thm:lebesgue-stieltjes-signed}
\end{theorem}

\begin{proof}
Write \(F\) as the difference \(F = F_1 - F_2\) of two monotonically increasing and bounded functions \(F_1, F_2 : [a, b] \to \mathbb{R}\) (Theorem 3.3, p119 of \autocite{steinRealAnalysisMeasure2005}). Take the right limit on \(F(x) = F_1(x) - F_2(x)\) to further assume that \(F_1, F_2\) are right-continuous. Extend the domain of \(F_i\) for \(i=1, 2\) to \(\mathbb{R}\) by letting \(F_i(x) = F_i(a)\) for \(x < a\) and \(F_i(x) = F_i(b)\) for \(x > b\). The measures \(dF_1\) and \(dF_2\) on \(\mathbb{R}\) are well-defined by \Cref{thm:lebesgue-stieltjes} and bounded on \([a, b]\). So the bounded signed measure \(dF := dF_1 - dF_2\) is well-defined and satisfies \(dF((c, d]) = F(d) - F(c)\) for any half-open subinterval \((c, d]\) of \([a, b]\). To check \(dF(\left\{ a \right\}) = 0\), check \(dF_i(\left\{ a \right\}) = F_i(a) - F_i(a-) = 0\). Appeal to \Cref{lem:measure-interval-uniqueness} for the uniqueness of \(dF\).
\end{proof}

We now allow the domain of \(F : S^1 \to \mathbb{R}\) to be \(S^1\). Define \(q : [0, 2\pi] \to S^1\) as the quotient map \(q(t) = t + 2\pi \mathbb{Z}\) identifying the endpoints \(0\) and \(2\pi\). We will say that \(F\) is \emph{of bounded variation} if and only if \(F \circ q : [0, 2\pi] \to \mathbb{R}\) is of bounded variation. It is natural to define \(dF\) using \(d(F \circ q)\) where \(F \circ q : [0, 2\pi] \to \mathbb{R}\). Recall the convention that the interval \((t_1, t_2]\) of \(\mathbb{R}\) is used to denote the corresponding interval of \(S^1\) mapped under \(\mathbb{R} \to S^1\).

\begin{theorem}

For any right-continuous function \(F : S^1 \to \mathbb{R}\) of bounded variation, there exists a unique bounded signed measure \(dF\) on \(S^1\) such that for any half-open interval \((t_1, t_2]\) of \(S^1\) with \(t_1 < t_2 \leq t_1 + 2\pi\), we have \(dF((t_1, t_2]) = F(t_2) - F(t_1)\). Moreover, such \(dF\) is unique, and is the pushforward of \(d(F \circ q)\) under the quotient map \(q : [0, 2\pi] \to S^1\).

\label{thm:lebesgue-stieltjes-circle}
\end{theorem}

\begin{proof}
The function \(F \circ q : [0, 2\pi] \to \mathbb{R}\) is right-continuous and of bounded variation. So its Lebesgue-Stieltjes measure \(d(F \circ q)\) on \([0, 2\pi]\) is well-defined by \Cref{thm:lebesgue-stieltjes-signed}. Now let \(\mu\) be the pushforward of \(d(F \circ q)\) under the map \(q\). Take any interval \(I = (t_1, t_2]\) of \(S^1\) with \(0 \leq t_1 < t_2 \leq 2\pi\). We have
\[
\mu(I) = d(F \circ q)(q^{-1}(I)) = d(F \circ q)((t_1, t_2]) = F(t_1) - F(t_2)
\]
because \(q^{-1}(I)\) is either \((t_1, t_2]\) (if \(t_2 < 2\pi\)) or \(\left\{ 0 \right\} \cup (t_1, t_2]\) (if \(t_2 = 2\pi\)), and \(d(F \circ q)(\left\{ 0 \right\}) = 0\) by \Cref{thm:lebesgue-stieltjes-signed}. Now it suffices to show \(\mu((t_1, t_2]) = F(t_2) - F(t_1)\) for the case \(0 \leq t_1 \leq 2\pi \leq t_2 \leq 4\pi\) where the interval wraps around \(0 = 2\pi\) in \(S^1\). We have
\[
\mu((t_1, t_2]) = \mu((t_1, 2\pi]) + \mu((2\pi, t_2]) = F(2\pi) - F(t_1) + F(t_2) - F(2\pi) = F(t_2) - F(t_1)
\]
thus completing the proof. Appeal to \Cref{lem:measure-interval-uniqueness} for the uniqueness of such \(dF\) on \(S^1\).
\end{proof}

The Lebesgue-Stieltjes measure \(dF\) can be used as a justification of the differential of \(F\). It satisfies the following properties that a differential is expected to satisfy.

\begin{proposition}

The Lebesgue-Stieltjes measure \(dF\) on domain \(\mathbb{R}\), \([a, b]\), or \(S^1\) is linear with respect to \(F\).

\label{pro:lebesgue-stieltjes-linear}
\end{proposition}

\begin{proof}
Let \(F_1, F_2\) be arbitrary right-continuous functions of bounded variation, and \(a, b\) be arbitrary real values. Observe that \(d(aF_1 + bF_2)\) and \(a \cdot dF_1 + b \cdot dF_2\) both agree on any half-open interval \(I\), and use \Cref{lem:measure-interval-uniqueness} to conclude that they are equal.
\end{proof}

\begin{proposition}

Let \(F, G\) be real-valued functions on the domain \(X\) which is either \(S^1\) or a closed interval of \(\mathbb{R}\). Assume that both \(F\) and \(G\) are of bounded variation. Assume that \(F\) is continuous and \(G\) is right-continuous. Then \(d(FG) = G dF + F dG\).

\label{pro:lebesgue-stieltjes-product}
\end{proposition}

\begin{proof}
We first prove the case where \(X\) is a closed interval \(I\). By Proposition 4.5, Chapter 0 of \autocite{revuzContinuousMartingalesBrownian1999}, we have the integration by parts
\[
\int_{(a, b]} G(x)\, dF(x) + \int_{(a, b]} F(x-) \, dG(x) = F(b) G(b) - F(a) G(a).
\]
for any endpoints \(a \leq b\) in \(I\). Now use \(F(x-) = F(x)\) and \Cref{lem:measure-interval-uniqueness} to conclude \(d(FG) = G dF + F dG\). For the case where the domain is \(X = S^1\), use the same argument for \(F \circ q\) and \(G \circ q\) with domain \(X = [0, 2\pi]\), and apply \Cref{thm:lebesgue-stieltjes-circle}.
\end{proof}

\subsection{Surface area measure}
\label{sec:surface-area-measure}
\begin{definition}

The \emph{surface area measure} \(\sigma_K\) of a planar convex body \(K\) is a measure on \(S^1\) defined in p214 of \cite{schneider_2013}, which is also denoted as \(S_1(K, -)\) or \(S_K\) in p464 of \cite{schneider_2013}.

\label{def:surface-area-measure}
\end{definition}

For any convex body \(K\), the surface area measure \(\sigma_K\) essentially measures the side lengths of \(K\). For example, if \(K\) is a convex polygon, then \(\sigma_K\) is a discrete measure such that the measure \(\sigma_K\left( t \right)\) at point \(t\) is the length of the edge \(e_K(t)\). Assume the other case where for every \(t \in S^1\), the tangent line \(l_K(t)\) always meets \(K\) at a single point \(v(t)\), so that \(\partial K\) is parametrized smoothly by \(v_K(t)\) for \(t \in S^1\). Then it turns out that \(\sigma_K(dt) = R(t) dt\) where \(R(t) = \left\lVert v'(t) \right\rVert\) is the radius of curvature of \(\partial K\) at \(v(t)\).

We now collect theorems on the surface area measure \(\sigma_K\).

\begin{theorem}

(Equation (8.23), p464 of \cite{schneider_2013}) The surface area measure \(\sigma_K\) is convex-linear with respect to \(K\).

\label{thm:surface-area-measure-convex-linear}
\end{theorem}

Recall that the \emph{mixed volume} \(V(K_1, K_2)\) of any two planar convex bodies \(K_1\) and \(K_2\), defined in Theorem 5.1.7 in p280 of \cite{schneider_2013}, is a non-negative value with the following properties.

\begin{theorem}

(Theorem 5.1.7 and Equation 5.27 of \cite{schneider_2013}) The followings are true for any planar convex bodies \(K, K_1, K_2\).

\begin{enumerate}
\def\labelenumi{\arabic{enumi}.}
\tightlist
\item
  \(V(K, K) = |K|\)
\item
  \(V(K_1, K_2)\) is bilinear in \(K_1\) and \(K_2\) (\Cref{def:convex-body-linear}).
\item
  \(V(K_1, K_2) = V(K_2, K_1)\)
\end{enumerate}

\label{thm:mixed-volume}
\end{theorem}

We have the following representation of mixed volume in terms of support function and surface area measure. Remark that for any measurable function \(f\) on a space \(X\) and a measure \(\sigma\) on \(X\), we denote the integral of \(f\) with respect to \(\sigma\) as \(\left< f, \sigma \right>_{X} = \int_{x \in X} f(x)\,\sigma(dx)\).

\begin{theorem}

(Equation 5.19 of \cite{schneider_2013}) The mixed volume \(V(K_1, K_2)\) of any two planar convex bodies \(K_1\) and \(K_2\) can be represented as the following.
\[
V(K_1, K_2) = \left< p_{K_1}, \sigma_{K_2} \right>_{S^1} / 2
\]
Consequently, the area \(|K|\) of any planar convex body \(K\) can be represented as the following.
\[
|K| = V(K, K) = \left< p_K, \sigma_K \right>_{S^1} / 2
\]

\label{thm:surface-area-measure-area}
\end{theorem}

We prove the following important \emph{vertex equality} of \(K\) using \(\sigma_K\).

\begin{definition}

For any measure \(\sigma\) on \(S^1\) and a Borel subset \(A\) of \(S^1\), define the \emph{restriction} \(\sigma|_A\) of \(\sigma\) to \(A\) as the measure on \(S^1\) defined as \(\sigma|_A(X) = \sigma(A \cap X)\).

\label{def:measure-restriction}
\end{definition}

\begin{theorem}

For every interval \((t_1, t_2]\) in \(S^1\) of length \(\leq 2\pi\), we have the following equality.
\[
v_K^+(t_2) - v_K^+(t_1) = \int_{t \in (t_1, t_2]} v_t \, \sigma_K(dt)
\]

\label{thm:boundary-measure}
\end{theorem}

\begin{proof}
First we observe that the equality holds when \(K\) is a polygon. In this case, for every \(t\) the value \(\sigma_K(t)\) is nonzero if and only if it measures the length of a proper edge \(e_K(t)\). So the right-hand side measures the sum of all vectors from vertex \(v_K^-(t)\) to \(v_K^+(t)\) along the proper edges \(e_K(t)\) with angles \(t \in (t_1, t_2]\). The sum in the right-hand side is then the vector from \(v_K^+(t_1)\) to \(v_K^+(t_2)\), justifying the equality for polygon \(K\).

Now we prove the equality for general \(K\). As in the proof of Theorem 8.3.3, p466 of \cite{schneider_2013}, we can take a series \(K_1, K_2, \dots\) of polygons converging to \(K\) in the Hausdorff distance such that \(e_{K_n}(t_i) = e_{K}(t_i)\) for all \(n = 1, 2, \dots\) and \(i = 1, 2\). In particular, we have \(v_{K_n}^{\pm}(t_i) = v_{K}^{\pm}(t_i)\) and \(\sigma_{K_n}(\{t_i\}) = \sigma_{K}(\{t_i\})\) for all \(n = 1, 2, \dots\) and \(i = 1, 2\). By Theorem 4.2.1, p212 of \cite{schneider_2013}, the measures \(\sigma_{K_n}\) converge to \(\sigma_K\) weakly as \(n \to \infty\).

Define \(U\) as the open set \(S^1 \setminus \left\{ t_1, t_2 \right\}\) of \(S\), and \(V\) as the open interval \((t_1, t_2)\) of \(S^1\). Define \(\mu_n\) and \(\mu\) as the restriction of \(\sigma_{K_n}\) and \(\sigma_K\) to \(U\), then \(\mu_n\) converges to \(\mu\) weakly as \(n \to \infty\) because \(\sigma_{K_n}(\{t_i\}) = \sigma_{K}(\{t_i\})\) for \(i = 1, 2\). Define \(\lambda_n\) and \(\lambda\) as the restriction of \(\sigma_{K_n}\) and \(\sigma_K\) to \(V\). We want to prove that \(\lambda_n \to \lambda\) weakly as \(n \to \infty\). Take any continuity set \(X\) of \(\lambda\) so that \(\lambda(\partial X) = 0\) and thus \(\mu(\partial X \cap V) = 0\). Because \(\partial(X \cap V) \subseteq (\partial X \cap V) \cup \partial V\), and both \(\mu(\partial X \cap V)\) and \(\mu(\partial V)\) are zero, the set \(X \cap V\) is a continuity set of \(\mu\). So \(\mu_n(X \cap V) \to \mu(X \cap V)\) and thus \(\lambda_n(X) \to \lambda(X)\) as \(n \to \infty\). This completes the proof that \(\lambda_n \to \lambda\) weakly as \(n \to \infty\).

We finally take the limit \(n \to \infty\) to the equality
\[
v_{K_n}^+(t_2) - v_{K_n}^+(t_1) = \int_{t \in (t_1, t_2]} v_t \, \sigma_{K_n}(dt)
\]
for polygons \(K_n\). The left-hand side is equal to \(v_K^+(t_2) - v_K^+(t_1)\) by the way how we took \(K_n\). The right-hand side is equal to
\[
(v_{K_n}^+(t_2) - v_{K_n}^-(t_2)) + \int_{t \in S^1} v_t \, \lambda_n(dt)
\]
and by \(v_{K_n}^{\pm}(t_i) = v_{K}^{\pm}(t_i)\) and the weak convergence \(\lambda_n \to \lambda\), the expression converges to
\[
(v_{K}^+(t_2) - v_{K}^-(t_2)) + \int_{t \in S^1} v_t \, \lambda(dt) = \int_{t \in (t_1, t_2]} v_t\, \sigma_{K}(dt)
\]
thus completing the proof for general \(K\).
\end{proof}

\Cref{thm:boundary-measure} has the following concise representation in differentials via the Lebesgue-Stieltjes measure. Note that \(v_t = (-\sin t, \cos t)\) is a unit vector, and \(v_K^+(t)\) is the vertex of \(K\).

\begin{proposition}

We have \(dv_K^+(t) = v_t \sigma_K(dt)\).

\label{pro:boundary-measure-differential}
\end{proposition}

That is, if we write the coordinates of \(v_K^+(t)\) as \((x(t), y(t))\), then the Lebesgue-Stieltjes measure \(dx\) and \(dy\) of \(x(t)\) and \(y(t)\) are \(-\sin t \cdot \sigma_K(dt)\) and \(\cos t \cdot \sigma_K(dt)\) respectively. Note that \(dx\) and \(dy\) are well-defined because \(v_K^+(t)\) is of bounded variation (\Cref{thm:vertex-bounded-variation}) and right-continuous (\Cref{cor:vertex-right-continuous}).

\begin{proof}[Proof of \Cref{pro:boundary-measure-differential}]
Observe that the two pairs of measures \(dv_K^+(t)\) and \(v_t \sigma_K(dt)\) agree on any half-open intervals of \(S^1\) by \Cref{thm:boundary-measure}. Appeal to \Cref{lem:measure-interval-uniqueness} to show the equality.
\end{proof}

Surface area measure at a single point \(t\) measures the length of the edge \(e_K(t)\).

\begin{theorem}

\(\sigma_K(\left\{ t \right\})\) is equal to the length of the edge \(e_K(t)\). Moreover, \(v_K^+(t) = v_K^-(t) + \sigma_K( \left\{ t \right\} ) v_t\).

\label{thm:surface-area-singleton}
\end{theorem}

\begin{proof}
Let \(t_2 = t\) and send \(t_1 \to t^-\) in \Cref{thm:boundary-measure}. Then by \Cref{thm:limits-converging-to-vertex} we get the equation \(v_K^+(t) = v_K^-(t) + \sigma_K(\left\{ t \right\}) v_t\).
\end{proof}

\begin{proposition}

Except for a countable number of values of \(t \in S^1\), we have \(v_K^-(t) = v_K^+(t)\).

\label{pro:surface-area-singleton-almost-everywhere}
\end{proposition}

\begin{proof}
Since \(\sigma_K\) is a finite measure on \(S^1\), \(\sigma_K(\left\{ t \right\})\) is zero except for a countable number of values of \(t \in S^1\). Apply \Cref{thm:surface-area-singleton} to such \(t\) with \(\sigma_K(\left\{ t \right\}) = 0\).
\end{proof}

We also have the following variants of \Cref{thm:boundary-measure} that work for closed and open intervals of \(S^1\) respectively.

\begin{corollary}

For every interval \([t_1, t_2]\) in \(S^1\) of length \(< 2\pi\), we have the following equality.
\[
v_K^+(t_2) - v_K^-(t_1) = \int_{t \in [t_1, t_2]} v_t \, \sigma_K(dt)
\]

\label{cor:boundary-measure-closed}
\end{corollary}

\begin{proof}
Add \Cref{thm:surface-area-singleton} with \(t=t_1\) to \Cref{thm:boundary-measure}.
\end{proof}

\begin{corollary}

For every interval \((t_1, t_2)\) in \(S^1\) of length \(\leq 2\pi\), we have the following equality.
\[
v_K^-(t_2) - v_K^+(t_1) = \int_{t \in (t_1, t_2)} v_t \, \sigma_K(dt)
\]

\label{cor:boundary-measure-open}
\end{corollary}

\begin{proof}
Subtract \Cref{thm:surface-area-singleton} with \(t=t_2\) from \Cref{thm:boundary-measure}.
\end{proof}

We can also measure the width of \(K\) using the surface area measure \(\sigma_K\).

\begin{corollary}

For any angle \(t \in S^1\), the width \(p_K(t) + p_K(t + \omega)\) of \(K\) measured in the direction of \(u_t\) is equal to the following.
\[
\int_{u \in (t, t + \pi)} \sin(u - t) \, \sigma_K(dt)
\]

\label{cor:boundary-measure-width}
\end{corollary}

\begin{proof}
Apply \(t_1 = t, t_2 = t + \pi\) to \Cref{thm:boundary-measure} and take the dot product with \(-u_t\).
\end{proof}

\subsection{Parametrization of boundary}
\label{sec:parametrization-of-boundary}
If \(K\) has nonempty interior, it occurs naturally that the boundary \(\partial K\) is a Jordan curve bounding \(K\) in its interior. So for any different \(p, q \in \partial K\), we can think of the Jordan arc \(\mathbf{b}\) connecting \(p\) and \(q\) along the boundary \(\partial K\) in counterclockwise direction. However, to rigorously justify that the curve area functional \(\mathcal{I}(\mathbf{b})\) of \(\mathbf{b}\) is well-defined and relates to the surface area measure \(\sigma_K\) (\Cref{thm:param-curve-area-functional}), we need to contruct an explicit rectifiable parametrization of \(\mathbf{b}\) which requires some work.

For every \(t_0 \in \mathbb{R}\) and \(t_1 \in [t_0, t_0 + 2\pi]\), we will define \(\mathbf{b}_K^{t_0, t_1}\) as essentially the arc-length parametrization of the curve connecting \(v_K^+(t_0)\) to \(v_K^+(t_1)\) along the boundary \(\partial K\) counterclockwise. The full \Cref{def:boundary-segment-parametrization} of \(\mathbf{b}_{K}^{t_0, t_1}\) is technical will be given much later. Instead, we start by stating the properties \(\mathbf{b}_K^{t_0, t_1}\) that agrees with our intuition that we will prove rigorously. Note that in the theorems below we allow \(K\) to have empty interior.

\begin{theorem}

Assume arbitrary \(t_0 \in \mathbb{R}\) and \(t_1 \in [t_0, t_0 + 2\pi]\). Then \(\mathbf{b}_K^{t_0, t_1}\) is an arc-length parametrization of the \(\left\{ v_K^+(t_0) \right\} \bigcup \cup_{t \in (t_0, t_1]} e_K(t)\) from point \(v_K^+(t_1)\) to \(v_K^+(t_2)\).

\label{thm:param-segment}
\end{theorem}

\begin{theorem}

Assume arbitrary \(t_0 \in \mathbb{R}\) and \(t_1 \in [t_0, t_0 + 2\pi]\). Then the curve \(\mathbf{b}_K^{t_0, t_1}\) have length \(\sigma_K((t_0, t_1])\).

\label{thm:param-segment-length}
\end{theorem}

\begin{theorem}

Assume arbitrary \(t_0, t_1, t_2\) such that \(t_0 \leq t_1 \leq t_2 \leq t_0 + 2\pi\). Then \(\mathbf{b}_{K}^{t_0, t_2}\) is the concatenation of \(\mathbf{b}_{K}^{t_0, t_1}\) and \(\mathbf{b}_{K}^{t_1, t_2}\).

\label{thm:param-concatenation}
\end{theorem}

\begin{theorem}

Assume arbitrary \(t_0 \in \mathbb{R}\) and \(t_1 \in [t_0, t_0 + 2\pi]\). Then the curve area functional of \(\mathbf{b}_K^{t_0, t_1}\) can be represented in two different ways:
\[
\mathcal{I} \left( \mathbf{b}_{K}^{t_0, t_1} \right) = \frac{1}{2} \int_{(t_0, t_1]}p_K(t)\,\sigma_K(dt) = \frac{1}{2} \int_{(t_0, t_1]} v_K^+(t) \times d v_K^+(t)
\]

\label{thm:param-curve-area-functional}
\end{theorem}

\begin{theorem}

For every \(t \in \mathbb{R}\), we have \(|K| = \mathcal{I}\left( \mathbf{b}_K^{t, t + 2\pi} \right)\).

\label{thm:param-positive-area}
\end{theorem}

\begin{proof}
This is a corollary of \Cref{thm:surface-area-measure-area} and \Cref{thm:param-curve-area-functional}.
\end{proof}

We will also show that \(\mathbf{b}_K^{t_0, t_1}\) is one of: a Jordan arc, a Jordan curve, or a single point (\Cref{cor:param-positive-jordan}). We first recall the difference between a Jordan arc and curve (p170 of \autocite{apostolMathematicalAnalysisModern}).

\begin{definition}

A \emph{Jordan curve} is a curve parametrized by continuous \(\mathbf{p} : [a, b] \to \mathbb{R}^2\) such that \(a<b\), \(\mathbf{p}(a) = \mathbf{p}(b)\) and \(\mathbf{p}\) being injective on \([a, b)\).

\label{def:jordan-curve}
\end{definition}

\begin{definition}

A \emph{Jordan arc} is a curve parametrized by continuous and injective \(\mathbf{p} : [a, b] \to \mathbb{R}^2\) such that \(a<b\).

\label{def:jordan-arc}
\end{definition}

In order for \(\partial K\) to be a Jordan curve, \(K\) has to have nonempty interior. For the notion of the orientation of a Jordan curve, we refer to p170 of \autocite{apostolMathematicalAnalysisModern}. The following theorem shows that \(\mathbf{b}_K^{t, t + 2\pi}\) is the unique parametrization of \(\partial K\) as a positively-oriented Jordan curve with \(v_K^+(t)\) as its endpoints.

\begin{theorem}

Assume that \(K\) have nonempty interior. For every \(t \in \mathbb{R}\), the curve \(\mathbf{b}_K^{t, t + 2\pi}\) is a positively oriented arc-length parametrization of the boundary \(\partial K\) as a Jordan curve that starts and ends with the point \(v_K^+(t)\).

\label{thm:param-positive-jordan}
\end{theorem}

Now the following is a corollary of \Cref{thm:param-concatenation} and \Cref{thm:param-positive-jordan}.

\begin{corollary}

Assume that \(K\) have nonempty interior. Assume arbitrary \(t_0 \in \mathbb{R}\) and \(t_1 \in [t_0, t_0 + 2\pi]\). Then \(\mathbf{b}_K^{t_0, t_1}\) is one of: a Jordan arc, a Jordan curve, or a single point.

\label{cor:param-positive-jordan}
\end{corollary}

\subsection{Definition of parametrization}

\begin{definition}

Denote the \emph{perimeter} of \(K\) as \(B_K = \sigma_K\left( S^1 \right)\).

\label{def:convex-body-perimeter}
\end{definition}

Fix an arbitrary convex body \(K\) and the starting angle \(t_0 \in \mathbb{R}\). Our goal to construct an arc-length parametrization \(\mathbf{b}_{K}^{t_0} : [0, B_K] \to \mathbb{R}^2\) of the boundary \(\partial K\) starting with the point \(v_K^+(t_0)\). Take an arbitrary point \(p\) on the boundary \(\partial K\). Let \(s \in [0, B_K]\) be the arc length from \(v_K^+(t_0)\) to \(p\) along \(\partial K\), so that we want \(p = \mathbf{b}_{K}^{t_0}(s)\) in the end. As \(p\) is in \(\partial K\), it is inside the tangent line \(l_K(t)\) for some angle \(t \in (t_0, t_0 + 2\pi]\). Now the arc length \(s \in [0, B_K]\) and the tangent line angle \(t \in (t_0, t_0 + 2\pi]\) are the two different variables attached to \(p \in \partial K\).

Unfortunately, the relation between \(s\) and \(t\) cannot be simply described as a function from one to another. A single value of \(s\) may correspond to multiple values of \(t\) (if \(p\) is a sharp corner of angle \(< \pi\)), and likewise a single value of \(t\) may correspond to multiple values of \(s\) (if \(p\) is on the edge \(e_K(t)\) which is a proper line segment). We need the language of generalized inverse (e.g. \autocite{fortelleStudyGeneralizedInverses}) to describe the relationship between \(s\) and \(t\).

The map \(g_K^{t_0}\) is defined so that it sends \(t\) to the largest possible corresponding \(s\).

\begin{definition}

Define \(g_K^{t_0} :[t_0, t_0 + 2\pi] \to [0, B_K]\) as \(g_{K}^{t_0}(t) = \sigma_K\left( (t_0, t] \right)\).

\label{def:conversion-ts}
\end{definition}

The map \(f_K^{t_0}\) is defined so that sends \(s\) to the smallest possible corresponding \(t\).

\begin{definition}

Define \(f_K^{t_0} : [0, B_K] \to [t_0, t_0 + 2\pi]\) as \(f_K^{t_0}(s) = \min \left\{ t \geq t_0 : \sigma_K((t_0, t]) \geq s \right\}\).

\label{def:conversion-st}
\end{definition}

It is rudimentary to check that \(f_K^{t_0}\) is well-defined. We remark that \(f_K^{t_0}\) is the minimum inverse \(g_K^{t_0\wedge}\) of \(g_K^{t_0}\) as defined in \autocite{fortelleStudyGeneralizedInverses}. Note the following.

\begin{proposition}

The functions \(f_K^{t_0}\) and \(g_{K}^{t_0}\) are monotonically increasing.

\label{pro:conversion-monotone}
\end{proposition}

\begin{proof}
That \(g_K^{t_0}(t)\) is increasing is immediate from \Cref{def:conversion-ts}. For any \(t_1 < t_2\), observe
\[
\left\{ t_1 \geq t_0 : \sigma_K((t_0, t]) \geq s \right\} \subseteq \left\{ t_2 \geq t_0 : \sigma_K((t_0, t]) \geq s \right\}
\]
so by \Cref{def:conversion-st} we have \(f_K^{t_0}(t_1) \leq f_K^{t_0}(t_2)\).
\end{proof}

The following can be checked using \Cref{def:conversion-st}.

\begin{proposition}

We have \(f_K^{t_0}(0) = t_0\) and \(f_K^{t_0}(s) > t_0\) for all \(s > 0\).

\label{pro:conversion-st-nonzero}
\end{proposition}

\begin{proof}
That \(f_K^{t_0}(0) = t_0\) is immediate from \Cref{def:conversion-st}. If \(s > 0\), then any \(t \geq t_0\) satisfying \(\sigma_K((t_0, t]) \geq s\) has to satisfy \(t > t_0\), so we have \(f_K^{t_0}(s) > t_0\).
\end{proof}

We will often write \(f_K^{t_0}\) and \(g_K^{t_0}\) as simply \(f\) and \(g\) in proofs because our \(K\) and \(t_0\) are fixed. With the converstions between \(s\) and \(t\) prepared (\(f\) maps \(s\) to \(t\), and \(g\) maps \(t\) to \(s\)), the path \(\mathbf{b}_{K}^{t_0}\) can be defined by integrating the unit vector \(u_t\) for each \(s\).

\begin{definition}

Define \(\mathbf{b}_{K}^{t_0} : [0, B_K] \to \mathbb{R}^2\) as the absolutely continuous (and thus rectifiable) function with the initial condition \(\mathbf{b}_{K}^{t_0}(0) = v_K^+(t_0)\) and the derivative \(\left(\mathbf{b}_{K}^{t_0}\right)'(s) = v_{f_K^{t_0}(s)}\) almost everywhere. That is:
\[
\mathbf{b}_{K}^{t_0}(s) := v_K^+(t_0) + \int_{s' \in (0, s]} v_{f_K^{t_0}(s')} \, ds'
\]

\label{def:parametrization-formal}
\end{definition}

Note that the function \(f_{K}^{t_0}\) is monotonically increasing, so the integral in \Cref{def:parametrization-formal} is well-defined.

\begin{proposition}

The function \(\mathbf{b}_{K}^{t_0} : [0, B_K] \to \mathbb{R}^2\) is an arc-length parametrization.

\label{pro:parametrization-arc-length}
\end{proposition}

\begin{proof}
Length of an absolutely continuous curve~\(\mathbf{x} : [a, b] \to \mathbb{R}^2\)~is the integral of~\(| | \mathbf{x}'(s) | |\) from \(s=a\) to \(s=b\) \autocite{jones2001lebesgue}. For \(\mathbf{x} = \mathbf{b}_K^{t_0}\), we have \(| | \mathbf{x}'(s) | | = 1\) for almost every \(s\) by \Cref{def:parametrization-formal}, thus completing the proof.
\end{proof}

We define \(\mathbf{b}_K^{t_0, t_1}\) as an initial segment of \(\mathbf{b}_K^{t_0}\) that ends with \(v_K^+(t_1)\).

\begin{definition}

For any \(t_0, t_1 \in \mathbb{R}\) such that \(t_1 \in [t_0, t_0 + 2 \pi]\), define \(\mathbf{b}_{K}^{t_0, t_1}\) as the curve \(\mathbf{b}_{K}^{t_0} (s)\) restricted on the interval \(s \in [0, g_{K}^{t_0}(t_1)]\).

\label{def:boundary-segment-parametrization}
\end{definition}

\subsection{Theorems on parametrization}

We now show that \(\mathbf{b}_K^{t_0}\) does parametrize our boundary \(\partial K\) as intended. We prepare three technical lemmas that handle conversions between \(s\) and \(t\).

\begin{lemma}

The followings hold.

\begin{enumerate}
\def\labelenumi{\arabic{enumi}.}
\tightlist
\item
  For any \(t_1 \in (t_0, t_0 + 2\pi]\), we have \(\left(f_{K}^{t_0}\right)^{-1}([t_0, t_1]) = [0, \sigma_K\left( (t_0, t_1] \right)] = [0, g_{K}^{t_0}(t_1)]\).
\item
  Moreover, the set \(\left( f_{K}^{t_0} \right)^{-1} (\left\{ t_1 \right\})\) is either \([g_{K}^{t_0}(t_1-), g_{K}^{t_0}(t_1)]\) or \((g_{K}^{t_0}(t_1-), g_{K}^{t_0}(t_1)]\).
\end{enumerate}

\label{lem:parametrization-set-calculation}
\end{lemma}

\begin{proof}
Write \(f_K^{t_0}\) and \(g_K^{t_0}\) as simply \(f\) and \(g\). The first statement comes from manipulating the definitions as the following.

\begin{align*}
f^{-1}([t_0, t_1]) & = \left\{ s \in [0, B_K] : \min \left\{ t \geq t_0 : \sigma_K\left( (t_0, t] \right) \geq s \right\} \in [t_0, t_1] \right\} \\
& = \left\{ s \in [0, B_K] :  \sigma\left( (t_0, t_1] \right) \geq s \right\} \\
& = [0, \sigma_K((t_0, t_1])] = [0, g(t_1)]
\end{align*}
Now send \(t \to t_1^-\) in the equality \(f^{-1}([t_0, t]) = [0, g(t)]\) to obtain that \(f^{-1}([t_0, t_1)) = \bigcup_{t < t_1} [0, g(t)]\) is either \([0, g(t_1-))\) or \([0, g(t_1-)]\). Then use \(f^{-1} (\left\{ t_1 \right\}) = f^{-1}([t_0, t_1]) \setminus f^{-1}([t_0, t_1))\) to get the second statement.
\end{proof}

\begin{lemma}

The measure \(\sigma_K\) on \((t_0, t_0 + 2 \pi]\) is the pushforward of the Lebesgue measure on \((0, B_K]\) with respect to the map \(f_{K}^{t_0} : (0, B_K] \to (t_0, t_0 + 2 \pi]\) restricted to \((0, B_K]\).

\label{lem:parametrization-pushforward}
\end{lemma}

\begin{proof}
Write \(f_K^{t_0}\) as \(f\). Observe that \(f\) restricted to \((0, B_K]\) has range in \((t_0, t_0 + 2\pi]\) by \Cref{pro:conversion-st-nonzero}. The first statement of \Cref{lem:parametrization-set-calculation} then shows that the measure \(\sigma_K\) on \((t_0, t_0 + 2 \pi]\) and the pushforward of the Lebesgue measure on \((0, B_K]\) with respect to \(f : (0, B_K] \to (t_0, t_0 + 2 \pi]\) agree on every closed interval \((t_0, t]\) for all \(t \in (t_0, t_0 + 2\pi]\).
\end{proof}

\begin{lemma}

\(\mathbf{b}_{K}^{t_0}(g_{K}^{t_0}(t)) = v_{K}^+(t)\) for all \(t \in [t_0, t_0 + 2\pi]\) and \(\mathbf{b}_{K}^{t_0}(g_{K}^{t_0}(t-)) = v_{K}^-(t)\) for all \(t \in (t_0, t_0 + 2\pi]\). Moreover, for all \(t \in (t_0, t_0 + 2\pi]\) the function \(\mathbf{b}_{K}^{t_0}\) restricted to the interval \([g_{K}^{t_0}(t_1-), g_{K}^{t_0}(t_1)]\) is the arc-length parametrization of the edge \(e_K(t)\) from vertex \(v_K^-(t)\) to \(v_K^+(t)\).

\label{lem:parametrization-vertex}
\end{lemma}

\begin{proof}
Write \(f_K^{t_0}\) and \(g_K^{t_0}\) as simply \(f\) and \(g\). By \Cref{lem:parametrization-pushforward} and \Cref{thm:boundary-measure}, we have the following calculation.

\begin{align*}
\mathbf{b}_{K}^{t_0} (g(t)) & = v_K^+(t_0) + \int_{s' \in (0, g(t)]} v_{f(s')} \, ds' \\
& = v_K^+(t_0) + \int_{s' \in f^{-1}([t_0, t])} v_{f(s')} \, ds' \\
& = v_K^+(t_0) + \int_{t \in(t_0, t]} v_t \, \sigma(dt) = v^+_K(t)
\end{align*}
For the proof of \(\mathbf{b}_{K}^{t_0}(g_{K}^{t_0}(t-)) = v_{K}^-(t)\), send \(t' \to t^-\) in the equation \(\mathbf{b}_{K}^{t_0}(g_{K}^{t_0}(t')) = v_{K}^+(t')\) and use \Cref{thm:limits-converging-to-vertex}. By the second statement of \Cref{lem:parametrization-set-calculation}, the value \(f(s')\) on the interval \(s' \in (g(t-), g(t)]\) is always equal to \(t\). So the derivative of \(\mathbf{b}_K^{t_0}(s')\) restricted to the interval \([g(t-), g(t)]\) is almost everywhere equal to \(v_t\), and \(\mathbf{b}_{K}^{t_0}\) is the arc-length parametrization of the edge \(e_K(t)\) from vertex \(v_K^-(t)\) to \(v_K^+(t)\) on the interval \([g(t-), g(t)]\).
\end{proof}

We now prove the claimed theorems on \(\mathbf{b}_K^{t_0, t_1}\). That \(\mathbf{b}_K^{t_0, t_1}\) is injective will be proved later.

\begin{proof}[Proof of \Cref{thm:param-segment}]
Write \(f_K^{t_0}\) and \(g_K^{t_0}\) as simply \(f\) and \(g\). By \Cref{lem:parametrization-set-calculation}, the interval \([0, g(t_1)]\) is equal to the inverse image \(f^{-1} ([t_0, t_1])\), and so is the disjoint union of the singleton \(f^{-1} (\left\{ t_0 \right\}) = \left\{ 0 \right\}\) and the intervals \(f ^{-1} (\left\{ t \right\})\) whose closure is \([g(t-), g(t)]\) for all \(t \in (t_0, t_1]\). Under the map \(\mathbf{b}_{K}^{t_0}\), the singleton \(\left\{ 0 \right\}\) maps to \(\left\{ v_K^+(t_0) \right\}\) and the set \([g(t-), g(t)]\) maps to \(e_K(t)\) for all \(t \in (t_0, t_1]\) by \Cref{lem:parametrization-vertex}. This proves that the image of the interval \([0, g(t_1)]\) under the map \(\mathbf{b}_{K}^{t_0}\) is the set \(\left\{ v_K^+(t_0) \right\} \cup \bigcup_{t \in (t_0, t_1]} e_K(t)\).
\end{proof}

\begin{proof}[Proof of \Cref{thm:param-segment-length}]
This comes from \Cref{pro:parametrization-arc-length} and that the domain \([0, g_{K}^{t_0}(t_1)]\) of \(\mathbf{b}_K^{t_0, t_1}\) has length \(\sigma_K((t_0, t_1])\).
\end{proof}

\begin{proof}[Proof of \Cref{thm:param-concatenation}]
Write \(f_K^{t_0}\) and \(g_K^{t_0}\) as simply \(f\) and \(g\). The curve \(\mathbf{b}_{K}^{t_0, t_1}\) is an initial part of the curve \(\mathbf{b}_{K}^{t_0, t_2}\). So it remains to show that \(\mathbf{b}_{K}^{t_0}\) restricted to the interval \([g(t_1), g(t_2)]\) is the same as \(\mathbf{b}_{K}^{t_1}\) restricted to \([0, g_K^{t_1}(t_2)]\), with the domain shifted to right by \(g(t_1)\). Observe \(g(t_1) + g_K^{t_1}(t_2) = g(t_2)\) by \Cref{def:conversion-ts} and additivity of \(\sigma_K\). The initial point of the two curves is equal to \(v_K^+(t_1)\) by \Cref{lem:parametrization-vertex}. We show that the derivatives of \(\mathbf{b}_{K}^{t_0}(t + g(t_1))\) and \(\mathbf{b}_{K}^{t_1}(t)\) match for all \(t \in [0, g(t_2) - g(t_1)]\). By \Cref{def:parametrization-formal}, we only need to check \(f(t + g(t_1)) = f_K^{t_1} (t)\). This immediately follows from \Cref{def:conversion-st}.
\end{proof}

\begin{proof}[Proof of \Cref{thm:param-curve-area-functional}]
Write \(f_K^{t_0}\) and \(g_K^{t_0}\) as simply \(f\) and \(g\). Take any \(s \in (0, g(t_1)]\) and let \(t = f(s)\). Observe that by \Cref{pro:conversion-st-nonzero}, we have \(t \in (t_0, t_1]\) and \(s\) is in \(f^{-1}(\left\{ t \right\})\) which is either \((g(t_1 -), g(t_1)]\) or \([g(t_1 -), g(t_1)]\) by \Cref{lem:parametrization-set-calculation}. Then as \(\mathbf{b}_{K}^{t_0} (s) \in e_K(t)\) by \Cref{lem:parametrization-vertex}, we have \(\mathbf{b}_{K}^{t_0} (s) \times v_{t} = p_K(t)\). So we have the following.

\begin{align*}
\mathcal{I} \left( \mathbf{b}_{K}^{t_0, t_1} \right) & = \frac{1}{2} \int_{s \in (0, g(t_1)]} \mathbf{b}_{K}^{t_0} (s) \times v_{f(s)} \, ds \\
& = \frac{1}{2} \int_{s \in f^{-1}((t_0, t_1])} p_K(f(s)) \, ds \\
& = \frac{1}{2} \int_{t \in(t_0, t_1]} p_K(t) \, \sigma(dt)
\end{align*}
The first equality above uses \Cref{def:parametrization-formal}. The second equality above uses \Cref{lem:parametrization-set-calculation} and \(\mathbf{b}_{K}^{t_0} (s) \times v_{t} = p_K(t)\). The last equality above uses \Cref{lem:parametrization-pushforward}. This proves the first equality stated in the theorem. To show the second stated equality, check \(v_K(t) \times dv_K^+(t) = v_K^+(t) \times v_{t} \sigma_K(dt) = p_K(t) \sigma(dt)\) by \Cref{pro:boundary-measure-differential}.
\end{proof}

\subsection{Injectivity of parametrization}

Proof of \Cref{thm:param-positive-jordan} requires a bit of preparation. The boundary \(\partial K\) is the union of all the edges.

\begin{theorem}

Let \(K\) be any convex body. Then the topological boundary \(\partial K\) of \(K\) as a subset of \(\mathbb{R}^2\) is the union of all edges \(\cup_{t \in S^1} e_K(t)\).

\label{thm:boundary-is-union-all-edges}
\end{theorem}

\begin{proof}
Let \(E = \cup_{t \in S^1} e_K(t)\). \(E \subseteq \partial K\) comes from \(E \subseteq K\) and that any point in \(E\) is on some tangent line of \(K\) so its neighborhood contains a point outside \(K\). Now take any point \(p \in \partial K\). As \(K\) is closed we have \(p \in K\). So \(p \cdot u_t \leq p_K(t)\) for any \(t \in S^1\). Assume that the equality does not hold for any \(t \in S^1\). Then by compactness of \(S^1\) and continuity of \(p_K\) there is some \(\epsilon > 0\) such that \(\epsilon \leq p_K(t) - p\cdot u_t\) for any \(t\). This implies that the ball of radius \(\epsilon\) centered at \(p\) is contained in \(K\). This contradicts \(p \in \partial K\). So it should be that there is some \(t \in S^1\) such that \(p \cdot u_t = p_K(t)\). That is, \(p \in e_K(t)\).
\end{proof}

We define the following segment of \(\partial K\) as well.

\begin{definition}

For any \(t_0, t_1 \in \mathbb{R}\) such that \(t_1 \in [t_0, t_0 + 2 \pi]\), define \(\mathbf{b}_{K}^{t_0, t_1-}\) as the curve \(\mathbf{b}_{K}^{t_0} (s)\) restricted on the interval \(s \in [0, g_{K}^{t_0}(t_1-)]\).

\label{def:param-segment-open}
\end{definition}

By \Cref{lem:parametrization-vertex}, the curve \(\mathbf{b}_K^{t_0, t_1-}\) ends with the vertex \(v_K^-(t_1)\). Moreover, we have the following corollary of the same \Cref{lem:parametrization-vertex}.

\begin{corollary}

For any \(t_0, t_1 \in \mathbb{R}\) such that \(t_1 \in [t_0, t_0 + 2 \pi]\), \(\mathbf{b}_K^{t_0, t_1}\) is the concatenation of \(\mathbf{b}_K^{t_0, t_1 - }\) and the arc-length parametrization of \(e_K(t_1)\) from \(v_K^-(t_1)\) to \(v_K^+(t_1)\).

\label{cor:param-segment-open}
\end{corollary}

By \Cref{def:parametrization-formal} we have \(\left(\mathbf{b}_{K}^{t_0}\right)'(s) = u_{f_K^{t_0}(s)}\) for almost every \(s\), and by \Cref{pro:conversion-st-nonzero} and \Cref{lem:parametrization-set-calculation} we have \(t_0 < f_K^{t_0}(s) < t_1\) for every \(0 < s < g_{K}^{t_0}(t_1-)\). Thus we have the following:

\begin{corollary}

Let \(t_0, t_1 \in \mathbb{R}\) be arbitrary such that \(t_1 \in [t_0, t_0 + 2 \pi]\). Then for almost every \(s\), the derivative \(\left( \mathbf{b}_{K}^{t_0, t_1-} \right)'(s)\) is equal to \(u_t\) for some \(t \in (t_0, t_1)\).

\label{cor:param-segment-open-deriv}
\end{corollary}

We use the following lemma to determine the orientation of a Jordan curve.

\begin{lemma}

Let \(p\) and \(q\) be two different points of \(\mathbb{R}^2\). Define the closed half-planes \(H_0\) and \(H_1\) as the closed half-planes separated by the line \(l\) connecting \(p\) and \(q\), so that for any point \(x\) in the interior of \(H_0\) (resp. \(H_1\)) the points \(x, p, q\) are in clockwise (resp. counterclockwise) order. If a Jordan curve \(J\) consists of the join of two arcs \(\Gamma_0\) and \(\Gamma_1\), where \(\Gamma_0\) connects \(p\) to \(q\) inside \(H_0\), and \(\Gamma_1\) connects \(q\) to \(p\) inside \(H_1\), then \(J\) is positively oriented.

\label{lem:orientation}
\end{lemma}

\begin{proof}
(sketch) We first show that it is safe to assume the case where \(J\) only intersects \(l\) at two points \(p\) and \(q\). Observe that \(H_i\) has a deformation retract to some subset \(S_i \subseteq H_i\) with \(S_i \cap l = \left\{ p, q \right\}\) (push the three segments of \(l \setminus \{p, q\}\) towards the interior of \(H_i\)). Using the retracts, we may continuously deform the arcs \(\Gamma_0\) and \(\Gamma_1\) inside \(S_0\) and \(S_1\) respectively without chainging the orientation of \(J\). Now take any point \(r\) inside the segment connecting \(p\) and \(q\). Continuously move a point \(x\) inside \(J\) in the orientation of \(J\) starting with \(x = p\). As \(x\) moves along \(\Gamma_0\) from \(p\) to \(q\) the argument of \(x\) with respect to \(r\) increases by \(\pi\). And as \(x\) moves along \(\Gamma_1\) the argument of \(x\) with respect to \(r\) again increases by \(\pi\). So the total increase in the argument of \(x \in J\) is \(2\pi\) and \(J\) is positively oriented.
\end{proof}

Now we are ready to prove \Cref{thm:param-positive-jordan}.

\begin{proof}
(of \Cref{thm:param-positive-jordan})

That \(\mathbf{b}_K^{t, t + 2\pi}\) is an arc-length parametrization of \(\partial K\) comes from \Cref{thm:param-segment} and \Cref{thm:boundary-is-union-all-edges}.

We now show that \(\mathbf{b}_K^{t, t + 2\pi}\) is a Jordan curve. By \Cref{thm:param-concatenation} the curve \(\mathbf{b}_K^{t, t + 2\pi}\) is the concatenation of two curves \(\mathbf{b}_K^{t, t + \pi}\) and \(\mathbf{b}_K^{t + \pi, t + 2\pi}\) connecting \(p = v_{K}^+(t)\) and \(q = v_K^+(t + \pi)\) and vice versa. As \(K\) has nonempty interior, the width of \(K\) measured in the direction of \(u_t\) is strictly positive, and the point \(p\) is strictly further than the point \(q\) in the direction of \(u_t\).

We first show that the curve \(\mathbf{b}_K^{t, t + \pi}\) is a Jordan arc from \(p\) to \(q\). The curve \(\mathbf{b}_K^{t, t + \pi}\) is the join of the curve \(\mathbf{b}_K^{t, t + \pi-}\) and \(e_{K}(t + \pi)\) by \Cref{cor:param-segment-open}. Also, by \Cref{cor:param-segment-open-deriv}, the derivative of \(\mathbf{b}_K^{t, t + \pi-}(s) \cdot u_t\) with respect to \(s\) is strictly positive for almost every \(s\), so the curve \(\mathbf{b}_K\) is moving strictly in the direction of \(-u_t\). This with the fact that \(e_K(t + \pi)\) is parallel to \(v_t\) shows that the curve \(\mathbf{x}_{K, t, t + \pi}\) is injective and thus a Jordan arc. A similar argument shows that \(\mathbf{b}_K^{t + \pi, t + 2\pi}\) is also a Jordan arc.

Define the closed half-planes \(H_0\) and \(H_1\) as the half-planes divided by the line \(l\) connecting \(p\) and \(q\), so that for any point \(x\) in the interior of \(H_0\) (resp. \(H_1\)) the points \(x, p, q\) are in clockwise (resp. counterclockwise) order. Observe that \(\mathbf{b}_K^{t, t + \pi}\) (resp. \(\mathbf{b}_K^{t + \pi, t + 2\pi}\)) is in \(H_0\) (resp. \(H_1\)) by \Cref{thm:param-segment}. Let \(\mathbf{b}\) be either of the curves \(\mathbf{b}_K^{t, t + \pi}\) or \(\mathbf{b}_K^{t + \pi, t + 2\pi}\). Call the line segment connecting \(p\) and \(q\) as \(pq\). Then \(\mathbf{b}\) is either \(pq\) (in case \(\mathbf{b}\) passes through a point \(r\) strictly on \(pq\)) or a curve connecting \(p\) and \(q\) through the interior of \(H_0\) (or \(H_1\)). In any case, the curves \(\mathbf{b}_K^{t, t + \pi}\) and \(\mathbf{b}_K^{t + \pi, t + 2\pi}\) only overlap at the endpoints \(p\) and \(q\) because \(K\) has nonempty interior, showing that \(\mathbf{b}_K^{t, t + 2\pi}\) is a Jordan curve. That \(\mathbf{b}_K^{t, t + 2\pi}\) is positively oriented is a consequence of \Cref{lem:orientation}.
\end{proof}

\subsection{Closed interval}

We define the closed-interval variant \(\mathbf{b}_K^{t_0-, t_1}\) of the curve \(\mathbf{b}_K^{t_0, t_1}\) as essentially the arc-length parametrization of the curve connecting \(v_K^-(t_0)\) to \(v_K^+(t_1)\) along the boundary \(\partial K\) counterclockwise.

\begin{definition}

For every \(t_0 \in \mathbb{R}\) and \(t_1 \in [t_0, t_0 + 2\pi)\) define \(\mathbf{b}_K^{t_0-, t_1}\) as the concatenation of the arc-length parametrization of the edge \(e_K(t_0)\) from \(v_K^-(t_0)\) to \(v_K^+(t_1)\) and the curve \(\mathbf{b}_K^{t_0, t_1}\).

\label{def:closed-param}
\end{definition}

This follows from \Cref{thm:param-segment}.

\begin{corollary}

Assume arbitrary \(t_0 \in \mathbb{R}\) and \(t_1 \in [t_0, t_0 + 2\pi)\). Then \(\mathbf{b}_K^{t_0-, t_1}\) is an arc-length parametrization of the set \(\cup_{t \in [t_0, t_1]} e_K(t)\) from point \(v_K^-(t_0)\) to \(v_K^+(t_1)\).

\label{cor:closed-param-segment}
\end{corollary}

This follows from \Cref{thm:param-segment-length}.

\begin{corollary}

Assume arbitrary \(t_0 \in \mathbb{R}\) and \(t_1 \in [t_0, t_0 + 2\pi)\). Then the curve \(\mathbf{b}_K^{t_0-, t_1}\) have length \(\sigma_K([t_0, t_1])\).

\label{cor:closed-param-segment-length}
\end{corollary}

This follows from \Cref{thm:param-concatenation}.

\begin{corollary}

Assume arbitrary \(t_0, t_1, t_2\) such that \(t_0 \leq t_1 \leq t_2 < t_0 + 2\pi\). Then \(\mathbf{b}_{K}^{t_0-, t_2}\) is the concatenation of \(\mathbf{b}_{K}^{t_0-, t_1}\) and \(\mathbf{b}_{K}^{t_1, t_2}\).

\label{cor:closed-param-concatenation}
\end{corollary}

This follows from \Cref{thm:param-curve-area-functional} and \Cref{thm:surface-area-singleton}.

\begin{corollary}

Assume arbitrary \(t_0 \in \mathbb{R}\) and \(t_1 \in [t_0, t_0 + 2\pi)\). Then we have:
\[
\mathcal{I} \left( \mathbf{b}_{K}^{t_0-, t_1} \right) = \frac{1}{2} \int_{[t_0, t_1]}p_K(t)\,\sigma_K(dt)
\]

\label{cor:closed-param-curve-area-functional}
\end{corollary}

\begin{theorem}

Assume that \(K\) have nonempty interior. Assume arbitrary \(t_0 \in \mathbb{R}\) and \(t_1 \in [t_0, t_0 + 2\pi)\). Then \(\mathbf{b}_K^{t_0-, t_1}\) is one of: a Jordan arc, a Jordan curve, or a single point.

\label{thm:closed-param-positive-jordan}
\end{theorem}

\begin{proof}
Take an arbitrary \(t_{-1}\) so that \(t_{-1} < t_0 \leq t_1 < t_{-1} + 2\pi\). Then \(\mathbf{b}_K^{t_{-1}, t_1}\) is the concatenation of \(\mathbf{b}_K^{t_{-1}, t_0-}\), \(e_K(t_0)\), and \(\mathbf{b}_K^{t_0, t_1}\) by \Cref{thm:param-concatenation} and \Cref{cor:param-segment-open}. Then by \Cref{def:closed-param} the concatenation of \(e_K(t_0)\), and \(\mathbf{b}_K^{t_0, t_1}\) is \(\mathbf{b}_K^{t_0-, t_1}\). Now by \Cref{cor:param-positive-jordan} and that \(\mathbf{b}_K^{t_0-, t_1}\) is a part of \(\mathbf{b}_K^{t_{-1}, t_1}\), we prove the theorem.
\end{proof}

\subsection{Normal angles}
\label{sec:normal-angles}
This section defines the set of \emph{normal angles} of a convex body \(K\).

\begin{definition}

Define the set of \emph{normal angles} \(\mathbf{n}(K)\) as the support of the surface area measure \(\sigma_K\) on \(S^1\).

\label{def:convex-set-support}
\end{definition}

If \(K\) is a convex polygon, \(\mathbf{n}(K)\) is the collection of all angles \(t\) such that each \(u_t\) is a normal vector of a proper edge of \(K\). The notion \(\mathbf{n}(K)\) generalizes this to arbitrary convex body \(K\). For an example, take the semicircle \(K = \left\{ (x, y) : x^2 + y^2 \leq 1, y \geq 0 \right\}\). Then the normal angles of \(K\) is the set \([0, \pi] \cup \{3\pi/2\}\).

We now collect theorems on \(\mathbf{n}(K)\).

\begin{lemma}

Let \((t_1, t_2)\) be any open interval of \(S^1\) of length \(< \pi\). Then for every \(t \in S^1 \setminus (t_1, t_2)\), we have \(v_K(t_1, t_2) \in H_K(t)\).

\label{lem:vertex-in-half-plane}
\end{lemma}

\begin{proof}
Let \(p = v_K(t_1, t_2)\). We can either assume \(t_1 - \pi < t < t_1\) or \(t_2 < t < t_2 + \pi\). In the first case, the points \(v_K(t_1, t)\) and \(p\) are on the line \(l_K(t_1)\) and \(p\) is further than \(v_K(t_1, t)\) in the direction of \(v_{t_1}\). Since \(v_K(t_1, t) \in H_K(t)\) we now should have \(p \in H_K(t)\). In the second case, the points \(v_K(t, t_2)\) and \(p\) are on the line \(l_K(t_2)\) and \(v_K(t, t_2)\) is further than \(p\) in the direction of \(v_{t_2}\). Since \(v_K(t, t_2) \in H_K(t)\) we now have \(p \in H_K(t)\).
\end{proof}

\begin{theorem}

Let \(K\) be a convex body, and let \((t_1, t_2)\) be any open interval of \(S^1\) of length \(< \pi\). The followings are equivalent.

\begin{enumerate}
\def\labelenumi{\arabic{enumi}.}
\tightlist
\item
  \((t_1, t_2)\) is disjoint from \(\mathbf{n}(K)\)
\item
  There is a single point \(p\) so that we have \(v_K^+(t) = v_K^-(t) = p\) for all \(t \in (t_1, t_2)\).
\item
  Every tangent line \(l_K(t)\) passes through a common point \(p\) for \(t \in [t_1, t_2]\).
\item
  \(v_K(t_1, t_2) \in K\)
\end{enumerate}

\label{thm:convex-set-support-disjoint}
\end{theorem}

\begin{proof}
(1 \(\Rightarrow\) 2) Let \(p = v_K^+(t_1)\). Then \(v_K^-(t_2) = p\) as well by \Cref{cor:boundary-measure-open}. We also have \(p = v_K^{\pm}(t)\) for every \(t \in (t_1, t_2)\) by \Cref{thm:boundary-measure} on the interval \((t_1, t]\) and \Cref{cor:boundary-measure-open} on the interval \((t_1, t)\).

(2 \(\Rightarrow\) 1) By \Cref{thm:limits-converging-to-vertex} we also have \(v_K^+(t_1) = v_K^-(t_2) = p\). By \Cref{cor:boundary-measure-open} we have the integral \(\int_{t \in (t_1, t_2)} v_t \, \sigma_K(dt) = v_K^-(t_2) - v_K^+(t_1)\) equal to \(0\). Now \(\sigma_K\) has to be zero on \((t_1, t_2)\), or otherwise the integral taken the dot product with \(-u_{t_1}\) should be nonzero as well.

(2 \(\Rightarrow\) 3) follows from that every edge \(e_K(t) = l_K(t) \cap K\) is the segment connecting \(v_K^-(t)\) to \(v_K^+(t)\).

(3 \(\Rightarrow\) 4) The point \(p\) that every tangent line \(l_K(t)\) of \(t \in [t_1, t_2]\) pass through should be \(l_K(t_1) \cap l_K(t_2) = v_K(t_1, t_2)\). We have \(p \in H_K(t)\) for all \(t \in [t_1, t_2]\). We also have \(p \in H_K(t)\) for all \(t \in S^1 \setminus (t_1, t_2)\) by \Cref{lem:vertex-in-half-plane}. Now \(p \in \bigcap_{t \in S^1} H_K(t) = K\).

(4 \(\Rightarrow\) 2) Let \(p := v_K(t_1, t_2)\) and define the cone \(F = H_K(t_1) \cap H_K(t_2)\) pointed at \(p\). Take any \(t \in (t_1, t_2)\). Then the value of \(z \cdot u_t\) over all \(z \in F\) has a unique maximum at \(z = p\). Because \(p \in K \subseteq F\), the value of \(z \cdot u_t\) over all \(z \in K\) also has a unique maximum at \(z = p\). This means that \(e_K(t) = \left\{ p \right\}\), completing the proof.
\end{proof}

\begin{theorem}

Let \(\Pi\) be any closed subset of \(S^1\) such that \(S^1 \setminus \Pi\) is a disjoint union of open intervals of length \(< \pi\). Then for any convex body \(K\), the followings are equivalent.

\begin{enumerate}
\def\labelenumi{\arabic{enumi}.}
\tightlist
\item
  \(K = \bigcap_{t \in \Pi} H_K(t)\)
\item
  \(\mathbf{n}(K)\) is contained in \(\Pi\).
\end{enumerate}

\label{thm:convex-set-support}
\end{theorem}

\begin{proof}
(1 \(\Rightarrow\) 2) Let \((t_1, t_2)\) be any connected component of \(S^1 \setminus \Pi\). Then the interval has length \(< \pi\) by assumption. Now take any \(t \in (t_1, t_2)\). The vertex \(v_K(t_1, t_2)\) is in \(\bigcap_{t \in \Pi} H_K(t) = K\) by \Cref{lem:vertex-in-half-plane}. So by \Cref{thm:convex-set-support-disjoint}, \((t_1, t_2)\) is disjoint from \(\mathbf{n}(K)\). Since \((t_1, t_2)\) was an arbitrary connected component of \(S^1 \setminus \Pi\), we are done.

(2 \(\Rightarrow\) 1) It suffices to show that \(\bigcap_{u \in \Pi} H_K(u) \subseteq H_K(t)\) for all \(t \in S^1\). Once this is shown, we can take intersection over all \(t \in S^1\) to conclude \(K \subseteq \bigcap_{u \in \Pi} H_K(u) \subseteq K\).

If \(t \in \Pi\), then we obviously have \(\bigcap_{u \in \Pi} H_K(u) \subseteq H_K(t)\) so the proof is done. Now take any \(t \in S^1 \setminus \Pi\) and let \((t_1, t_2)\) be the connected component of \(S^1 \setminus \Pi\) containing \(t\). By condition 4 of \Cref{thm:convex-set-support-disjoint} the half-plane \(H_K(t)\) contains the intersection \(F := H_K(t_1) \cap H_K(t_2)\). Observe \(t_1, t_2 \in \Pi\). So \(\bigcap_{u \in \Pi} H_K(u) \subseteq F \subseteq H_K(t)\) for all \(t \in S^1 \setminus \Pi\), completing the proof.
\end{proof}

The following theorem is known as the Gauss-Minkowski theorem (\autocite{marckert2014compact}; Theorem 8.3.1, p465 of \cite{schneider_2013}). It gives a bijection between any convex body \(K\) and its boundary measure \(\sigma_K\).

\begin{theorem}

(Gauss-Minkowski) For any finite Borel measure \(\sigma\) on \(S^1\) with \(\int_{S^1} v_t \, \sigma (dt) = 0\) there is a unique convex body \(K\) with \(\sigma_K = \sigma\) up to translations of \(K\).

\label{thm:gauss-minkowski}
\end{theorem}

By \Cref{def:convex-set-support}, we immediately get the following restriction of \Cref{thm:gauss-minkowski} in normal angles.

\begin{corollary}

Let \(\Pi\) be any closed subset of \(S^1\). For any finite Borel measure \(\sigma\) on \(\Pi\) such that \(\int_{\Pi} v_t\,\sigma(dt) = 0\), there is a convex body \(K\) with normal angles \(\mathbf{n}(K)\) in \(\Pi\) such that \(\sigma_K|_{\Pi} = \sigma\) (\Cref{def:measure-restriction}), which is unique up to translations of \(K\).

\label{cor:supported-gauss-measure}
\end{corollary}

\subsection{Mamikon's theorem}
\label{sec:mamikon's-theorem}
We prove a generalized version of Mamikon’s theorem \autocite{mnatsakanianAnnularRingsEqual1997} that works for the boundary segment \(\mathbf{b}_K^{t_0, t_1}\) (\Cref{sec:parametrization-of-boundary}) of any convex body \(K\) with a potentially non-differentiable boundary.

\begin{theorem}

Let \(K\) be an arbitrary convex body. Let \(t_0, t_1 \in \mathbb{R}\) be any angles such that \(t_0 < t_1 \leq t_0 + 2 \pi\). Note that \(\mathbf{b}_K^{t_0, t_1}\) is the counterclockwise curve along \(\partial K\) from \(p := v_K^+(t_0)\) to \(q := v_K^+(t_1)\). Let \(\mathbf{y} : [t_0, t_1] \to \mathbb{R}^2\) be any curve that is continous and rectifiable, such that for all \(t \in [t_0, t_1]\) the point \(\mathbf{y}(t)\) is always on the tangent line \(l_K(t)\). Consequently, there is a measurable function \(f : [t_0, t_1] \to \mathbb{R}\) such that \(\mathbf{y}(t) = v_K^+(t) + f(t)v_t\) for all \(t \in [t_0, t_1]\). We have the equality
\[
\mathcal{I}(\mathbf{y}) + \mathcal{I}\left( \mathbf{y}(t_1), q \right) - \mathcal{I}\left( \mathbf{b}_K^{t_0, t_1} \right)  - \mathcal{I}\left( \mathbf{y}(t_0), p \right) =  \frac{1}{2}\int_{t_0}^{t_1} f(t) ^2 \, dt.
\]

\label{thm:mamikon}
\end{theorem}

\begin{proof}
For this proof only, let \(\mathbf{x} := v_K^+\) be the alias of \(v_K^+ : [t_0, t_1] \to \mathbb{R}^2\). First, we prove a differential version of the theorem by calculating the differentials on the interval \((t_0, t_1]\). Note that \(\mathbf{y}\) is continuous by definition and \(\mathbf{x}\) is right-continuous by \Cref{thm:limits-converging-to-vertex}, so that \(f\) and \(\mathbf{y} \times \mathbf{x}\) are also right-continuous on \((t_0, t_1]\). So the differential \(d(\mathbf{y} \times \mathbf{x})\) makes sense as a Lebesgue-Stieltjes measure on \((t_0, t_1]\) (\Cref{sec:lebesgue-stieltjes-measure}). We have the chain of equalities

\begin{align*}
& \phantom{{} = .} \mathbf{y}(t) \times d \mathbf{y}(t) - \mathbf{x}(t) \times d \mathbf{x}(t) + d \left( \mathbf{y}(t) \times \mathbf{x}(t) \right)  \\
& = \mathbf{y}(t) \times d \mathbf{y}(t) - \mathbf{x}(t) \times d \mathbf{x}(t) + \left( d \mathbf{y}(t) \times \mathbf{x}(t) + \mathbf{y}(t) \times d \mathbf{x} (t) \right)  \\
& = \left( \mathbf{y}(t) - \mathbf{x}(t) \right) \times d\left( \mathbf{y}(t) + \mathbf{x}(t) \right)  \\
& = \left( \mathbf{y}(t) - \mathbf{x}(t) \right) \times d\left( \mathbf{y}(t) - \mathbf{x}(t) \right)  \\
& = f(t) u_t \times d\left( f(t) u_t \right) = f(t) u_t \times ( u_t df(t) + f(t) v_t dt) = f(t)^2 dt
\end{align*}

of measures on \((t_0, t_1]\). The first equality uses the product rule of differentials (\Cref{pro:lebesgue-stieltjes-product}). The second equality is bilinearity of \(\times\) (note that \(d \mathbf{y}(t) \times \mathbf{x}(t) = - \mathbf{x}(t) \times d \mathbf{y}(t)\) by antisymmetry of \(\times\)). As we have \(d \mathbf{x}(t) = \sigma(dt)v_t\) by \Cref{pro:boundary-measure-differential} and \(\mathbf{y}(t) - \mathbf{x}(t) = f(t)v_t\), they are parallel and we get \((\mathbf{y}(t) - \mathbf{x}(t)) \times d \mathbf{x}(t) = 0\) which is used in the third equality. The last chain of equalities are basic calculations.

If we integrate the differential formula above on the whole interval \((t_0, t_1]\), the terms \(\mathbf{y}(t) \times d \mathbf{y}(t)\) and \(\mathbf{x}(t) \times d \mathbf{x}(t)\) becomes \(2 \mathcal{I}(\mathbf{y})\) and \(2 \mathcal{I}(\mathbf{x})\) by \Cref{def:curve-area-functional} and \Cref{thm:param-curve-area-functional} respectively. The Lebesgue-Stieltjes measure \(d(\mathbf{y}(t) \times \mathbf{x}(t))\) integrates to the difference \(2 \mathcal{I} \left( \mathbf{y}(t_1), v_K^+(t_1) \right) - 2 \mathcal{I} \left( \mathbf{y}(t_0), v_K^+(t_0) \right)\). So the integral matches twice the left-hand side of the claimed equality in \Cref{thm:mamikon}, completing the proof.
\end{proof}

We have the following variant of \Cref{thm:mamikon} on the curve \(\mathbf{b}_K^{t_0 -, t_1}\) (\Cref{def:closed-param}) as well.

\begin{theorem}

Let \(K\) be an arbitrary convex body. Let \(t_0, t_1 \in \mathbb{R}\) be any angles such that \(t_0 < t_1 < t_0 + 2 \pi\). Note that \(\mathbf{b}_K^{t_0 -, t_1}\) is a curve along \(\partial K\) from \(p := v_K^-(t_0)\) to \(q := v_K^+(t_1)\). Let \(\mathbf{y} : [t_0, t_1] \to \mathbb{R}^2\) and \(f : [t_0, t_1] \to \mathbb{R}\) be as in \Cref{thm:mamikon}. Then we have
\[
\mathcal{I}(\mathbf{y}) + \mathcal{I} \left( \mathbf{y}(t_1), q \right) - \mathcal{I}(\mathbf{b}_K^{t_0 -, t_1}) - \mathcal{I} \left( \mathbf{y}(t_0), p \right) =  \frac{1}{2}\int_{t_0}^{t_1} f(t) ^2 \, dt.
\]

\label{thm:mamikon-closed}
\end{theorem}

\begin{proof}
Apply \Cref{thm:mamikon} to \(\mathbf{b}_K^{t_0, t_1}\), and use that \(\mathbf{b}_{K}^{t_0 -, t_1}\) is the join of \(e_{K}(t_0)\) and \(\mathbf{b}_K^{t_0, t_1}\) (\Cref{def:closed-param}).
\end{proof}

Now we prove a variant of \Cref{thm:mamikon} where the curve \(\mathbf{y}(t)\) parametrizes a segment of the tangent line \(l_K(t_1)\) of \(K\). To do so, we need to prepare some notation.

\begin{definition}

Let \(t, t' \in S^1\) be arbitrary such that \(t' \neq t, t + \pi\). Define \(\tau_K(t, t')\) as the unique value \(\alpha\) such that \(v_K(t, t') = v_K^+(t) + \alpha v_t\).

\label{def:tangent-leg-length}
\end{definition}

Note that \(v_K(t, t')\) was defined as the intersection of \(l_K(t)\) and \(l_K(t')\) (\Cref{def:convex-body-tangent-lines-intersection}). So fixing the line \(l_K(t')\) at angle \(t'\), the value \(\tau_K(t, t')\) measures the distance from \(v_K^+(t)\) to \(v_K(t, t')\) along the line \(l_K(t)\) at angle \(t\). Such a value \(\alpha\) exists because the points \(v_K(t, t')\) and \(v_K^+(t)\) are on the line \(l_K(t)\). Linearity of \(\tau_K(t, t')\) comes from \Cref{lem:tangent-lines-intersection-linear} and \Cref{cor:vertex-linear}.

\begin{corollary}

Let \(t, t' \in S^1\) be arbitrary such that \(t' \neq t, t + \pi\). Then \(\tau_K(t, t')\) is linear with respect to \(K\).

\label{cor:tangent-line-length-linear}
\end{corollary}

Now this variant of Mamikon’s theorem measures the area between a segment \(\mathbf{b}_K^{t_0, t_1}\) of \(\partial K\) and the tangent line \(l_K(t_1)\).

\begin{theorem}

Let \(K\) be an arbitrary convex body. Let \(t_0, t_1 \in \mathbb{R}\) be the angles such that \(t_0 < t_1 < t_0 + \pi\). Note that \(\mathbf{b}_K^{t_0, t_1}\) is the counterclockwise curve along \(\partial K\) from \(p := v_K^+(t_0)\) to \(q := v_K^+(t_1)\). Let \(r = l_K(t_0) \cap l_K(t_1)\). Then we have
\[
\mathcal{I}(r, q) - \mathcal{I}\left( \mathbf{b}_K^{t_0, t_1} \right) - \mathcal{I}\left(r, p \right) =  \frac{1}{2}\int_{t_0}^{t_1} \tau_{K}(t, t_1) ^2 \, dt.
\]

\label{thm:mamikon-tangent-line}
\end{theorem}

\begin{proof}
Define \(\mathbf{y} : [t_0, t_1] \to \mathbb{R}^2\) as \(\mathbf{y}(t) = v_K(t, t_1)\) for every \(t < t_1\) and \(\mathbf{y}(t_1) = v_K^-(t_1)\). Then \(\mathbf{y}\) is absolutely continuous by \Cref{thm:tangent-line-parametrization} and parametrizes the line segment from \(r\) to \(v_K^-(t_1)\). By \Cref{def:tangent-leg-length}, the function \(f(t) = \tau_K(t, t_1)\) satisfies \(\mathbf{y}(t) = v_K^+(t) + f(t) v_t\). Now apply \Cref{thm:mamikon} to the curves \(\mathbf{b}_{K}^{t_0, t_1}\), \(\mathbf{y}\) and function \(f\) to get
\[
\mathcal{I}(\mathbf{y}) + \mathcal{I}\left( v_K^-(t_1), q \right) - \mathcal{I}\left( \mathbf{b}_K^{t_0, t_1} \right)  - \mathcal{I}\left( r, p \right) = \frac{1}{2}\int_{t_0}^{t_1} \tau_{K}(t, t_1) ^2 \, dt.
\]
Now use \(\mathcal{I}(\mathbf{y}) = \mathcal{I}(r, v_K^-(t_1))\) (see the remark before \Cref{def:curve-area-functional-segment}) to conclude the proof.
\end{proof}





\section*{Acknowledgements}

The author acknowledges support from the Korea Foundation for Advanced Studies during the completion of this research.


\printbibliography

\end{document}
