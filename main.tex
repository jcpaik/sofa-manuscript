\documentclass[a4paper]{amsart}

\usepackage{textcomp}
\usepackage{amsmath}
\usepackage{amsthm}
\usepackage{amsfonts}
\usepackage{hyperref}
\usepackage{cleveref}
\usepackage{enumitem}
\usepackage{svg}
\usepackage{float}

\usepackage[backend=bibtex]{biblatex}
\addbibresource{refs.bib}

\theoremstyle{plain}
\newtheorem{theorem}{Theorem}
\newtheorem{lemma}{Lemma}
\newtheorem{corollary}{Corollary}
\newtheorem{remark}{Remark}
\theoremstyle{definition}
\newtheorem{definition}{Definition}

\floatplacement{figure}{h}

\providecommand{\tightlist}{%
  \setlength{\itemsep}{0pt}\setlength{\parskip}{0pt}}
\makeatletter
\def\maxwidth{\ifdim\Gin@nat@width>\linewidth\linewidth\else\Gin@nat@width\fi}
\def\maxheight{\ifdim\Gin@nat@height>\textheight\textheight\else\Gin@nat@height\fi}
\makeatother
% Scale images if necessary, so that they will not overflow the page
% margins by default, and it is still possible to overwrite the defaults
% using explicit options in \includegraphics[width, height, ...]{}
\setkeys{Gin}{width=\maxwidth,height=\maxheight,keepaspectratio}

\title[Moving a Sofa I: A Concave Upper Bound of Sofa Area]{Moving a Sofa I: A Concave Upper Bound of Sofa Area}

\author{Jineon Baek}
\address{University of Michigan, Department of Mathematics,
2074 East Hall,
530 Church Street,
Ann Arbor, MI 48109-1043}
\email{jineon@umich.edu}

\begin{document}

\maketitle

\begin{abstract}
The \emph{Moving Sofa Problem} proposed by Leo Moser in 1966 asks for the maximum area of a planar shape called \emph{sofa} that can be moved around a right-angled corner in a hallway of unit width.
While Gerver constructed a sofa of area $\mu_G = 2.2195\dots$ and conjectured that his sofa attains the maximum area, the problem remains unsolved to this date.
In this paper, we propose the \emph{injectivity hypothesis}, a speculated property of the sofa with maximum area, and construct a concave upper bound $\mathcal{A}_1$ of the sofa area assuming the hypothesis.
By maximizing $\mathcal{A}_1$, we obtain a much sharper upper bound $1 + \pi^2/8 = 2.2337\dots$ of the area compared to the previous upper bound of $2.37$ by Romik and Kallus.
This paper is a part of an ongoing attempt towards Gerver's conjecture,
and we provide an overview of the prospective computer-assisted proof of the injectivity hypothesis,
with details to be presented in an upcoming paper within this series.
\end{abstract}

\chapter{Introduction}
\label{sec:introduction}
In this section, we set up basic notations, definitions and conventions that will be assumed thoroughout the rest of the document. We also gather the notions that will be defined later for easier reference. Definitions not listed in this section will be always referenced before its first use in a proof.

This section is comprehensive and not meant to be read in one setting. Instead, the reader may start by reading only the definitions related to moving sofas, shapes, and convex bodies and move on, referring to parts of this section later as needed.
\section{Moving Sofa Problem}
\label{sec:moving-sofa-problem}
Moving a large couch through a narrow hallway requires a well-planned pivoting. The \emph{moving sofa problem}, first published by Leo Moser in 1966 \autocite{moser1966problem}, is asked in a two-dimensional idealization of such a situation:

\begin{quote}
What is the largest area \(\mu_{\text{max}}\) of a connected shape that can move around the right-angled corner of a hallway with unit width?
\end{quote}

More precisely, define the hallway \(L\) as the union \(L = L_H \cup L_V\) of sets \(L_H = (-\infty, 1] \times [0, 1]\) and \(L_V = [0, 1] \times (-\infty, 1]\) representing the horizontal and vertical side of \(L\) respectively. A \emph{moving sofa} \(S\) may be defined as a connected subset of \(L_H\) that can be moved inside \(L\) by a continuous rigid motion to a subset of \(L_V\). It is known that there exists a moving sofa attaining the maximum area \(\mu_{\text{max}}\) \autocite{gerverMovingSofaCorner1992,croft2012unsolved}, but the precise value of \(\mu_{\text{max}}\) remains unknown despite decades of partial progress \autocite{hammersley1968enfeeblement,gerverMovingSofaCorner1992,romikDifferentialEquationsExact2018,kallusImprovedUpperBounds2018}.

The best bounds currently known on \(\mu_{\max}\) are summarized as
\begin{equation}
\label{eqn:best-bounds}
\mu_G := 2.2195\dots \leq \mu_{\max} \leq 2.37.
\end{equation}
The lower bound \(2.2195\dots \leq \mu_{\max}\) comes from Gerver’s sofa \(S_G\) of area \(\mu_G :=
2.2195\dots\) constructed in 1994 \autocite{gerverMovingSofaCorner1992} (see \Cref{fig:gerver}).
Gerver derived his sofa from local optimality considerations\footnote{Gerver assumed five stages of
the movement of a sofa to derive his sofa \(S_G\) \autocite{gerverMovingSofaCorner1992}. While his
sofa \(S_G\) is locally optimal (Theorem 2 of \autocite{gerverMovingSofaCorner1992}), this does not
eliminate the possibility of a maximum-area sofa with a different kind of movement. Romik’s
simplified derivation of \(S_G\) in \autocite{romikDifferentialEquationsExact2018} also relies on
the same assumption (Equation 24, p324 of \autocite{romikDifferentialEquationsExact2018}). So their
derivations do not constitute a full proof of Gerver’s conjecture \(\mu_{\max} = \mu_G\).} and
conjectured \(\mu_G = \mu_{\max}\) that his sofa attains the maximum area. Approximate solutions
found by computer experiments are consistent with Gerver’s conjecture.\footnote{Wagner used Monte
Carlo simulation to find an approximate solution (Figure 2 of \autocite{wagner1976sofa}) that
resembles Gerver’s sofa in shape. More recent approximate solutions, as found by Gibbs
\autocite{gibbsComputationalStudySofas2014} in 2014 and Batsch \autocite{batschNumericalApproachAnalysing2022} in 2022, deviate in area from Gerver’s sofa by small margins of \(1.7 \times 10^{-7}\) and \(5.7 \times 10^{-9}\) respectively.} On the other hand, the upper bound \(\mu_{\max} \leq 2.37\) was proved by Kallus and Romik using extensive computer assistance \autocite{kallusImprovedUpperBounds2018}.

\begin{figure}
\centering
\includesvg[width=0.7\textwidth,height=\textheight]{images/gerver-sofa.svg}
\caption{Gerver’s sofa \(S_G\). The ticks denote the endpoints of 18 analytic curves and segments constituting the boundary of \(S_G\) (see \autocite{romikDifferentialEquationsExact2018} for details). The lower portion of \(S_G\) is made of two small ‘tails’ (depicted red) and one large ‘core’ (depicted blue).}
\label{fig:gerver}
\end{figure}

We prove that any moving sofa satisfying a certain property, named as the \emph{injectivity condition} (\Cref{def:injectivity}), has an area at most \(1 + \pi^2/8 = 2.2337\dots\) (\Cref{thm:main}). This upper bound, while conditional, is much closer to the lower bound \(2.2195\dots\) of Gerver than the upper bound \(2.37\) of Kallus and Romik. Since Gerver’s sofa \(S_G\) satisfies the injectivity condition, our conditional upper bound is effective on the domain containing the conjectured optimum \(S_G\). The proof does not rely on any computer assistance, and reframes the problem to a convex quadratic optimization problem as a new approach. Then we use a calculus of variation based on the Brunn-Minkowski theory on convex bodies to prove the upper bound \(1 + \pi^2/8\). We also conjecture the \emph{injectivity hypothesis} (\Cref{con:injectivity}) that there exists a maximum-area moving sofa satisfying the injectivity condition. With our result, proving the injectivity hypothesis amounts to improving the upper bound to \(\mu_{\max} \leq 1 + \pi^2/8\).


\section{Moving Hallway Problem}
\label{sec:moving-hallway-problem}
\input{out/01. Introduction/02. Moving hallway problem.tex}

\section{Main Result}
\label{sec:main-result}
\input{out/01. Introduction/03. Main result.tex}

\section{Overview of Chapters}
\label{sec:overview-of-chapters}
\Cref{sec:notations-and-conventions} contains basic definitions that will be used thoroughout this paper. Since \Cref{sec:notations-and-conventions} is comprehensive, the reader is not expected to read everything in \Cref{sec:notations-and-conventions} at one setting. Instead, the reader can start off by reading the definitions on moving sofas and convex bodies, and later refer to the parts of this section as needed.

\Cref{sec:monotone-sofas} proves \Cref{thm:monotonization-is-sofa} that any moving sofa can be enlarged to a monotone sofa, and proves structural \Cref{thm:monotone-sofa-structure} and \Cref{thm:niche-in-cap} on monotone sofas. Using this, \Cref{sec:sofa-area-functional-a} reduces the moving sofa problem to the maximization of the \emph{sofa area functional} \(\mathcal{A} : \mathcal{K}_\omega \to \mathbb{R}\) on the space of caps \(\mathcal{K}_\omega\). \Cref{sec:conditional-upper-bound-a1} proves the main \Cref{thm:main} by establishing the upper bound \(1 + \omega^2/2\) of the sofa area \(\mathcal{A}\) using the upper bound \(\mathcal{A}_1\). Each chapter starts with an overview of the chapter.

\Cref{sec:convex-bodies} proves numerous properties of an arbitrary planar convex body \(K\) that
we will use thoroughly in this paper. So the logical ordering of this paper is
\Cref{sec:notations-and-conventions}, followed by \Cref{sec:convex-bodies}, then the chapters
starting \Cref{sec:monotone-sofas}. A logically inclined reader may read in this ordering to verify
the correctness of all arguments. On the other hand, readers who are interested in the overall idea
may either start by reading the chapters in order and refer to the appendix when needed, or skim the introduction of each section of \Cref{sec:convex-bodies} and then read the rest of the paper.




\chapter{Notations and conventions}
\label{sec:notations-and-conventions}
In this section, we set up basic notations, definitions and conventions that will be assumed thoroughout the rest of the document. We also gather the notions that will be defined later for easier reference. Definitions not listed in this section will be always referenced before its first use in a proof.

This section is comprehensive and not meant to be read in one setting. Instead, the reader may start by reading only the definitions related to moving sofas, shapes, and convex bodies and move on, referring to parts of this section later as needed.
\section{Notations}
\label{sec:notations}
\input{out/10. Notations and conventions/01. Notations.tex}

\section{Moving Sofa}
\label{sec:moving-sofa}
\begin{definition}

The \emph{hallway} \(L = L_H \cup L_V\) is the union of sets \(L_H = (-\infty, 1] \times [0, 1]\) and \(L_V = [0, 1] \times (-\infty, 1]\), each representing the horizontal and vertical side of \(L\) respectively.

\label{def:hallway}
\end{definition}

\begin{definition}

Define the unit-width horizontal and vertical strips \(H = \mathbb{R} \times [0, 1]\) and \(V = [0, 1] \times \mathbb{R}\) respectively.

\label{def:strip}
\end{definition}

In the introduction, we gave a definition of a moving sofa \(S\) as a subset of \(L_H\). However, the condition that \(S\) should be confined in \(L_H\) is a bit restrictive for our future use. So we will also call any translation of such \(S \subseteq L_H\) a \emph{moving sofa} as well without loss of generality.

\begin{definition}

A \emph{moving sofa} \(S\) is a connected, nonempty and compact subset of \(\mathbb{R}^2\), such that a translation of \(S\) is a subset of \(L_H\) that admits a continuous rigid motion inside \(L\) from \(L_H\) to \(L_V\).

\label{def:sofa}
\end{definition}

It is safe assume that a moving sofa is always closed, since for any subset of \(L\) its closure is also contained in \(L\). We also define the rotation angle \(\omega\) of a moving sofa \(S\).

\begin{definition}

Say that a moving sofa \(S\) have the \emph{rotation angle} \(\omega \in (0, \pi/2]\) if the continuous rigid motion of a translate of \(S\) from \(L_H\) to \(L_V\) inside \(L\) rotates the body clockwise by \(\omega\) in its full movement.

\label{def:rotation-angle}
\end{definition}

With the result of \autocite{kallusImprovedUpperBounds2018} that \(\omega \in [81.203\dots^\circ, 90^\circ]\) for a maximum-area moving sofa, we will always assume that a moving sofa have rotation angle \(\omega \in (0, \pi/2]\). For each rotation angle \(\omega\), we define the following notions for future use.

\begin{definition}

Define \(R_\theta : \mathbb{R}^2 \to \mathbb{R}^2\) as the rotation map of \(\mathbb{R}^2\) around the origin by a counterclockwise angle of \(\theta \in S^1\).

\label{def:rotation-map}
\end{definition}

\begin{definition}

For any \(\omega \in (0, \pi/2]\), define the \emph{parallelogram} \(P_\omega = H \cap R_\omega(V)\) with \emph{rotation angle} \(\omega\).\footnote{If \(\omega = \pi/2\), then the set \(P_{\pi/2} = H\) is technically not a parallelogram. We will however call it as the parallelogram with rotation angle \(\pi/2\).}

\label{def:parallelogram}
\end{definition}

\begin{definition}

For any \(\omega \in (0, \pi/2]\), define the set \(J_\omega = [0, \omega] \cup [\pi/2, \omega + \pi/2]\).

\label{def:j-cap}
\end{definition}

For reference, the notion of \emph{standard position} will be defined in \Cref{def:standard-position}. The \emph{monotonization} \(\mathcal{M}(S)\), \emph{cap} \(\mathcal{C}(S)\), and \emph{niche} \(\mathcal{N}(K)\) of a moving sofa \(S\) will be defined in \Cref{def:monotonization}, \Cref{def:cap-sofa}, and \Cref{def:niche} respectively.

\section{Convex Body}
\label{sec:convex-body}
\begin{definition}

In this paper, a \emph{shape} \(S\) is a nonempty and compact subset of \(\mathbb{R}^2\).

\label{def:shape}
\end{definition}

\begin{definition}

For any angle \(t\) in \(S^1\) or \(\mathbb{R}\), define the unit vectors \(u_t = \left( \cos t, \sin t \right)\) and \(v_t = \left( -\sin t,\cos t \right)\).

\label{def:frame}
\end{definition}

Any line on \(\mathbb{R}^2\) can be described by the angle \(t\) of its normal vector \(u_t\) and its (signed) distance from the origin.

\begin{definition}

For any angle \(t\) in \(S^1\) and a value \(h \in \mathbb{R}\), define the line \(l(t, h)\) with the \emph{normal angle} \(t\) and the signed distance \(h\) from the origin as
\[
l(t, h) = \left\{ p \in \mathbb{R}^2 : p \cdot u_t = h \right\}.
\]

\label{def:line}
\end{definition}

A line on \(\mathbb{R}^2\) divids the plane into two half-planes. Following \Cref{def:line}, we also give a name to one of the half-planes in the direction of \(-u_t\).

\begin{definition}

For any angle \(t\) in \(S^1\) and a value \(h \in \mathbb{R}\), define the closed \emph{half-plane} \(H(t, h)\) with the boundary \(l(t, h)\) as
\[
H(t, h) = \left\{ p \in \mathbb{R}^2 : p \cdot u_t \leq h \right\}.
\]
We say that the closed half-plane \(H(t, h)\) has the \emph{normal angle} \(t\).

\label{def:half-plane}
\end{definition}

Fix a shape \(S\) and angle \(t \in S^1\). Take a sufficiently large \(h \in \mathbb{R}\) so that \(H(t, h) \supseteq S\). As we decrease \(h\) continuously, the line \(l(t, h)\) will get close to \(S\) until it makes contact with \(S\) for the first time. We define the value of \(h\), tangent line \(l(t, h)\), tangent half-plane \(H(t, h)\) as the following \Cref{def:support-function}, \Cref{def:tangent-line} and \Cref{def:tangent-half-plane}.

\begin{definition}

For any shape \(S\), define its \emph{support function} \(p_S : S^1 \to \mathbb{R}\) as the value \(p_S(t) = \sup \left\{ p \cdot u_t : p \in S \right\}\).

\label{def:support-function}
\end{definition}

\begin{definition}

For any shape \(S\) and angle \(t \in S^1\), define the \emph{tangent line} \(l_S(t)\) of \(S\) with \emph{normal angle} \(t\) as the line \(l_S(t) := l(t, p_S(t))\).

\label{def:tangent-line}
\end{definition}

\begin{definition}

For any shape \(S\) and angle \(t \in S^1\), define the \emph{tangent half-plane} \(H_S(t)\) of \(S\) with \emph{normal angle} \(t\) as the line \(H_S(t) := H(t, p_S(t))\).

\label{def:tangent-half-plane}
\end{definition}

Observe that the support function \(p_S(t)\) measures the signed distance from the origin \((0, 0)\) to the tangent line \(l_S(t)\) of \(S\) with the normal vector \(u_t\) directing outwards from \(S\). Support function and tangent lines of \(S\) are usually studied when \(S\) is a convex body (e.g.~p45 of \cite{schneider_2013}), but in this paper we generalize the notion to arbitrary shape \(S\).

The following notion of \emph{width} along a direction is also studied for convex bodies (e.g.~p49 of \cite{schneider_2013}).

\begin{definition}

For any shape \(S\) and angle \(t\) in \(S^1\) or \(\mathbb{R}\), the \emph{width} of \(S\) along the direction of unit vector \(u_t\) is defined as \(p_S(t) + p_S(t + \pi)\).

\label{def:width}
\end{definition}

Geometrically, the width of \(S\) along \(u_t\) measures the distance between the parallel tangent lines \(l_S(t)\) and \(l_S(t + \pi)\) of \(S\).

We adopt the following definition of a convex body (p8 of \cite{schneider_2013}).

\begin{definition}

A \emph{convex body} \(K\) is a nonempty, compact, and convex subset of \(\mathbb{R}^2\).

\label{def:convex-body}
\end{definition}

Many authors often also include the condition that \(K^\circ\) is nonempty, but we allow \(K^\circ\) to be empty (that is, \(K\) can be a closed line segment or a point).

In this paper only, we use the following notions of \emph{vertices} and \emph{edges} of a planar convex body \(K\).

\begin{definition}

For any convex body \(K\) and \(t \in S^1\), define the \emph{edge} \(e_K(t)\) of \(K\) as the intersection of \(K\) with the tangent line \(l_K(t)\).

\label{def:convex-body-edge}
\end{definition}

\begin{definition}

For any convex body \(K\) and \(t \in S^1\), let \(v_K^+(t)\) and \(v_K^-(t)\) be the endpoints of the edge \(e_K(t)\) such that \(v_K^+(t)\) is positioned farthest in the direction of \(v_t\) and \(v_K^-(t)\) is positioned farthest in the opposite direction of \(v_t\). We call \(v_K^{\pm}(t)\) the \emph{vertices} of \(K\).

\label{def:convex-body-vertex}
\end{definition}

It is possible that the edge \(e_K(t)\) can be a single point. In such case, the tangent line \(l_K(t)\) touches \(K\) at the single point \(v_K^+(t) = v_K^-(t)\). In fact, it turns out that this holds for every \(t \in S^1\) except for a countable number of values of \(t\) (\Cref{pro:surface-area-singleton-almost-everywhere}).

\begin{figure}
\centering
\includesvg[width=0.5\textwidth,height=\textheight]{images/convex-body.svg}
\caption{A convex body \(K\) with its edge, vertices, tangent line, and half-plane.}
\label{fig:convex-body}
\end{figure}

For reference, the notion of cap \(K\) as a kind of convex bodies is defined in \Cref{def:cap}. The space of all caps \(\mathcal{K}_\omega\) with rotation angle \(\omega\) is defined in \Cref{def:cap-space}. The vertices \(A_K^{\pm}(t)\), \(C_K^{\pm}(t)\) of a cap \(K\) is defined in \Cref{def:cap-vertices}. The upper boundary \(\delta K\) of a cap \(K\) is defined in \Cref{def:upper-boundary-of-cap}.

\section{Area Functionals}
\label{sec:area-functionals}
\input{out/10. Notations and conventions/04. Area functionals.tex}

\section{Parts of Hallway}
\label{sec:parts-of-hallway}
\input{out/10. Notations and conventions/05. Parts of hallway.tex}



\chapter{Monotone sofas}
\label{sec:monotone-sofas}
In this section, we rigorously define what is a \emph{monotone sofa}. \Cref{thm:monotonization-is-sofa} shows that the process named \emph{monotonization} enlarges any moving sofa \(S\) to a larger moving sofa \(\mathcal{M}(S)\) as described in \Cref{eqn:monotone} of \Cref{sec:moving-hallway-problem}. A monotone sofa is then simply defined as the monotonization \(\mathcal{M}(S)\) of some moving sofa \(S\) (\Cref{def:monotone-sofa}). Proving the connectedness of \(\mathcal{M}(S)\) (\Cref{thm:monotonization-is-connected}) will be the key step in establishing \Cref{thm:monotonization-is-sofa}.

\Cref{thm:monotonization-structure} shows that any monotone sofa \(S\) is equal to \(K \setminus \mathcal{N}(K)\), where \(K = \mathcal{C}(S)\) is a convex set called the \emph{cap of} \(S\) (\Cref{thm:cap-hallway-intersection}), and \(\mathcal{N}(K)\) is a subset of \(K\) called the \emph{niche} determined by the cap \(K\) (\Cref{def:niche}). Then we show that the niche \(\mathcal{N}(K)\) is always contained in the cap \(K\) of sofa \(S\) (\Cref{thm:niche-in-cap}). With this, the area \(|S| = |K| - |\mathcal{N}(K)|\) of a monotone sofa \(S\) can be understood in terms of cap \(K\) and niche \(\mathcal{N}(K)\) separately.
\subsection{Tangent Hallway}
\label{sec:tangent-hallway}
\subsection{Tangent hallway}

Define the \emph{tangent hallways} for a shape \(S\) (that is, any nonempty compact subset \(S\) of \(\mathbb{R}^2\) by \Cref{def:shape}).

\begin{definition}

For any shape \(S\) and angle \(t \in S^1\), define the \emph{tangent hallway} \(L_S(t)\) of \(S\) with angle \(t\) as
\[
L_S(t) = R_t(L) + (p_S(t) - 1)  u_t + (p_S(t + \pi/2) - 1) v_t.
\]

\label{def:tangent-hallway}
\end{definition}

Note that \(R_t\) is the rotation of \(\mathbb{R}^2\) along the origin by a counterclockwise angle of \(t\) (\Cref{def:rotation-map}). The equation of \(L_S(t)\) in \Cref{def:tangent-hallway} is determined by the following defining property of \(L_S(t)\).

\begin{definition}

Here, a \emph{rigid transformation} \(f : \mathbb{R}^2 \to \mathbb{R}^2\) on \(\mathbb{R}^2\) is the composition \(z \mapsto R_t(z) + q\) of translation by a vector \(q \in \mathbb{R}^2\) and rotation by an angle \(t \in S^1\) along the origin. We also say that a shape \(S'\) is a rigid transformation \emph{of} another shape \(S\) if there exists a rigid transformation \(f : \mathbb{R}^2 \to \mathbb{R}^2\) such that \(S' = f(S)\).

\label{def:rigid-transformation}
\end{definition}

\begin{proposition}

For any shape \(S\) and angle \(t \in S^1\), the tangent hallway \(L_S(t)\) is the unique rigid transformation of \(L\) rotated counterclockwise by \(t\), such that the outer walls of \(L_S(t)\) corresponding to the outer walls \(a\) and \(c\) of \(L\) are the tangent lines \(l_S(t)\) and \(l_S(t + \pi/2)\) of \(S\) respectively.

\label{pro:tangent-hallway}
\end{proposition}

\begin{proof}
Let \(c_1\) and \(c_2\) be arbitrary real values. Then \(L' = R_t(L) + c_1 u_t + c_2 v_t\) is an arbitrary rigid transformation of \(L\) rotated counterclockwise by \(t\). The outer walls of \(L'\) corresponding to the outer walls \(a\) and \(c\) of \(L\) (\Cref{def:hallway-walls}) are \(l(t, c_1 + 1)\) and \(l(t + \pi/2, c_2 + 1)\) respectively. They match with the tangent lines \(l_S(t) = l(t, p_S(t))\) and \(l_S(t + \pi/2) = l(t + \pi/2, p_S(t + \pi/2))\) of \(S\) if and only if \(c_1 = p_S(t) - 1\) and \(c_2 = p_S(t + \pi/2) - 1\). That is, if and only if \(L' = L_S(t)\).
\end{proof}

Name the parts of tangent hallway \(L_S(t)\) according to the parts of \(L\) (\Cref{def:hallway-corners}, \Cref{def:hallway-walls}, and \Cref{def:hallway-regions}) for future use.

\begin{definition}

For any shape \(S\) and angle \(t \in S^1\), define the rigid transformation \(f_{S, t} : \mathbb{R}^2 \to \mathbb{R}^2\) as
\[
f_{S, t}(z) = R_t(z) + (p_S(t) - 1)  u_t + (p_S(t + \pi/2) - 1) v_t
\]
so that \(f_{S, t}\) maps \(L\) to \(L_S(t)\).

\label{def:tangent-hallway-map}
\end{definition}

\begin{definition}

For any shape \(S\) and angle \(t \in S^1\), let \(\mathbf{x}_S(t), \mathbf{y}_S(t), a_S(t), b_S(t), c_S(t), d_S(t), Q^+_S(t), Q^-_S(t)\) be the parts of \(L_S(t)\) corresponding to the parts \(\mathbf{x}, \mathbf{y}, a, b, c, d, Q^+, Q^-\) of \(L\) respectively. That is, for any \(? = \mathbf{x}, \mathbf{y}, a, b, c, d, Q^+, Q^-\), let \(?_S(t) := f_{S, t}(?)\).

\label{def:rotating-hallway-parts}
\end{definition}

\begin{proposition}

We have \(L_S(t) = Q_S^+(t) \setminus Q_S^-(t)\) and \(Q^+_S(t) = H_S(t) \cap H_S(t + \pi/2)\). Also we have the following representations of the parts purely in terms of the supporting function \(p_S\) of \(S\).

\begin{gather*}
\mathbf{x}_S(t) = (p_S(t) - 1) u_t + (p_S(t + \pi/2) - 1) v_t \\
\mathbf{y}_S(t) = p_S(t) u_t + p_S(t + \pi/2) v_t \\
a_S(t) = l_S(t) = l(t, p_S(t)) \\
b_S(t) \subseteq l(t, p_S(t) - 1) \\
c_S(t) = l_S(t + \pi/2) = l(t + \pi/2, p_S(t + \pi/2)) \\
d_S(t) \subseteq l(t + \pi/2, p_S(t + \pi/2) - 1) \\
Q_S^+(t) = H(t, p_S(t)) \cap H(t + \pi/2, p_S(t + \pi/2)) \\
Q_S^-(t) = H(t, p_S(t) - 1)^{\circ} \cap H(t + \pi/2, p_S(t + \pi/2))^{\circ}
\end{gather*}

\label{pro:rotating-hallway-parts}
\end{proposition}

\begin{proof}
The formulas for \(\mathbf{x}_S(t)\) and \(\mathbf{y}_S(t)\) are obtained by letting \(z\) equal to \(\mathbf{x} = (0, 0)\) or \(\mathbf{y} = (1, 1)\) in the equation of \Cref{def:tangent-hallway-map}. The formulas for \(a_S(t), b_S(t), c_S(t)\), and \(d_S(t)\) follows from the proof of \Cref{pro:tangent-hallway}. The equality \(L_S(t) = Q_S^+(t) \setminus Q_S^-(t)\) follows from mapping \(L = Q^+ \setminus Q^-\) under the transformation \(f_{S, t}\). The equality \(Q^+_S(t) = H_S(t) \cap H_S(t + \pi/2)\) follows from that \(Q^+_S(t)\) is a cone bounded by tangent lines \(a_S(t) = l_S(t)\) and \(c_S(t) = l_S(t + \pi/2)\) as in the proof of \Cref{pro:tangent-hallway}. The formulas for \(Q_S^-(t)\) and \(Q_S^+(t)\) in terms of \(p_S\) now follow from \Cref{def:tangent-half-plane} and that \(Q_S^-(t)\) is bounded by \(b_S(t)\) and \(d_S(t)\).
\end{proof}

Assume that a rigid transformation \(L'\) of \(L\) rotated counterclockwise by an angle of \(t \in S^1\) contains a shape \(S\). By translating the outer walls of \(L'\) towards \(S\) until they make contact with \(S\), we can see that the tangent hallway \(L_S(t)\) also contains \(S\).

\begin{proposition}

Let \(S\) be any shape contained in a translation of \(R_t(L)\) with angle \(t \in S^1\). Then the tangent hallway \(L_S(t)\) with angle \(t\) also contains \(S\).

\label{pro:tangent-hallway-contains}
\end{proposition}

\begin{proof}
Assume that the translation \(L'\) of \(R_t(L)\) contains \(S\). Then while keeping \(S\) inside \(L'\), we can push \(L'\) towards \(S\) in the directions \(-u_t\) and \(-v_t\) until the outer walls of the final \(L' = L_S(t)\) make contact with \(S\). The pushed hallway \(L_S(t)\) still contains \(S\) because the directions \(-u_t\) and \(-v_t\) of the movement only push the inner walls of \(L'\) away from \(S\).
\end{proof}

\subsection{Moving Hallway Problem}

By our \Cref{def:sofa} of a moving sofa \(S\), any translation of \(S\) is also a valid moving sofa. Without loss of generality, we will always assume that a moving sofa \(S\) is in \emph{standard position} by translating it.

\begin{definition}

A moving sofa \(S\) with rotation angle \(\omega \in (0, \pi/2]\) is in \emph{standard position} if \(p_S(\omega) = p_S(\pi/2) = 1\).

\label{def:standard-position}
\end{definition}

\begin{proposition}

For any angle \(\omega \in (0, \pi/2]\) and shape \(S\), there is a translation \(S'\) of \(S\) such that \(p_{S'}(\omega) = p_{S'}(\pi/2) = 1\) which is (i) unique if \(\omega < \pi/2\), or (ii) unique up to horizontal translations if \(\omega = \pi/2\).

\label{pro:standard-position-shape}
\end{proposition}

\begin{proof}
Since the support function \(p_{S'}(t)\) measures the signed distance from origin to tangent line \(l_{S'}(t)\) (see the remark above \Cref{def:support-function}), the translation \(S'\) of \(S\) satisfies the condition \(p_{S'}(\omega) = p_{S'}(\pi/2) = 1\) if and only if the lines \(l(\omega, 1)\) and \(l(\pi/2, 1)\) are tangent to \(S'\) and \(S'\) is below the lines. Translate \(S\) below the lines \(l(\omega, 1)\) and \(l(\pi/2, 1)\) so that it makes contact with the two lines. If \(\omega < \pi/2\), then the constraints determine the unique location of \(S'\). If \(\omega = \pi/2\), then the two lines are equal to the horizontal line \(y=1\), and \(S'\) can move freely horizontally as long as the line \(y=1\) makes contact with \(S'\) from above.
\end{proof}

Assume any moving sofa \(S\) with rotation angle \(\omega \in (0, \pi/2]\). By \Cref{pro:standard-position-shape} any moving sofa can be put in standard position by translating it. Gerver also observed in \autocite{gerverMovingSofaCorner1992} that \(S\) should be contained in the tangent hallways \(L_S(t)\) for all \(t \in [0, \omega]\) (\Cref{pro:tangent-hallway-contains}). We summarize the full details of Gerver’s observation (line 18-22, p269; line 24-31, p270 of \autocite{gerverMovingSofaCorner1992}) in the following theorem.

\begin{theorem}

Let \(\omega \in (0, \pi/2]\) be an arbitrary angle. For a connected shape \(S\), the following conditions are equivalent.

\begin{enumerate}
\def\labelenumi{\arabic{enumi}.}
\tightlist
\item
  \(S\) is a moving sofa with rotation angle \(\omega\).
\item
  \(S\) is contained in a translation of \(H\) and \(R_\omega(V)\). Also, for every \(t \in [0, \omega]\), \(S\) is contained in a translation of \(R_t(L)\), the hallway rotated counterclockwise by an angle of \(t\).
\item
  Let \(S'\) be any translation of \(S\) such that \(p_{S'}(\omega) = p_{S'}(\pi/2) = 1\). Then (i) \(S' \subseteq P_\omega\) (\Cref{def:parallelogram}), (ii) \(S' \subseteq L_{S'}(t)\) for all \(t \in [0, \omega]\), and (iii) \(S'\) is a moving sofa with rotation angle \(\omega\) in standard position.
\end{enumerate}

\label{thm:moving-sofa-iff-hallway-intersection}
\end{theorem}

\begin{proof}
(1 \(\Rightarrow\) 2) Consider the movement of \(S\) inside the hallway \(L\). For any angle \(t \in [0, \omega]\), there is a moment where the sofa \(S\) is rotated clockwise by an angle of \(t\) inside \(L\), by the intermediate value theorem on the angle of rotation of \(S\) inside \(L\). Viewing this from the perspective of the sofa \(S\), \(S\) is contained in some translation of \(L\) rotated \emph{counterclockwise} by an arbitrary \(t \in [0, \omega]\). Likewise, by looking at the initial (resp. final) position of \(S\) inside \(L_H\) (resp. \(L_V\)) from the perspective of \(S\), the set \(S\) should be contained in a translation of \(H\) and \(R_\omega(V)\) respectively.

(2 \(\Rightarrow\) 3) Take any \(S\) satisfying (2) and its arbitrary translation \(S'\) satisfying \(p_{S'}(\omega) = p_{S'}(\pi/2) = 1\) which is the premise of (3). Then the translate \(S'\) of \(S\) also satisfies (2). So without loss of generality, we can simply assume \(S' = S\) and show (i), (ii) and (iii). Since \(S\) is contained in a translation of \(H\) and \(R_\omega(V)\), the width of \(S\) along the direction of \(u_\omega\) and \(v_0\) (\Cref{def:width}) are at most 1. So \(p_S(\omega) = p_S(\pi/2) = 1\) implies (i) \(S \subseteq P_\omega\). \Cref{pro:tangent-hallway-contains} implies (ii) \(S \subseteq L_S(t)\). It remains to show (iii) that \(S\) is a moving sofa.

Because the support function \(p_S(t)\) of \(S\) is continuous, the tangent hallway \(L_S(t)\) moves continuously with respect to \(t\) by \Cref{def:tangent-hallway}. For every \(t \in [0, \omega]\), let \(g_t := f_{S, t}^{-1}\) be the unique rigid transformation that maps \(L_S(t)\) to \(L\). Then the rigid transformation \(S_t := g_t(S)\) of \(S\) also changes continuously with respect to \(t\). Because \(L_S(0)\) is a translation of \(L\) by letting \(t=0\) in \Cref{def:tangent-hallway}, \(g_0\) is a translation and so \(S_0\) is a translation of \(S\). Mapping \(S \subseteq L_S(t)\) under \(g_t\) we have \(S_t \subseteq L\). So \(S_t\) over the angle \(t \in [0, \omega]\) as time is a continuous movement of a translation \(S_0\) of \(S\) inside \(L\).

It remains to show that \(S_0 \subseteq H\) and \(S_\omega \subseteq V\). Because \(p_S(\pi/2) = 1\), \(L_S(0)\) is a translation of \(L\) along the direction \(u_0\), and the map \(g_0\) is also a translation along the direction \(u_0\). Because \(S \subseteq H\), we also have \(S_0 = g_0(S) \subseteq H\). Likewise, since \(p_S(\omega) = 1\) the hallway \(L_S(\omega)\) is a translation of \(L\) along the direction \(v_\omega\). So the map \(g_\omega\) is the composition of a translation along the direction \(v_\omega\) and \(R_{-\omega}\). Because \(S \subseteq R_\omega(V)\), we also have \(S_\omega = g_\omega(S) \subseteq g_\omega(R_\omega(V)) = V\).

(3 \(\Rightarrow\) 1) By \Cref{pro:standard-position-shape}, any connected shape \(S\) have a translation \(S'\) that satisfies the premise \(p_{S'}(\omega) = p_{S'}(\pi/2) = 1\) of (3). So \(S'\) is a moving sofa by (3), and its translation \(S\) is a moving sofa as well.
\end{proof}

\subsection{Monotonization}
\label{sec:monotonization}
We now define the notion of monotone sofas and establish \Cref{thm:monotonization-is-sofa} that a moving sofa is contained in a larger monotone sofa. Define the \emph{monotonization} of any moving sofa \(S\) in standard position as the following set.

\begin{definition}

Let \(S\) be any moving sofa with rotation angle \(\omega \in (0, \pi/2]\) in standard position. The \emph{monotonization} of \(S\) is the intersection
\[
\mathcal{M}(S) = P_\omega \cap \bigcap_{0 \leq t \leq \omega} L_S(t).
\]

\label{def:monotonization}
\end{definition}

Compare the equation in \Cref{def:monotonization} to \Cref{eqn:monotone} in \Cref{sec:moving-hallway-problem}. The paralleogram \(P_\omega\) is the intersection of \(H\) and \(V_\omega\) (\Cref{def:parallelogram}), and the tangent hallways \(L_S(t)\) are the rotating hallways \(L_t\) making contact with \(S\) in the outer walls as described in \Cref{sec:moving-hallway-problem}. Condition 3 of \Cref{thm:moving-sofa-iff-hallway-intersection} implies that the set \(\mathcal{M}(S)\) contains \(S\).

\begin{corollary}

\(\mathcal{M}(S) \supseteq S\) for any moving sofa \(S\) in standard position.

\label{cor:monotonization-is-larger}
\end{corollary}

We will establish the connectedness of \(\mathcal{M}(S)\).

\begin{theorem}

Let \(S\) be a moving sofa with rotation angle \(\omega \in (0, \pi/2]\) in standard position. Then the monotonization \(\mathcal{M}(S)\) is connected.

\label{thm:monotonization-is-connected}
\end{theorem}

Once the connectedness of \(\mathcal{M}(S)\) is established, we can immediately show that the monotonization \(\mathcal{M}(S)\) is a moving sofa containing the initial moving sofa \(S\).

\begin{theorem}

Let \(S\) be any moving sofa with rotation angle \(\omega \in (0, \pi/2]\) in standard position. The monotonization \(\mathcal{M}(S)\) of \(S\) is a moving sofa containing \(S\) with the same rotation angle \(\omega\) in standard position.

\label{thm:monotonization-is-sofa}
\end{theorem}

\begin{proof}
By \Cref{thm:monotonization-is-connected}, the shape \(\mathcal{M}(S)\) is connected. By \Cref{def:monotonization}, the set \(\mathcal{M}(S)\) is contained \(P_\omega\) and \(L_S(t)\) for all \(t \in [0, \omega]\), so it satisfies the second condition of \Cref{thm:moving-sofa-iff-hallway-intersection}. So the set \(\mathcal{M}(S)\) is a moving sofa with rotation angle \(\omega\). \(\mathcal{M}(S)\) contains \(S\) by \Cref{cor:monotonization-is-larger}. From \(S \subseteq \mathcal{M}(S) \subseteq P_\omega\) and
\[
p_S(\omega) = p_{P_\omega}(\omega) = p_S(\pi/2) = p_{P_\omega}(\pi/2) = 1
\]
we have \(p_{\mathcal{M}(S)}(\omega) = p_{\mathcal{M}(S)}(\pi/2) = 1\). So \(\mathcal{M}(S)\) is in standard position.
\end{proof}

Now any monotonization of a moving sofa is also a moving sofa. Call the resulting monotonization a \emph{monotone sofa}.

\begin{definition}

A \emph{monotone sofa} is the monotonization of some moving sofa with rotation angle \(\omega \in (0, \pi/2]\) in standard position.

\label{def:monotone-sofa}
\end{definition}

The moving sofa problem asks for the largest-area moving sofas. So \Cref{thm:monotonization-is-sofa} tells us that we only need to consider monotone sofas for the problem. The rest of this \Cref{sec:monotonization} proves \Cref{thm:monotonization-is-connected} as promised.

\subsection{\texorpdfstring{Proof of \Cref{thm:monotonization-is-connected}}{Proof of }}

We prepare the following terminologies.

\begin{definition}

Let \(S\) be any moving sofa with rotation angle \(\omega \in (0, \pi/2]\) in standard position. Define the set
\[
\mathcal{C}(S) = P_\omega \cap \bigcap_{0 \leq t \leq \omega} Q^+_S(t).
\]

\label{def:cap-sofa}
\end{definition}

The set \(\mathcal{C}(S)\) will later be called as the \emph{cap} of \(S\) (\Cref{thm:cap-hallway-intersection}) after defining the notion of cap in \Cref{def:cap}. We don’t need this notion of cap as of now.

\begin{definition}

Say that a set \(X \subseteq \mathbb{R}^2\) is \emph{closed in the direction of} vector \(v \in \mathbb{R}^2\) if, for any \(x \in X\) and \(\lambda \geq 0\), we have \(x + \lambda v \in X\).

\label{def:closed-in-direction}
\end{definition}

\begin{definition}

Any line \(l\) of \(\mathbb{R}^2\) divides the plane into two half-planes. If \(l\) is not parallel to the \(y\)-axis, call the \emph{left side} (resp. \emph{right side}) of \(l\) as the closed half-plane with boundary \(l\) containing the point \(- Nu_0\) (resp. \(Nu_0\)) for a sufficiently large \(N\). If a point \(p\) is on the left (resp. right) side of \(l\) and not on the boundary \(l\), we say that \(p\) is \emph{strictly on the left} (resp. \emph{right}) \emph{side} of \(l\).

\label{def:line-half-plane-directions}
\end{definition}

We also prepare a lemma.

\begin{lemma}

Let \(S\) be any moving sofa with rotation angle \(\omega \in [0, \pi/2]\) in standard position. Then the support functions \(p_S\), \(p_{\mathcal{M}(S)}\), and \(p_{\mathcal{C}(S)}\) of \(S\), \(\mathcal{M}(S)\) and \(\mathcal{C}(S)\) agree on the set \(J_\omega\).

\label{lem:cap-same-support-function}
\end{lemma}

\begin{proof}
We have \(S \subseteq \mathcal{M}(S) \subseteq \mathcal{C}(S)\) by \Cref{cor:monotonization-is-larger} and \(L_S(t) \subset Q_S^+(t)\). So it remains to show \(p_{\mathcal{C}(S)}(t) \leq p_S(t)\) for every \(t\) in \(J_\omega\). By the definition of \(\mathcal{C}(S)\) we have \(S \subseteq \mathcal{C}(S) \subseteq H_S(t)\). So we have \(p_{\mathcal{C}(S)}(t) \leq p_S(t)\) indeed.
\end{proof}

A moving sofa \(S\) and its cap \(\mathcal{C}(S)\) shares the same tangent hallways \(L_S(t) = L_{\mathcal{C}(S)}(t)\).

\begin{proposition}

For any moving sofa \(S\) with rotation angle \(\omega \in [0, \pi/2]\) in standard position, the tangent hallway \(L_S(t)\) of \(S\) and the tangent hallway \(L_K(t)\) of set \(K = \mathcal{C}(S)\) are equal for every \(t \in [0, \omega]\).

\label{pro:cap-same-tangent-hallway}
\end{proposition}

\begin{proof}
The tangent hallways \(L_X(t)\) of \(X = S, K\) depend solely on the values of the support function \(p_X\) of \(X\) on \(J_\omega\), by the equation of \(L_X(t)\) in \Cref{def:tangent-hallway}. The support functions of \(S\) and \(K\) match on the set \(J_\omega\) by \Cref{lem:cap-same-support-function}, so the result follows.
\end{proof}

We are now ready to show that \(\mathcal{M}(S)\) is connected.

\begin{proof}[Proof of \Cref{thm:monotonization-is-connected}]
Define the set \(X := \bigcup_{0 \leq t \leq \omega} Q^-_S(t)\). By plugging the equation \(L_S(t) = Q_S^+(t) \setminus Q_S^-(t)\) to \Cref{def:monotonization}, we have \(\mathcal{M}(S) = \mathcal{C}(S) \setminus X\). Observe that \(\mathcal{C}(S)\) is a convex body containing \(S\) (say, by \Cref{cor:monotonization-is-larger} and \(\mathcal{M}(S) \subseteq \mathcal{C}(S)\)).

Fix an arbitrary point \(p\) in \(\mathcal{M}(S)\). Take an arbitrary angle \(\theta \in [\omega, \pi/2]\). Observe that the set \(X = \bigcup_{t \in [0, \omega]} Q^-_S(t)\) is closed in the direction of \(-u_\theta\) (\Cref{def:closed-in-direction}) for all angle \(\theta \in [\omega, \pi/2]\), since each \(Q_S^-(t)\) is closed in the direction of \(-u_\theta\). Take the line \(l_\theta\) passing the point \(p\) in the direction of \(u_\theta\). The set \(s_\theta = l_\theta \cap \mathcal{M}(S)\) contains \(p\), and \(s_\theta\) is a nonempty line segment because \(s_\theta\) is the line segment \(l_\theta \cap \mathcal{C}(S)\) minus the half-line \(l_\theta \setminus X\). If the line \(l_\theta\) meets \(S\) for any \(\theta \in [\omega, \pi/2]\), then \(p\) is connected to \(S\) along the line segment \(s_\theta\) inside \(\mathcal{M}(S)\) and the proof is done. Our goal now is to prove that there is some \(\theta \in [\omega, \pi/2]\) such that \(l_\theta\) meets \(S\).

Assume by contradiction that for every \(\theta \in [\omega, \pi/2]\) the line \(l_\theta\) is disjoint from \(S\). By \Cref{lem:cap-same-support-function}, we have \(l_{\mathcal{M}(S)}(t) = l_S(t)\) for every \(t \in J_\omega = [0, \omega] \cup [\pi/2, \pi/2 + \omega]\). Because \(p \in \mathcal{M}(S)\), the line \(l_{\pi/2}\) passing through \(p\) is either equal to \(l_{\mathcal{M}(S)}(0) = l_S(0)\) or strictly on the left side of \(l_{S}(0)\). If \(l_{\pi/2} = l_S(0)\) then \(l_{\pi/2}\) contains some point of \(S\) contradicting our assumption. So the line \(l_{\pi/2}\) is strictly on the left side of \(l_{S}(0)\), and there is a point of \(S\) strictly on the right side of \(l_{\pi/2}\). Likewise, as \(p \in \mathcal{M}(S)\), the line \(l_{\omega}\) that passes through \(p\) is either equal to \(l_{\mathcal{M}(S)}(\omega + \pi/2) = l_S(\omega + \pi/2)\) or strictly on the right side of \(l_S(\omega + \pi/2)\). The line \(l_\omega\) cannot be equal to \(l_S(\omega + \pi/2)\) because we assumed that \(l_\omega\) is disjoint from \(S\). So the line \(l_{\omega}\) is strictly on the right side of \(l_S(\omega + \pi/2)\), and there is a point of \(S\) strictly on the left side of \(l_{\omega}\).

Because the line \(l_\theta\) is disjoint from \(S\) for any \(\theta \in [\omega, \pi/2]\), the set \(S\) is inside the set \(Y = \mathbb{R}^2 \setminus \bigcup_{\theta \in [\omega, \pi/2]} l_\theta\). Note that \(Y\) has exactly two connected components \(Y_L\) and \(Y_R\) on the left and the right side of the lines \(l_\theta\) respectively. As there is a point of \(S\) strictly on the right side of \(l_{\pi/2}\), the set \(S \cap Y_R\) is nonempty. As there is also a point of \(S\) strictly on the left side of \(l_\omega\), the set \(S \cap Y_L\) is also nonempty. We get contradiction as \(S\) should be a connected subset of \(Y\).
\end{proof}

\subsection{Structure of a monotone sofa}
\label{sec:structure-of-a-monotone-sofa}
Here, we show that any monotone sofa \(S\) is always equal to a \emph{cap} \(K\) minus its \emph{niche} \(\mathcal{N}(K)\) (\Cref{thm:monotone-sofa-structure}; see \Cref{fig:monotone-sofa} in \Cref{sec:moving-hallway-problem}).

Define a \emph{cap} as a convex body satisfying certain properties.

\begin{definition}

A \emph{cap} \(K\) with \emph{rotation angle} \(\omega \in (0, \pi/2]\) is a convex body such that the followings hold.

\begin{enumerate}
\def\labelenumi{\arabic{enumi}.}
\tightlist
\item
  \(p_K(\omega) = p_K(\pi/2) = 1\) and \(p_K(\pi + \omega) = p_K(3\pi/2) = 0\).
\item
  \(K\) is an intersection of closed half-planes with normal angles (\Cref{def:half-plane}) in \(J_\omega \cup \{\pi + \omega, 3\pi/2\}\).
\end{enumerate}

\label{def:cap}
\end{definition}

Geometrically, the first condition of \Cref{def:cap} states that \(K\) is contained in the parallelogram \(P_\omega\) making contact with all sides of \(P_\omega\). By \Cref{thm:convex-set-support}, the second condition of \Cref{def:cap} is equivalent to saying that the \emph{normal angles} \(\mathbf{n}(K)\) of \(K\) (\Cref{def:convex-set-support}) is contained in the set \(J_\omega \cup \{\pi + \omega, 3\pi/2\}\). See \Cref{sec:normal-angles} for a quick overview of \(\mathbf{n}(K)\).

We will show that the set \(\mathcal{C}(S)\) in \Cref{def:cap-sofa} is a cap with rotation angle \(\omega\). This justifies calling \(\mathcal{C}(S)\) \emph{the cap of} \(S\) associated to \(S\).

\begin{theorem}

The set \(\mathcal{C}(S)\) in \Cref{def:cap-sofa} is a cap with rotation angle \(\omega\) as in \Cref{def:cap}. With this, call \(\mathcal{C}(S)\) the \emph{cap of the moving sofa} \(S\).

\label{thm:cap-hallway-intersection}
\end{theorem}

We postpone the proof of \Cref{thm:cap-hallway-intersection} at the end of this \Cref{sec:structure-of-a-monotone-sofa}. Define the \emph{niche} \(\mathcal{N}(K)\) associated to any cap \(K\).

\begin{definition}

For any angle \(\omega \in [0, \pi/2]\), define the \emph{fan} \(F_\omega = H(\pi+\omega, 0) \cap H(3\pi/2, 0)\) with angle \(\omega\) as the convex cone pointed at the origin, bounded from below by the bottom edges \(l(\omega, 0)\) and \(l(3\pi/2, 0)\) of the parallelogram \(P_\omega\).

\label{def:fan}
\end{definition}

\begin{definition}

Let \(K\) be any cap with rotation angle \(\omega \in [0, \pi/2]\). Define the \emph{niche} of \(K\) as
\[
\mathcal{N}(K) := F_{\omega} \cap \bigcup_{0 \leq t \leq \omega} Q^-_K(t).
\]

\label{def:niche}
\end{definition}

Now we establish the structure of any monotonization of a sofa.

\begin{theorem}

Let \(S\) be a moving sofa with rotation angle \(\omega \in (0, \pi/2]\) in standard position. The monotonization \(\mathcal{M}(S)\) of \(S\) is equal to \(K \setminus \mathcal{N}(K)\), where \(K = \mathcal{C}(S)\) is the cap of sofa \(S\) and \(\mathcal{N}(K)\) is the niche of the cap \(K\).

\label{thm:monotonization-structure}
\end{theorem}

\begin{proof}
Let \(K = \mathcal{C}(S)\) be the cap of \(S\). By breaking down each \(L_S(t)\) into \(Q_S^+(t) \setminus Q_S^-(t)\), the monotonization \(\mathcal{M}(S)\) of \(S\) can be represented as the following subtraction of two sets.
\begin{equation}
\label{eqn:monotonization}
\begin{split}
\mathcal{M}(S) & = P_\omega \cap \bigcap_{0 \leq t \leq \omega} L_S(t) \\
& = \left( P_\omega \cap \bigcap_{0 \leq t \leq \omega} Q^+_S(t) \right) \setminus \left( F_\omega \cap \bigcup_{0 \leq t \leq \omega} Q^-_S(t) \right)
\end{split}
\end{equation}
By \Cref{pro:cap-same-tangent-hallway} we have \(Q_S^-(t) = Q_K^-(t)\). So we have \(\mathcal{M}(S) = K \setminus \mathcal{N}(K)\) by the definitions of \(K\) and \(\mathcal{N}(K)\).
\end{proof}

\begin{remark}

\Cref{eqn:monotonization} can understood intuitively as the following (see \Cref{fig:monotone-sofa}). The cap \(K\) is a convex body bounded from below by the edges of fan \(F_\omega\), and from above by the outer walls \(a_S(t)\) and \(c_S(t)\) of \(L_S(t)\). Imagine the set \(K\) as a block of clay that rotates inside the hallway \(L\) in the clockwise angle of \(t \in [0, \omega]\) while always touching the outer walls \(a\) and \(c\) of \(L\). As \(K\) rotates inside \(L\), the inner corner of \(L\) carves out the niche \(\mathcal{N}(K)\) which is the regions bounded by inner walls \(b_S(t)\) and \(d_S(t)\) of \(L_S(t)\) from \(K\). After the full movement of \(K\), the final clay \(K \setminus \mathcal{N}(K)\) is a moving sofa \(\mathcal{M}(S)\).

\label{rem:cap-niche-intuition}
\end{remark}

A moving sofa \(S\) and its monotonization \(\mathcal{M}(S)\) shares the same cap.

\begin{proposition}

For any moving sofa \(S\) with rotation angle \(\omega \in [0, \pi/2]\) in standard position, we have \(\mathcal{C}(\mathcal{M}(S)) = \mathcal{C}(S)\).

\label{pro:monotonization-cap}
\end{proposition}

\begin{proof}
By \Cref{def:cap-sofa} and \Cref{pro:rotating-hallway-parts}, the set \(\mathcal{C}(X)\) of \(X = S\) or \(\mathcal{M}(S)\) depend only on the values of the support function \(p_X\) on \(J_\omega\). The support functions of \(S\) and \(\mathcal{M}(S)\) match on \(J_\omega\) by \Cref{lem:cap-same-support-function}, completing the proof.
\end{proof}

We will use the following intrinsic variant of \Cref{thm:monotonization-structure} to represent any monotone sofa \(S\) as its cap minus niche.

\begin{theorem}

Let \(S\) be any monotone sofa with rotation angle \(\omega \in (0, \pi/2]\). Then \(S\) is in standard position and \(S = K \setminus \mathcal{N}(K)\), where \(K := \mathcal{C}(S)\) is the cap of \(S\) with rotation angle \(\omega\), and \(\mathcal{N}(K)\) is the niche of the cap \(K\).

\label{thm:monotone-sofa-structure}
\end{theorem}

\begin{proof}
Let \(S = \mathcal{M}(S')\) be any monotone sofa, so that it is the monotonization of a moving sofa \(S'\) in standard position. Then \(K := \mathcal{C}(S) = \mathcal{C}(S')\) by \Cref{pro:monotonization-cap}. Now apply \Cref{thm:monotonization-structure} to \(\mathcal{M}(S')\) to conclude that \(S = \mathcal{M}(S') = \mathcal{C}(S') \setminus \mathcal{N}(\mathcal{C}(S')) = K \setminus \mathcal{N}(K)\).
\end{proof}

Note that this variant does not mention anything about monotonization. In particular, by \Cref{thm:monotone-sofa-structure} any monotone sofa \(S\) can be recovered from its cap \(K = \mathcal{C}(S)\).

Monotization \(S \mapsto \mathcal{M}(S)\) is a process that enlarges any moving sofa \(S\) by \Cref{thm:monotonization-is-sofa}. Moreover, if \(S\) is already monotone (so that \(S = \mathcal{M}(S')\) for some \(S'\)), then the monotonization fixes \(S\).

\begin{theorem}

For any monotone sofa \(S\), we have \(\mathcal{M}(S) = S\).

\label{thm:monotonization-fixpoint}
\end{theorem}

\begin{proof}
Since \(S\) is monotone, \(S = \mathcal{M}(S')\) for some other moving sofa \(S'\). Now check
\[
\mathcal{M}(S) = \mathcal{C}(S) \setminus \mathcal{N}(\mathcal{C}(S)) = \mathcal{C}(S') \setminus \mathcal{N}(\mathcal{C}(S')) = \mathcal{M}(S') = S
\]
which holds from \Cref{thm:monotonization-structure} and \Cref{pro:monotonization-cap}.
\end{proof}

Thus, the monotonization \(S \mapsto \mathcal{M}(S)\) can be said as a ‘projection’ from all moving sofas to monotone sofas, in the sense that \(\mathcal{M}\) is a surjective map that fixes monotone sofas.

\subsection{Proof of \Cref{thm:cap-hallway-intersection}}

If \(\omega = \pi / 2\), then the set \(P_\omega\) is the horizontal strip \(H\). If \(\omega < \pi/2\), \(P_\omega\) is a proper parallelogram with the following points as vertices.

\begin{definition}

Let \(O = (0, 0)\) be the origin. For any angle \(\omega \in (0, \pi/2]\), define the point \(o_\omega = (\tan(\omega/2), 1)\).

\label{def:parallelogram-vertices}
\end{definition}

Note that if \(\omega < \pi/2\), then \(O\) is the lower-left corner of \(P_\omega\) and \(o_{\omega} = l(\omega, 1) \cap l(\pi/2, 1)\) is the upper-right corner of \(P_\omega\). Define the following subset of \(P_\omega\).

\begin{definition}

Let \(\omega \in (0, \pi/2]\) be arbitrary. Define \(M_\omega\) as the convex hull of the points \(O, o_\omega, o_\omega-u_\omega, o_\omega-v_0\).

\label{def:middle-set}
\end{definition}

Geometrically, \(M_\omega\) is a subset of \(P_\omega\) enclosed by the perpendicular legs from \(o_\omega\) to the bottom sides \(l(\omega, 0)\) and \(l(\pi/2, 0)\) of \(P_\omega\). We also introduce the following terminology.

\begin{definition}

Say that a point \(p_1\) is \emph{further than} (resp. \emph{strictly further than}) the point \(p_2\) \emph{in the direction} of nonzero vector \(v \in \mathbb{R}^2\) if \(p_1 \cdot v \geq p_2 \cdot v\) (resp. \(p_1 \cdot v > p_2 \cdot v\)).

\label{def:further-in-direction}
\end{definition}

We show the following lemma.

\begin{lemma}

If \(\omega < \pi/2\), then the set \(\mathcal{C}(S)\) in \Cref{def:cap-sofa} contains \(M_\omega\).

\label{lem:cap-contains-middle-set}
\end{lemma}

\begin{proof}
Since \(p_S(\omega) = p_S(\pi/2) = 1\), we can take points \(q\) and \(r\) of \(S\) so that \(q\) is on the line \(l(\pi/2, 1)\) further than \(o_\omega\) in the direction of \(-u_0\), and \(r\) is on the line \(l(\omega, 1)\) further than \(o_\omega\) in the direction of \(-v_\omega\). Take an arbitrary \(t \in [0, \omega]\). Because \(Q^+_S(t)\) is a right-angled convex cone with normal vectors \(u_t\) and \(v_t\) containing \(q\) and \(r\), \(Q_S^+(t)\) also contains \(o_\omega\). Because \(Q_S^+(t)\) contains \(o_\omega\) and is closed in the direction of \(-u_t\) and \(-v_t\) (\Cref{def:closed-in-direction}), \(Q_S^+(t)\) contains \(M_\omega\) as a subset. So the intersection \(\mathcal{C}(S)\) of \(P_\omega\) and \(Q_S^+(t)\) contains \(M_\omega\).
\end{proof}

We finish the proof of \Cref{thm:cap-hallway-intersection}.

\begin{proof}[Proof of \Cref{thm:cap-hallway-intersection}]
Let \(S\) be any moving sofa with rotation angle \(\omega \in (0, \pi/2]\) in standard position. Let \(K = \mathcal{C}(S)\). That \(S \subseteq K\) is an immediate consequence of the third condition of \Cref{thm:moving-sofa-iff-hallway-intersection}. We now show that \(K\) is a cap with rotation angle \(\omega\).

Assume the case \(\omega < \pi/2\). Then by \Cref{def:cap-sofa} and \Cref{lem:cap-contains-middle-set} we have \(M_\omega \subseteq K \subseteq P_\omega\), and the support function of \(M_\omega\) and \(P_\omega\) agree on the angles \(\omega, \pi/2, \omega + \pi, 3\pi/2\). So the first condition of \Cref{def:cap} is satisfied. Now assume \(\omega = \pi/2\). Since \(S \subseteq K \subseteq H\) and \(S\) is in standard position we have \(p_S(\pi/2) = p_K(\pi/2) = 1\). With \(p_K(\pi/2) = 1\), take the point \(z \in K\) on the line \(y=1\). Let \(X := \bigcap_{t \in [0, \pi/2]} Q_S^+(t)\), then by the definition of \(K\) we have \(K = H \cap X\). Since \(X\) is closed in the direction of \(-v_0\) (\Cref{def:closed-in-direction}), the point \(z' := z - (0, 1)\) is also in \(X\). So \(z' \in H \cap X = K\) and \(z'\) is on the line \(y=0\). This implies that \(p_K(3\pi/2) = 0\). So the first condition of \Cref{def:cap} is true.

The set \(P_\omega\) is the intersection of four half-planes with normal angles \(\omega, \pi/2, \pi + \omega, 3\pi/2\). The set \(Q_S^+(t)\) is an intersection of two half-planes with normal angles \(t\) and \(t + \pi/2\). Now the second condition of \Cref{def:cap} follows.
\end{proof}

\subsection{Cap contains niche}
\label{sec:cap-contains-niche}
We will now establish the following theorem.

\begin{theorem}

For any monotone sofa \(S\) with cap \(K = \mathcal{C}(S)\), the cap \(K\) contains the niche \(\mathcal{N}(K)\).

\label{thm:niche-in-cap}
\end{theorem}

Note that \(S = K \setminus \mathcal{N}(K)\) by \Cref{thm:monotone-sofa-structure}. With \Cref{thm:niche-in-cap}, the area \(|S| = |K| - |\mathcal{N}(K)|\) of a monotone sofa can be understood separately in terms of its cap and niche.

\begin{remark}

In spite of \Cref{thm:niche-in-cap}, a general cap \(K\) following \Cref{def:cap} may not always contain its niche \(\mathcal{N}(K)\) in \Cref{def:niche}. For an example, take \(K = [0, 100] \times [0, 1]\) with rotation angle \(\omega = \pi/2\). Then \(K\) is too wide and the inner quadrant \(Q_K^-(\pi/4)\) of \(L_K(\pi/4)\) pushes out of \(K\), so we have \(\mathcal{N}(K) \not\subseteq K\). In this case, the cap \(K\) is never the cap \(\mathcal{C}(S)\) associated to a particular moving sofa \(S\) as in \Cref{thm:cap-hallway-intersection}. \Cref{thm:monotonization-connected-iff} identifies the exact condition of \(K\) where \(\mathcal{N}(K) \subseteq K\).

\label{rem:niche-not-in-cap}
\end{remark}

\subsection{Geometric definitions on cap and niche}

We need a handful of geometric definitions on a cap \(K\) to prove \Cref{thm:niche-in-cap}. We will also use them throughout the rest of the document as well. Define the \emph{vertices} of a cap \(K\).

\begin{definition}

Let \(K\) be a cap with rotation angle \(\omega\). For any \(t \in [0, \omega]\), define the vertices \(A^+_K(t) = v^+_K(t)\), \(A^-_K(t) = v^-_K(t)\), \(C^+_K(t) = v^+_K(t + \pi/2)\), and \(C^-_K(t) = v^-_K(t + \pi/2)\) of \(K\).

\label{def:cap-vertices}
\end{definition}

Note that the outer wall \(a_K(t)\) (resp. \(c_K(t)\)) of \(L_K(t)\) is in contact with the cap \(K\) at the vertices \(A_K^+(t)\) and \(A_K^-(t)\) (resp. \(C_K^+(t)\) and \(C_K^-(t)\)) respectively. We also define the \emph{upper boundary} of a cap \(K\).

\begin{definition}

For any cap \(K\) with rotation angle \(\omega\), define the \emph{upper boundary} \(\delta K\) of \(K\) as the set \(\delta K = \bigcup_{t \in [0, \omega + \pi/2]} e_K(t)\).

\label{def:upper-boundary-of-cap}
\end{definition}

For any cap \(K\) with rotation angle \(\omega\), the upper boundary \(\delta K\) is exactly the points of \(K\) making contact with the outer walls \(a_K(t)\) and \(c_K(t)\) of tangent hallways \(L_K(t)\) for every \(t \in [0, \omega]\). We collect some observations on \(\delta K\).

\begin{proposition}

Let \(K\) be a cap with rotation angle \(\omega\). The set \(K \setminus \delta K\) is the interior of \(K\) in the subset topology of \(F_\omega\).

\label{pro:upper-boundary-interior}
\end{proposition}

\begin{proof}
Since \(K\) and \(F_\omega\) are closed in \(\mathbb{R}^2\), the set \(K\) is closed in the subset topology of \(F_\omega\). Let \(X\) be the boundary of \(K\) in the subset topology of \(F_\omega\), then we have \(X \subseteq K\) because \(K\) is closed in the subset topology of \(F_\omega\). We will show that \(\delta K\) is equal to \(X\), then it follows that the set \(K \setminus \delta K\) is the interior of \(K\) in the subset topology of \(F_\omega\).

We show \(\delta K \subseteq X\) and \(X \subseteq \delta K\) respectively. Take any point \(z\) of \(\delta K\). Then \(z \in e_K(t)\) for some \(t \in [0, \omega + \pi/2]\). Since \(K\) is a planar convex body, for any \(\epsilon > 0\) the point \(z' = z + \epsilon u_t\) is not in \(K\). Since the set \(F_\omega\) is closed in the direction of \(u_t\) (\Cref{def:closed-in-direction}), the point \(z'\) is also in \(F_\omega\). Thus we have a point \(z'\) in the neighborhood of \(z\) which is outside \(K\), and \(\delta K\) is a subset of \(X\).

On the other hand, take any point \(z\) of \(X\) and assume by contradiction that \(z \in K \setminus \delta K\). Then for every \(t \in [0, \omega + \pi/2]\) we have \(z \not \in e_K(t)\) so that \(z \cdot u_t < p_K(t)\). Since \(p_K\) is continuous, the value \(p_K(t) - z \cdot u_t\) has a global lower bound \(\epsilon > 0\) on the compact interval \([0, \omega + \pi/2]\). So an open ball \(U\) of radius \(\epsilon\) centered at \(z\) is contained in the half-space \(H_K(t)\) for all \(t \in [0, \omega + \pi/2]\). Now \(U \cap F_\omega \subseteq K\) and so \(z \not\in X\), leading to contradiction.
\end{proof}

Geometrically, the upper boundary \(\delta K\) is an arc from \(A_K^-(0)\) to \(C_K^+(\omega)\) in the counterclockwise direction along the boundary \(\partial K\) of \(K\). This is rigorously justified by the following consequence of \Cref{cor:closed-param-segment}. For full details, read the introduction of \Cref{sec:parametrization-of-boundary}.

\begin{corollary}

Let \(K\) be a cap with rotation angle \(\omega\). The upper boundary \(\delta K\) admits an absolutely-continuous, arc-length parametrization \(\mathbf{b}_K^{0-, \pi/2 + \omega}\) (\Cref{def:closed-param}) from \(A_K^-(0)\) to \(C_K^+(\omega)\) in the counterclockwise direction along \(\partial K\).

\label{cor:upper-boundary-param}
\end{corollary}

We also give name to the convex polygons \(F_\omega \cap Q^-_K(t)\) whose union over all \(t \in [0, \omega]\) constitutes the niche \(\mathcal{N}(K)\).

\begin{definition}

For any cap \(K\) with rotation angle \(\omega\), define \(T_K(t) = F_\omega \cap Q^-_K(t)\) as the \emph{wedge} of \(K\) with angle \(t \in [0, \omega]\).

\label{def:wedge}
\end{definition}

\begin{proposition}

For any cap \(K\) with rotation angle \(\omega\), we have \(\mathcal{N}(K) = \cup_{t \in [0, \omega]} T_K(t)\).

\label{pro:wedge}
\end{proposition}

\begin{proof}
Immediate from \Cref{def:niche}.
\end{proof}

We give names to the parts of the wedge \(T_K(t)\).

\begin{definition}

For any cap \(K\) with rotation angle \(\omega\) and \(t \in (0, \omega)\), define \(W_K(t)\) as the intersection of lines \(b_K(t)\) and \(l(\pi, 0)\). Define \(w_K(t) = (A_K^-(0) - W_K(t)) \cdot u_0\) as the signed distance from point \(W_K(t)\) and the vertex \(A_K^-(0)\) along the line \(l(\pi, 0)\) in the direction of \(u_0\).

Likewise, define \(Z_K(t)\) as the intersection of lines \(d_K(t)\) and \(l(\omega, 0)\). Define \(z_K(t) = (C_K^+(\omega) - Z_K(t)) \cdot v_\omega\) as the signed length between \(Z_K(t)\) and the vertex \(C_K^+(\omega)\) along the line \(l(\omega, 0)\) in the direction of \(v_\omega\).

\label{def:wedge-side-lengths}
\end{definition}

Note that if the wedge \(T_K(t)\) contains the origin \(O\), then \(T_K(t)\) is a quadrilateral with vertices \(O, W_K(t), Z_K(t)\), and \(\mathbf{x}_K(t)\), and the points \(W_K(t)\) and \(Z_K(t)\) are the leftmost and rightmost point of \(\overline{T_K(t)}\) respectively.

\subsection{Controlling the wedge inside cap}

To show \(\mathcal{N}(K) \subseteq K\) we need to control each wedge \(T_K(t)\) inside \(K\). First we show \(w_K(t), z_K(t) \geq 0\). This controls the endpoints \(W_K(t)\) and \(Z_K(t)\) of \(T_K(t)\) inside \(K\).

\begin{lemma}

Let \(K\) be any cap with rotation angle \(\omega\). For any angle \(t \in (0, \omega)\), we have \(w_K(t), z_K(t) \geq 0\).

\label{lem:wedge-ends-in-cap}
\end{lemma}

\begin{proof}
To show that \(w_K(t) \geq 0\), we need to show that the point \(A_K^-(0)\) is further than the point \(W_K(t)\) in the direction of \(u_0\) (see \Cref{def:further-in-direction} for the terminology). The point \(q := a_K(t) \cap l(\pi/2, 1)\) is further than \(W_K(t) = b_K(t) \cap l(\pi/2, 0)\) in the direction of \(u_0\), because the lines \(a_K(t)\) and \(b_K(t)\) form the boundary of a unit-width vertical strip rotated counterclockwise by \(t\). The point \(A^-_K(t)\) is further than \(q = l_K(t) \cap l_K(\pi/2)\) in the direction of \(u_0\) because \(K\) is a convex body. Finally, the point \(A^-_K(0)\) is further than \(A_K^-(t)\) in the direction of \(u_0\) because \(K\) is a convex body. Summing up, the points \(W_K(t), q, A_K^-(t), A_K^-(0)\) are aligned in the direction of \(u_0\), completing the proof. A symmetric argument will show that the points \(Z_K(t)\), \(r := c_K(t) \cap l(\omega, 1)\), \(C_K^+(t)\), \(C_K^+(\omega)\) are aligned in the direction of \(v_\omega\), proving \(z_K(t) \geq 0\).
\end{proof}

\begin{corollary}

Let \(K\) be any cap with rotation angle \(\omega\). Then \(A^-_K(0), C^+_K(\omega) \in K \setminus \mathcal{N}(K)\).

\label{cor:cap-ends-not-in-niche}
\end{corollary}

\begin{proof}
We only need to show that \(A^-_K(0), C^+_K(\omega)\) are not in \(\mathcal{N}(K)\). That is, for any \(t \in (0, \omega)\), neither points are in \(T_K(t)\). Since \(w_K(t) \geq 0\) by \Cref{lem:wedge-ends-in-cap}, the point \(A_K^-(0)\) is on the right side of the boundary \(b_K(t)\) of \(T_K(t)\). So \(A_K^-(0) \not\in T_K(t)\). Similarly, \(z_K(t) \geq 0\) implies \(C_K^+(\omega) \not\in T_K(t)\).
\end{proof}

We then show that if the corner \(\mathbf{x}_K(t)\) is inside \(K\), then the whole wedge \(T_K(t)\) is always inside \(K\).

\begin{lemma}

Fix any cap \(K\) with rotation angle \(\omega \in [0, \pi/2]\) and an angle \(t \in (0, \omega)\). If the inner corner \(\mathbf{x}_K(t)\) is in \(K\), then the wedge \(T_K(t)\) is a subset of \(K\).

\label{lem:niche-in-cap}
\end{lemma}

\begin{proof}
Assume \(\mathbf{x}_K(t) \in K\). If \(\omega = \pi/2\), then by \(\mathbf{x}_K(t) \in K\), the wedge \(T_K(t)\) is the triangle with vertices \(W_K(t)\), \(\mathbf{x}_K(t)\), and \(Z_K(t)\) in counterclockwise order. Note also that \(W_K(t)\) is further than \(Z_K(t)\) in the direction of \(u_0\) (\Cref{def:further-in-direction}). As \(w_K(t), z_K(t) \geq 0\), this implies that all the three vertices of \(T_K(t)\) are in \(K\).

If \(\omega < \pi/2\), we divide the proof into four cases on whether the origin \(O\) lies strictly below the lines \(b_K(t)\) and \(d_K(t)\) or not respectively.

\begin{itemize}
\tightlist
\item
  If \((0, 0)\) lies on or above both \(b_K(t)\) and \(d_K(t)\), then we get contradiction as the corner \(\mathbf{x}_K(t)\) should be outside the interior \(F_\omega^\circ\) of fan \(F_\omega\), but \(\mathbf{x}_K(t) \in K\).
\item
  If \((0, 0)\) lies on or above \(b_K(t)\) but lies strictly below \(d_K(t)\), then \(T_K(t)\) is a triangle with vertices \(\mathbf{x}_K(t)\), \(Z_K(t)\) and the intersection \(p := l(\omega, 0) \cap b_K(t)\). In this case, the point \(p\) is in the line segment connecting \(Z_K(t)\) and \((0, 0)\). Also, as \(z_K(t) \geq 0\) (\Cref{lem:wedge-ends-in-cap}) the point \(Z_K(t)\) lies in the segment connecting \(C^+_K(\omega)\) and the origin \((0, 0)\). So the points \(\mathbf{x}_K(t), Z_K(t), p\) are in \(K\) and by convexity of \(K\) we have \(T \subseteq K\).
\item
  The case where \((0, 0)\) lies strictly below \(b_K(t)\) but lies on or above \(d_K(t)\) can be handed by an argument symmetric to the previous case.
\item
  If \((0, 0)\) lies strictly below both \(b_K(t)\) and \(d_K(t)\), then \(T_K(t)\) is a quadrilateral with vertices \(\mathbf{x}_K(t)\), \(Z_K(t)\), \(W_K(t)\) and \((0, 0)\). As \(w_K(t) \geq 0\) (resp. \(z_K(t) \geq 0\)) by \Cref{lem:wedge-ends-in-cap}, the point \(W_K(t)\) (resp. \(Z_K(t)\)) is in the line segment connecting \((0, 0)\) and \(A^-_K(0)\) (resp. \(C^+_K(\omega)\)). So all the vertices of \(T_K(t)\) are in \(K\), and \(T_K(t)\) is in \(K\) by convexity.
\end{itemize}

\end{proof}

\subsection{\texorpdfstring{Equivalent conditions for \(\mathcal{N}(K) \subseteq K\)}{Equivalent conditions for \textbackslash mathcal\{N\}(K) \textbackslash subseteq K}}

Now we prove \Cref{thm:niche-in-cap}. In fact, we identify the exact condition where \(\mathcal{N}(K) \subseteq K\) for a general cap \(K\) following \Cref{def:cap}.

\begin{theorem}

Let \(K\) be any cap with rotation angle \(\omega\). Then the followings are all equivalent.

\begin{enumerate}
\def\labelenumi{\arabic{enumi}.}
\tightlist
\item
  \(\mathcal{N}(K) \subseteq K\)
\item
  \(\mathcal{N}(K) \subseteq K \setminus \delta K\)
\item
  For every \(t \in [0, \omega]\), either \(\mathbf{x}_K(t) \not\in F_\omega^\circ\) or \(\mathbf{x}_K(t) \in K\).
\item
  The set \(S = K \setminus \mathcal{N}(K)\) is connected.
\end{enumerate}

\label{thm:monotonization-connected-iff}
\end{theorem}

\begin{proof}
The conditions (1) and (2) are equivalent because the niche \(\mathcal{N}(K)\) is open in the subset topology of \(F_\omega\) by \Cref{def:niche}, and the set \(K \setminus \delta K\) is the interior of \(K\) in the subset topology of \(F_\omega\) by \Cref{pro:upper-boundary-interior}.

(1 \(\Rightarrow\) 3) We will prove the contraposition and assume \(\mathbf{x}_K(t) \in F_\omega^\circ \setminus K\). Then a neighborhood of \(\mathbf{x}_K(t)\) is inside \(F_\omega\) and disjoint from \(K\), so a subset of \(T_K(t)\) is outside \(K\), showing \(\mathcal{N}(K) \not\subseteq K \setminus \delta K\).

(3 \(\Rightarrow\) 1) If \(\mathbf{x}_K(t) \not \in F_\omega^\circ\), then \(T_K(t)\) is an empty set. If \(\mathbf{x}_K(t) \in K\), then by \Cref{lem:niche-in-cap} we have \(T_K(t) \subseteq K\).

(2 \(\Rightarrow\) 4) As \(\delta K\) is disjoint from \(\mathcal{N}(K)\), we have \(\delta K \subseteq S\). We show that \(S\) is connected. First, note that \(\delta K\) is connected by \Cref{cor:upper-boundary-param}. Next, take any point \(p \in S\). Take the half-line \(r\) starting from \(p\) in the upward direction \(v_0\). Then \(r\) touches a point in \(\delta K\) as \(p \in K\). Moreover, \(r\) is disjoint from \(\mathcal{N}(K)\) as the set \(\mathcal{N}(K) \cup (\mathbb{R}^2 \setminus F_\omega)\) is closed in the direction \(-v_0\) (\Cref{def:closed-in-direction}). Now \(r \cap K\) is a line segment inside \(S\) that connects the arbitrary point \(p \in S\) to a point in \(\delta K\). So \(S\) is connected.

(4 \(\Rightarrow\) 3) Assume by contradiction that \(\mathbf{x}_K(t) \in F_\omega^\circ \setminus K\) for some \(t \in [0, \omega]\). Then it should be that \(t \neq 0\) or \(\omega\). We first show that the vertical line \(l\) passing through \(\mathbf{x}_K(t)\) in the direction of \(v_0\) is disjoint from \(S\). The ray with initial point \(\mathbf{x}_K(t)\) and direction \(v_0\) is disjoint from \(K\) as the set \(F_\omega^\circ \setminus K\) is closed in the direction \(v_0\). The ray with initial point \(\mathbf{x}_K(t)\) and direction \(-v_0\) is not in \(S\) because \(\mathbf{x}_K(t)\) is the corner of \(Q_K^-(t)\), and \(Q_K^-(t)\) is closed in the direction of \(-v_0\). So the vertical line \(l\) passing through \(\mathbf{x}_K(t)\) does not overlap with \(S\).

Now separate the horizontal strip \(H\) into two chunks by the vertical line \(l\) passing through \(\mathbf{x}_K(t)\). As \(S\) is connected, \(S\) should lie either strictly on left or strictly on right of \(l\). As \(\mathbf{x}(t)\) lies strictly inside \(F_\omega\), the point \(W_K(t)\) is strictly further than \(\mathbf{x}(t)\) in the direction of \(u_0\), and by \Cref{lem:wedge-ends-in-cap} the point \(A_K^-(0)\) is further than \(W_K(t)\) in the direction of \(u_0\). So the endpoint \(A_K^-(0)\) of \(K\) lies strictly on the right side of \(l\). Similarly, the point \(Z_K(t)\) is strictly further than \(\mathbf{x}_K(t)\) in the direction of \(-u_0\), and by \Cref{lem:wedge-ends-in-cap} the point \(C_K^+(\omega)\) is further than \(W_K(t)\) in the direction of \(-u_0\). So the endpoint \(C^+_K(\omega)\) of \(K\) lies strictly on the left side of \(l\). As the endpoints \(A^-_K(0)\) and \(C^+_K(\omega)\) are in \(K \setminus \mathcal{N}(K)\) by \Cref{cor:cap-ends-not-in-niche}, and the line \(l\) separates the two points, the set \(K \setminus \mathcal{N}(K)\) is disconnected.
\end{proof}

\Cref{thm:niche-in-cap} is an immediate consequence of \Cref{thm:monotonization-connected-iff}.

\begin{proof}[Proof of \Cref{thm:niche-in-cap}]
We have \(S = K \setminus \mathcal{N}(K)\) by \Cref{thm:monotone-sofa-structure}. In particular, \(K \setminus \mathcal{N}(K)\) is a moving sofa so it is connected. Use that condition 4 implies condition 1 in \Cref{thm:monotonization-connected-iff} to complete the proof.
\end{proof}



\chapter{Sofa area functional $\mathcal{A}$}
\label{sec:sofa-area-functional-a}
In this section, we set up basic notations, definitions and conventions that will be assumed thoroughout the rest of the document. We also gather the notions that will be defined later for easier reference. Definitions not listed in this section will be always referenced before its first use in a proof.

This section is comprehensive and not meant to be read in one setting. Instead, the reader may start by reading only the definitions related to moving sofas, shapes, and convex bodies and move on, referring to parts of this section later as needed.


\chapter{Conditional upper bound $\mathcal{A}_1$}
\label{sec:conditional-upper-bound-a1}
In this section, we set up basic notations, definitions and conventions that will be assumed thoroughout the rest of the document. We also gather the notions that will be defined later for easier reference. Definitions not listed in this section will be always referenced before its first use in a proof.

This section is comprehensive and not meant to be read in one setting. Instead, the reader may start by reading only the definitions related to moving sofas, shapes, and convex bodies and move on, referring to parts of this section later as needed.
\subsection{Definition of A1}
\label{sec:definition-of-a1}
\input{out/25. Conditional upper bound A1/01. Definition of A1.tex}

\subsection{Calculus of variation}
\label{sec:calculus-of-variation}
\input{out/25. Conditional upper bound A1/02. Calculus of variation.tex}

\subsection{Boundary measure}
\label{sec:boundary-measure}
We will show that \(\mathcal{A}_1 : \mathcal{K}_\omega \to \mathbb{R}\) is a quadratic functional.

\begin{theorem}

For any \(\omega \in (0, \pi/2]\), the functional \(\mathcal{A}_1 : \mathcal{K}_{\omega} \to \mathbb{R}\) is quadratic.

\label{thm:a1-quadratic}
\end{theorem}

To establish \Cref{thm:a1-quadratic}, we will define the \emph{boundary measure} \(\beta_K\) of \(K \in \mathcal{K}_\omega\) and utilize it. Also, we will establish a correspondence between any cap \(K \in \mathcal{K}_\omega\) and its boundary measure \(\beta_K\) (\Cref{thm:boundary-measure-cap} and \Cref{thm:cap-from-boundary-measure}).

\subsubsection{Convex-linear values of cap}

We observe that a lot of values defined on the cap \(K \in \mathcal{K}_\omega\) is convex-linear with respect to \(K\). A reader interested in the details of proofs can read \Cref{sec:vertex-and-support-function} for the full details.

\begin{theorem}

The following values are convex-linear with respect to \(K \in \mathcal{K}_\omega\).

\begin{itemize}
\tightlist
\item
  Support function \(p_K\)
\item
  Vertices \(A^{\pm}_K(t)\) and \(C^{\pm}_K(t)\) for a fixed \(t \in [0, \omega]\)
\item
  The inner and outer corner \(\mathbf{x}_K(t)\) and \(\mathbf{y}_K(t)\) of the tangent hallway with any angle \(t \in [0, \omega]\)
\item
  The points \(W_K(t)\), \(Z_K(t)\) and the values \(w_K(t)\), \(z_K(t)\) for a fixed \(t \in (0, \omega)\)
\item
  The perpendicular leg lengths \(g^{\pm}_K(t)\) and \(h^{\pm}_K(t)\) for all \(t \in [0, \omega]\)
\end{itemize}

\label{thm:cap-convex-linear}
\end{theorem}

\begin{proof}
Use \Cref{pro:support-function-linear} for \(p_K\), \Cref{cor:vertex-linear} for \(A^{\pm}_K(t)\) and \(C^{\pm}_K(t)\), \Cref{lem:tangent-lines-intersection-linear} for \(\mathbf{y}_K(t), W_K(t)\), and \(Z_K(t)\). Use the equality \(\mathbf{y}_K(t) = \mathbf{x}_K(t) + u_t + v_t\) for \(\mathbf{x}_K(t)\), the equalities in \Cref{def:wedge-side-lengths} for \(w_K(t)\) and \(z_K(t)\), and the equalities in \Cref{def:cap-tangent-arm-length} for \(g^{\pm}_K(t)\) and \(h^{\pm}_K(t)\).
\end{proof}

\Cref{thm:cap-convex-linear} in particular establishes that the curve area functional \(\mathcal{I}(\mathbf{x}_K)\) (\Cref{def:curve-area-functional}) is quadratic with respect to \(K\).

\begin{corollary}

The value \(\mathcal{I}(\mathbf{x}_K)\) of a cap \(K \in \mathcal{K}_\omega\) is quadratic with respect to \(K\).

\label{cor:inner-corner-quadratic}
\end{corollary}

\subsubsection{Boundary Measure}

We now define the \emph{boundary measure} \(\beta_K\) of a cap \(K\) as the restriction of the \emph{surface measure} \(\sigma_K\) of \(K\) (\Cref{def:surface-area-measure}).

\begin{definition}

For any cap \(K \in \mathcal{K}_\omega\) with rotation angle \(\omega\), define the \emph{boundary measure} \(\beta_K\) of \(K\) on the set \(J_\omega\) (\Cref{def:j-cap}) as the surface area measure \(\sigma_K\) of \(K\) restricted to \(J_\omega\).

\label{def:boundary-measure}
\end{definition}

See \Cref{sec:surface-area-measure} for a brief introduction on the surface area measure \(\sigma_K\). The boundary measure \(\beta_K\) of cap \(K\) describes the information of length of the upper boundary \(\delta K\). For example, let \(K = [0, 1]^2 \cup \left\{ (x, y) : x \leq 0, y \geq 0, x^2 + y^2 \leq 1 \right\}\) be a cap with rotation angle \(\pi/2\), which is the union of a unit square and a quarter-circle of radius one. Then the boundary measure \(\beta_K\) is a measure on \(J_{\pi/2} = [0, \pi]\) such that \(\beta_K\left( \left\{ 0 \right\} \right) = \beta_K\left( \left\{ \pi/2 \right\} \right) = 1\), and \(\beta_K\) equal to zero on the interval \((0, \pi/2)\) and the Lebesgue measure \(\beta_K(dt) = dt\) on the interval \((\pi/2, \pi)\). We now collect the properties of \(\beta_K\).

\begin{proposition}

The boundary measure \(\beta_K\) is convex-linear with respect to \(K \in \mathcal{K}_\omega\).

\label{pro:boundary-measure-linear}
\end{proposition}

\begin{proof}
Immediate from \Cref{thm:surface-area-measure-convex-linear}.
\end{proof}

\begin{proposition}

For any cap \(K \in \mathcal{K}_\omega\), we have
\[
|K| = \left< p_K, \beta_K \right>_{J_\omega}.
\]

\label{pro:boundary-measure-area}
\end{proposition}

\begin{proof}
By \Cref{thm:surface-area-measure-area} we have \(|K| = \left< p_K, \sigma_K \right>_{S^1}\). Apply \Cref{thm:convex-set-support} to the second condition of \Cref{def:cap} to obtain that \(\sigma_K\) is supported on the set \(J_{\omega} \cup \left\{ \omega + \pi, 3\pi/2 \right\}\). The first condition of \Cref{def:cap} gives \(p_K(\omega + \pi) = p_K(3\pi/2) = 0\). From these, we have \(|K| = \left< p_K, \sigma_K \right>_{S^1} = \left< p_K, \beta_K \right>_{J_\omega}\).
\end{proof}

Now the quadraticity of \(|K|\) comes from convex-linearity of \(p_K\) (\Cref{pro:support-function-linear}) and \(\beta_K\) (\Cref{pro:boundary-measure-linear}) with respect to \(K\).

\begin{corollary}

The area \(|K|\) of a cap \(K \in \mathcal{K}_{\omega}\) is a quadratic functional on \(\mathcal{K}_\omega\)

\label{cor:area-quadratic-functional}
\end{corollary}

The quadraticity of \(\mathcal{A}_1\) is now obtained.

\begin{proof}[Proof of \Cref{thm:a1-quadratic}]
Immediate consequence of \Cref{cor:inner-corner-quadratic} and \Cref{cor:area-quadratic-functional}.
\end{proof}

Gauss-Minkowski theorem (\Cref{thm:gauss-minkowski}) states that any convex set \(K\), up to translation, corresponds one-to-one to a measure \(\sigma\) on \(S^1\) such that \(\int_{S^1}u_t\,\sigma(dt) = 0\) by taking the surface area measure \(\sigma = \sigma_K\). Using this correspondence, we can always construct a bijection between a cap \(K \in \mathcal{K}_\omega\) and its boundary measure \(\beta = \beta_K\).

\begin{theorem}

For any cap \(K \in \mathcal{K}_\omega\) with rotation angle \(\omega\), its boundary measure \(\beta_K\) satisfies the following equations.
\[
\int_{t \in [0, \omega]} \cos(t) \, \beta_K(dt) = 1 \qquad \int_{t \in [\pi/2, \omega + \pi/2]} \cos\left( \omega + \pi/2 - t \right)  \, \beta_K(dt) = 1
\]

\label{thm:boundary-measure-cap}
\end{theorem}

\begin{proof}
By the second condition of \Cref{def:cap} and \Cref{thm:convex-set-support}, we have \(\mathbf{n}(K) \subseteq J_\omega \cup \left\{ \pi + \omega, 3\pi/2 \right\}\). Now by \Cref{thm:convex-set-support-disjoint}, that the interval \((-\pi/2, 0)\) of \(S^1\) is disjoint from \(\Pi\) implies that the point \(A_K^-(0)\) is on the line \(l_K(3\pi/2)\) which is \(y=0\). Likewise, that the interval \((\omega, \pi/2)\) of \(S^1\) is disjoint from \(\Pi\) implies that the point \(A_K^+(\omega)\) is on the line \(l_K(\pi/2)\) which is \(y=1\). By \Cref{cor:boundary-measure-closed} we have
\[
\int_{t \in [0, \omega]} v_t \, \beta_K(dt) = A^+_K(\omega) - A^-_K(0)
\]
and by taking the dot product with \(v_0\), we have the first equality. The second equality can be proved similarly by measuring the displacement from \(C_K^+(\omega)\) to \(C_K^-(0)\) along the direction \(u_\omega\).
\end{proof}

\begin{theorem}

Take arbitrary \(\omega \in (0, \pi/2]\). Conversely to \Cref{thm:boundary-measure-cap}, let \(\beta\) be a measure on \(J_\omega\) that satisfies the following equations.
\[
\int_{t \in [0, \omega]} \cos(t) \, \beta(dt) = 1 \qquad \int_{t \in [\pi/2, \omega + \pi/2]} \cos\left( \omega + \pi/2 - t \right)  \, \beta(dt) = 1
\]
Then there exists a cap \(K \in \mathcal{K}_\omega\) such that \(\beta_K = \beta\). Such \(K\) is unique if \(\omega < \pi/2\), and unique up to horizontal translation if \(\omega = \pi/2\).

\label{thm:cap-from-boundary-measure}
\end{theorem}

\begin{proof}
We first show that there is a unique extension \(\sigma\) of \(\beta\) on the set \(\Pi = J_\omega \cup \{\pi + \omega, 3\pi/2\}\) such that \(\int_{t \in \Pi} v_t \, \sigma(dt) = 0\). The values of \(\sigma\) are determined on \(J_\omega\), and we need to find the values of \(\sigma(\left\{ \pi + \omega \right\})\) and \(\sigma(\left\{ 3 \pi/2 \right\})\) that satisfies the equation \(\int_{t \in \Pi} v_t \, \sigma(dt) = 0\).

If \(\omega = \pi/2\), then by subtracting the two equations in \Cref{thm:cap-from-boundary-measure} we have \(\int_{t \in [0, \pi]} \cos(t)\,\beta(dt) = 0\). So the equation \(\int_{t \in \Pi} v_t \, \sigma(dt) = 0\) becomes \(\sigma(\left\{ 3\pi/2 \right\}) = \int_{t \in [0, \pi]} \sin (t) \,\beta(dt)\) which immediately gives a unique solution \(\sigma\).

Now assume \(\omega < \pi/2\). Let \(A := \int_{t \in [0, \omega]}\sin(t)\,\beta(dt) \geq 0\), then we have \(\int_{t \in [0, \omega]} v_t \,\beta(dt) = - A u_0 + v_0\) by the first equality of \Cref{thm:cap-from-boundary-measure}. Likewise, if we let \(B := \int_{t \in [\pi/2, \omega + \pi/2]} \sin(\omega + \pi/2 - t)\,\beta(dt) \geq 0\), then we have \(\int_{t \in [\pi/2, \omega + \pi/2]}v_t\,\beta(dt) = B v_\omega - u_\omega\) by the second equality of \Cref{thm:cap-from-boundary-measure}. Now the equation \(\int_{t \in \Pi} v_t \, \sigma(dt) = 0\) we are solving for becomes
\[
(-Au_0 + v_0) + (Bv_\omega - u_\omega) + \sigma\left( \left\{ 3\pi/2 \right\}  \right)  u_0 - \sigma\left( \left\{ \pi + \omega \right\}  \right)  v_\omega = 0
\]
and \(\sigma(\left\{ \pi + \omega \right\}) = B + v_\omega \cdot o_\omega \geq 0\) and \(\sigma(\left\{ 3 \pi/2 \right\}) = A + u_0 \cdot o_\omega \geq 0\) (remark that \(o_\omega\) is in \Cref{def:parallelogram-vertices}) gives the unique solution of \(\sigma\).

We now use \Cref{cor:supported-gauss-measure} on the measure \(\sigma\) extended on the set \(\Pi\). There is a unique convex body \(K\) up to translation so that \(\mathbf{n}(K) \subseteq \Pi\) (see \Cref{def:convex-set-support}) and \(\sigma_K|_{\Pi} = \sigma\). Our goal now is to translate \(K\) so that it is a cap with rotation angle \(\omega\). Since \(\mathbf{n}(K) \subseteq \Pi\), any translate \(K\) satisfy the second condition of cap in \Cref{def:cap}. It remains to prove the first condition of \Cref{def:cap}.

The width of \(K\) along the directions \(u_\omega\) and \(v_0\) are equal to 1 by applying the equations given in \Cref{thm:cap-from-boundary-measure} to \Cref{cor:boundary-measure-width} with angles \(t = \omega, \pi/2\). If \(\omega = \pi/2\), we only need \(\beta_K(\pi/2) = 1\) for \(K\) to satisfy the first condition of \Cref{def:cap}, and such a cap \(K\) is unique up to horizontal translation. If \(\omega < \pi/2\), we need both \(\beta_K(\pi/2) = \beta_K(\omega) = 1\) to satisfy the first condition of \Cref{def:cap}, so such a cap \(K\) exists uniquely among all translates.
\end{proof}

\subsection{Concavity of A1}
\label{sec:concavity-of-a1}
\input{out/25. Conditional upper bound A1/10. Concavity of A1.tex}

\subsection{Directional derivative of A1}
\label{sec:directional-derivative-of-a1}
\input{out/25. Conditional upper bound A1/15. Directional derivative of A1.tex}

\subsection{Maximizer of A1}
\label{sec:maximizer-of-a1}
\input{out/25. Conditional upper bound A1/20. Maximizer of A1.tex}



\appendix
\chapter{Convex bodies}
\label{sec:convex-bodies}
Fix an arbitrary planar convex body \(K\), which by \Cref{def:convex-body} is a nonempty, compact and convex subset of \(\mathbb{R}^2\). This appendix defines and proves numerous properties of \(K\). For the ease of understanding, it is recommended to read only the parts of this appendix when needed. However, an interested reader can read this appendix from beginning to end to verify the correctness of all proofs and theorems. Note that we allow \(K\) to have empty interior. If a theorem requires \(K\) to have nonempty interior, we state it explicitly.
\subsection{Vertex and support function}
\label{sec:vertex-and-support-function}
\input{out/A1. Convex bodies/05. Vertex and support function.tex}

\subsection{Lebesgue-Stieltjes measure}
\label{sec:lebesgue-stieltjes-measure}
\input{out/A1. Convex bodies/07. Lebesgue-Stieltjes measure.tex}

\subsection{Surface area measure}
\label{sec:surface-area-measure}
\input{out/A1. Convex bodies/10. Surface area measure.tex}

\subsection{Parametrization of boundary}
\label{sec:parametrization-of-boundary}
If \(K\) has nonempty interior, it occurs naturally that the boundary \(\partial K\) is a Jordan curve bounding \(K\) in its interior. So for any different \(p, q \in \partial K\), we can think of the Jordan arc \(\mathbf{b}\) connecting \(p\) and \(q\) along the boundary \(\partial K\) in counterclockwise direction. However, to rigorously justify that the curve area functional \(\mathcal{I}(\mathbf{b})\) of \(\mathbf{b}\) is well-defined and relates to the surface area measure \(\sigma_K\) (\Cref{thm:param-curve-area-functional}), we need to contruct an explicit rectifiable parametrization of \(\mathbf{b}\) which requires some work.

For every \(t_0 \in \mathbb{R}\) and \(t_1 \in [t_0, t_0 + 2\pi]\), we will define \(\mathbf{b}_K^{t_0, t_1}\) as essentially the arc-length parametrization of the curve connecting \(v_K^+(t_0)\) to \(v_K^+(t_1)\) along the boundary \(\partial K\) counterclockwise. The full \Cref{def:boundary-segment-parametrization} of \(\mathbf{b}_{K}^{t_0, t_1}\) is technical will be given much later. Instead, we start by stating the properties \(\mathbf{b}_K^{t_0, t_1}\) that agrees with our intuition that we will prove rigorously. Note that in the theorems below we allow \(K\) to have empty interior.

\begin{theorem}

Assume arbitrary \(t_0 \in \mathbb{R}\) and \(t_1 \in [t_0, t_0 + 2\pi]\). Then \(\mathbf{b}_K^{t_0, t_1}\) is an arc-length parametrization of the \(\left\{ v_K^+(t_0) \right\} \bigcup \cup_{t \in (t_0, t_1]} e_K(t)\) from point \(v_K^+(t_1)\) to \(v_K^+(t_2)\).

\label{thm:param-segment}
\end{theorem}

\begin{theorem}

Assume arbitrary \(t_0 \in \mathbb{R}\) and \(t_1 \in [t_0, t_0 + 2\pi]\). Then the curve \(\mathbf{b}_K^{t_0, t_1}\) have length \(\sigma_K((t_0, t_1])\).

\label{thm:param-segment-length}
\end{theorem}

\begin{theorem}

Assume arbitrary \(t_0, t_1, t_2\) such that \(t_0 \leq t_1 \leq t_2 \leq t_0 + 2\pi\). Then \(\mathbf{b}_{K}^{t_0, t_2}\) is the concatenation of \(\mathbf{b}_{K}^{t_0, t_1}\) and \(\mathbf{b}_{K}^{t_1, t_2}\).

\label{thm:param-concatenation}
\end{theorem}

\begin{theorem}

Assume arbitrary \(t_0 \in \mathbb{R}\) and \(t_1 \in [t_0, t_0 + 2\pi]\). Then the curve area functional of \(\mathbf{b}_K^{t_0, t_1}\) can be represented in two different ways:
\[
\mathcal{I} \left( \mathbf{b}_{K}^{t_0, t_1} \right) = \frac{1}{2} \int_{(t_0, t_1]}p_K(t)\,\sigma_K(dt) = \frac{1}{2} \int_{(t_0, t_1]} v_K^+(t) \times d v_K^+(t)
\]

\label{thm:param-curve-area-functional}
\end{theorem}

\begin{theorem}

For every \(t \in \mathbb{R}\), we have \(|K| = \mathcal{I}\left( \mathbf{b}_K^{t, t + 2\pi} \right)\).

\label{thm:param-positive-area}
\end{theorem}

\begin{proof}
This is a corollary of \Cref{thm:surface-area-measure-area} and \Cref{thm:param-curve-area-functional}.
\end{proof}

We will also show that \(\mathbf{b}_K^{t_0, t_1}\) is one of: a Jordan arc, a Jordan curve, or a single point (\Cref{cor:param-positive-jordan}). We first recall the difference between a Jordan arc and curve (p170 of \autocite{apostolMathematicalAnalysisModern}).

\begin{definition}

A \emph{Jordan curve} is a curve parametrized by continuous \(\mathbf{p} : [a, b] \to \mathbb{R}^2\) such that \(a<b\), \(\mathbf{p}(a) = \mathbf{p}(b)\) and \(\mathbf{p}\) being injective on \([a, b)\).

\label{def:jordan-curve}
\end{definition}

\begin{definition}

A \emph{Jordan arc} is a curve parametrized by continuous and injective \(\mathbf{p} : [a, b] \to \mathbb{R}^2\) such that \(a<b\).

\label{def:jordan-arc}
\end{definition}

In order for \(\partial K\) to be a Jordan curve, \(K\) has to have nonempty interior. For the notion of the orientation of a Jordan curve, we refer to p170 of \autocite{apostolMathematicalAnalysisModern}. The following theorem shows that \(\mathbf{b}_K^{t, t + 2\pi}\) is the unique parametrization of \(\partial K\) as a positively-oriented Jordan curve with \(v_K^+(t)\) as its endpoints.

\begin{theorem}

Assume that \(K\) have nonempty interior. For every \(t \in \mathbb{R}\), the curve \(\mathbf{b}_K^{t, t + 2\pi}\) is a positively oriented arc-length parametrization of the boundary \(\partial K\) as a Jordan curve that starts and ends with the point \(v_K^+(t)\).

\label{thm:param-positive-jordan}
\end{theorem}

Now the following is a corollary of \Cref{thm:param-concatenation} and \Cref{thm:param-positive-jordan}.

\begin{corollary}

Assume that \(K\) have nonempty interior. Assume arbitrary \(t_0 \in \mathbb{R}\) and \(t_1 \in [t_0, t_0 + 2\pi]\). Then \(\mathbf{b}_K^{t_0, t_1}\) is one of: a Jordan arc, a Jordan curve, or a single point.

\label{cor:param-positive-jordan}
\end{corollary}

\subsubsection{Definition of Parametrization}

\begin{definition}

Denote the \emph{perimeter} of \(K\) as \(B_K = \sigma_K\left( S^1 \right)\).

\label{def:convex-body-perimeter}
\end{definition}

Fix an arbitrary convex body \(K\) and the starting angle \(t_0 \in \mathbb{R}\). Our goal to construct an arc-length parametrization \(\mathbf{b}_{K}^{t_0} : [0, B_K] \to \mathbb{R}^2\) of the boundary \(\partial K\) starting with the point \(v_K^+(t_0)\). Take an arbitrary point \(p\) on the boundary \(\partial K\). Let \(s \in [0, B_K]\) be the arc length from \(v_K^+(t_0)\) to \(p\) along \(\partial K\), so that we want \(p = \mathbf{b}_{K}^{t_0}(s)\) in the end. As \(p\) is in \(\partial K\), it is inside the tangent line \(l_K(t)\) for some angle \(t \in (t_0, t_0 + 2\pi]\). Now the arc length \(s \in [0, B_K]\) and the tangent line angle \(t \in (t_0, t_0 + 2\pi]\) are the two different variables attached to \(p \in \partial K\).

Unfortunately, the relation between \(s\) and \(t\) cannot be simply described as a function from one to another. A single value of \(s\) may correspond to multiple values of \(t\) (if \(p\) is a sharp corner of angle \(< \pi\)), and likewise a single value of \(t\) may correspond to multiple values of \(s\) (if \(p\) is on the edge \(e_K(t)\) which is a proper line segment). We need the language of generalized inverse (e.g. \autocite{fortelleStudyGeneralizedInverses}) to describe the relationship between \(s\) and \(t\).

The map \(g_K^{t_0}\) is defined so that it sends \(t\) to the largest possible corresponding \(s\).

\begin{definition}

Define \(g_K^{t_0} :[t_0, t_0 + 2\pi] \to [0, B_K]\) as \(g_{K}^{t_0}(t) = \sigma_K\left( (t_0, t] \right)\).

\label{def:conversion-ts}
\end{definition}

The map \(f_K^{t_0}\) is defined so that sends \(s\) to the smallest possible corresponding \(t\).

\begin{definition}

Define \(f_K^{t_0} : [0, B_K] \to [t_0, t_0 + 2\pi]\) as \(f_K^{t_0}(s) = \min \left\{ t \geq t_0 : \sigma_K((t_0, t]) \geq s \right\}\).

\label{def:conversion-st}
\end{definition}

It is rudimentary to check that \(f_K^{t_0}\) is well-defined. We remark that \(f_K^{t_0}\) is the minimum inverse \(g_K^{t_0\wedge}\) of \(g_K^{t_0}\) as defined in \autocite{fortelleStudyGeneralizedInverses}. Note the following.

\begin{proposition}

The functions \(f_K^{t_0}\) and \(g_{K}^{t_0}\) are monotonically increasing.

\label{pro:conversion-monotone}
\end{proposition}

\begin{proof}
That \(g_K^{t_0}(t)\) is increasing is immediate from \Cref{def:conversion-ts}. For any \(t_1 < t_2\), observe
\[
\left\{ t_1 \geq t_0 : \sigma_K((t_0, t]) \geq s \right\} \subseteq \left\{ t_2 \geq t_0 : \sigma_K((t_0, t]) \geq s \right\}
\]
so by \Cref{def:conversion-st} we have \(f_K^{t_0}(t_1) \leq f_K^{t_0}(t_2)\).
\end{proof}

The following can be checked using \Cref{def:conversion-st}.

\begin{proposition}

We have \(f_K^{t_0}(0) = t_0\) and \(f_K^{t_0}(s) > t_0\) for all \(s > 0\).

\label{pro:conversion-st-nonzero}
\end{proposition}

\begin{proof}
That \(f_K^{t_0}(0) = t_0\) is immediate from \Cref{def:conversion-st}. If \(s > 0\), then any \(t \geq t_0\) satisfying \(\sigma_K((t_0, t]) \geq s\) has to satisfy \(t > t_0\), so we have \(f_K^{t_0}(s) > t_0\).
\end{proof}

We will often write \(f_K^{t_0}\) and \(g_K^{t_0}\) as simply \(f\) and \(g\) in proofs because our \(K\) and \(t_0\) are fixed. With the converstions between \(s\) and \(t\) prepared (\(f\) maps \(s\) to \(t\), and \(g\) maps \(t\) to \(s\)), the path \(\mathbf{b}_{K}^{t_0}\) can be defined by integrating the unit vector \(u_t\) for each \(s\).

\begin{definition}

Define \(\mathbf{b}_{K}^{t_0} : [0, B_K] \to \mathbb{R}^2\) as the absolutely continuous (and thus rectifiable) function with the initial condition \(\mathbf{b}_{K}^{t_0}(0) = v_K^+(t_0)\) and the derivative \(\left(\mathbf{b}_{K}^{t_0}\right)'(s) = v_{f_K^{t_0}(s)}\) almost everywhere. That is:
\[
\mathbf{b}_{K}^{t_0}(s) := v_K^+(t_0) + \int_{s' \in (0, s]} v_{f_K^{t_0}(s')} \, ds'
\]

\label{def:parametrization-formal}
\end{definition}

Note that the function \(f_{K}^{t_0}\) is monotonically increasing, so the integral in \Cref{def:parametrization-formal} is well-defined.

\begin{proposition}

The function \(\mathbf{b}_{K}^{t_0} : [0, B_K] \to \mathbb{R}^2\) is an arc-length parametrization.

\label{pro:parametrization-arc-length}
\end{proposition}

\begin{proof}
Length of an absolutely continuous curve~\(\mathbf{x} : [a, b] \to \mathbb{R}^2\)~is the integral of~\(| | \mathbf{x}'(s) | |\) from \(s=a\) to \(s=b\) \autocite{jones2001lebesgue}. For \(\mathbf{x} = \mathbf{b}_K^{t_0}\), we have \(| | \mathbf{x}'(s) | | = 1\) for almost every \(s\) by \Cref{def:parametrization-formal}, thus completing the proof.
\end{proof}

We define \(\mathbf{b}_K^{t_0, t_1}\) as an initial segment of \(\mathbf{b}_K^{t_0}\) that ends with \(v_K^+(t_1)\).

\begin{definition}

For any \(t_0, t_1 \in \mathbb{R}\) such that \(t_1 \in [t_0, t_0 + 2 \pi]\), define \(\mathbf{b}_{K}^{t_0, t_1}\) as the curve \(\mathbf{b}_{K}^{t_0} (s)\) restricted on the interval \(s \in [0, g_{K}^{t_0}(t_1)]\).

\label{def:boundary-segment-parametrization}
\end{definition}

\subsubsection{Theorems on Parametrization}

We now show that \(\mathbf{b}_K^{t_0}\) does parametrize our boundary \(\partial K\) as intended. We prepare three technical lemmas that handle conversions between \(s\) and \(t\).

\begin{lemma}

The followings hold.

\begin{enumerate}
\def\labelenumi{\arabic{enumi}.}
\tightlist
\item
  For any \(t_1 \in (t_0, t_0 + 2\pi]\), we have \(\left(f_{K}^{t_0}\right)^{-1}([t_0, t_1]) = [0, \sigma_K\left( (t_0, t_1] \right)] = [0, g_{K}^{t_0}(t_1)]\).
\item
  Moreover, the set \(\left( f_{K}^{t_0} \right)^{-1} (\left\{ t_1 \right\})\) is either \([g_{K}^{t_0}(t_1-), g_{K}^{t_0}(t_1)]\) or \((g_{K}^{t_0}(t_1-), g_{K}^{t_0}(t_1)]\).
\end{enumerate}

\label{lem:parametrization-set-calculation}
\end{lemma}

\begin{proof}
Write \(f_K^{t_0}\) and \(g_K^{t_0}\) as simply \(f\) and \(g\). The first statement comes from manipulating the definitions as the following.

\begin{align*}
f^{-1}([t_0, t_1]) & = \left\{ s \in [0, B_K] : \min \left\{ t \geq t_0 : \sigma_K\left( (t_0, t] \right) \geq s \right\} \in [t_0, t_1] \right\} \\
& = \left\{ s \in [0, B_K] :  \sigma\left( (t_0, t_1] \right) \geq s \right\} \\
& = [0, \sigma_K((t_0, t_1])] = [0, g(t_1)]
\end{align*}
Now send \(t \to t_1^-\) in the equality \(f^{-1}([t_0, t]) = [0, g(t)]\) to obtain that \(f^{-1}([t_0, t_1)) = \bigcup_{t < t_1} [0, g(t)]\) is either \([0, g(t_1-))\) or \([0, g(t_1-)]\). Then use \(f^{-1} (\left\{ t_1 \right\}) = f^{-1}([t_0, t_1]) \setminus f^{-1}([t_0, t_1))\) to get the second statement.
\end{proof}

\begin{lemma}

The measure \(\sigma_K\) on \((t_0, t_0 + 2 \pi]\) is the pushforward of the Lebesgue measure on \((0, B_K]\) with respect to the map \(f_{K}^{t_0} : (0, B_K] \to (t_0, t_0 + 2 \pi]\) restricted to \((0, B_K]\).

\label{lem:parametrization-pushforward}
\end{lemma}

\begin{proof}
Write \(f_K^{t_0}\) as \(f\). Observe that \(f\) restricted to \((0, B_K]\) has range in \((t_0, t_0 + 2\pi]\) by \Cref{pro:conversion-st-nonzero}. The first statement of \Cref{lem:parametrization-set-calculation} then shows that the measure \(\sigma_K\) on \((t_0, t_0 + 2 \pi]\) and the pushforward of the Lebesgue measure on \((0, B_K]\) with respect to \(f : (0, B_K] \to (t_0, t_0 + 2 \pi]\) agree on every closed interval \((t_0, t]\) for all \(t \in (t_0, t_0 + 2\pi]\).
\end{proof}

\begin{lemma}

\(\mathbf{b}_{K}^{t_0}(g_{K}^{t_0}(t)) = v_{K}^+(t)\) for all \(t \in [t_0, t_0 + 2\pi]\) and \(\mathbf{b}_{K}^{t_0}(g_{K}^{t_0}(t-)) = v_{K}^-(t)\) for all \(t \in (t_0, t_0 + 2\pi]\). Moreover, for all \(t \in (t_0, t_0 + 2\pi]\) the function \(\mathbf{b}_{K}^{t_0}\) restricted to the interval \([g_{K}^{t_0}(t_1-), g_{K}^{t_0}(t_1)]\) is the arc-length parametrization of the edge \(e_K(t)\) from vertex \(v_K^-(t)\) to \(v_K^+(t)\).

\label{lem:parametrization-vertex}
\end{lemma}

\begin{proof}
Write \(f_K^{t_0}\) and \(g_K^{t_0}\) as simply \(f\) and \(g\). By \Cref{lem:parametrization-pushforward} and \Cref{thm:boundary-measure}, we have the following calculation.

\begin{align*}
\mathbf{b}_{K}^{t_0} (g(t)) & = v_K^+(t_0) + \int_{s' \in (0, g(t)]} v_{f(s')} \, ds' \\
& = v_K^+(t_0) + \int_{s' \in f^{-1}([t_0, t])} v_{f(s')} \, ds' \\
& = v_K^+(t_0) + \int_{t \in(t_0, t]} v_t \, \sigma(dt) = v^+_K(t)
\end{align*}
For the proof of \(\mathbf{b}_{K}^{t_0}(g_{K}^{t_0}(t-)) = v_{K}^-(t)\), send \(t' \to t^-\) in the equation \(\mathbf{b}_{K}^{t_0}(g_{K}^{t_0}(t')) = v_{K}^+(t')\) and use \Cref{thm:limits-converging-to-vertex}. By the second statement of \Cref{lem:parametrization-set-calculation}, the value \(f(s')\) on the interval \(s' \in (g(t-), g(t)]\) is always equal to \(t\). So the derivative of \(\mathbf{b}_K^{t_0}(s')\) restricted to the interval \([g(t-), g(t)]\) is almost everywhere equal to \(v_t\), and \(\mathbf{b}_{K}^{t_0}\) is the arc-length parametrization of the edge \(e_K(t)\) from vertex \(v_K^-(t)\) to \(v_K^+(t)\) on the interval \([g(t-), g(t)]\).
\end{proof}

We now prove the claimed theorems on \(\mathbf{b}_K^{t_0, t_1}\). That \(\mathbf{b}_K^{t_0, t_1}\) is injective will be proved later.

\begin{proof}[Proof of \Cref{thm:param-segment}]
Write \(f_K^{t_0}\) and \(g_K^{t_0}\) as simply \(f\) and \(g\). By \Cref{lem:parametrization-set-calculation}, the interval \([0, g(t_1)]\) is equal to the inverse image \(f^{-1} ([t_0, t_1])\), and so is the disjoint union of the singleton \(f^{-1} (\left\{ t_0 \right\}) = \left\{ 0 \right\}\) and the intervals \(f ^{-1} (\left\{ t \right\})\) whose closure is \([g(t-), g(t)]\) for all \(t \in (t_0, t_1]\). Under the map \(\mathbf{b}_{K}^{t_0}\), the singleton \(\left\{ 0 \right\}\) maps to \(\left\{ v_K^+(t_0) \right\}\) and the set \([g(t-), g(t)]\) maps to \(e_K(t)\) for all \(t \in (t_0, t_1]\) by \Cref{lem:parametrization-vertex}. This proves that the image of the interval \([0, g(t_1)]\) under the map \(\mathbf{b}_{K}^{t_0}\) is the set \(\left\{ v_K^+(t_0) \right\} \cup \bigcup_{t \in (t_0, t_1]} e_K(t)\).
\end{proof}

\begin{proof}[Proof of \Cref{thm:param-segment-length}]
This comes from \Cref{pro:parametrization-arc-length} and that the domain \([0, g_{K}^{t_0}(t_1)]\) of \(\mathbf{b}_K^{t_0, t_1}\) has length \(\sigma_K((t_0, t_1])\).
\end{proof}

\begin{proof}[Proof of \Cref{thm:param-concatenation}]
Write \(f_K^{t_0}\) and \(g_K^{t_0}\) as simply \(f\) and \(g\). The curve \(\mathbf{b}_{K}^{t_0, t_1}\) is an initial part of the curve \(\mathbf{b}_{K}^{t_0, t_2}\). So it remains to show that \(\mathbf{b}_{K}^{t_0}\) restricted to the interval \([g(t_1), g(t_2)]\) is the same as \(\mathbf{b}_{K}^{t_1}\) restricted to \([0, g_K^{t_1}(t_2)]\), with the domain shifted to right by \(g(t_1)\). Observe \(g(t_1) + g_K^{t_1}(t_2) = g(t_2)\) by \Cref{def:conversion-ts} and additivity of \(\sigma_K\). The initial point of the two curves is equal to \(v_K^+(t_1)\) by \Cref{lem:parametrization-vertex}. We show that the derivatives of \(\mathbf{b}_{K}^{t_0}(t + g(t_1))\) and \(\mathbf{b}_{K}^{t_1}(t)\) match for all \(t \in [0, g(t_2) - g(t_1)]\). By \Cref{def:parametrization-formal}, we only need to check \(f(t + g(t_1)) = f_K^{t_1} (t)\). This immediately follows from \Cref{def:conversion-st}.
\end{proof}

\begin{proof}[Proof of \Cref{thm:param-curve-area-functional}]
Write \(f_K^{t_0}\) and \(g_K^{t_0}\) as simply \(f\) and \(g\). Take any \(s \in (0, g(t_1)]\) and let \(t = f(s)\). Observe that by \Cref{pro:conversion-st-nonzero}, we have \(t \in (t_0, t_1]\) and \(s\) is in \(f^{-1}(\left\{ t \right\})\) which is either \((g(t_1 -), g(t_1)]\) or \([g(t_1 -), g(t_1)]\) by \Cref{lem:parametrization-set-calculation}. Then as \(\mathbf{b}_{K}^{t_0} (s) \in e_K(t)\) by \Cref{lem:parametrization-vertex}, we have \(\mathbf{b}_{K}^{t_0} (s) \times v_{t} = p_K(t)\). So we have the following.

\begin{align*}
\mathcal{I} \left( \mathbf{b}_{K}^{t_0, t_1} \right) & = \frac{1}{2} \int_{s \in (0, g(t_1)]} \mathbf{b}_{K}^{t_0} (s) \times v_{f(s)} \, ds \\
& = \frac{1}{2} \int_{s \in f^{-1}((t_0, t_1])} p_K(f(s)) \, ds \\
& = \frac{1}{2} \int_{t \in(t_0, t_1]} p_K(t) \, \sigma(dt)
\end{align*}
The first equality above uses \Cref{def:parametrization-formal}. The second equality above uses \Cref{lem:parametrization-set-calculation} and \(\mathbf{b}_{K}^{t_0} (s) \times v_{t} = p_K(t)\). The last equality above uses \Cref{lem:parametrization-pushforward}. This proves the first equality stated in the theorem. To show the second stated equality, check \(v_K(t) \times dv_K^+(t) = v_K^+(t) \times v_{t} \sigma_K(dt) = p_K(t) \sigma(dt)\) by \Cref{pro:boundary-measure-differential}.
\end{proof}

\subsubsection{Injectivity of Parametrization}

Proof of \Cref{thm:param-positive-jordan} requires a bit of preparation. The boundary \(\partial K\) is the union of all the edges.

\begin{theorem}

Let \(K\) be any convex body. Then the topological boundary \(\partial K\) of \(K\) as a subset of \(\mathbb{R}^2\) is the union of all edges \(\cup_{t \in S^1} e_K(t)\).

\label{thm:boundary-is-union-all-edges}
\end{theorem}

\begin{proof}
Let \(E = \cup_{t \in S^1} e_K(t)\). \(E \subseteq \partial K\) comes from \(E \subseteq K\) and that any point in \(E\) is on some tangent line of \(K\) so its neighborhood contains a point outside \(K\). Now take any point \(p \in \partial K\). As \(K\) is closed we have \(p \in K\). So \(p \cdot u_t \leq p_K(t)\) for any \(t \in S^1\). Assume that the equality does not hold for any \(t \in S^1\). Then by compactness of \(S^1\) and continuity of \(p_K\) there is some \(\epsilon > 0\) such that \(\epsilon \leq p_K(t) - p\cdot u_t\) for any \(t\). This implies that the ball of radius \(\epsilon\) centered at \(p\) is contained in \(K\). This contradicts \(p \in \partial K\). So it should be that there is some \(t \in S^1\) such that \(p \cdot u_t = p_K(t)\). That is, \(p \in e_K(t)\).
\end{proof}

We define the following segment of \(\partial K\) as well.

\begin{definition}

For any \(t_0, t_1 \in \mathbb{R}\) such that \(t_1 \in [t_0, t_0 + 2 \pi]\), define \(\mathbf{b}_{K}^{t_0, t_1-}\) as the curve \(\mathbf{b}_{K}^{t_0} (s)\) restricted on the interval \(s \in [0, g_{K}^{t_0}(t_1-)]\).

\label{def:param-segment-open}
\end{definition}

By \Cref{lem:parametrization-vertex}, the curve \(\mathbf{b}_K^{t_0, t_1-}\) ends with the vertex \(v_K^-(t_1)\). Moreover, we have the following corollary of the same \Cref{lem:parametrization-vertex}.

\begin{corollary}

For any \(t_0, t_1 \in \mathbb{R}\) such that \(t_1 \in [t_0, t_0 + 2 \pi]\), \(\mathbf{b}_K^{t_0, t_1}\) is the concatenation of \(\mathbf{b}_K^{t_0, t_1 - }\) and the arc-length parametrization of \(e_K(t_1)\) from \(v_K^-(t_1)\) to \(v_K^+(t_1)\).

\label{cor:param-segment-open}
\end{corollary}

By \Cref{def:parametrization-formal} we have \(\left(\mathbf{b}_{K}^{t_0}\right)'(s) = u_{f_K^{t_0}(s)}\) for almost every \(s\), and by \Cref{pro:conversion-st-nonzero} and \Cref{lem:parametrization-set-calculation} we have \(t_0 < f_K^{t_0}(s) < t_1\) for every \(0 < s < g_{K}^{t_0}(t_1-)\). Thus we have the following:

\begin{corollary}

Let \(t_0, t_1 \in \mathbb{R}\) be arbitrary such that \(t_1 \in [t_0, t_0 + 2 \pi]\). Then for almost every \(s\), the derivative \(\left( \mathbf{b}_{K}^{t_0, t_1-} \right)'(s)\) is equal to \(u_t\) for some \(t \in (t_0, t_1)\).

\label{cor:param-segment-open-deriv}
\end{corollary}

We use the following lemma to determine the orientation of a Jordan curve.

\begin{lemma}

Let \(p\) and \(q\) be two different points of \(\mathbb{R}^2\). Define the closed half-planes \(H_0\) and \(H_1\) as the closed half-planes separated by the line \(l\) connecting \(p\) and \(q\), so that for any point \(x\) in the interior of \(H_0\) (resp. \(H_1\)) the points \(x, p, q\) are in clockwise (resp. counterclockwise) order. If a Jordan curve \(J\) consists of the join of two arcs \(\Gamma_0\) and \(\Gamma_1\), where \(\Gamma_0\) connects \(p\) to \(q\) inside \(H_0\), and \(\Gamma_1\) connects \(q\) to \(p\) inside \(H_1\), then \(J\) is positively oriented.

\label{lem:orientation}
\end{lemma}

\begin{proof}
(sketch) We first show that it is safe to assume the case where \(J\) only intersects \(l\) at two points \(p\) and \(q\). Observe that \(H_i\) has a deformation retract to some subset \(S_i \subseteq H_i\) with \(S_i \cap l = \left\{ p, q \right\}\) (push the three segments of \(l \setminus \{p, q\}\) towards the interior of \(H_i\)). Using the retracts, we may continuously deform the arcs \(\Gamma_0\) and \(\Gamma_1\) inside \(S_0\) and \(S_1\) respectively without chainging the orientation of \(J\). Now take any point \(r\) inside the segment connecting \(p\) and \(q\). Continuously move a point \(x\) inside \(J\) in the orientation of \(J\) starting with \(x = p\). As \(x\) moves along \(\Gamma_0\) from \(p\) to \(q\) the argument of \(x\) with respect to \(r\) increases by \(\pi\). And as \(x\) moves along \(\Gamma_1\) the argument of \(x\) with respect to \(r\) again increases by \(\pi\). So the total increase in the argument of \(x \in J\) is \(2\pi\) and \(J\) is positively oriented.
\end{proof}

Now we are ready to prove \Cref{thm:param-positive-jordan}.

\begin{proof}[Proof of \Cref{thm:param-positive-jordan}]
That \(\mathbf{b}_K^{t, t + 2\pi}\) is an arc-length parametrization of \(\partial K\) comes from \Cref{thm:param-segment} and \Cref{thm:boundary-is-union-all-edges}.

We now show that \(\mathbf{b}_K^{t, t + 2\pi}\) is a Jordan curve. By \Cref{thm:param-concatenation} the curve \(\mathbf{b}_K^{t, t + 2\pi}\) is the concatenation of two curves \(\mathbf{b}_K^{t, t + \pi}\) and \(\mathbf{b}_K^{t + \pi, t + 2\pi}\) connecting \(p = v_{K}^+(t)\) and \(q = v_K^+(t + \pi)\) and vice versa. As \(K\) has nonempty interior, the width of \(K\) measured in the direction of \(u_t\) is strictly positive, and the point \(p\) is strictly further than the point \(q\) in the direction of \(u_t\).

We first show that the curve \(\mathbf{b}_K^{t, t + \pi}\) is a Jordan arc from \(p\) to \(q\). The curve \(\mathbf{b}_K^{t, t + \pi}\) is the join of the curve \(\mathbf{b}_K^{t, t + \pi-}\) and \(e_{K}(t + \pi)\) by \Cref{cor:param-segment-open}. Also, by \Cref{cor:param-segment-open-deriv}, the derivative of \(\mathbf{b}_K^{t, t + \pi-}(s) \cdot u_t\) with respect to \(s\) is strictly positive for almost every \(s\), so the curve \(\mathbf{b}_K\) is moving strictly in the direction of \(-u_t\). This with the fact that \(e_K(t + \pi)\) is parallel to \(v_t\) shows that the curve \(\mathbf{x}_{K, t, t + \pi}\) is injective and thus a Jordan arc. A similar argument shows that \(\mathbf{b}_K^{t + \pi, t + 2\pi}\) is also a Jordan arc.

Define the closed half-planes \(H_0\) and \(H_1\) as the half-planes divided by the line \(l\) connecting \(p\) and \(q\), so that for any point \(x\) in the interior of \(H_0\) (resp. \(H_1\)) the points \(x, p, q\) are in clockwise (resp. counterclockwise) order. Observe that \(\mathbf{b}_K^{t, t + \pi}\) (resp. \(\mathbf{b}_K^{t + \pi, t + 2\pi}\)) is in \(H_0\) (resp. \(H_1\)) by \Cref{thm:param-segment}. Let \(\mathbf{b}\) be either of the curves \(\mathbf{b}_K^{t, t + \pi}\) or \(\mathbf{b}_K^{t + \pi, t + 2\pi}\). Call the line segment connecting \(p\) and \(q\) as \(pq\). Then \(\mathbf{b}\) is either \(pq\) (in case \(\mathbf{b}\) passes through a point \(r\) strictly on \(pq\)) or a curve connecting \(p\) and \(q\) through the interior of \(H_0\) (or \(H_1\)). In any case, the curves \(\mathbf{b}_K^{t, t + \pi}\) and \(\mathbf{b}_K^{t + \pi, t + 2\pi}\) only overlap at the endpoints \(p\) and \(q\) because \(K\) has nonempty interior, showing that \(\mathbf{b}_K^{t, t + 2\pi}\) is a Jordan curve. That \(\mathbf{b}_K^{t, t + 2\pi}\) is positively oriented is a consequence of \Cref{lem:orientation}.
\end{proof}

\subsubsection{Parametrization on Closed Interval}

We define the closed-interval variant \(\mathbf{b}_K^{t_0-, t_1}\) of the curve \(\mathbf{b}_K^{t_0, t_1}\) as essentially the arc-length parametrization of the curve connecting \(v_K^-(t_0)\) to \(v_K^+(t_1)\) along the boundary \(\partial K\) counterclockwise.

\begin{definition}

For every \(t_0 \in \mathbb{R}\) and \(t_1 \in [t_0, t_0 + 2\pi)\) define \(\mathbf{b}_K^{t_0-, t_1}\) as the concatenation of the arc-length parametrization of the edge \(e_K(t_0)\) from \(v_K^-(t_0)\) to \(v_K^+(t_1)\) and the curve \(\mathbf{b}_K^{t_0, t_1}\).

\label{def:closed-param}
\end{definition}

This follows from \Cref{thm:param-segment}.

\begin{corollary}

Assume arbitrary \(t_0 \in \mathbb{R}\) and \(t_1 \in [t_0, t_0 + 2\pi)\). Then \(\mathbf{b}_K^{t_0-, t_1}\) is an arc-length parametrization of the set \(\cup_{t \in [t_0, t_1]} e_K(t)\) from point \(v_K^-(t_0)\) to \(v_K^+(t_1)\).

\label{cor:closed-param-segment}
\end{corollary}

This follows from \Cref{thm:param-segment-length}.

\begin{corollary}

Assume arbitrary \(t_0 \in \mathbb{R}\) and \(t_1 \in [t_0, t_0 + 2\pi)\). Then the curve \(\mathbf{b}_K^{t_0-, t_1}\) have length \(\sigma_K([t_0, t_1])\).

\label{cor:closed-param-segment-length}
\end{corollary}

This follows from \Cref{thm:param-concatenation}.

\begin{corollary}

Assume arbitrary \(t_0, t_1, t_2\) such that \(t_0 \leq t_1 \leq t_2 < t_0 + 2\pi\). Then \(\mathbf{b}_{K}^{t_0-, t_2}\) is the concatenation of \(\mathbf{b}_{K}^{t_0-, t_1}\) and \(\mathbf{b}_{K}^{t_1, t_2}\).

\label{cor:closed-param-concatenation}
\end{corollary}

This follows from \Cref{thm:param-curve-area-functional} and \Cref{thm:surface-area-singleton}.

\begin{corollary}

Assume arbitrary \(t_0 \in \mathbb{R}\) and \(t_1 \in [t_0, t_0 + 2\pi)\). Then we have:
\[
\mathcal{I} \left( \mathbf{b}_{K}^{t_0-, t_1} \right) = \frac{1}{2} \int_{[t_0, t_1]}p_K(t)\,\sigma_K(dt)
\]

\label{cor:closed-param-curve-area-functional}
\end{corollary}

\begin{theorem}

Assume that \(K\) have nonempty interior. Assume arbitrary \(t_0 \in \mathbb{R}\) and \(t_1 \in [t_0, t_0 + 2\pi)\). Then \(\mathbf{b}_K^{t_0-, t_1}\) is one of: a Jordan arc, a Jordan curve, or a single point.

\label{thm:closed-param-positive-jordan}
\end{theorem}

\begin{proof}
Take an arbitrary \(t_{-1}\) so that \(t_{-1} < t_0 \leq t_1 < t_{-1} + 2\pi\). Then \(\mathbf{b}_K^{t_{-1}, t_1}\) is the concatenation of \(\mathbf{b}_K^{t_{-1}, t_0-}\), \(e_K(t_0)\), and \(\mathbf{b}_K^{t_0, t_1}\) by \Cref{thm:param-concatenation} and \Cref{cor:param-segment-open}. Then by \Cref{def:closed-param} the concatenation of \(e_K(t_0)\), and \(\mathbf{b}_K^{t_0, t_1}\) is \(\mathbf{b}_K^{t_0-, t_1}\). Now by \Cref{cor:param-positive-jordan} and that \(\mathbf{b}_K^{t_0-, t_1}\) is a part of \(\mathbf{b}_K^{t_{-1}, t_1}\), we prove the theorem.
\end{proof}

\subsection{Normal angles}
\label{sec:normal-angles}
\input{out/A1. Convex bodies/20. Normal angles.tex}

\subsection{Mamikon's theorem}
\label{sec:mamikon's-theorem}
\input{out/A1. Convex bodies/25. Mamikon's theorem.tex}





\printbibliography

\end{document}
