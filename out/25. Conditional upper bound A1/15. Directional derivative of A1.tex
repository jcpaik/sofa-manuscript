In this section, we calculate the directional derivative \(D\mathcal{A}_1(K; -)\) of \(\mathcal{A}_1\) (\Cref{def:convex-space-directional-derivative}) at any \(K \in \mathcal{K}_{\omega}\). As \(\mathcal{A}_1(K) = \left| K \right| - \mathcal{I}(\mathbf{x}_K)\), we calculate the directional derivative of \(|K|\) and \(\mathcal{I}(\mathbf{x}_K)\) separately.

First we calculate the derivative of the area \(|K|\) with respect to \(K\).

\begin{theorem}

Let \(K\) and \(K'\) be arbitrary convex bodies. Then we have
\[
\left. \frac{d}{d \lambda} \right|_{\lambda=0} \left| (1-\lambda) K + \lambda K' \right|  = \left< p_{K'} - p_K , \beta_K \right>_{S^1}.
\]

\label{thm:convex-body-area-variation}
\end{theorem}

\begin{proof}
For any convex body \(K\) we have \(|K| = V(K, K)\) where \(V\) is the mixed volume of two planar convex bodies. So by applying \Cref{lem:derivative-calculation} to \(|K| = V(K, K)\) and using that \(V(K, K') = V(K', K)\), we have the following.
\[
\left. \frac{d}{d \lambda} \right|_{\lambda=0} \left| (1-\lambda) K + \lambda K' \right| = 2V(K', K) - 2V(K, K)  
\]
By applying \Cref{thm:surface-area-measure-area} we get the result.
\end{proof}

We now calculate the derivative of \(\mathcal{I}(\mathbf{x}_K)\) with respect to \(K\). We have the following general calculation for the curve area functional \(\mathcal{I}(\mathbf{x})\).

\begin{theorem}

Let \(\mathbf{x}_1, \mathbf{x}_2 : [a, b]\to\mathbb{R}^2\) be two rectifiable curves. Then the following holds.
\[
\left. \frac{d}{d \lambda} \right|_{\lambda = 0} \mathcal{I}((1 - \lambda)\mathbf{x}_1 + \lambda \mathbf{x}_2) = \left[ \int_a^b (\mathbf{x}_2(t) - \mathbf{x}_1(t))  \times d\mathbf{x}_1 (t) \right] +  \mathcal{I}(\mathbf{x}_1(b), \mathbf{x}_2(b)) - \mathcal{I}(\mathbf{x}_1(a), \mathbf{x}_1(a))
\]

\label{thm:variation-curve}
\end{theorem}

\begin{proof}
Consider the bilinear form \(\mathcal{J}(\mathbf{x}_1, \mathbf{x}_2) = \int_a ^b \mathbf{x}_1(t) \times d \mathbf{x}_2(t)\) on rectifiable \(\mathbf{x}_1, \mathbf{x}_2 : [a, b] \to \mathbb{R}^2\). Apply \Cref{lem:derivative-calculation} to \(2\mathcal{I}(\mathbf{x}) = \mathcal{J}(\mathbf{x}, \mathbf{x})\) to get
\begin{equation}
\label{eqn:variation-curve}
\left. \frac{d}{d \lambda} \right|_{\lambda = 0} 2\mathcal{I}((1 - \lambda)\mathbf{x}_1 + \lambda \mathbf{x}_2) = \mathcal{J}(\mathbf{x}_1, \mathbf{x}_2) + \mathcal{J}(\mathbf{x}_2, \mathbf{x}_1) - 2 \mathcal{J}(\mathbf{x}_1, \mathbf{x}_1).
\end{equation}
Using the integration by parts (\Cref{pro:lebesgue-stieltjes-product}), we have
\[
\int_a^b \mathbf{x}_1(t) \times d \mathbf{x}_2(t) = \mathbf{x}_1 (b) \times \mathbf{x}_2(b) - \mathbf{x}_1(a) \times \mathbf{x}_2(a) + \int_a^b \mathbf{x}_2(t) \times d\mathbf{x}_1 (t)
\]
or
\[
\mathcal{J}(\mathbf{x}_1, \mathbf{x}_2) = 2\mathcal{I}(\mathbf{x}_1(b), \mathbf{x}_2(b)) - 2\mathcal{I}(\mathbf{x}_1(a) - \mathbf{x}_2(a)) + \mathcal{J}(\mathbf{x}_2, \mathbf{x}_1).
\]
Plug this back in \Cref{eqn:variation-curve} and rearrange to get the claimed equality in \Cref{thm:variation-curve}.
\end{proof}

We introduce a measure \(\iota_K\) to calculate the derivative of \(\mathcal{I}(\mathbf{x}_K)\) with respect to cap \(K\).

\begin{definition}

For any cap \(K \in \mathcal{K}_{\omega}\), define the function \(i_K : J_\omega \to \mathbb{R}\) as \(i_K(t) = h_K^+(t) - 1\) and \(i_K(t + \pi / 2) = g^+_K(t) - 1\) for every \(t \in [0, \omega]\). Define \(\iota_K\) as the measure on \(J_\omega\) derived from the density function \(i_K\). That is, \(\iota_K(dt) = i_K(t) dt\).

\label{def:i-cap}
\end{definition}

\Cref{def:i-cap} is motivated by the following lemma.

\begin{lemma}

Let \(I \subseteq [0, \omega]\) be an arbitrary Borel subset. Let \(K_1, K_2 \in \mathcal{K}_{\omega}\) be arbitrary. Then the following holds.
\[
\int_{t \in I} \mathbf{x}_{K_1}(t) \times d \mathbf{x}_{K_2} (t) = \left< p_{K_1} - 1, \iota_{K_2} \right>_{I \cup (I + \pi/2)} 
\]

\label{lem:i-cap}
\end{lemma}

\begin{proof}
By \Cref{thm:inner-corner-deriv} and \Cref{pro:arm-length-unsigned}, the derivative of \(\mathbf{x}_{K_2}(t)\) with respect to \(t\) exists almost everywhere and is the following.
\[
\mathbf{x}'_{K_2}(t) = -(g_{K_2}^+(t) - 1) u_t + (h_{K_2}^+(t) - 1) v_t
\]
Meanwhile, we have the following.
\[
\mathbf{x}_{K_1}(t) = (p_{K_1} (t) - 1) u_t + 
(p_{K_1} (t + \pi / 2) - 1) v_t
\]
So the cross-product \(\mathbf{x}_{K_1}(t) \times \mathbf{x}_{K_2}'(t)\) is equal to the following almost everywhere.
\[
(h_{K_2}^+(t) - 1) (p_{K_1} (t) - 1) + (g_{K_2}^+(t) - 1) (p_{K_1} (t + \pi / 2) - 1)
\]
Now the left-hand side is equal to
\[
\int_{t \in J} (h_{K_2}^+(t) - 1) (p_{K_1} (t) - 1) + (g_{K_2}^+(t) - 1) (p_{K_1} (t + \pi / 2) - 1) \, dt
\]
and by \Cref{def:i-cap} this integral is equal to \(\left< p_{K_1} - 1, \iota_{K_2} \right>_{I \cup (I + \pi/2)}\).
\end{proof}

\begin{theorem}

Let \(K_1\) and \(K_2\) be two caps in \(\mathcal{K}_{\omega}\). Then we have the following.
\[
D\mathcal{A}_1(K_1; K_2) = \left. \frac{d}{d\lambda} \right|_{\lambda = 0} \mathcal{A}_1((1 - \lambda) K_1 + \lambda K_2)
= \left< p_{K_2} - p_{K_1}, \beta_{K_1} - \iota_{K_1} \right>_{J_\omega}
\]

\label{thm:variation-a1}
\end{theorem}

\begin{proof}
We have \(\mathcal{A}_1(K) = |K| - \mathcal{I}(\mathbf{x}_K)\). Apply \Cref{thm:convex-body-area-variation} to the term \(|K|\) to have the following.
\[
\left. \frac{d}{d \lambda} \right|_{\lambda=0} \left| (1-\lambda) K_1 + \lambda K_2 \right|  = \left< p_{K_2} - p_{K_1} , \beta_{K_1} \right>_{S^1}
\]
Note that \(\beta_{K_1}\) and \(\beta_{K_2}\) are supported on the set \(J_\omega \cup \left\{ \pi + \omega, 3\pi/2 \right\}\), and both \(p_{K_2}\) and \(p_{K_1}\) have function value equal to 1 on the set \(\left\{ \pi + \omega, 3\pi/2 \right\}\). So we have \(\left< p_{K_2} - p_{K_1} , \beta_{K_1} \right>_{S^1} = \left< p_{K_2} - p_{K_1} , \beta_{K_1} \right>_{J_\omega}\).

Apply \Cref{thm:variation-curve} to the term \(\mathcal{I}(\mathbf{x}_K)\), and use that the points \(O\), \(\mathbf{x}_{K}(0)\), and \(A^-_K(0)\) (respectively, the points \(O\), \(\mathbf{x}_K(\omega)\), and \(C_K^+(\omega)\)) are colinear to get the following.

\begin{align*}
\left. \frac{d}{d \lambda} \right|_{\lambda = 0} \mathcal{I}((1 - \lambda)\mathbf{x}_{K_1} + \lambda \mathbf{x}_{K_2}) & = \int_0^{\omega} (\mathbf{x}_{K_2}(t) - \mathbf{x}_{K_1}(t)) \times d\mathbf{x}_{K_1}(t) \\
& = \left< p_{K_2} - p_{K_1}, \iota_{K_1} \right> 
\end{align*}
The second equality comes from applying \Cref{lem:i-cap} twice. Subtract the derivates of \(|K|\) and \(\mathcal{I}(\mathbf{x}_K)\) above to conclude the proof.
\end{proof}

We explain the intuitive meaning of \Cref{thm:variation-a1} by comparing it to the local optimization argument of Theorem 2 in \autocite{romikDifferentialEquationsExact2018}. Assume for the sake of explanation that \(S\) is a monotone sofa of rotation angle \(\pi/2\) with cap \(K\), such that the niche \(\mathcal{N}(K)\) is exactly the region bounded by the curve \(\mathbf{x}_K(t)\). Assume that our cap \(K\) has vertices \(A_K(t) = A_K^{\pm}(t)\) and \(C_K(t) = C_K^{\pm}(t)\) continuously differentiable with respect to \(t\). Assume also that \(\mathbf{x}_K(t)\) is continuously differentiable.

Take an arbitrary angle \(t_0\) and fix small positive \(\delta\) and \(\epsilon\). In \autocite{romikDifferentialEquationsExact2018}, Romik pertubed the sofa \(S\) to obtain a new sofa \(S'\) as the following. Initially, the monotone sofa \(S = H \cap \bigcap_{0 \leq t \leq \pi/2} L_K(t)\) is the intersection of rotating hallways \(L_K(t)\). For every \(t \in [0, \pi/2]\), Romik pertubed each hallway \(L_K(t)\) to a new hallways \(L'(t)\) as the following.

\begin{itemize}
\tightlist
\item
  For every \(t \in [t_0, t_0 + \delta]\), let \(L'(t) = L_K(t) + \epsilon u_t\).
\item
  For every other \(t\), let \(L'(t) = L_K(t)\).
\end{itemize}

That is, we move \(L_S(t)\) in the direction of \(\epsilon u_t\) for only \(t \in [t_0, t_0 + \delta]\). Now define \(S' = H \cap \bigcap_{0 \leq u \leq \pi/2} L'(u)\) so that \(S'\) is a sofa which is a slight perturbation of \(S\). If \(S\) attains the maximum area, it should be that \(S'\) has area equal to or less than \(S\).

We now compare the area of \(S\) and \(S'\). As we perturb \(S\) to \(S'\), some area is gained near \(A_K(t_0)\) as the walls \(a_K(t)\) are pushed in the direction of \(\epsilon u_t\) for \(t \in [t_0, t_0 + \delta]\). The gain near \(A_K(t_0)\) is approximately \(\epsilon \delta \left\lVert A_K'(t_0) \right\rVert\), as the shape of the gain is approximately a rectangle of sides \(A_K(t_0 + \delta) - A_K(t_0) \simeq \delta \left\lVert A_K'(t_0) \right\rVert v_0\) and \(\epsilon u_{t_0}\). Likewise, some area is lost near \(\mathbf{x}_K(t_0)\) as we perturb \(S\) to \(S'\) as the corners \(\mathbf{x}_K(t)\) are pushed in the direction of \(\epsilon u_t\) for \(t \in [t_0, t_0 + \delta]\). The loss near \(\mathbf{x}_K(t_0)\) is \(\epsilon \delta \mathbf{x}_K'(t_0) \cdot v_{t_0}\) as the shape of the loss is approximately a parallelogram of sides \(\delta \mathbf{x}_K'(t_0)\) and \(\epsilon u_{t_0}\). So the total gain of area from \(S\) to \(S'\) is approximately \(\epsilon \delta \left( \left\lVert A_K'(t_0) \right\rVert - \mathbf{x}'(t_0) \cdot v_{t_0} \right)\). In \autocite{romikDifferentialEquationsExact2018}, Romik solved for the critical condition \(\left\lVert A_K'(t_0) \right\rVert = \mathbf{x}_K'(t_0) \cdot v_{t_0}\) (and another condition \(\left\lVert C_K'(t_0) \right\rVert = - \mathbf{x}_K'(t_0) \cdot u_{t_0}\) obtained by perturbing each \(L_K(t)\) in the orthogonal direction of \(\epsilon v_{t_0}\)) to derive an ordinary differential equation of \(\mathbf{x}_K\) (ODE3 of Theorem 2, \autocite{romikDifferentialEquationsExact2018}).

We now observe that this total gain of area \(\epsilon \delta \left\lVert A_K'(t) \right\rVert - \epsilon \delta \mathbf{x}'(t) \cdot v_t\) is captured in \Cref{thm:variation-a1}. The perturbation of hallways from \(L_K(t)\) to \(L'(t)\) in \(\epsilon u_t\) can be described in terms of their support functions as \(p_{K'} = p_K + \epsilon 1_{[t, t + \delta]}\). Correspondingly, the value \(\left< p_{K'} - p_{K}, \beta_{K} \right>_{[0, \pi]} = \epsilon \beta_{K}((t, t + \delta])\) is approximately \(\epsilon \delta \left\lVert A_K'(t) \right\rVert\) which is equal to the gain of \(|K|\) near \(A_K(t)\). The value \(\left< p_{K'} - p_{K}, \iota_{K} \right>_{[0, \pi]} = \epsilon \iota_{K}((t, t + \delta])\), by \Cref{thm:inner-corner-deriv} and \Cref{def:i-cap}, is approximately \(\epsilon \delta \mathbf{x}_{K}'(t) \cdot v_t\) which is equal to the loss of area by the gain of \(\mathcal{N}(K)\) near \(\mathbf{x}_K(t)\). Summing up, we have \(\left< p_{K'} - p_{K}, \beta_{K} - \iota_{K} \right>_{[0, \pi]}\) measuring the total difference in the area of \(S\).

To summarize, the values \((p_{K'} - p_{K})(t)\) and \((p_{K'} - p_{K})(t + \pi/2)\) measures the movement of \(\mathbf{x}_K(t)\) along the direction \(u_t\) and \(v_t\) respectively. Then the measure \(\beta_{K}\) near \(t\) and \(t + \pi/2\) respectively measures the differential side lengths of the boundary of \(K\) near \(A_K(t)\) and \(C_K(t)\) respectively. Likewise, \(i_K(t)\) and \(i_K(t + \pi/2)\) measures the component of \(\mathbf{x}'(t)\) in direction of \(v_t\) and \(-u_t\) respectively. The formula in \Cref{thm:variation-a1} multiplies the contribution of change in \(p_K\) with , to get the change in total area of \(S\) for angle \(t\). Then the formula in \Cref{thm:variation-a1} sums this change in area over all \(t\) to get the derivative of \(\mathcal{A}_1(K)\) with respect to \(K\).