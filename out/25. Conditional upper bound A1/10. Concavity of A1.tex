We now show that \(\mathcal{A}_1\) is concave.

\begin{theorem}

For every rotation angle \(\omega \in (0, \pi/2]\), the functional \(\mathcal{A}_1 : \mathcal{K}_\omega \to \mathbb{R}\) is concave.

\label{thm:a1-negative-semidefinite}
\end{theorem}

\emph{Mamikon’s theorem} \autocite{mnatsakanianAnnularRingsEqual1997} is used to prove \Cref{thm:a1-negative-semidefinite}. To explain Mamikon’s theorem, first assume an arbitrary convex body \(K\) (see \Cref{fig:mamikon}). Also, for every angle \(t\) in a fixed interval \([t_0, t_1]\), assume a tangent segment \(s(t)\) of \(K\) of length \(f(t)\). The segment \(s(t)\) has an endpoint \(v_K^+(t)\) on \(\partial K\), and the other endpoint \(\mathbf{y}(t)\) on the tangent line \(l_K(t)\) of \(K\) with angle \(t\) from the \(y\)-axis. Mamikon’s theorem states that the area swept by the segment \(s(t)\) from \(t=t_1\) to \(t=t_2\) has an area of \(\frac{1}{2}\int_{t_1}^{t_2} f(t)\,dt\).

\begin{figure}
\centering
\includesvg[width=0.7\textwidth,height=\textheight]{images/mamikon.svg}
\caption{An illustration of Mamikon’s theorem (\Cref{thm:mamikon}).}
\label{fig:mamikon}
\end{figure}

For the proof of \Cref{thm:a1-negative-semidefinite}, we will use \Cref{thm:mamikon} and \Cref{thm:mamikon-closed} which are the precise statements of Mamikon’s theorem that work for any convex body \(K\) with potentially non-differentiable boundaries (see \Cref{sec:mamikon's-theorem} for details). Using Mamikon’s theorem, we will express the area \(\mathcal{S}(K)\) of a. Then we will use the following \Cref{lem:sum-of-squares} to show that \(\mathcal{S}(K)\) is \emph{concave} with respect to \(K\). By showing that \(\mathcal{S}(K) + \mathcal{A}_1(K)\) is linear with respect to \(K\),

\begin{lemma}

Let \(h : \mathcal{K} \to V\) be a convex-linear map from a convex space \(\mathcal{K}\) to a real vector space \(V\) with inner product \(\left< -, - \right>_V\). Then the quadratic form \(f\) on \(\mathcal{K}\) defined as \(f(K) = \left< h(K), h(K) \right>_V\) is convex.

\label{lem:sum-of-squares}
\end{lemma}

\begin{proof}
Take arbitrary \(K_1, K_2 \in \mathcal{K}\). Fixing \(K_1\) and \(K_2\), observe that \(f(c_\lambda(K_1, K_2))\) is a quadratic polynomial of \(\lambda \in [0, 1]\) with the leading coefficient \(\left\lVert h(K_2) - h(K_1) \right\rVert_V^2\) of \(\lambda^2\), by expanding the term \(h(c_\lambda(K_1, K_2)) = h(K_1) + \lambda (h(K_2) - h(K_1))\) with respect to the inner product \(\left< -, - \right>_V\). This shows the convexity of \(f\) along the line segment connecting \(K_1\) and \(K_2\). Since \(K_1\) and \(K_2\) are chosen arbitrarily, \(f\) is convex.
\end{proof}

\begin{figure}
\centering
\includesvg[width=0.5\textwidth,height=\textheight]{images/mamikon-sofa.svg}
\caption{Mamikon’s theorem applied to the region with area \(\mathcal{S}(K)\), bounded from below by the upper boundary \(\delta K\) of cap \(K\), and bounded from above by outer corner \(\mathbf{y}_K : [0, \omega] \to \mathbb{R}^2\) of tangent hallways \(L_K(t)\).}
\label{fig:mamikon-sofa}
\end{figure}

\begin{proof}[Proof of \Cref{thm:a1-negative-semidefinite}]
(See \Cref{fig:mamikon-sofa}) We will define \(\mathcal{S}(K)\) as the area of the region bounded from below by the upper boundary \(\delta K\) of cap \(K\), and bounded from above by curve \(\mathbf{y}_K : [0, \omega] \to \mathbb{R}^2\). We will express \(\mathcal{S}(K)\) as an integral of squares using Mamikon’s theorem, then use \Cref{lem:sum-of-squares} to show that \(\mathcal{S}(K)\) is convex with respect to \(K\). Then we will show that \(\mathcal{S}(K) + \mathcal{A}_1(K)\) is linear with respect to \(K\). This will complete the proof of concavity of \(\mathcal{A}_1(K)\).

We first show \(|K| = \mathcal{I}(\delta K)\), where \(\delta K\) is the upper boundary of \(K\). By \Cref{cor:upper-boundary-param}, \(\delta K\) is the segment \(\mathbf{b}_K^{0-, \pi/2 + \omega}\) of the whole boundary \(\partial K\). By \Cref{thm:param-positive-jordan} and \Cref{thm:param-positive-area}, we have \(|K| = \mathcal{I}(\partial K)\). By the second condition of \Cref{def:cap}, the boundary \(\partial K\) is the concatenation of \(\delta K\) and two line segments \(e_K(3\pi/2)\), \(e_K(\pi + \omega)\) if \(\omega < \pi/2\), or one line segment \(e_K(3\pi/2)\) if \(\omega = \pi/2\). In either case, the line segments are aligned with the origin \(O\), so their curve area functionals (\Cref{def:curve-area-functional-segment}) are zero. This shows \(|K| = \mathcal{I}(\delta K)\) as we wanted.

Next, we subdivide the upper boundary \(\delta K = \mathbf{b}_K^{0-, \pi/2+\omega}\) into two curves \(\mathbf{b}_1 := \mathbf{b}_K^{0-, \omega}\) and \(\mathbf{b}_2 := \mathbf{b}_K^{\omega, \omega + \pi/2}\) using \Cref{cor:closed-param-concatenation}. The curve \(\mathbf{b}_1\) is the path from \(A_K^-(0)\) to \(A_K^+(\omega)\) along \(\delta K\), and the curve \(\mathbf{b}_2\) is the path from \(A_K^+(\omega)\) to \(C_K^+(\omega)\) along \(\delta K\). Now we have \(|K| = \mathcal{I}(\delta K) = \mathcal{I}(\mathbf{b}_1) + \mathcal{I}(\mathbf{b}_2)\).

We now show that the area \(\mathcal{S}(K)\) of the region bounded from below by \(\delta K\) and bounded from above by \(\mathbf{y}_K : [0, \omega] \to \mathbb{R}^2\) is convex with respect to \(K\). The region is enclosed two curves \(\delta K\), \(\mathbf{y}_K\) and two line segment from \(\mathbf{y}_K(\omega)\) to \(C_K^+(\omega)\) and from \(A_K^-(0)\) to \(\mathbf{y}_K(0)\) respectively. Accordingly, define \(\mathcal{S}(K)\) as the value
\[
\mathcal{S}(K) := \mathcal{I}(\mathbf{y}_K) + \mathcal{I}(\mathbf{y}_K(\omega), C_K^+(\omega)) - \mathcal{I}(\delta K) - \mathcal{I}(\mathbf{y}_K(0), A_K^-(0)).
\]

We will express \(\mathcal{S}(K)\) as a sum of integrals of squares by stitching two instances of Mamikon’s theorem on curves \(\mathbf{b}_1\) and \(\mathbf{b}_2\) respectively. First, apply \Cref{thm:mamikon-closed} to the curve \(\mathbf{b}_1\) and the outer corner \(\mathbf{y}_K(t)\) for \(t \in [0, \omega]\). Note that the value \(h_K^+(t)\) in \Cref{def:cap-tangent-arm-length} measures the distance from the point \(v_K^+(t)\) on \(\mathbf{b}_1\) to \(\mathbf{y}_K(t)\). Now we get

\begin{align*}
\mathcal{I}(\mathbf{y}_K) + \mathcal{I}(\mathbf{y}_K(\omega), A_K^+(\omega)) - \mathcal{I}(\mathbf{b}_1) - \mathcal{I}(\mathbf{y}_K(0), A_K^-(0)) & = \frac{1}{2} \int_0^\omega h^+_K(t)^2 \, dt.
\end{align*}
Second, apply \Cref{thm:mamikon-tangent-line} to the curve \(\mathbf{b}_2\) and the tangent line \(c_K(\omega)\) (which is \(l_K(\pi/2 + \omega)\) by \Cref{pro:rotating-hallway-parts}) of \(K\) with angle range \(t \in [\omega, \pi/2 + \omega]\), to get

\begin{align*}
\mathcal{I}(\mathbf{y}_K(\omega), C^+_K(\omega)) -
\mathcal{I}(\mathbf{b}_2) - 
\mathcal{I}(\mathbf{y}_K(\omega), A_K^+(\omega))
& = \frac{1}{2} \int_{\omega}^{\pi/2 + \omega} \tau_K(t, \omega + \pi/2)^2 \, dt.
\end{align*}
Note that the value \(\tau_K(t, \omega + \pi/2)\) measures the distance from the point \(v_K^+(t)\) on \(\mathbf{b}_2\) to the intersection \(l_K(t) \cap c_K(\omega)\) (\Cref{def:tangent-leg-length}).

Add the two equations from Mamikon’s theorem and use \(\mathcal{I}(\delta K) = \mathcal{I}(\mathbf{b}_1) + \mathcal{I}(\mathbf{b}_2)\) to express \(\mathcal{S}(K)\) as a sum of integrals of squares.
\begin{equation}
\label{eqn:sk}
\begin{split}
\mathcal{S}(K) & = \frac{1}{2} \int_0^\omega h^+_K(t)^2 \, dt +  \frac{1}{2} \int_{\omega}^{\pi/2 + \omega} \tau_K(t, \omega + \pi/2)^2 \, dt
\end{split}
\end{equation}
Note that the base \(h_K^+(t)\) and \(\tau_K(t, \omega + \pi/2)\) of integrands are convex-linear with respect to \(K\) by \Cref{thm:cap-convex-linear} and \Cref{cor:tangent-line-length-linear}. So the term \(\mathcal{S}(K)\) is convex with respect to \(K\) by \Cref{lem:sum-of-squares}.

Next, we show that \(\mathcal{S}(K) + \mathcal{A}_1(K)\) is convex-linear with respect to \(K\). Using \(\mathcal{A}_1(K) = |K| - \mathcal{I}(\mathbf{x}_K) = \mathcal{I}(\delta K) - \mathcal{I}(\mathbf{x}_K)\), we have
\[
\mathcal{S}(K) + \mathcal{A}_1(K) = \left( \mathcal{I}(\mathbf{y}_K) - \mathcal{I}(\mathbf{x}_K) \right) + \mathcal{I}(\mathbf{y}_K(\omega), C_K^+(\omega)) - \mathcal{I}(\mathbf{y}_K(0), A_K^-(0)).
\]
Observe that the points \(O, \mathbf{y}_K(\omega), C_K^+(\omega)\) form the vertices of a right-angled triangle of height 1 along the direction \(u_\omega\) with base \(p_K(\omega + \pi/2)\). So the term \(\mathcal{I}(\mathbf{y}_K(\omega), C_K^+(\omega)) = p_K(\pi/2 + \omega) / 2\) is linear with respect to \(K\) (\Cref{pro:support-function-linear}). Likewise, the points \(O, \mathbf{y}_K(0), A_K^-(0)\) form the vertices of a right-angled triangle of height 1 along the direction \(v_0\) with base \(p_K(0)\). So the term \(\mathcal{I}(\mathbf{y}_K(0), A_K^-(0)) = -p_K(\omega)/2\) is linear with respect to \(K\) (\Cref{pro:support-function-linear}). It remains to show that \(\mathcal{I}(\mathbf{y}_K) - \mathcal{I}(\mathbf{x}_K)\) is linear with respect to \(K\). Note that the difference \(\mathbf{y}_K(t) - \mathbf{x}_K(t) = u_t + v_t\) of \(\mathbf{y}_K\) and \(\mathbf{x}_K\) is constant with respect to \(K\). Write \(u_t + v_t\) as \(c_t\) for simplicity, then we have
\[
\begin{split}
\mathcal{I}(\mathbf{y}_K) & = \frac{1}{2} \int_{0}^\omega \mathbf{y}_K(t) \times d \mathbf{y}_K(t) \\
& = \frac{1}{2} \int_{0}^\omega (\mathbf{x}_K(t) + c_t) \times d (\mathbf{x}_K(t) + c_t)  \\
& = \mathcal{I}(\mathbf{x}_K) + \frac{1}{2} \left( \int_{0}^\omega c_t \times d \mathbf{x}_K(t) 
+ \int_{0}^\omega \mathbf{x}_K(t) \times d c_t + \int_{0}^\omega c_t \times d c_t \right) 
\end{split}
\]
so \(\mathcal{I}(\mathbf{y}_K) - \mathcal{I}(\mathbf{x}_K)\) is convex-linear with respect to \(K\), by the convex-linearity of \(\mathbf{x}_K\) (\Cref{thm:cap-convex-linear}).

Since \(\mathcal{S}(K)\) is convex and \(\mathcal{S}(K) + \mathcal{A}_1(K)\) is convex-linear, \(\mathcal{A}_1(K)\) is concave with respect to \(K\).
\end{proof}