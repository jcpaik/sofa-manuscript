We first define the upper bound \(\mathcal{A}_1 : \mathcal{K}_\omega \to \mathbb{R}\) of the sofa area functional \(\mathcal{A}\). Recall the \Cref{def:curve-area-functional} of the \emph{curve area functional}
\[
\mathcal{I}(\mathbf{x}) := \frac{1}{2} \int_a^b \mathbf{x}(t) \times d\mathbf{x}(t) := \frac{1}{2} \int_a^b x(t) dy(t) - y(t) dx(t)
\]
for an arbitrary curve \(\Gamma\) with rectifiable parametrization \(\mathbf{x} : [a, b] \to \mathbb{R}^2\).. By Green’s theorem, we have the following.

\begin{proposition}

(Theorem 10.43, p289 of \autocite{apostol2000visual}) If \(\mathbf{x}\) is a Jordan curve oriented counterclockwise (resp. clockwise), \(\mathcal{I}(\mathbf{x})\) is the exact area of the region enclosed by \(\mathbf{x}\) (resp. the area with a negative sign).

\label{pro:curve-area-functional-area}
\end{proposition}

If \(\mathbf{x}\) is not closed (that is, \(\mathbf{x}(a) \neq \mathbf{x}(b)\)), the sofa area functional \(\mathcal{I}(\mathbf{x})\) measures the signed area of the region bounded by the curve \(\mathbf{x}\), and two line segments connecting the origin to \(\mathbf{x}(a)\) and \(\mathbf{x}(b)\) respectively. We also have the following additivity of \(\mathcal{I}\).

\begin{proposition}

If \(\gamma\) is the concatenation of two curves \(\alpha\) and \(\beta\) then \(\mathcal{I}(\gamma) = \mathcal{I}(\alpha) + \mathcal{I}(\beta)\).

\label{pro:curve-area-functional-additive}
\end{proposition}

For any \(\omega \in (0, \pi/2]\) and cap \(K \in \mathcal{K}_\omega\), we will define \(\mathcal{A}_1(K)\) as the area of \(K\) minus the area of the region enclosed by \(\mathbf{x}_K : [0, \omega]\). We will express the area enclosed by \(\mathbf{x}_K\) as \(\mathcal{I}(\mathbf{x}_K)\).

\begin{proposition}

For any \(\omega \in (0, \pi/2]\) and cap \(K \in \mathcal{K}_\omega\), the inner corner \(\mathbf{x}_K : [0, \omega] \to \mathbb{R}\) is Lipschitz.

\label{pro:inner-corner-lipschitz}
\end{proposition}

\begin{proof}
The support function \(p_K\) of \(K\) is Lipschitz (\Cref{thm:support-function-lipschitz}), so
\[
\mathbf{x}_K(t) = (p_K(t) - 1) u_t + (p_K(t + \pi/2) - 1) v_t
\]
is also Lipschitz.
\end{proof}

Thus \(\mathbf{x}_K\) is rectifiable and the value \(\mathcal{I}(\mathbf{x}_K)\) is well-defined. With this, define the functional \(\mathcal{A}_1 : \mathcal{K}_\omega \to \mathbb{R}\) as the following.

\begin{definition}

For any angle \(\omega \in (0, \pi/2]\) and cap \(K\) in \(\mathcal{K}_\omega\), define \(\mathcal{A}_{1, \omega}(K) = |K| - \mathcal{I}(\mathbf{x}_K)\). If the angle \(\omega\) is clear from the context, denote \(\mathcal{A}_{1, \omega}\) as simply \(\mathcal{A}_1\).

\label{def:a1}
\end{definition}

We now show that \(\mathcal{A}_1(K)\) is an upper bound of the area functional \(\mathcal{A}(K)\) if \(\mathbf{x}_K\) is injective and in the fan \(F_\omega\). Our key observation is the following.

\begin{lemma}

Let \(\omega \in (0, \pi/2]\) and \(K \in \mathcal{K}_{\omega}\) be arbitrary. Let \(\mathbf{z} : [t_0, t_1] \to \mathbb{R}^2\) be any open simple curve (that is, a curve with \(t_0 < t_1\) and injective parametrization \(\mathbf{z}\)) inside the set \(F_{\omega} \cap \bigcup_{0 \leq t \leq \omega} \overline{Q^-_K(t)}\), such that the starting point \(\mathbf{z}(t_0)\) is on the boundary \(l(\pi/2, 0) \cap F_\omega\) of \(F_\omega\), and the endpoint \(\mathbf{z}(t_1)\) is on the boundary \(l(\omega, 0) \cap F_\omega\) of \(F_\omega\). Then we have \(\mathcal{I}(\mathbf{z}) \leq |\mathcal{N}(K)|\).

\label{lem:curve-area-functional-lower-bound}
\end{lemma}

\begin{proof}
Define \(\mathbf{b}\) as the curve from \(\mathbf{z}(t_1)\) to \(\mathbf{z}(t_0)\) along the boundary \(\partial F_\omega\) of fan \(F_\omega\) (so \(\mathbf{b}\) is either a segment or the concatenation of two segments). Since \(\mathbf{z}\) is injective, we have \(\mathbf{z}(t_0) \neq \mathbf{z}(t_1)\) so \(\mathbf{b}\) is also an open simple curve. For every \(\epsilon > 0\), define the closed curve \(\Gamma_\epsilon\) as the concatenation of the following curves in order: the curve \(\mathbf{z}(t)\), the vertical segment from \(\mathbf{z}(t_1)\) to \(\mathbf{z}(t_1) - (0, \epsilon)\), the curve \(\mathbf{b} - (0, \epsilon)\) shifted downwards by \(\epsilon\), and then the vertical segment from \(\mathbf{z}(t_0) - (0, \epsilon)\) to \(\mathbf{z}(t_0)\). The curve \(\Gamma_{\epsilon}\) is a Jordan curve because \(\mathbf{z}\) is an open simple curve inside \(F_\omega\). By Jordan curve theorem, the curve \(\Gamma_\epsilon\) encloses an open set \(\mathcal{N}_\epsilon\). Define \(\mathcal{N}_0\) as the intersection \(F_{\omega} \cap \mathcal{N}_{\epsilon}\), then \(\mathcal{N}_0\) is independent of the choice of \(\epsilon > 0\); for any \(\epsilon_1 > \epsilon_2 > 0\), there is a continuous deformation of \(\mathbb{R}^2\) that fixes \(F_\omega\) and shrinks \(\mathbb{R}^2 \setminus F_\omega\) vertically so that it shrinks \(\Gamma_{\epsilon_1}\) to \(\Gamma_{\epsilon_2}\). Moreover, \(\mathcal{N}_{\epsilon}\) is the disjoint union of \(\mathcal{N}_0\) and the fixed region below \(\partial F_\omega\) of area \(\left| \mathbf{z}(t_1) - \mathbf{z}(t_0) \right| \epsilon\).

We have \(\left| \mathcal{N}_\epsilon \right| = \left| \mathcal{I}(\Gamma_\epsilon) \right|\) by Green’s theorem on \(\Gamma_\epsilon\) regardless of the orientation of \(\Gamma_\epsilon\). By sending \(\epsilon \to 0\), we have \(\left| \mathcal{N}_0 \right| = \left| \mathcal{I}(\mathbf{z}) \right|\). We now show \(\mathcal{N}_0 \subseteq \mathcal{N}(K)\) which finishes the proof. Take any \(p \in \mathcal{N}_0\). Take the ray \(r\) emanating from \(p\) in the direction \(v_0\), then it should cross some point \(q \neq p\) in the curve \(\mathbf{z}\). As \(\mathbf{z}\) is inside the set \(F_{\omega} \cap \bigcup_{0 \leq t \leq \omega} \overline{Q^-_K(t)}\), the point \(q\) is contained in \(F_{\omega} \cap \overline{Q_K^-(t)}\) for some \(0 \leq t \leq \omega\). We have \(t \neq 0, \omega\) because \(q\) is strictly above the boundary of \(F_\omega\), and for \(t=0, \pi/2\) the set \(Q^-_K(t)\) is either on or below \(\partial F_\omega\). Because the point \(p\) is in \(F_{\omega}\) and strictly below the point \(q\), it should be that \(p\) is contained in \(F_{\pi/2} \cap Q_K^-(t)\). So the point \(p\) is in the niche \(\mathcal{N}(K)\), and we have \(\mathcal{N}_0 \subseteq \mathcal{N}(K)\).
\end{proof}

We can freely choose the curve \(\mathbf{z}\) inside the set \(F_{\pi/2} \cap \bigcup_{0 \leq t \leq \pi/2} \overline{Q^-_K(t)}\). In this paper, we simply choose \(\mathbf{z} = \mathbf{x}_K\) and get the following.

\begin{theorem}

For any \(\omega \in (0, \pi/2]\) and \(K \in \mathcal{K}_{\omega}\), if the curve \(\mathbf{x}_K : [0, \omega] \to \mathbb{R}^2\) is injective and in \(F_\omega\), we have \(\mathcal{A}(K) \leq \mathcal{A}_1(K)\).

\label{thm:a1-upper-bound}
\end{theorem}

\begin{proof}
Since \(\mathbf{x}_K(t) \in \overline{Q_K^-(t)}\) for all \(t \in [0, \omega]\) and we assumed that \(\mathbf{x}_K(t) \in F_\omega\), the curve \(z := \mathbf{x}_K\) is an open simple curve inside \(F_{\omega} \cap \bigcup_{0 \leq t \leq \omega} \overline{Q^-_K(t)}\). Also, by \(p_K(\omega) = p_K(\pi/2) = 1\) we have \(\mathbf{x}_K(0) \in l(\pi/2, 0)\) and \(\mathbf{x}_K(\omega) \in l(\omega, 0)\). So the curve \(\mathbf{z} := \mathbf{x}_K\) satisfies the condition of \Cref{lem:curve-area-functional-lower-bound}, and we have \(\mathcal{I}(\mathbf{x}_K) \leq |\mathcal{N}(K)|\). So we have
\[
\mathcal{A}(K) = |K| - |\mathcal{N}(K)| \leq |K| - \mathcal{I}(\mathbf{x}_K) = \mathcal{A}_1(K)
\]
which finishes the proof.
\end{proof}

\subsection{Derivative of the inner corner}

The formula (\Cref{def:curve-area-functional}) of \(\mathcal{I}(\mathbf{x}_K)\) requires us to take derivative of the inner corner \(\mathbf{x}_K\). We end this subsection by calculating the derivative of \(\mathbf{x}_K\) explicitly, which will be used later. Define the \emph{arm lengths} \(g_K^{\pm}(t)\) and \(h_K^{\pm}(t)\) of tangent hallways of a cap \(K\) as the following.

\begin{definition}

Let \(K \in \mathcal{K}_\omega\) be arbitrary. For any \(t \in [0, \omega]\), let \(g_K^+(t)\) (resp. \(g_K^-(t)\)) be the unique real value such that \(\mathbf{y}_K(t) = A^+_K(t) + g_K^+(t) v_t\) (resp. \(\mathbf{y}(t) = A^-_K(t) + g_K^-(t) v_t\)). Similarly, let \(h_K^+(t)\) (resp. \(h_K^-(t)\)) be the unique real value such that \(\mathbf{y}_K(t) = C^+_K(t) + h_K^+(t) u_t\) (resp. \(\mathbf{y}(t) = C^-_K(t) + h_K^-(t) u_t\)).

\label{def:cap-tangent-arm-length}
\end{definition}

Observe that the point \(\mathbf{y}_K(t)\) and the vertices \(C_K^{\pm}(t)\) are on the tangent line \(c_K(t)\) in the direction of \(u_t\). Likewise \(y_K(t)\) and \(A_K^{\pm}(t)\) are on the tangent line \(a_K(t)\) in the direction of \(v_t\). So \Cref{def:cap-tangent-arm-length} is well-defined. For the same reason, we also have the following equations.

\begin{proposition}

Let \(K \in \mathcal{K}_\omega\) and \(t \in [0, \omega]\) be arbitrary. Then \(g_K^{\pm}(t) = \left( C_K^{\pm}(t) - A_K^{\pm}(t) \right) \cdot v_t\) and \(h_K^{\pm}(t) = (A_K^{\pm}(t) - C_K^{\pm}(t)) \cdot u_t\).

\label{pro:cap-tangent-arm-length}
\end{proposition}

The derivative of \(\mathbf{x}_K\) can be expressed in terms of the arm lengths of \(K\).

\begin{theorem}

For any cap \(K \in \mathcal{K}_\omega\), the right derivative of the outer and inner corner \(\mathbf{y}_K(t)\) exists for any \(0 \leq t < \omega\) and is equal to the following.

\begin{align*}
\partial^+ \mathbf{y}_K(t) = -g_K^+(t) u_t + h_K^+(t) v_t \qquad \partial^+ \mathbf{x}_K(t) = -(g_K^+(t) - 1) u_t + (h_K^+(t) - 1) v_t
\end{align*}
Likewise, the left derivative of \(\mathbf{y}_K\) and \(\mathbf{x}_K\) exists for all \(0 < t \leq \omega\) and is equal to the following.

\begin{align*}
\partial^- \mathbf{y}_K(t) = -g_K^-(t) u_t + h_K^-(t) v_t \qquad \partial^+ \mathbf{x}_K(t) = -(g_K^-(t) - 1) u_t + (h_K^-(t) - 1) v_t
\end{align*}

\label{thm:inner-corner-deriv}
\end{theorem}

\begin{proof}
Fix an arbitrary cap \(K\) and omit the subscript \(K\) in vertices \(\mathbf{y}_K(t)\), \(\mathbf{x}_K(t)\) and tangent lines \(a_K(t)\). Take any \(0 \leq t < \omega\) and set \(s = t + \delta\) for sufficiently small and arbitrary \(\delta > 0\). We evaluate \(\partial^+ \mathbf{y}(t) = \lim_{\delta \rightarrow 0^+}(\mathbf{y}(s) - \mathbf{y}(t)) / \delta\). Define \(A_{t, s} = a(t) \cap a(s)\). Since \(A_{t, s}\) is on the lines \(a(t)\) and \(a(s)\), it satisfies both \(A_{t, s} \cdot u_t = \mathbf{y}(t) \cdot u_t\) and \(A_{t, s} \cdot u_s = \mathbf{y}(s) \cdot u_s\). Rewrite \(u_s = \cos \delta \cdot u_t + \sin \delta \cdot v_t\) on the second equation and we have

\begin{align*}
	A_{t, s} \cos \delta \cdot u_t + A_{t, s} \sin \delta \cdot v_t =  	\cos \delta (\mathbf{y}(s) \cdot u_t) + \sin \delta (\mathbf{y}(s) \cdot v_t).
\end{align*}
Group by \(\cos \delta\) and \(\sin \delta\) and substitute \(A_{t, s} \cdot u_t\) with \(\mathbf{y}(t) \cdot u_t\), then
\[ \cos \delta (\mathbf{y}(s) \cdot u_t - \mathbf{y}(t) \cdot u_t)
	= \sin \delta (A_{t, s}  (s) \cdot v_t - \mathbf{y}(s) \cdot v_t) .
	\]
Divide by \(\delta\) and send \(\delta \to 0^+\). We get the following limit as \(A_{t, s} \to A^+(t)\) (\Cref{thm:limits-converging-to-vertex}).
\[ \partial^+ (\mathbf{y}(t) \cdot u_t)  = (A^+(t) - \mathbf{y}(t)) \cdot v_t = - g^+(t)\]
A similar argument can be applied to show \(\partial^+ (\mathbf{y}(t) \cdot v_t) = h^+(t)\) and thus the first equation of the theorem. The right derivative of \(\mathbf{x}_K(t)\) comes from \(\mathbf{x}_K(t) = \mathbf{y}_K(t) - u_t - v_t\). A symmetric argument calculates the left derivative of \(\mathbf{y}_K\) and \(\mathbf{x}_K\).
\end{proof}

We prepare some observations on arm lengths that will be used later.

\begin{remark}

The resulting equation \(\partial^+ \mathbf{x}_K(t) = -(g_K^+(t) - 1) u_t + (h_K^+(t) - 1) v_t\) in \Cref{thm:inner-corner-deriv} can be interpreted intuitively as the following. Imagine the hallway \(L_K(t)\) where we increment \(t\) slightly by \(\epsilon > 0\). If \(\epsilon\) is very small, the wall \(c_K(t)\) rotates around the pivot \(C_K^+(t)\). As \(\mathbf{x}_K(t)\) rotates differentially with the pivot \(C_K^+(t)\) as center, the \(v_t\) component \(h_K^+(t) - 1\) of the derivative \(\partial^+ \mathbf{x}_K(t)\) is the distance from the pivot \(C_K^+(t)\) to \(\mathbf{x}_K(t)\) measured in the direction of \(u_t\). The \(u_t\) component \(-(g_K^+(t) - 1)\) of \(\partial^+ \mathbf{x}_K(t)\) can be interpreted similarly as the distance from pivot \(A^+_K(t)\) to \(\mathbf{x}_K(t)\) along the direction \(v_t\).

\label{rem:inner-corner-deriv}
\end{remark}

Except for a countable number of \(t\), we do not need to differentiate \(g_K^+(t)\) and \(g_K^-(t)\) (and likewise for \(h_K^{\pm}(t)\)).

\begin{definition}

For any cap \(K\) of rotation angle \(\omega \in (0, \pi/2]\) and any angle \(t \in [0, \omega]\), if \(g_K^+(t) = g_K^-(t)\) (resp. \(h_K^+(t) = h_K^-(t)\)) then simply denote the matching value as \(g_K(t)\) (resp. \(h_K(t)\)).

\label{def:arm-length-unsinged}
\end{definition}

\begin{proposition}

Let \(K\) be any cap of rotation angle \(\omega \in (0, \pi/2]\) and take any angle \(t \in [0, \omega]\). The condition \(g_K^+(t) = g_K^-(t)\) (resp. \(h_K^+(t) = h_K^-(t)\)) holds if and only if \(\beta_K(\left\{ t \right\}) = 0\) (resp. \(\beta_K(\left\{ t + \pi/2 \right\}) = 0\)) by \Cref{thm:surface-area-singleton}. So \(g_K\) and \(h_K\) are almost everywhere defined and integrable functions on \([0, \omega]\).

\label{pro:arm-length-unsigned}
\end{proposition}

By \Cref{pro:cap-tangent-arm-length} and \Cref{thm:limits-converging-to-vertex} we have the following.

\begin{corollary}

Let \(K\) be any cap of rotation angle \(\omega \in (0, \pi/2]\) and take any angle \(t \in [0, \omega]\). If \(t > 0\), then \(g_K^{\pm}(u) \to g_K^-(t)\) and \(h_K^{\pm}(u) \to h_K^-(t)\) as \(u \to t^-\). If \(t < \omega\), then \(g_K^{\pm}(u) \to g_K^+(t)\) and \(h_K^{\pm}(u) \to h_K^+(t)\) as \(u \to t^+\).

\label{cor:arm-length-continuity}
\end{corollary}