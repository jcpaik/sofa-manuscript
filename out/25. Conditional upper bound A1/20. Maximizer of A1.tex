We now solve for the maximizer \(K = K_{\omega, 1}\) of our concave quadratic upper bound \(\mathcal{A}_1 : \mathcal{K}_\omega \to \mathbb{R}\). We do this by solving for the cap \(K\) that satisfies \(\beta_{K} = \iota_{K}\) on the set \(J_\omega \setminus \left\{ \omega, \pi/2 \right\}\). Once we find such \(K\), then it happens that the directional derivative \(D\mathcal{A}_1(K; K') = 0\) for every other cap \(K'\) by \Cref{thm:variation-a1} because \(p_{K'}(\omega) = p_K(\omega) = p_{K'}(\pi/2) = p_K(\pi/2) = 1\). Then by \Cref{thm:quadratic-variation} the cap \(K\) attains the maximum value of \(\mathcal{A}_1\).

As we have seen in the previous subsection, the equation \(\beta_{K} = \iota_{K}\) on the set \(J_\omega \setminus \left\{ \omega, \pi/2 \right\}\) can be compared to the local optimality equation (ODE3) in Theorem 2 of \autocite{romikDifferentialEquationsExact2018}. However, unlike \autocite{romikDifferentialEquationsExact2018} which solved the equation for the inner corner \(\mathbf{x}_K\), we will solve for the arm lengths \(g_K^{\pm}(t)\) and \(h_K^{\pm}(t)\) to find \(K\). This will lead to a simpler set of differential equations to solve. From now on, let \(K \in \mathcal{K}_{\omega}\) be an arbitrary cap with rotation angle \(\omega \in (0, \pi/2]\).

The following theorem shows that under sufficient conditions, we do not need to differentiate \(g_K^+(t)\) and \(g_K^+(t)\) (resp. \(h_K^+(t)\) and \(h_K^-(t)\)) and also can calculate the derivatives of \(g_K\) and \(h_K\) in terms of \(\beta_K\).

\begin{theorem}

Assume that there is a open interval \(U\) in \((0, \pi/2)\) and a continuous function \(f : U \cup \left( U + \pi/2 \right) \to \mathbb{R}\) such that the measure \(\beta_K\) has density function \(f\) on \(U \cup (U + \pi/2)\). That is, we have \(\beta_K(X) = \int_X f(x)\,dx\) for every Borel subset \(X \subseteq U \cup (U + \pi/2)\). Then we have \(g_K^+(t) = g_K^-(t)\) and \(h_K^+(t) = h_K^-(t)\) for every \(t \in U\) so the function \(g_K(t)\) and \(h_K(t)\) are well-defined on \(t \in U\). Moreover, \(g_{K}'(t) = -f(t) + h_{K}(t)\) and \(h_K'(t) = f(t + \pi/2) - g_K(t)\) for every \(t \in U\).

\label{thm:arm-length-derivative}
\end{theorem}

\begin{proof}
We have \(g_K(t) = g_K^{\pm}(t)\) and \(h_K(t) = h_K^{\pm}(t)\) for all \(t \in U\) by \Cref{pro:arm-length-unsigned}. To calculate the derivatives, first apply \Cref{thm:boundary-measure} to \Cref{pro:cap-tangent-arm-length} to get the following equations for all \(t \in U\).
\[
g_{K}(t) = \int_{t}^{t+\pi/2} \cos \left( u - t \right) \beta_K(du)
\]
\[
h_{K}(t) = \int_{t}^{t+\pi/2} \sin \left( u - t \right) \beta_K(du)
\]
Differentiate them at \(t \in U\) using Leibniz integral rule to complete the proof.
\[
g_{K}'(t) = -f(t) + \int_{t}^{t+\pi/2} \sin (u-t)\, \beta_K(du) = -f(t) + h_{K}(t) 
\]
\[
h_{K}'(t) = f\left( t + \pi/2 \right) - \int_{t}^{t+\pi/2} \cos (u-t)\, \beta_K(du) = f(t + \pi/2) - g_{K}(t)
\]

\end{proof}

This theorem is a converse of \Cref{thm:arm-length-derivative} that calculates \(\beta_K\) from continuously differentiable \(g_K\) and \(h_K\).

\begin{theorem}

Assume there is a open interval \(U\) in \((0, \pi/2)\) so that \(g_K(t)\) and \(h_K(t)\) are well-defined and continuously differentiable on \(U\). Define the continous function \(f\) on \(U \cup (U + \pi/2)\) as \(f(t) = h_K(t) - g_K'(t)\) and \(f(t + \pi/2) = g_K(t) + h'_K(t)\) for all \(t \in U\). Then the measure \(\beta_K\) has density function \(f\) on \(U \cup (U + \pi/2)\).

\label{thm:arm-length-derivative-converse}
\end{theorem}

\begin{proof}
By \Cref{thm:inner-corner-deriv}, \(\mathbf{y}_K(t)\) has continuous differentiation \(-g_K(t) u_t + h_K(t) v_t\) on \(t \in U\). So \(A_K^{\pm}(t) = \mathbf{y}_K(t) - g_K(t) u_t\) and \(C_K^{\pm}(t) = \mathbf{y}_K(t) - h_K(t) v_t\) are continuously differentiable on \(t \in U\) too. Then by \Cref{pro:boundary-measure-differential} and \Cref{def:boundary-measure} the boundary measure \(\beta_K\) has a continuous density function \(f_0\) on \(U \cup (U + \pi/2)\), where \(f_0(t) = A_K'(t) \cdot v_t\) and \(f_0(t + \pi/2) = - C_K'(t) \cdot v_t\). Now by \Cref{thm:arm-length-derivative} we should have \(f_0 = f\), completing the proof.
\end{proof}

Now we solve the equation \(\beta_K = \iota_K\) on any open set \((t_1, t_2) \cup (t_1 + \pi/2, t_2 + \pi/2)\) in terms of \(\beta_K\).

\begin{theorem}

Let \(0 \leq t_1 < t_2 \leq \omega\) be two arbitrary angles. Let \(U = (t_1, t_2)\) and \(X = U \cup (U + \pi/2)\). Then the followings are equivalent.

\begin{enumerate}
\def\labelenumi{\arabic{enumi}.}
\tightlist
\item
  We have \(\beta_{K} = \iota_{K}\) on the set \(X\)
\item
  We have \(g_K(t) = a + t\) and \(h_K(t) = b - t\) for some constants \(a, b \in \mathbb{R}\) on \(t \in U\).
\end{enumerate}

\label{thm:balanced-ACx}
\end{theorem}

\begin{proof}
Assume (1) that \(\beta_K = \iota_K\) on \(X\). The measure of \(\beta_K = \iota_K\) is zero for every point of \(X\) by \Cref{def:i-cap}. So we have \(g_K(t) = g^\pm_K(t)\) and \(h_K(t) = h_K^{\pm}(t)\) for every \(t \in U\) by \Cref{pro:arm-length-unsigned}. Also, \(g_K(t)\), \(h_K(t)\) are continuous with respect to \(t \in U\) by \Cref{cor:arm-length-continuity}. Now since \(\beta_K = \iota_K\) has the continuous density function \(i_K\) on \(X\), we can apply \Cref{thm:arm-length-derivative} to \(K\). By applying so, we have

\begin{gather}
g_{K}'(t) = -i_K(t) + h_{K}(t) = 1 \\
h_K'(t) = i_K(t + \pi/2) - g_K(t) = -1
\end{gather}
on \(t \in U\) and this immediately proves (2).

Now assume (2). By \Cref{thm:arm-length-derivative-converse} the measure \(\beta_K\) should have density function \(f(t) = b - t - 1\) and \(f(t + \pi/2) = a + t + 1\) over all \(t \in U\). This matches with the density function of \(\iota_K\) in \Cref{def:i-cap}, completing the proof of (1).
\end{proof}

We now solve for the equation \(\beta_K = \iota_K\) on the set \(J_\omega \setminus \left\{ \omega, \pi/2 \right\}\) by solving for \(\beta_K\). Note that our derivation aims to not solve the equation completely, but to derive enough properties of such \(K\) to get a single value of \(K\). First, we should have \(\beta_K(\left\{ 0 \right\}) = \beta_K(\left\{ \omega + \pi/2 \right\}) = 0\) because \(0, \omega + \pi/2 \in J_\omega \setminus \left\{ \omega, \pi/2 \right\}\) and \(\iota_K\) have measure zero on every singleton. So the values \(g_K(0)\) and \(h_K(\omega)\) exist, and we have \(g_K(0) = h_K(\omega) = 1\) as the width of \(K\) measured in the direction of \(u_\omega\) or \(v_0\) is one. So by \Cref{thm:balanced-ACx} with \(t_1 = 0\) and \(t_2 = \omega\) we should have \(g_K(t) = t + 1\) and \(h_K(t) = \omega - t + 1\) on the set \(t \in (0, \omega)\). Now by \Cref{def:i-cap}, the measure \(\beta_K\) has density \(\beta_K(dt) = (\omega - t) dt\) and \(\beta_K(dt + \pi/2) = t dt\) on \(t \in (0, \omega)\). It remains to find the values of \(\beta_K\) on the points \(\omega\) and \(\pi/2\). The measure \(\beta_K\) has to satisfy the equations in \Cref{thm:boundary-measure-cap}. Since we have calculations
\begin{gather*}
\label{eqn:k1-conditions}
\int_{t \in [0, \omega)} (\omega - t) \cos t  \, dt = 1 - \cos \omega \\
\int_{t \in (\pi/2, \omega + \pi/2]} (t - \pi/2) \cos\left( \omega + \pi/2 - t \right)  \, dt = 1 - \cos \omega
\end{gather*}
we should have \(\beta_K(\left\{ \omega \right\}) = \beta_K(\left\{ \pi/2 \right\}) = 1\) if \(\omega < \pi/2\). Motivated by the calculations made here, we define the following candidate \(K = K_{\omega, 1}\)

\begin{definition}

Define the cap \(K = K_{\omega, 1} \in \mathcal{K}_{\omega}\) with rotation angle \(\omega \in (0, \pi/2]\) as the unique cap with following boundary measure \(\beta_{K}\).

\begin{enumerate}
\def\labelenumi{\arabic{enumi}.}
\tightlist
\item
  \(\beta_{{K}}(dt) = (\omega -t)dt\) on \(t \in [0, \omega)\) and \(\beta_{K}(dt + \pi/2) = t dt\) on \(t \in (0, \omega]\).
\item
  If \(\omega < \pi/2\), \(\beta_K(\left\{ \omega \right\}) = \beta_K(\left\{ \pi/2 \right\}) = 1\).
\item
  If \(\omega = \pi/2\), \(\beta_K(\left\{ \pi/2 \right\}) = 2\) and \(v_K^+(\pi/2) = (-1, 1)\) and \(v_K^-(\pi/2) = (1, 1)\).
\end{enumerate}

\label{def:maximum-presofa-a1}
\end{definition}

Let us justify the unique existence of such \(K_{\omega, 1}\). Observe that the two equations in \Cref{thm:cap-from-boundary-measure} are true by \Cref{eqn:k1-conditions}. So if \(\omega < \pi/2\), the unique existence of \(K_{\omega, 1}\) comes from the statement of \Cref{thm:cap-from-boundary-measure} immediately. If \(\omega = \pi/2\), then the additional constraints \(v_K^+(\pi/2) = (-1, 1)\) and \(v_K^-(\pi/2) = (1, 1)\) fixes the single \(K\) among all possible horizontal translates. So the uniqueness of \(K\) is also guaranteed.

We now check back that the solution \(K = K_{\omega, 1}\) indeed satisfies the equation \(\beta_K = \iota_K\) on \(J_\omega \setminus \left\{ \omega, \pi/2 \right\}\).

\begin{theorem}

The cap \(K_{\omega, 1}\) maximizes \(\mathcal{A}_1 : \mathcal{K}_{\omega} \to \mathbb{R}\).

\label{thm:maximum-presofa-a1}
\end{theorem}

\begin{proof}
Let \(K := K_{\omega, 1}\). First we verify that \(K\) satisfies the equation \(\beta_K = \iota_K\) on \(J_\omega \setminus \{\omega, \pi/2\}\). Because \(\beta_K(\left\{ 0 \right\}) = \beta_K(\left\{ \omega + \pi/2 \right\}) = 0\), the values \(g_K(0)\) and \(h_K(\omega)\) exist. We have \(g_K(0) = 1\) and \(h_K(\omega) = 1\) as the width of \(K\) measured in the direction of \(u_\omega\) and \(v_0\) are one. Then by \Cref{thm:arm-length-derivative} on the interval \((0, \omega)\) we have

\begin{align*}
g_{K}'(t) & = -\left( \omega - t \right)  + h_{K}(t) \\
h_K'(t) & = t - g_K(t)
\end{align*}
on \(t \in (0, \omega)\). This imply \(g''_K(t) = 1 + t - g_K(t)\) on \(t \in (0, \omega)\). So we have \(g_K(t) = 1 + t + C_1 \sin t + C_2 \cos t\) for some constants \(C_1, C_2\) on \(t \in (0, \omega)\). Because \(g_K(t) \to g_K(0) = 1\) as \(t \to 0\) by \Cref{cor:arm-length-continuity}, we should have \(C_2 = 0\). Because \(h_K(t) \to h_K(\omega) = 1\) as \(t \to \omega\) by \Cref{cor:arm-length-continuity}, we have \(g'_K(t) \to 1\) as \(t \to \omega\). This then imply \(1 + C_1 \sin \omega = 1\) and \(C_1 = 0\). Now \(g_K(t) = 1 + t\) and correspondingly \(h_K(t) = 1 + \omega - t\) on \(t \in (0, \omega)\). As \(\iota_K\) is defined from the values of \(g_K\) and \(h_K\) (\Cref{def:i-cap}), we can verify \(\beta_K = \iota_K\) on \(J_\omega \setminus \{\omega, \pi/2\}\).

Take any \(K' \in \mathcal{K}_\omega\). Apply \(\beta_K = \iota_K\) on \(J_\omega \setminus \{\omega, \pi/2\}\) and \(p_K(\omega) = p_K(\pi/2) = p_{K'}(\omega) = p_{K'}(\pi/2) = 1\) to \Cref{thm:variation-a1}, to get \(D\mathcal{A}_1(K; K') = 0\). The functional \(\mathcal{A}_1\) is concave by \Cref{thm:a1-negative-semidefinite}. So \Cref{thm:quadratic-variation} and \(D\mathcal{A}_1(K; -) = 0\) implies that \(K = K_{\omega, 1}\) is the maximizer of \(\mathcal{A}_1\).
\end{proof}

A consequence of the proof of \Cref{thm:maximum-presofa-a1} is:

\begin{corollary}

For \(K = K_{\omega, 1}\), we have \(g_K(t) = 1 + t\) and \(h_K(t) = 1 + \omega - t\) for all \(t \in (0, \omega)\).

\label{cor:maximum-presofa-a1-arm-length}
\end{corollary}

Now we compute the maximum value of \(\mathcal{A}_1\).

\begin{theorem}

The maximum value \(\mathcal{A}_1(K_{\omega, 1})\) of \(\mathcal{A}_1\) is \(1 + \omega^2/2\).

\label{thm:maximum-presofa-a1-area}
\end{theorem}

\begin{proof}
Let \(K = K_{\omega, 1}\). We will exploit the mirror symmetry of \(K_{\omega, 1}\) along the line \(l\) connecting \(O\) to \(o_\omega\) (\Cref{def:parallelogram-vertices}). The line \(l\) divides \(K_{\omega, 1}\) into two pieces which are mirror images to each other. Call the piece on the right side of \(l\) as \(K_h\). Observe that the boundary of the half-piece \(K_h\) consists of the curve from \(A_K^-(0)\) to \(o_\omega\), and two segments from \(O\) to \(A_K^-(0)\) and \(o_\omega\) respectively. So \(p_{K}(t) = p_{K_h}(t)\) for \(t \in [0, \omega]\) and \(\beta_K\) and \(\sigma_{K_h}\) agree on \([0, \omega)\). Observe that \(\sigma_{K_h}\left( \left\{ \omega \right\} \right) = 1\) no matter if either \(\omega < \pi/2\) or \(\omega = \pi/2\), unlike the value \(\beta_K(\left\{ \omega \right\})\) that may change depending on \(\omega\). We also have \(|K| = 2 |K_h|\).

Now we compute the value of \(p_{K}(t) = p_{K_h}(t)\) for \(t \in [0, \omega]\). For the second equality, we are using \Cref{cor:boundary-measure-closed} with \(t_1 = t\) and \(t_2 = \omega\).

\begin{align*}
p_{K_h}(t) - o_\omega \cdot u_t & = (A^-_{K_h}(t) - o_\omega) \cdot u_t =  \\
& = \sin(\omega - t) + \int_{u \in [t, \omega] } (\omega - u) \sin \left( u - t \right) \, du \\
& = \omega - t
\end{align*}
So \(p_{K}(t) = p_{K_h}(t) = \omega - t + o_\omega \cdot u_t\). By the symmetry of \(K\) along \(l\) we also have \(p_K(t + \pi/2) = t + o_{\omega} \cdot v_t\).

Now calculate half the area of \(K\).

\begin{align*}
|K_h| = \frac{1}{2} \int_{t \in [0, \omega]} p_{K_h}(t) \, \beta(dt) & = 
\frac{1}{2} + \frac{1}{2} \int_{t \in [0, \omega]} \left( \omega - t + o_\omega \cdot u_t \right)  (\omega-t) \, dt \\
& = \frac{1}{2} + \frac{1}{2} \left( \omega^3 / 3 + o_\omega \cdot \int_{0}^{\omega} u_t (\omega - t)\, dt \right) 
\end{align*}
Define \(R := o_\omega \cdot \int_{0}^{\omega} u_t (\omega - t)\, dt\). Multiplying by 2, we get \(|K| = 1 + \omega^3 / 3 + R\)

Next, we compute the curve area functional \(\mathcal{I}(\mathbf{x}_{K})\). We have
\[
\mathbf{x}_{K}(t) = (p_{K}(t) - 1)u_t + (p_{K}(t + \pi/2) - 1) v_t
\]
by \Cref{pro:rotating-hallway-parts}. We also have
\begin{equation}
\label{eqn:a1-maximizer-x-deriv}
\mathbf{x}'_{K}(t) = -(g_{K}^+(t) - 1) \cdot u_t + (h_{K}^+(t) - 1) \cdot v_t = 
- t \cdot u_t + (\omega - t) \cdot v_t.
\end{equation}
by \Cref{thm:inner-corner-deriv} and \Cref{cor:maximum-presofa-a1-arm-length}. Now compute \(\mathcal{I}(\mathbf{x}_K)\).

\begin{align*}
\mathcal{I}(\mathbf{x}_{K}) & = \frac{1}{2} \int_0^\omega (p_{K}(t) - 1)(\omega - t) + (p_{K}(t+\pi/2) - 1) t \, dt  \\
& = \frac{1}{2} \int_0^\omega (\omega - t + o_{\omega} \cdot u_t - 1) (\omega - t) \, dt + 
\frac{1}{2} \int_0^\omega (t + o_{\omega} \cdot v_t - 1) t \, dt \\
& = \omega^3 / 3 - \omega^2 / 2 + R
\end{align*}
Finally, we compute \(\mathcal{A}_1(K) = |K| - \mathcal{I}(\mathbf{x}_{K}) = 1 + \omega^2 / 2\).
\end{proof}

We finish the proof of the main \Cref{thm:main-cap}.

\begin{proof}[Proof of \Cref{thm:main-cap}]
Assume any cap \(K \in \mathcal{K}_\omega\) with rotation angle \(\omega \in (0, \pi/2]\) and the rotation path \(\mathbf{x}_K : [0, \omega] \to \mathbb{R}^2\) injective and on the fan \(F_\omega\). Then by \Cref{thm:a1-upper-bound} we have \(\mathcal{A}(K) \leq \mathcal{A}_1(K)\). By \Cref{thm:maximum-presofa-a1-area} we have \(\mathcal{A}_1(K) \leq 1 + \omega^2/2\), completing the proof.
\end{proof}

We justify the description of the maximizer \(S_1\) with cap \(K_{\pi/2, 1}\) in \Cref{sec:main-result}. The right side of the cap \(K_{\pi/2, 1}\) is parametrized by the curve \(\gamma : [0, \pi/2] \to \mathbb{R}^2\) with \(\gamma(0) = (1, 1)\) and \(\gamma'(t) = t(\cos t, -\sin t)\), from \((\pi/2 - t, \pi/2)\) the description of the on the interval \Cref{cor:boundary-measure-open} The description of the inner corner \(\mathbf{x}_{S_1} : [0, \pi/2] \to \mathbb{R}^2\) with \(\mathbf{x}_{S_1}(0) = (\pi/2-1, 0)\) and
\[
\mathbf{x}_{S_1}'(t) = -t (\cos t, \sin t) + (\pi/2- t) (-\sin t, \cos t)
\]
comes from \Cref{eqn:a1-maximizer-x-deriv}.