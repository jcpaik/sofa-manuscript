Moving a large couch through a narrow hallway requires a well-planned pivoting. The \emph{moving sofa problem}, first published by Leo Moser in 1966 \autocite{moser1966problem}, is asked in a two-dimensional idealization of such a situation:

\begin{quote}
What is the largest area \(\mu_{\text{max}}\) of a connected shape that can move around the right-angled corner of a hallway with unit width?
\end{quote}

More precisely, define the hallway \(L\) as the union \(L = L_H \cup L_V\) of sets \(L_H = (-\infty, 1] \times [0, 1]\) and \(L_V = [0, 1] \times (-\infty, 1]\) representing the horizontal and vertical side of \(L\) respectively. A \emph{moving sofa} \(S\) may be defined as a connected subset of \(L_H\) that can be moved inside \(L\) by a continuous rigid motion to a subset of \(L_V\). It is known that there exists a moving sofa attaining the maximum area \(\mu_{\text{max}}\) \autocite{gerverMovingSofaCorner1992,croft2012unsolved}, but the precise value of \(\mu_{\text{max}}\) remains unknown despite decades of partial progress \autocite{hammersley1968enfeeblement,gerverMovingSofaCorner1992,romikDifferentialEquationsExact2018,kallusImprovedUpperBounds2018}.

The best bounds currently known on \(\mu_{\max}\) are summarized as
\begin{equation}
\label{eqn:best-bounds}
\mu_G := 2.2195\dots \leq \mu_{\max} \leq 2.37.
\end{equation}
The lower bound \(2.2195\dots \leq \mu_{\max}\) comes from Gerver’s sofa \(S_G\) of area \(\mu_G := 2.2195\dots\) constructed in 1994 \autocite{gerverMovingSofaCorner1992} (see \Cref{fig:gerver}). Gerver derived his sofa from local optimality considerations\footnote{Gerver assumed five stages of the movement of a sofa to derive his sofa \(S_G\) \autocite{gerverMovingSofaCorner1992}. While his sofa \(S_G\) is locally optimal (Theorem 2 of \autocite{gerverMovingSofaCorner1992}), this does not eliminate the possibility of a maximum-area sofa with a different kind of movement. Romik’s simplified derivation of \(S_G\) in \autocite{romikDifferentialEquationsExact2018} also relies on the same assumption (Equation 24, p324 of \autocite{romikDifferentialEquationsExact2018}). So their derivations do not constitute a full proof of Gerver’s conjecture \(\mu_{\max} = \mu_G\).} and conjectured \(\mu_{\max} = \mu_G\) that his sofa attains the maximum area. Approximate solutions found by computer experiments are consistent with Gerver’s conjecture.\footnote{Wagner used Monte Carlo simulation to find an approximate solution (Figure 2 of \autocite{wagner1976sofa}) that resembles Gerver’s sofa in shape. More recent approximate solutions, as found by Gibbs \autocite{gibbsComputationalStudySofas2014} in 2014 and Batsch \autocite{batschNumericalApproachAnalysing2022} in 2022, deviate in area from Gerver’s sofa by small margins of \(1.7 \times 10^{-7}\) and \(5.7 \times 10^{-9}\) respectively.}

\begin{figure}
\centering
\includesvg[width=1\textwidth,height=\textheight]{images/gerver-full.svg}
\caption{Gerver’s sofa \(S_G\). The ticks denote the endpoints of 18 analytic curves and segments constituting the boundary of \(S_G\) (see \autocite{romikDifferentialEquationsExact2018} for details). The lower portion of \(S_G\) is made of two small ‘tails’ (depicted red) and one large ‘core’ (depicted blue).}
\label{fig:gerver}
\end{figure}

On the other hand, the upper bound \(\mu_{\max} \leq 2.37\) was proved by Kallus and Romik \autocite{kallusImprovedUpperBounds2018}. If Gerver’s conjecture \(\mu_{\max} = \mu_G\) is true, then the remaining task would be to bring the upper bound of \(\mu_{\max}\) down to the lower bound \(\mu_G\). However, not many methods are known for bounding \(\mu_{\max}\) from above. All known upper bounds of \(\mu_{\max}\), including that of Kallus and Romik, approximate the moving sofa by a polygonal intersection \(S_\Theta\) of the horizontal strip \(H = \mathbb{R} \times [0, 1]\) and copies of hallway \(L\) rotated counterclockwise by each angle in a finite set \(\Theta \subset (0, \pi/2)\) (see \Cref{fig:polygon-sofa}).\footnote{The polygonal intersection \(S_\Theta = H \cap \bigcap_{t \in \Theta} L_t\) is the overestimation of the monotone sofa \(\mathcal{M}(S) = H \cap V_\omega \cap \bigcap_{0 \leq t \leq \omega} L_t\) in \Cref{eqn:monotone}.} The bound \(\mu_{\max} \leq 2 \sqrt{2} = 2.828\dots\) by Hammersley \autocite{hammersley1968enfeeblement} is obtained by taking a single angle \(\Theta = \left\{ \pi/4 \right\}\), and the bound \(\mu_{\max} \leq 2.37\) by Kallus and Romik \autocite{kallusImprovedUpperBounds2018} is achieved by taking five specific angles \(\Theta\) and estimating the area of \(S_\Theta\) from above with extensive computer assistance.

\begin{figure}
\centering
\includesvg[width=0.7\textwidth,height=\textheight]{images/polygon-sofa.svg}
\caption{Polygonal intersection \(S_\Theta\) used by the upper bound \(2 \sqrt{2} = 2.828\dots\) of Hammersley \autocite{hammersley1968enfeeblement}, and \(2.37\) of Kallus and Romik \autocite{kallusImprovedUpperBounds2018}.}
\label{fig:polygon-sofa}
\end{figure}

We develop a new approach for bounding \(\mu_{\max}\) from above by interpreting the moving sofa problem as an infinite-dimensional convex quadratic programming. Consequently, we prove that any moving sofa satisfying a certain property, named as the \emph{injectivity condition} (\Cref{def:injectivity}), has an area at most \(1 + \pi^2/8 = 2.2337\dots\) (\Cref{thm:main}). This conditional upper bound does not rely on any computer assistance, while being much closer to the lower bound \(2.2195\dots\) of Gerver than the computer-assisted upper bound \(2.37\) of Kallus and Romik. Gerver’s sofa \(S_G\), the conjectured optimum, satisfies the injectivity condition in particular. We also conjecture the \emph{injectivity hypothesis} (\Cref{con:injectivity}) that there exists a maximum-area moving sofa satisfying the injectivity condition. With our result, proving the injectivity hypothesis would imply the unconditional upper bound \(\mu_{\max} \leq 1 + \pi^2/8\).

The idea for proving the main \Cref{thm:main} is to overestimate the area of a moving sofa \(S\) by ignoring the effect of the inner walls to the moving sofa \(S\) (compare the left side of \Cref{fig:presofa} to \Cref{fig:gerver}). The overestimated area \(\mathcal{A}_1(K)\) (\Cref{fig:a1-upper-bound}) turns out to be a quadratic functional on a convex body \(K\) that we call the \emph{cap} of \(S\). We establish the concavity of \(\mathcal{A}_1\) using Mamikon’s theorem \autocite{mnatsakanianAnnularRingsEqual1997}, a theorem in classical geometry (\Cref{fig:mamikon-sofa}). Then we introduce a calculus of variations based on the Brunn-Minkowski theory to find a global optimum \(K_1\) of \(\mathcal{A}_1\). The optimum of \(\mathcal{A}_1\) gives an unmovable sofa \(S_1\) of area \(1 + \pi^2/8 = 2.2337\dots\) and width \(\pi\) very close to the area of Gerver’s sofa \(S_G\) (see the right side of \Cref{fig:presofa}).

\begin{figure}
\centering
\includesvg[width=1\textwidth,height=\textheight]{images/presofa-combined.svg}
\caption{The maximizing (unmovable) sofa \(S_1\) of upper bound \(\mathcal{A}_1\) of area \(1 + \pi^2/8 = 2.2337\dots\). The regions below two tails (red curves) stick out of the hallway \(L\) during the movement of \(S_1\) in \(L\) (left). The shape \(S_1\) is very similar to Gerver’s sofa \(S_G\) whose boundary is drawn in dotted lines (right).}
\label{fig:presofa}
\end{figure}