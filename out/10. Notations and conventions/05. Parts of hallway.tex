We give names to the different parts of the hallway \(L\) for future reference.

\begin{definition}

Let \(\mathbf{x} = (0, 0)\) and \(\mathbf{y} = (1, 1)\) be the inner and outer corner of \(L\) respectively.

\label{def:hallway-corners}
\end{definition}

\begin{definition}

Let \(a\) and \(c\) be the lines \(x=1\) and \(y=1\) representing the \emph{outer walls} of \(L\) passing through \(\mathbf{y}\). Let \(b\) and \(d\) be the half-lines \((-\infty, 0] \times \left\{ 0 \right\}\) and \(\left\{ 0 \right\} \times (-\infty, 0]\) representing the \emph{inner walls} of \(L\) emanating from the inner corner \(\mathbf{x}\).

\label{def:hallway-walls}
\end{definition}

\begin{definition}

Let \(Q^+ = (-\infty, 1]^2\) be the closed quarter-plane bounded by outer walls \(a\) and \(c\). Let \(Q^- = (-\infty, 0)^2\) be the open quarter-plane bounded by inner walls \(b\) and \(d\), so that \(L = Q^+ \setminus Q^-\).

\label{def:hallway-regions}
\end{definition}

\begin{figure}
\centering
\includesvg[width=0.4\textwidth,height=\textheight]{images/hallway-detailed.svg}
\caption{The standard hallway \(L\) and its parts.}
\label{fig:hallway-detailed}
\end{figure}

For reference, the \emph{tangent hallway} \(L_S(t)\) of a shape \(S\) will be defined in \Cref{def:tangent-hallway}. The corresponding parts \(\mathbf{x}_S(t), \mathbf{y}_S(t), a_S(t), b_S(t), c_S(t), d_S(t), Q^+_S(t), Q^-_S(t)\) of the tangent hallway will be defined in \Cref{def:rotating-hallway-parts}.