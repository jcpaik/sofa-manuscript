\begin{definition}

The \emph{hallway} \(L = L_H \cup L_V\) is the union of sets \(L_H = (-\infty, 1] \times [0, 1]\) and \(L_V = [0, 1] \times (-\infty, 1]\), each representing the horizontal and vertical side of \(L\) respectively.

\label{def:hallway}
\end{definition}

\begin{definition}

Define the unit-width horizontal and vertical strips \(H = \mathbb{R} \times [0, 1]\) and \(V = [0, 1] \times \mathbb{R}\) respectively.

\label{def:strip}
\end{definition}

In the introduction, we gave a definition of a moving sofa \(S\) as a subset of \(L_H\). However, the condition that \(S\) should be confined in \(L_H\) is a bit restrictive for our future use. So we will also call any translation of such \(S \subseteq L_H\) a \emph{moving sofa} as well without loss of generality.

\begin{definition}

A \emph{moving sofa} \(S\) is a connected, nonempty and compact subset of \(\mathbb{R}^2\), such that a translation of \(S\) is a subset of \(L_H\) that admits a continuous rigid motion inside \(L\) from \(L_H\) to \(L_V\).

\label{def:sofa}
\end{definition}

It is safe assume that a moving sofa is always closed, since for any subset of \(L\) its closure is also contained in \(L\). We also define the rotation angle \(\omega\) of a moving sofa \(S\).

\begin{definition}

Say that a moving sofa \(S\) have the \emph{rotation angle} \(\omega \in (0, \pi/2]\) if the continuous rigid motion of a translate of \(S\) from \(L_H\) to \(L_V\) inside \(L\) rotates the body clockwise by \(\omega\) in its full movement.

\label{def:rotation-angle}
\end{definition}

With the result of \autocite{kallusImprovedUpperBounds2018} that \(\omega \in [81.203\dots^\circ, 90^\circ]\) for a maximum-area moving sofa, we will always assume that a moving sofa have rotation angle \(\omega \in (0, \pi/2]\). For each rotation angle \(\omega\), we define the following notions for future use.

\begin{definition}

Define \(R_\theta : \mathbb{R}^2 \to \mathbb{R}^2\) as the rotation map of \(\mathbb{R}^2\) around the origin by a counterclockwise angle of \(\theta \in S^1\).

\label{def:rotation-map}
\end{definition}

\begin{definition}

For any \(\omega \in (0, \pi/2]\), define the \emph{parallelogram} \(P_\omega = H \cap R_\omega(V)\) with \emph{rotation angle} \(\omega\).\footnote{If \(\omega = \pi/2\), then the set \(P_{\pi/2} = H\) is technically not a parallelogram. We will however call it as the parallelogram with rotation angle \(\pi/2\).}

\label{def:parallelogram}
\end{definition}

\begin{definition}

For any \(\omega \in (0, \pi/2]\), define the set \(J_\omega = [0, \omega] \cup [\pi/2, \omega + \pi/2]\).

\label{def:j-cap}
\end{definition}

For reference, the notion of \emph{standard position} of a moving sofa will be defined in \Cref{def:standard-position}. The \emph{monotonization} \(\mathcal{M}(S)\), \emph{cap} \(\mathcal{C}(S)\), and \emph{niche} \(\mathcal{N}(K)\) of a moving sofa \(S\) will be defined in \Cref{def:monotonization}, \Cref{def:cap-sofa}, and \Cref{def:niche} respectively.