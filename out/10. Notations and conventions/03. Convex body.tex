\begin{definition}

In this paper, a \emph{shape} \(S\) is a nonempty and compact subset of \(\mathbb{R}^2\).

\label{def:shape}
\end{definition}

\begin{definition}

For any angle \(t\) in \(S^1\) or \(\mathbb{R}\), define the unit vectors \(u_t = \left( \cos t, \sin t \right)\) and \(v_t = \left( -\sin t,\cos t \right)\).

\label{def:frame}
\end{definition}

Any line on \(\mathbb{R}^2\) can be described by the angle \(t\) of its normal vector \(u_t\) and its (signed) distance from the origin.

\begin{definition}

For any angle \(t\) in \(S^1\) and a value \(h \in \mathbb{R}\), define the line \(l(t, h)\) with the \emph{normal angle} \(t\) and the signed distance \(h\) from the origin as
\[
l(t, h) = \left\{ p \in \mathbb{R}^2 : p \cdot u_t = h \right\}.
\]

\label{def:line}
\end{definition}

A line on \(\mathbb{R}^2\) divids the plane into two half-planes. Following \Cref{def:line}, we also give a name to one of the half-planes in the direction of \(-u_t\).

\begin{definition}

For any angle \(t\) in \(S^1\) and a value \(h \in \mathbb{R}\), define the closed \emph{half-plane} \(H(t, h)\) with the boundary \(l(t, h)\) as
\[
H(t, h) = \left\{ p \in \mathbb{R}^2 : p \cdot u_t \leq h \right\}.
\]
We say that the closed half-plane \(H(t, h)\) has the \emph{normal angle} \(t\).

\label{def:half-plane}
\end{definition}

Fix a shape \(S\) and angle \(t \in S^1\). Take a sufficiently large \(h \in \mathbb{R}\) so that \(H(t, h) \supseteq S\). As we decrease \(h\) continuously, the line \(l(t, h)\) will get close to \(S\) until it makes contact with \(S\) for the first time. We define the value of \(h\), tangent line \(l(t, h)\), tangent half-plane \(H(t, h)\) as the following \Cref{def:support-function}, \Cref{def:tangent-line} and \Cref{def:tangent-half-plane}.

\begin{definition}

For any shape \(S\), define its \emph{support function} \(p_S : S^1 \to \mathbb{R}\) as the value \(p_S(t) = \sup \left\{ p \cdot u_t : p \in S \right\}\).

\label{def:support-function}
\end{definition}

\begin{definition}

For any shape \(S\) and angle \(t \in S^1\), define the \emph{tangent line} \(l_S(t)\) of \(S\) with \emph{normal angle} \(t\) as the line \(l_S(t) := l(t, p_S(t))\).

\label{def:tangent-line}
\end{definition}

\begin{definition}

For any shape \(S\) and angle \(t \in S^1\), define the \emph{tangent half-plane} \(H_S(t)\) of \(S\) with \emph{normal angle} \(t\) as the line \(H_S(t) := H(t, p_S(t))\).

\label{def:tangent-half-plane}
\end{definition}

Observe that the support function \(p_S(t)\) measures the signed distance from the origin \((0, 0)\) to the tangent line \(l_S(t)\) of \(S\) with the normal vector \(u_t\) directing outwards from \(S\). Support function and tangent lines of \(S\) are usually studied when \(S\) is a convex body (e.g.~p45 of \cite{schneider_2013}), but in this paper we generalize the notion to arbitrary shape \(S\).

The following notion of \emph{width} along a direction is also studied for convex bodies (e.g.~p49 of \cite{schneider_2013}).

\begin{definition}

For any shape \(S\) and angle \(t\) in \(S^1\) or \(\mathbb{R}\), the \emph{width} of \(S\) along the direction of unit vector \(u_t\) is defined as \(p_S(t) + p_S(t + \pi)\).

\label{def:width}
\end{definition}

Geometrically, the width of \(S\) along \(u_t\) measures the distance between the parallel tangent lines \(l_S(t)\) and \(l_S(t + \pi)\) of \(S\).

We adopt the following definition of a convex body (p8 of \cite{schneider_2013}).

\begin{definition}

A \emph{convex body} \(K\) is a nonempty, compact, and convex subset of \(\mathbb{R}^2\).

\label{def:convex-body}
\end{definition}

Many authors often also include the condition that \(K^\circ\) is nonempty, but we allow \(K^\circ\) to be empty (that is, \(K\) can be a closed line segment or a point).

In this paper only, we use the following notions of \emph{vertices} and \emph{edges} of a planar convex body \(K\).

\begin{definition}

For any convex body \(K\) and \(t \in S^1\), define the \emph{edge} \(e_K(t)\) of \(K\) as the intersection of \(K\) with the tangent line \(l_K(t)\).

\label{def:convex-body-edge}
\end{definition}

\begin{definition}

For any convex body \(K\) and \(t \in S^1\), let \(v_K^+(t)\) and \(v_K^-(t)\) be the endpoints of the edge \(e_K(t)\) such that \(v_K^+(t)\) is positioned farthest in the direction of \(v_t\) and \(v_K^-(t)\) is positioned farthest in the opposite direction of \(v_t\). We call \(v_K^{\pm}(t)\) the \emph{vertices} of \(K\).

\label{def:convex-body-vertex}
\end{definition}

It is possible that the edge \(e_K(t)\) can be a single point. In such case, the tangent line \(l_K(t)\) touches \(K\) at the single point \(v_K^+(t) = v_K^-(t)\). In fact, it turns out that this holds for every \(t \in S^1\) except for a countable number of values of \(t\) (\Cref{pro:surface-area-singleton-almost-everywhere}).

\begin{figure}
\centering
\includesvg[width=0.5\textwidth,height=\textheight]{images/convex-body.svg}
\caption{A convex body \(K\) with its edge, vertices, tangent line, and half-plane.}
\label{fig:convex-body}
\end{figure}

For reference, the notion of cap \(K\) as a kind of convex bodies is defined in \Cref{def:cap}. The space of all caps \(\mathcal{K}_\omega\) with rotation angle \(\omega\) is defined in \Cref{def:cap-space}. The vertices \(A_K^{\pm}(t)\), \(C_K^{\pm}(t)\) of a cap \(K\) is defined in \Cref{def:cap-vertices}. The upper boundary \(\delta K\) of a cap \(K\) is defined in \Cref{def:upper-boundary-of-cap}.