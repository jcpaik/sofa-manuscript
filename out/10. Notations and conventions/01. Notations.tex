Denote the area (Borel measure) of a measurable \(X \subseteq \mathbb{R}^2\) as \(|X|\). For any subset \(X\) of \(\mathbb{R}^2\), denote the topological closure, boundary, and interior as \(\overline{X}\), \(\partial X\), and \(X^\circ\) respectively.

For a subset \(X\) of \(\mathbb{R}^2\) and a vector \(v\) in \(\mathbb{R}^2\), define the set \(X + v = \left\{ x + v : x \in X \right\}\). For any two subsets \(X, Y\) of \(\mathbb{R}^2\), the set \(X + Y = \left\{ x + y : x \in X, y \in Y \right\}\) denotes the Minkowski sum of \(X\) and \(Y\). For any subset \(X\) of \(\mathbb{R}^2\) and a real value \(a\), define the set \(aX = \left\{ ax : a \in X \right\}\).

We use the convention \(S^1 = \mathbb{R} / 2 \pi \mathbb{Z}\). For any function \(f\) on \(S^1\) and any \(t \in \mathbb{R}\), the notation \(f(t)\) denotes the value \(f(t + 2 \pi \mathbb{Z})\). That is, a real value coerces to a value in \(S^1\) when used as an argument of a function that takes a value in \(S^1\). We will often denote an interval of \(S^1\) by its lift under the canonical map \(\mathbb{R} \to \mathbb{R} / 2 \pi \mathbb{Z} = S^1\). Specifically, for any \(t_1 \in \mathbb{R}\) and \(t_2 \in (t_1, t_1 + 2\pi]\), the intervals \((t_1, t_2]\) and \([t_1, t_2)\) of \(\mathbb{R}\) are used to denote the corresponding intervals of \(S^1\) mapped under \(\mathbb{R} \to S^1\). Likewise, for any \(t_1 \in \mathbb{R}\) and \(t_2 \in [t_1, t_1 + 2\pi)\), the interval \([t_1, t_2]\) of \(\mathbb{R}\) is used to denote the corresponding interval of \(S^1\) mapped under \(\mathbb{R} \to S^1\).

For any function \(f : \mathbb{R} \to \mathbb{R}\) or \(f : S^1 \to \mathbb{R}\), \(f(t-)\) denotes the left limit of \(f\) at \(t\) and \(f(t+)\) denotes the right limit of \(f\) at \(t\). For any function \(f : X \to \mathbb{R}\) defined on some open subset \(X\) of either \(\mathbb{R}\) and \(S^1\), and \(t \in X\), define \(\partial^+f(t)\) and \(\partial^-f(t)\) as the right and left differentiation of \(f\) at \(t\) if they exists.

We denote the integral of a measurable function \(f\) with respect to a measure \(\mu\) on a set \(X\) as either \(\int_{x \in X} f(x) \, \mu(dx)\) or \(\left< f, \mu \right>_X\). The latter notation is used especially when we want to emphasize that the integral is bi-linear with respect to both \(f\) and \(\mu\).