In this section, we rigorously define what is a \emph{monotone sofa}. \Cref{thm:monotonization-is-sofa} shows that the process named \emph{monotonization} enlarges any moving sofa \(S\) to a larger moving sofa \(\mathcal{M}(S)\) as described in \Cref{eqn:monotone} of \Cref{sec:moving-hallway-problem}. A monotone sofa is then simply defined as the monotonization \(\mathcal{M}(S)\) of some moving sofa \(S\) (\Cref{def:monotone-sofa}). Proving the connectedness of \(\mathcal{M}(S)\) (\Cref{thm:monotonization-is-connected}) will be the key step in establishing \Cref{thm:monotonization-is-sofa}.

\Cref{thm:monotonization-structure} shows that any monotone sofa \(S\) is equal to \(K \setminus \mathcal{N}(K)\), where \(K = \mathcal{C}(S)\) is a convex set called the \emph{cap of} \(S\) (\Cref{thm:cap-hallway-intersection}), and \(\mathcal{N}(K)\) is a subset of \(K\) called the \emph{niche} determined by the cap \(K\) (\Cref{def:niche}). Then we show that the niche \(\mathcal{N}(K)\) is always contained in the cap \(K\) of sofa \(S\) (\Cref{thm:niche-in-cap}). With this, the area \(|S| = |K| - |\mathcal{N}(K)|\) of a monotone sofa \(S\) can be understood in terms of cap \(K\) and niche \(\mathcal{N}(K)\) separately.