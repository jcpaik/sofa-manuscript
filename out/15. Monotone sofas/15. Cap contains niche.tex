We will now establish the following theorem.

\begin{theorem}

For any monotone sofa \(S\) with cap \(K = \mathcal{C}(S)\), the cap \(K\) contains the niche \(\mathcal{N}(K)\).

\label{thm:niche-in-cap}
\end{theorem}

Note that \(S = K \setminus \mathcal{N}(K)\) by \Cref{thm:monotone-sofa-structure}. With \Cref{thm:niche-in-cap}, the area \(|S| = |K| - |\mathcal{N}(K)|\) of a monotone sofa can be understood separately in terms of its cap and niche.

\begin{remark}

In spite of \Cref{thm:niche-in-cap}, a general cap \(K\) following \Cref{def:cap} may not always contain its niche \(\mathcal{N}(K)\) in \Cref{def:niche}. For an example, take \(K = [0, 100] \times [0, 1]\) with rotation angle \(\omega = \pi/2\). Then \(K\) is too wide and the inner quadrant \(Q_K^-(\pi/4)\) of \(L_K(\pi/4)\) pushes out of \(K\), so we have \(\mathcal{N}(K) \not\subseteq K\). In this case, the cap \(K\) is never the cap \(\mathcal{C}(S)\) associated to a particular moving sofa \(S\) as in \Cref{thm:cap-hallway-intersection}. \Cref{thm:monotonization-connected-iff} identifies the exact condition of \(K\) where \(\mathcal{N}(K) \subseteq K\).

\label{rem:niche-not-in-cap}
\end{remark}

\subsubsection{Geometric Definitions on Cap and Niche}

We need a handful of geometric definitions on a cap \(K\) to prove \Cref{thm:niche-in-cap}. We will also use them throughout the rest of the document as well. Define the \emph{vertices} of a cap \(K\).

\begin{definition}

Let \(K\) be a cap with rotation angle \(\omega\). For any \(t \in [0, \omega]\), define the vertices \(A^+_K(t) = v^+_K(t)\), \(A^-_K(t) = v^-_K(t)\), \(C^+_K(t) = v^+_K(t + \pi/2)\), and \(C^-_K(t) = v^-_K(t + \pi/2)\) of \(K\).

\label{def:cap-vertices}
\end{definition}

Note that the outer wall \(a_K(t)\) (resp. \(c_K(t)\)) of \(L_K(t)\) is in contact with the cap \(K\) at the vertices \(A_K^+(t)\) and \(A_K^-(t)\) (resp. \(C_K^+(t)\) and \(C_K^-(t)\)) respectively. We also define the \emph{upper boundary} of a cap \(K\).

\begin{definition}

For any cap \(K\) with rotation angle \(\omega\), define the \emph{upper boundary} \(\delta K\) of \(K\) as the set \(\delta K = \bigcup_{t \in [0, \omega + \pi/2]} e_K(t)\).

\label{def:upper-boundary-of-cap}
\end{definition}

For any cap \(K\) with rotation angle \(\omega\), the upper boundary \(\delta K\) is exactly the points of \(K\) making contact with the outer walls \(a_K(t)\) and \(c_K(t)\) of tangent hallways \(L_K(t)\) for every \(t \in [0, \omega]\). We collect some observations on \(\delta K\).

\begin{proposition}

Let \(K\) be a cap with rotation angle \(\omega\). The set \(K \setminus \delta K\) is the interior of \(K\) in the subset topology of \(F_\omega\).

\label{pro:upper-boundary-interior}
\end{proposition}

\begin{proof}
Since \(K\) and \(F_\omega\) are closed in \(\mathbb{R}^2\), the set \(K\) is closed in the subset topology of \(F_\omega\). Let \(X\) be the boundary of \(K\) in the subset topology of \(F_\omega\), then we have \(X \subseteq K\) because \(K\) is closed in the subset topology of \(F_\omega\). We will show that \(\delta K\) is equal to \(X\), then it follows that the set \(K \setminus \delta K\) is the interior of \(K\) in the subset topology of \(F_\omega\).

We show \(\delta K \subseteq X\) and \(X \subseteq \delta K\) respectively. Take any point \(z\) of \(\delta K\). Then \(z \in e_K(t)\) for some \(t \in [0, \omega + \pi/2]\). Since \(K\) is a planar convex body, for any \(\epsilon > 0\) the point \(z' = z + \epsilon u_t\) is not in \(K\). Since the set \(F_\omega\) is closed in the direction of \(u_t\) (\Cref{def:closed-in-direction}), the point \(z'\) is also in \(F_\omega\). Thus we have a point \(z'\) in the neighborhood of \(z\) which is outside \(K\), and \(\delta K\) is a subset of \(X\).

On the other hand, take any point \(z\) of \(X\) and assume by contradiction that \(z \in K \setminus \delta K\). Then for every \(t \in [0, \omega + \pi/2]\) we have \(z \not \in e_K(t)\) so that \(z \cdot u_t < p_K(t)\). Since \(p_K\) is continuous, the value \(p_K(t) - z \cdot u_t\) has a global lower bound \(\epsilon > 0\) on the compact interval \([0, \omega + \pi/2]\). So an open ball \(U\) of radius \(\epsilon\) centered at \(z\) is contained in the half-space \(H_K(t)\) for all \(t \in [0, \omega + \pi/2]\). Now \(U \cap F_\omega \subseteq K\) and so \(z \not\in X\), leading to contradiction.
\end{proof}

Geometrically, the upper boundary \(\delta K\) is an arc from \(A_K^-(0)\) to \(C_K^+(\omega)\) in the counterclockwise direction along the boundary \(\partial K\) of \(K\). This is rigorously justified by the following consequence of \Cref{cor:closed-param-segment}. For full details, read the introduction of \Cref{sec:parametrization-of-boundary}.

\begin{corollary}

Let \(K\) be a cap with rotation angle \(\omega\). The upper boundary \(\delta K\) admits an absolutely-continuous, arc-length parametrization \(\mathbf{b}_K^{0-, \pi/2 + \omega}\) (\Cref{def:closed-param}) from \(A_K^-(0)\) to \(C_K^+(\omega)\) in the counterclockwise direction along \(\partial K\).

\label{cor:upper-boundary-param}
\end{corollary}

We also give name to the convex polygons \(F_\omega \cap Q^-_K(t)\) whose union over all \(t \in [0, \omega]\) constitutes the niche \(\mathcal{N}(K)\).

\begin{definition}

For any cap \(K\) with rotation angle \(\omega\), define \(T_K(t) = F_\omega \cap Q^-_K(t)\) as the \emph{wedge} of \(K\) with angle \(t \in [0, \omega]\).

\label{def:wedge}
\end{definition}

\begin{proposition}

For any cap \(K\) with rotation angle \(\omega\), we have \(\mathcal{N}(K) = \cup_{t \in [0, \omega]} T_K(t)\).

\label{pro:wedge}
\end{proposition}

\begin{proof}
Immediate from \Cref{def:niche}.
\end{proof}

We give names to the parts of the wedge \(T_K(t)\).

\begin{definition}

For any cap \(K\) with rotation angle \(\omega\) and \(t \in (0, \omega)\), define \(W_K(t)\) as the intersection of lines \(b_K(t)\) and \(l(\pi, 0)\). Define \(w_K(t) = (A_K^-(0) - W_K(t)) \cdot u_0\) as the signed distance from point \(W_K(t)\) and the vertex \(A_K^-(0)\) along the line \(l(\pi, 0)\) in the direction of \(u_0\).

Likewise, define \(Z_K(t)\) as the intersection of lines \(d_K(t)\) and \(l(\omega, 0)\). Define \(z_K(t) = (C_K^+(\omega) - Z_K(t)) \cdot v_\omega\) as the signed length between \(Z_K(t)\) and the vertex \(C_K^+(\omega)\) along the line \(l(\omega, 0)\) in the direction of \(v_\omega\).

\label{def:wedge-side-lengths}
\end{definition}

Note that if the wedge \(T_K(t)\) contains the origin \(O\), then \(T_K(t)\) is a quadrilateral with vertices \(O, W_K(t), Z_K(t)\), and \(\mathbf{x}_K(t)\), and the points \(W_K(t)\) and \(Z_K(t)\) are the leftmost and rightmost point of \(\overline{T_K(t)}\) respectively.

\subsubsection{Controlling the Wedge Inside Cap}

To show \(\mathcal{N}(K) \subseteq K\) we need to control each wedge \(T_K(t)\) inside \(K\). First we show \(w_K(t), z_K(t) \geq 0\). This controls the endpoints \(W_K(t)\) and \(Z_K(t)\) of \(T_K(t)\) inside \(K\).

\begin{lemma}

Let \(K\) be any cap with rotation angle \(\omega\). For any angle \(t \in (0, \omega)\), we have \(w_K(t), z_K(t) \geq 0\).

\label{lem:wedge-ends-in-cap}
\end{lemma}

\begin{proof}
To show that \(w_K(t) \geq 0\), we need to show that the point \(A_K^-(0)\) is further than the point \(W_K(t)\) in the direction of \(u_0\) (see \Cref{def:further-in-direction} for the terminology). The point \(q := a_K(t) \cap l(\pi/2, 1)\) is further than \(W_K(t) = b_K(t) \cap l(\pi/2, 0)\) in the direction of \(u_0\), because the lines \(a_K(t)\) and \(b_K(t)\) form the boundary of a unit-width vertical strip rotated counterclockwise by \(t\). The point \(A^-_K(t)\) is further than \(q = l_K(t) \cap l_K(\pi/2)\) in the direction of \(u_0\) because \(K\) is a convex body. Finally, the point \(A^-_K(0)\) is further than \(A_K^-(t)\) in the direction of \(u_0\) because \(K\) is a convex body. Summing up, the points \(W_K(t), q, A_K^-(t), A_K^-(0)\) are aligned in the direction of \(u_0\), completing the proof. A symmetric argument will show that the points \(Z_K(t)\), \(r := c_K(t) \cap l(\omega, 1)\), \(C_K^+(t)\), \(C_K^+(\omega)\) are aligned in the direction of \(v_\omega\), proving \(z_K(t) \geq 0\).
\end{proof}

\begin{corollary}

Let \(K\) be any cap with rotation angle \(\omega\). Then \(A^-_K(0), C^+_K(\omega) \in K \setminus \mathcal{N}(K)\).

\label{cor:cap-ends-not-in-niche}
\end{corollary}

\begin{proof}
We only need to show that \(A^-_K(0), C^+_K(\omega)\) are not in \(\mathcal{N}(K)\). That is, for any \(t \in (0, \omega)\), neither points are in \(T_K(t)\). Since \(w_K(t) \geq 0\) by \Cref{lem:wedge-ends-in-cap}, the point \(A_K^-(0)\) is on the right side of the boundary \(b_K(t)\) of \(T_K(t)\). So \(A_K^-(0) \not\in T_K(t)\). Similarly, \(z_K(t) \geq 0\) implies \(C_K^+(\omega) \not\in T_K(t)\).
\end{proof}

We then show that if the corner \(\mathbf{x}_K(t)\) is inside \(K\), then the whole wedge \(T_K(t)\) is always inside \(K\).

\begin{lemma}

Fix any cap \(K\) with rotation angle \(\omega \in [0, \pi/2]\) and an angle \(t \in (0, \omega)\). If the inner corner \(\mathbf{x}_K(t)\) is in \(K\), then the wedge \(T_K(t)\) is a subset of \(K\).

\label{lem:niche-in-cap}
\end{lemma}

\begin{proof}
Assume \(\mathbf{x}_K(t) \in K\). If \(\omega = \pi/2\), then by \(\mathbf{x}_K(t) \in K\), the wedge \(T_K(t)\) is the triangle with vertices \(W_K(t)\), \(\mathbf{x}_K(t)\), and \(Z_K(t)\) in counterclockwise order. Note also that \(W_K(t)\) is further than \(Z_K(t)\) in the direction of \(u_0\) (\Cref{def:further-in-direction}). As \(w_K(t), z_K(t) \geq 0\), this implies that all the three vertices of \(T_K(t)\) are in \(K\).

If \(\omega < \pi/2\), we divide the proof into four cases on whether the origin \(O\) lies strictly below the lines \(b_K(t)\) and \(d_K(t)\) or not respectively.

\begin{itemize}
\tightlist
\item
  If \((0, 0)\) lies on or above both \(b_K(t)\) and \(d_K(t)\), then we get contradiction as the corner \(\mathbf{x}_K(t)\) should be outside the interior \(F_\omega^\circ\) of fan \(F_\omega\), but \(\mathbf{x}_K(t) \in K\).
\item
  If \((0, 0)\) lies on or above \(b_K(t)\) but lies strictly below \(d_K(t)\), then \(T_K(t)\) is a triangle with vertices \(\mathbf{x}_K(t)\), \(Z_K(t)\) and the intersection \(p := l(\omega, 0) \cap b_K(t)\). In this case, the point \(p\) is in the line segment connecting \(Z_K(t)\) and \((0, 0)\). Also, as \(z_K(t) \geq 0\) (\Cref{lem:wedge-ends-in-cap}) the point \(Z_K(t)\) lies in the segment connecting \(C^+_K(\omega)\) and the origin \((0, 0)\). So the points \(\mathbf{x}_K(t), Z_K(t), p\) are in \(K\) and by convexity of \(K\) we have \(T \subseteq K\).
\item
  The case where \((0, 0)\) lies strictly below \(b_K(t)\) but lies on or above \(d_K(t)\) can be handed by an argument symmetric to the previous case.
\item
  If \((0, 0)\) lies strictly below both \(b_K(t)\) and \(d_K(t)\), then \(T_K(t)\) is a quadrilateral with vertices \(\mathbf{x}_K(t)\), \(Z_K(t)\), \(W_K(t)\) and \((0, 0)\). As \(w_K(t) \geq 0\) (resp. \(z_K(t) \geq 0\)) by \Cref{lem:wedge-ends-in-cap}, the point \(W_K(t)\) (resp. \(Z_K(t)\)) is in the line segment connecting \((0, 0)\) and \(A^-_K(0)\) (resp. \(C^+_K(\omega)\)). So all the vertices of \(T_K(t)\) are in \(K\), and \(T_K(t)\) is in \(K\) by convexity.
\end{itemize}

\end{proof}

\subsubsection{\texorpdfstring{Equivalent Conditions for \(\mathcal{N}(K) \subseteq K\)}{Equivalent Conditions for \textbackslash mathcal\{N\}(K) \textbackslash subseteq K}}

Now we prove \Cref{thm:niche-in-cap}. In fact, we identify the exact condition where \(\mathcal{N}(K) \subseteq K\) for a general cap \(K\) following \Cref{def:cap}.

\begin{theorem}

Let \(K\) be any cap with rotation angle \(\omega\). Then the followings are all equivalent.

\begin{enumerate}
\def\labelenumi{\arabic{enumi}.}
\tightlist
\item
  \(\mathcal{N}(K) \subseteq K\)
\item
  \(\mathcal{N}(K) \subseteq K \setminus \delta K\)
\item
  For every \(t \in [0, \omega]\), either \(\mathbf{x}_K(t) \not\in F_\omega^\circ\) or \(\mathbf{x}_K(t) \in K\).
\item
  The set \(S = K \setminus \mathcal{N}(K)\) is connected.
\end{enumerate}

\label{thm:monotonization-connected-iff}
\end{theorem}

\begin{proof}
The conditions (1) and (2) are equivalent because the niche \(\mathcal{N}(K)\) is open in the subset topology of \(F_\omega\) by \Cref{def:niche}, and the set \(K \setminus \delta K\) is the interior of \(K\) in the subset topology of \(F_\omega\) by \Cref{pro:upper-boundary-interior}.

(1 \(\Rightarrow\) 3) We will prove the contraposition and assume \(\mathbf{x}_K(t) \in F_\omega^\circ \setminus K\). Then a neighborhood of \(\mathbf{x}_K(t)\) is inside \(F_\omega\) and disjoint from \(K\), so a subset of \(T_K(t)\) is outside \(K\), showing \(\mathcal{N}(K) \not\subseteq K \setminus \delta K\).

(3 \(\Rightarrow\) 1) If \(\mathbf{x}_K(t) \not \in F_\omega^\circ\), then \(T_K(t)\) is an empty set. If \(\mathbf{x}_K(t) \in K\), then by \Cref{lem:niche-in-cap} we have \(T_K(t) \subseteq K\).

(2 \(\Rightarrow\) 4) As \(\delta K\) is disjoint from \(\mathcal{N}(K)\), we have \(\delta K \subseteq S\). We show that \(S\) is connected. First, note that \(\delta K\) is connected by \Cref{cor:upper-boundary-param}. Next, take any point \(p \in S\). Take the half-line \(r\) starting from \(p\) in the upward direction \(v_0\). Then \(r\) touches a point in \(\delta K\) as \(p \in K\). Moreover, \(r\) is disjoint from \(\mathcal{N}(K)\) as the set \(\mathcal{N}(K) \cup (\mathbb{R}^2 \setminus F_\omega)\) is closed in the direction \(-v_0\) (\Cref{def:closed-in-direction}). Now \(r \cap K\) is a line segment inside \(S\) that connects the arbitrary point \(p \in S\) to a point in \(\delta K\). So \(S\) is connected.

(4 \(\Rightarrow\) 3) Assume by contradiction that \(\mathbf{x}_K(t) \in F_\omega^\circ \setminus K\) for some \(t \in [0, \omega]\). Then it should be that \(t \neq 0\) or \(\omega\). We first show that the vertical line \(l\) passing through \(\mathbf{x}_K(t)\) in the direction of \(v_0\) is disjoint from \(S\). The ray with initial point \(\mathbf{x}_K(t)\) and direction \(v_0\) is disjoint from \(K\) as the set \(F_\omega^\circ \setminus K\) is closed in the direction \(v_0\). The ray with initial point \(\mathbf{x}_K(t)\) and direction \(-v_0\) is not in \(S\) because \(\mathbf{x}_K(t)\) is the corner of \(Q_K^-(t)\), and \(Q_K^-(t)\) is closed in the direction of \(-v_0\). So the vertical line \(l\) passing through \(\mathbf{x}_K(t)\) does not overlap with \(S\).

Now separate the horizontal strip \(H\) into two chunks by the vertical line \(l\) passing through \(\mathbf{x}_K(t)\). As \(S\) is connected, \(S\) should lie either strictly on left or strictly on right of \(l\). As \(\mathbf{x}(t)\) lies strictly inside \(F_\omega\), the point \(W_K(t)\) is strictly further than \(\mathbf{x}(t)\) in the direction of \(u_0\), and by \Cref{lem:wedge-ends-in-cap} the point \(A_K^-(0)\) is further than \(W_K(t)\) in the direction of \(u_0\). So the endpoint \(A_K^-(0)\) of \(K\) lies strictly on the right side of \(l\). Similarly, the point \(Z_K(t)\) is strictly further than \(\mathbf{x}_K(t)\) in the direction of \(-u_0\), and by \Cref{lem:wedge-ends-in-cap} the point \(C_K^+(\omega)\) is further than \(W_K(t)\) in the direction of \(-u_0\). So the endpoint \(C^+_K(\omega)\) of \(K\) lies strictly on the left side of \(l\). As the endpoints \(A^-_K(0)\) and \(C^+_K(\omega)\) are in \(K \setminus \mathcal{N}(K)\) by \Cref{cor:cap-ends-not-in-niche}, and the line \(l\) separates the two points, the set \(K \setminus \mathcal{N}(K)\) is disconnected.
\end{proof}

\Cref{thm:niche-in-cap} is an immediate consequence of \Cref{thm:monotonization-connected-iff}.

\begin{proof}[Proof of \Cref{thm:niche-in-cap}]
We have \(S = K \setminus \mathcal{N}(K)\) by \Cref{thm:monotone-sofa-structure}. In particular, \(K \setminus \mathcal{N}(K)\) is a moving sofa so it is connected. Use that condition 4 implies condition 1 in \Cref{thm:monotonization-connected-iff} to complete the proof.
\end{proof}