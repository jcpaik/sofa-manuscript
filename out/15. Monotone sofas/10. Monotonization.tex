We now define the notion of monotone sofas and establish \Cref{thm:monotonization-is-sofa} that a moving sofa is contained in a larger monotone sofa. Define the \emph{monotonization} of any moving sofa \(S\) in standard position as the following set.

\begin{definition}

Let \(S\) be any moving sofa with rotation angle \(\omega \in (0, \pi/2]\) in standard position. The \emph{monotonization} of \(S\) is the intersection
\[
\mathcal{M}(S) = P_\omega \cap \bigcap_{0 \leq t \leq \omega} L_S(t).
\]

\label{def:monotonization}
\end{definition}

Compare the equation in \Cref{def:monotonization} to \Cref{eqn:monotone} in \Cref{sec:moving-hallway-problem}. The paralleogram \(P_\omega\) is the intersection of \(H\) and \(V_\omega\) (\Cref{def:parallelogram}), and the tangent hallways \(L_S(t)\) are the rotating hallways \(L_t\) making contact with \(S\) in the outer walls as described in \Cref{sec:moving-hallway-problem}. Condition 3 of \Cref{thm:moving-sofa-iff-hallway-intersection} implies that the set \(\mathcal{M}(S)\) contains \(S\).

\begin{corollary}

\(\mathcal{M}(S) \supseteq S\) for any moving sofa \(S\) in standard position.

\label{cor:monotonization-is-larger}
\end{corollary}

We will establish the connectedness of \(\mathcal{M}(S)\).

\begin{theorem}

Let \(S\) be a moving sofa with rotation angle \(\omega \in (0, \pi/2]\) in standard position. Then the monotonization \(\mathcal{M}(S)\) is connected.

\label{thm:monotonization-is-connected}
\end{theorem}

Once the connectedness of \(\mathcal{M}(S)\) is established, we can immediately show that the monotonization \(\mathcal{M}(S)\) is a moving sofa containing the initial moving sofa \(S\).

\begin{theorem}

Let \(S\) be any moving sofa with rotation angle \(\omega \in (0, \pi/2]\) in standard position. The monotonization \(\mathcal{M}(S)\) of \(S\) is a moving sofa containing \(S\) with the same rotation angle \(\omega\) in standard position.

\label{thm:monotonization-is-sofa}
\end{theorem}

\begin{proof}
By \Cref{thm:monotonization-is-connected}, the shape \(\mathcal{M}(S)\) is connected. By \Cref{def:monotonization}, the set \(\mathcal{M}(S)\) is contained \(P_\omega\) and \(L_S(t)\) for all \(t \in [0, \omega]\), so it satisfies the second condition of \Cref{thm:moving-sofa-iff-hallway-intersection}. So the set \(\mathcal{M}(S)\) is a moving sofa with rotation angle \(\omega\). \(\mathcal{M}(S)\) contains \(S\) by \Cref{cor:monotonization-is-larger}. From \(S \subseteq \mathcal{M}(S) \subseteq P_\omega\) and
\[
p_S(\omega) = p_{P_\omega}(\omega) = p_S(\pi/2) = p_{P_\omega}(\pi/2) = 1
\]
we have \(p_{\mathcal{M}(S)}(\omega) = p_{\mathcal{M}(S)}(\pi/2) = 1\). So \(\mathcal{M}(S)\) is in standard position.
\end{proof}

Now any monotonization of a moving sofa is also a moving sofa. Call the resulting monotonization a \emph{monotone sofa}.

\begin{definition}

A \emph{monotone sofa} is the monotonization of some moving sofa with rotation angle \(\omega \in (0, \pi/2]\) in standard position.

\label{def:monotone-sofa}
\end{definition}

The moving sofa problem asks for the largest-area moving sofas. So \Cref{thm:monotonization-is-sofa} tells us that we only need to consider monotone sofas for the problem. The rest of this \Cref{sec:monotonization} proves \Cref{thm:monotonization-is-connected} as promised.

\subsubsection{\texorpdfstring{Proof of \Cref{thm:monotonization-is-connected}}{Proof of }}

We prepare the following terminologies.

\begin{definition}

Let \(S\) be any moving sofa with rotation angle \(\omega \in (0, \pi/2]\) in standard position. Define the set
\[
\mathcal{C}(S) = P_\omega \cap \bigcap_{0 \leq t \leq \omega} Q^+_S(t).
\]

\label{def:cap-sofa}
\end{definition}

The set \(\mathcal{C}(S)\) will later be called as the \emph{cap} of \(S\) (\Cref{thm:cap-hallway-intersection}) after defining the notion of cap in \Cref{def:cap}. We don’t need this notion of cap as of now.

\begin{definition}

Say that a set \(X \subseteq \mathbb{R}^2\) is \emph{closed in the direction of} vector \(v \in \mathbb{R}^2\) if, for any \(x \in X\) and \(\lambda \geq 0\), we have \(x + \lambda v \in X\).

\label{def:closed-in-direction}
\end{definition}

\begin{definition}

Any line \(l\) of \(\mathbb{R}^2\) divides the plane into two half-planes. If \(l\) is not parallel to the \(y\)-axis, call the \emph{left side} (resp. \emph{right side}) of \(l\) as the closed half-plane with boundary \(l\) containing the point \(- Nu_0\) (resp. \(Nu_0\)) for a sufficiently large \(N\). If a point \(p\) is on the left (resp. right) side of \(l\) and not on the boundary \(l\), we say that \(p\) is \emph{strictly on the left} (resp. \emph{right}) \emph{side} of \(l\).

\label{def:line-half-plane-directions}
\end{definition}

We also prepare a lemma.

\begin{lemma}

Let \(S\) be any moving sofa with rotation angle \(\omega \in [0, \pi/2]\) in standard position. Then the support functions \(p_S\), \(p_{\mathcal{M}(S)}\), and \(p_{\mathcal{C}(S)}\) of \(S\), \(\mathcal{M}(S)\) and \(\mathcal{C}(S)\) agree on the set \(J_\omega\).

\label{lem:cap-same-support-function}
\end{lemma}

\begin{proof}
We have \(S \subseteq \mathcal{M}(S) \subseteq \mathcal{C}(S)\) by \Cref{cor:monotonization-is-larger} and \(L_S(t) \subset Q_S^+(t)\). So it remains to show \(p_{\mathcal{C}(S)}(t) \leq p_S(t)\) for every \(t\) in \(J_\omega\). By the definition of \(\mathcal{C}(S)\) we have \(S \subseteq \mathcal{C}(S) \subseteq H_S(t)\). So we have \(p_{\mathcal{C}(S)}(t) \leq p_S(t)\) indeed.
\end{proof}

A moving sofa \(S\) and its cap \(\mathcal{C}(S)\) shares the same tangent hallways \(L_S(t) = L_{\mathcal{C}(S)}(t)\).

\begin{proposition}

For any moving sofa \(S\) with rotation angle \(\omega \in [0, \pi/2]\) in standard position, the tangent hallway \(L_S(t)\) of \(S\) and the tangent hallway \(L_K(t)\) of set \(K = \mathcal{C}(S)\) are equal for every \(t \in [0, \omega]\).

\label{pro:cap-same-tangent-hallway}
\end{proposition}

\begin{proof}
The tangent hallways \(L_X(t)\) of \(X = S, K\) depend solely on the values of the support function \(p_X\) of \(X\) on \(J_\omega\), by the equation of \(L_X(t)\) in \Cref{def:tangent-hallway}. The support functions of \(S\) and \(K\) match on the set \(J_\omega\) by \Cref{lem:cap-same-support-function}, so the result follows.
\end{proof}

We are now ready to show that \(\mathcal{M}(S)\) is connected.

\begin{proof}[Proof of \Cref{thm:monotonization-is-connected}]
Define the set \(X := \bigcup_{0 \leq t \leq \omega} Q^-_S(t)\). By plugging the equation \(L_S(t) = Q_S^+(t) \setminus Q_S^-(t)\) to \Cref{def:monotonization}, we have \(\mathcal{M}(S) = \mathcal{C}(S) \setminus X\). Observe that \(\mathcal{C}(S)\) is a convex body containing \(S\) (say, by \Cref{cor:monotonization-is-larger} and \(\mathcal{M}(S) \subseteq \mathcal{C}(S)\)).

Fix an arbitrary point \(p\) in \(\mathcal{M}(S)\). Take an arbitrary angle \(\theta \in [\omega, \pi/2]\). Observe that the set \(X = \bigcup_{t \in [0, \omega]} Q^-_S(t)\) is closed in the direction of \(-u_\theta\) (\Cref{def:closed-in-direction}) for all angle \(\theta \in [\omega, \pi/2]\), since each \(Q_S^-(t)\) is closed in the direction of \(-u_\theta\). Take the line \(l_\theta\) passing the point \(p\) in the direction of \(u_\theta\). The set \(s_\theta = l_\theta \cap \mathcal{M}(S)\) contains \(p\), and \(s_\theta\) is a nonempty line segment because \(s_\theta\) is the line segment \(l_\theta \cap \mathcal{C}(S)\) minus the half-line \(l_\theta \setminus X\). If the line \(l_\theta\) meets \(S\) for any \(\theta \in [\omega, \pi/2]\), then \(p\) is connected to \(S\) along the line segment \(s_\theta\) inside \(\mathcal{M}(S)\) and the proof is done. Our goal now is to prove that there is some \(\theta \in [\omega, \pi/2]\) such that \(l_\theta\) meets \(S\).

Assume by contradiction that for every \(\theta \in [\omega, \pi/2]\) the line \(l_\theta\) is disjoint from \(S\). By \Cref{lem:cap-same-support-function}, we have \(l_{\mathcal{M}(S)}(t) = l_S(t)\) for every \(t \in J_\omega = [0, \omega] \cup [\pi/2, \pi/2 + \omega]\). Because \(p \in \mathcal{M}(S)\), the line \(l_{\pi/2}\) passing through \(p\) is either equal to \(l_{\mathcal{M}(S)}(0) = l_S(0)\) or strictly on the left side of \(l_{S}(0)\). If \(l_{\pi/2} = l_S(0)\) then \(l_{\pi/2}\) contains some point of \(S\) contradicting our assumption. So the line \(l_{\pi/2}\) is strictly on the left side of \(l_{S}(0)\), and there is a point of \(S\) strictly on the right side of \(l_{\pi/2}\). Likewise, as \(p \in \mathcal{M}(S)\), the line \(l_{\omega}\) that passes through \(p\) is either equal to \(l_{\mathcal{M}(S)}(\omega + \pi/2) = l_S(\omega + \pi/2)\) or strictly on the right side of \(l_S(\omega + \pi/2)\). The line \(l_\omega\) cannot be equal to \(l_S(\omega + \pi/2)\) because we assumed that \(l_\omega\) is disjoint from \(S\). So the line \(l_{\omega}\) is strictly on the right side of \(l_S(\omega + \pi/2)\), and there is a point of \(S\) strictly on the left side of \(l_{\omega}\).

Because the line \(l_\theta\) is disjoint from \(S\) for any \(\theta \in [\omega, \pi/2]\), the set \(S\) is inside the set \(Y = \mathbb{R}^2 \setminus \bigcup_{\theta \in [\omega, \pi/2]} l_\theta\). Note that \(Y\) has exactly two connected components \(Y_L\) and \(Y_R\) on the left and the right side of the lines \(l_\theta\) respectively. As there is a point of \(S\) strictly on the right side of \(l_{\pi/2}\), the set \(S \cap Y_R\) is nonempty. As there is also a point of \(S\) strictly on the left side of \(l_\omega\), the set \(S \cap Y_L\) is also nonempty. We get contradiction as \(S\) should be a connected subset of \(Y\).
\end{proof}