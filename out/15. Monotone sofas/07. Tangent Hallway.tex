\subsection{Tangent hallway}

Define the \emph{tangent hallways} for a shape \(S\) (that is, any nonempty compact subset \(S\) of \(\mathbb{R}^2\) by \Cref{def:shape}).

\begin{definition}

For any shape \(S\) and angle \(t \in S^1\), define the \emph{tangent hallway} \(L_S(t)\) of \(S\) with angle \(t\) as
\[
L_S(t) = R_t(L) + (p_S(t) - 1)  u_t + (p_S(t + \pi/2) - 1) v_t.
\]

\label{def:tangent-hallway}
\end{definition}

Note that \(R_t\) is the rotation of \(\mathbb{R}^2\) along the origin by a counterclockwise angle of \(t\) (\Cref{def:rotation-map}). The equation of \(L_S(t)\) in \Cref{def:tangent-hallway} is determined by the following defining property of \(L_S(t)\).

\begin{definition}

Here, a \emph{rigid transformation} \(f : \mathbb{R}^2 \to \mathbb{R}^2\) on \(\mathbb{R}^2\) is the composition \(z \mapsto R_t(z) + q\) of translation by a vector \(q \in \mathbb{R}^2\) and rotation by an angle \(t \in S^1\) along the origin. We also say that a shape \(S'\) is a rigid transformation \emph{of} another shape \(S\) if there exists a rigid transformation \(f : \mathbb{R}^2 \to \mathbb{R}^2\) such that \(S' = f(S)\).

\label{def:rigid-transformation}
\end{definition}

\begin{proposition}

For any shape \(S\) and angle \(t \in S^1\), the tangent hallway \(L_S(t)\) is the unique rigid transformation of \(L\) rotated counterclockwise by \(t\), such that the outer walls of \(L_S(t)\) corresponding to the outer walls \(a\) and \(c\) of \(L\) are the tangent lines \(l_S(t)\) and \(l_S(t + \pi/2)\) of \(S\) respectively.

\label{pro:tangent-hallway}
\end{proposition}

\begin{proof}
Let \(c_1\) and \(c_2\) be arbitrary real values. Then \(L' = R_t(L) + c_1 u_t + c_2 v_t\) is an arbitrary rigid transformation of \(L\) rotated counterclockwise by \(t\). The outer walls of \(L'\) corresponding to the outer walls \(a\) and \(c\) of \(L\) (\Cref{def:hallway-walls}) are \(l(t, c_1 + 1)\) and \(l(t + \pi/2, c_2 + 1)\) respectively. They match with the tangent lines \(l_S(t) = l(t, p_S(t))\) and \(l_S(t + \pi/2) = l(t + \pi/2, p_S(t + \pi/2))\) of \(S\) if and only if \(c_1 = p_S(t) - 1\) and \(c_2 = p_S(t + \pi/2) - 1\). That is, if and only if \(L' = L_S(t)\).
\end{proof}

Name the parts of tangent hallway \(L_S(t)\) according to the parts of \(L\) (\Cref{def:hallway-corners}, \Cref{def:hallway-walls}, and \Cref{def:hallway-regions}) for future use.

\begin{definition}

For any shape \(S\) and angle \(t \in S^1\), define the rigid transformation \(f_{S, t} : \mathbb{R}^2 \to \mathbb{R}^2\) as
\[
f_{S, t}(z) = R_t(z) + (p_S(t) - 1)  u_t + (p_S(t + \pi/2) - 1) v_t
\]
so that \(f_{S, t}\) maps \(L\) to \(L_S(t)\).

\label{def:tangent-hallway-map}
\end{definition}

\begin{definition}

For any shape \(S\) and angle \(t \in S^1\), let \(\mathbf{x}_S(t), \mathbf{y}_S(t), a_S(t), b_S(t), c_S(t), d_S(t), Q^+_S(t), Q^-_S(t)\) be the parts of \(L_S(t)\) corresponding to the parts \(\mathbf{x}, \mathbf{y}, a, b, c, d, Q^+, Q^-\) of \(L\) respectively. That is, for any \(? = \mathbf{x}, \mathbf{y}, a, b, c, d, Q^+, Q^-\), let \(?_S(t) := f_{S, t}(?)\).

\label{def:rotating-hallway-parts}
\end{definition}

\begin{proposition}

We have \(L_S(t) = Q_S^+(t) \setminus Q_S^-(t)\) and \(Q^+_S(t) = H_S(t) \cap H_S(t + \pi/2)\). Also we have the following representations of the parts purely in terms of the supporting function \(p_S\) of \(S\).

\begin{gather*}
\mathbf{x}_S(t) = (p_S(t) - 1) u_t + (p_S(t + \pi/2) - 1) v_t \\
\mathbf{y}_S(t) = p_S(t) u_t + p_S(t + \pi/2) v_t \\
a_S(t) = l_S(t) = l(t, p_S(t)) \\
b_S(t) \subseteq l(t, p_S(t) - 1) \\
c_S(t) = l_S(t + \pi/2) = l(t + \pi/2, p_S(t + \pi/2)) \\
d_S(t) \subseteq l(t + \pi/2, p_S(t + \pi/2) - 1) \\
Q_S^+(t) = H(t, p_S(t)) \cap H(t + \pi/2, p_S(t + \pi/2)) \\
Q_S^-(t) = H(t, p_S(t) - 1)^{\circ} \cap H(t + \pi/2, p_S(t + \pi/2))^{\circ}
\end{gather*}

\label{pro:rotating-hallway-parts}
\end{proposition}

\begin{proof}
The formulas for \(\mathbf{x}_S(t)\) and \(\mathbf{y}_S(t)\) are obtained by letting \(z\) equal to \(\mathbf{x} = (0, 0)\) or \(\mathbf{y} = (1, 1)\) in the equation of \Cref{def:tangent-hallway-map}. The formulas for \(a_S(t), b_S(t), c_S(t)\), and \(d_S(t)\) follows from the proof of \Cref{pro:tangent-hallway}. The equality \(L_S(t) = Q_S^+(t) \setminus Q_S^-(t)\) follows from mapping \(L = Q^+ \setminus Q^-\) under the transformation \(f_{S, t}\). The equality \(Q^+_S(t) = H_S(t) \cap H_S(t + \pi/2)\) follows from that \(Q^+_S(t)\) is a cone bounded by tangent lines \(a_S(t) = l_S(t)\) and \(c_S(t) = l_S(t + \pi/2)\) as in the proof of \Cref{pro:tangent-hallway}. The formulas for \(Q_S^-(t)\) and \(Q_S^+(t)\) in terms of \(p_S\) now follow from \Cref{def:tangent-half-plane} and that \(Q_S^-(t)\) is bounded by \(b_S(t)\) and \(d_S(t)\).
\end{proof}

Assume that a rigid transformation \(L'\) of \(L\) rotated counterclockwise by an angle of \(t \in S^1\) contains a shape \(S\). By translating the outer walls of \(L'\) towards \(S\) until they make contact with \(S\), we can see that the tangent hallway \(L_S(t)\) also contains \(S\).

\begin{proposition}

Let \(S\) be any shape contained in a translation of \(R_t(L)\) with angle \(t \in S^1\). Then the tangent hallway \(L_S(t)\) with angle \(t\) also contains \(S\).

\label{pro:tangent-hallway-contains}
\end{proposition}

\begin{proof}
Assume that the translation \(L'\) of \(R_t(L)\) contains \(S\). Then while keeping \(S\) inside \(L'\), we can push \(L'\) towards \(S\) in the directions \(-u_t\) and \(-v_t\) until the outer walls of the final \(L' = L_S(t)\) make contact with \(S\). The pushed hallway \(L_S(t)\) still contains \(S\) because the directions \(-u_t\) and \(-v_t\) of the movement only push the inner walls of \(L'\) away from \(S\).
\end{proof}

\subsection{Moving Hallway Problem}

By our \Cref{def:sofa} of a moving sofa \(S\), any translation of \(S\) is also a valid moving sofa. Without loss of generality, we will always assume that a moving sofa \(S\) is in \emph{standard position} by translating it.

\begin{definition}

A moving sofa \(S\) with rotation angle \(\omega \in (0, \pi/2]\) is in \emph{standard position} if \(p_S(\omega) = p_S(\pi/2) = 1\).

\label{def:standard-position}
\end{definition}

\begin{proposition}

For any angle \(\omega \in (0, \pi/2]\) and shape \(S\), there is a translation \(S'\) of \(S\) such that \(p_{S'}(\omega) = p_{S'}(\pi/2) = 1\) which is (i) unique if \(\omega < \pi/2\), or (ii) unique up to horizontal translations if \(\omega = \pi/2\).

\label{pro:standard-position-shape}
\end{proposition}

\begin{proof}
Since the support function \(p_{S'}(t)\) measures the signed distance from origin to tangent line \(l_{S'}(t)\) (see the remark above \Cref{def:support-function}), the translation \(S'\) of \(S\) satisfies the condition \(p_{S'}(\omega) = p_{S'}(\pi/2) = 1\) if and only if the lines \(l(\omega, 1)\) and \(l(\pi/2, 1)\) are tangent to \(S'\) and \(S'\) is below the lines. Translate \(S\) below the lines \(l(\omega, 1)\) and \(l(\pi/2, 1)\) so that it makes contact with the two lines. If \(\omega < \pi/2\), then the constraints determine the unique location of \(S'\). If \(\omega = \pi/2\), then the two lines are equal to the horizontal line \(y=1\), and \(S'\) can move freely horizontally as long as the line \(y=1\) makes contact with \(S'\) from above.
\end{proof}

Assume any moving sofa \(S\) with rotation angle \(\omega \in (0, \pi/2]\). By \Cref{pro:standard-position-shape} any moving sofa can be put in standard position by translating it. Gerver also observed in \autocite{gerverMovingSofaCorner1992} that \(S\) should be contained in the tangent hallways \(L_S(t)\) for all \(t \in [0, \omega]\) (\Cref{pro:tangent-hallway-contains}). We summarize the full details of Gerver’s observation (line 18-22, p269; line 24-31, p270 of \autocite{gerverMovingSofaCorner1992}) in the following theorem.

\begin{theorem}

Let \(\omega \in (0, \pi/2]\) be an arbitrary angle. For a connected shape \(S\), the following conditions are equivalent.

\begin{enumerate}
\def\labelenumi{\arabic{enumi}.}
\tightlist
\item
  \(S\) is a moving sofa with rotation angle \(\omega\).
\item
  \(S\) is contained in a translation of \(H\) and \(R_\omega(V)\). Also, for every \(t \in [0, \omega]\), \(S\) is contained in a translation of \(R_t(L)\), the hallway rotated counterclockwise by an angle of \(t\).
\item
  Let \(S'\) be any translation of \(S\) such that \(p_{S'}(\omega) = p_{S'}(\pi/2) = 1\). Then (i) \(S' \subseteq P_\omega\) (\Cref{def:parallelogram}), (ii) \(S' \subseteq L_{S'}(t)\) for all \(t \in [0, \omega]\), and (iii) \(S'\) is a moving sofa with rotation angle \(\omega\) in standard position.
\end{enumerate}

\label{thm:moving-sofa-iff-hallway-intersection}
\end{theorem}

\begin{proof}
(1 \(\Rightarrow\) 2) Consider the movement of \(S\) inside the hallway \(L\). For any angle \(t \in [0, \omega]\), there is a moment where the sofa \(S\) is rotated clockwise by an angle of \(t\) inside \(L\), by the intermediate value theorem on the angle of rotation of \(S\) inside \(L\). Viewing this from the perspective of the sofa \(S\), \(S\) is contained in some translation of \(L\) rotated \emph{counterclockwise} by an arbitrary \(t \in [0, \omega]\). Likewise, by looking at the initial (resp. final) position of \(S\) inside \(L_H\) (resp. \(L_V\)) from the perspective of \(S\), the set \(S\) should be contained in a translation of \(H\) and \(R_\omega(V)\) respectively.

(2 \(\Rightarrow\) 3) Take any \(S\) satisfying (2) and its arbitrary translation \(S'\) satisfying \(p_{S'}(\omega) = p_{S'}(\pi/2) = 1\) which is the premise of (3). Then the translate \(S'\) of \(S\) also satisfies (2). So without loss of generality, we can simply assume \(S' = S\) and show (i), (ii) and (iii). Since \(S\) is contained in a translation of \(H\) and \(R_\omega(V)\), the width of \(S\) along the direction of \(u_\omega\) and \(v_0\) (\Cref{def:width}) are at most 1. So \(p_S(\omega) = p_S(\pi/2) = 1\) implies (i) \(S \subseteq P_\omega\). \Cref{pro:tangent-hallway-contains} implies (ii) \(S \subseteq L_S(t)\). It remains to show (iii) that \(S\) is a moving sofa.

Because the support function \(p_S(t)\) of \(S\) is continuous, the tangent hallway \(L_S(t)\) moves continuously with respect to \(t\) by \Cref{def:tangent-hallway}. For every \(t \in [0, \omega]\), let \(g_t := f_{S, t}^{-1}\) be the unique rigid transformation that maps \(L_S(t)\) to \(L\). Then the rigid transformation \(S_t := g_t(S)\) of \(S\) also changes continuously with respect to \(t\). Because \(L_S(0)\) is a translation of \(L\) by letting \(t=0\) in \Cref{def:tangent-hallway}, \(g_0\) is a translation and so \(S_0\) is a translation of \(S\). Mapping \(S \subseteq L_S(t)\) under \(g_t\) we have \(S_t \subseteq L\). So \(S_t\) over the angle \(t \in [0, \omega]\) as time is a continuous movement of a translation \(S_0\) of \(S\) inside \(L\).

It remains to show that \(S_0 \subseteq H\) and \(S_\omega \subseteq V\). Because \(p_S(\pi/2) = 1\), \(L_S(0)\) is a translation of \(L\) along the direction \(u_0\), and the map \(g_0\) is also a translation along the direction \(u_0\). Because \(S \subseteq H\), we also have \(S_0 = g_0(S) \subseteq H\). Likewise, since \(p_S(\omega) = 1\) the hallway \(L_S(\omega)\) is a translation of \(L\) along the direction \(v_\omega\). So the map \(g_\omega\) is the composition of a translation along the direction \(v_\omega\) and \(R_{-\omega}\). Because \(S \subseteq R_\omega(V)\), we also have \(S_\omega = g_\omega(S) \subseteq g_\omega(R_\omega(V)) = V\).

(3 \(\Rightarrow\) 1) By \Cref{pro:standard-position-shape}, any connected shape \(S\) have a translation \(S'\) that satisfies the premise \(p_{S'}(\omega) = p_{S'}(\pi/2) = 1\) of (3). So \(S'\) is a moving sofa by (3), and its translation \(S\) is a moving sofa as well.
\end{proof}