If \(K\) has nonempty interior, it occurs naturally that the boundary \(\partial K\) is a Jordan curve bounding \(K\) in its interior. So for any different \(p, q \in \partial K\), we can think of the Jordan arc \(\mathbf{b}\) connecting \(p\) and \(q\) along the boundary \(\partial K\) in counterclockwise direction. However, to rigorously justify that the curve area functional \(\mathcal{I}(\mathbf{b})\) of \(\mathbf{b}\) is well-defined and relates to the surface area measure \(\sigma_K\) (\Cref{thm:param-curve-area-functional}), we need to contruct an explicit rectifiable parametrization of \(\mathbf{b}\) which requires some work.

For every \(t_0 \in \mathbb{R}\) and \(t_1 \in [t_0, t_0 + 2\pi]\), we will define \(\mathbf{b}_K^{t_0, t_1}\) as essentially the arc-length parametrization of the curve connecting \(v_K^+(t_0)\) to \(v_K^+(t_1)\) along the boundary \(\partial K\) counterclockwise. The full \Cref{def:boundary-segment-parametrization} of \(\mathbf{b}_{K}^{t_0, t_1}\) is technical will be given much later. Instead, we start by stating the properties \(\mathbf{b}_K^{t_0, t_1}\) that agrees with our intuition that we will prove rigorously. Note that in the theorems below we allow \(K\) to have empty interior.

\begin{theorem}

Assume arbitrary \(t_0 \in \mathbb{R}\) and \(t_1 \in [t_0, t_0 + 2\pi]\). Then \(\mathbf{b}_K^{t_0, t_1}\) is an arc-length parametrization of the \(\left\{ v_K^+(t_0) \right\} \bigcup \cup_{t \in (t_0, t_1]} e_K(t)\) from point \(v_K^+(t_1)\) to \(v_K^+(t_2)\).

\label{thm:param-segment}
\end{theorem}

\begin{theorem}

Assume arbitrary \(t_0 \in \mathbb{R}\) and \(t_1 \in [t_0, t_0 + 2\pi]\). Then the curve \(\mathbf{b}_K^{t_0, t_1}\) have length \(\sigma_K((t_0, t_1])\).

\label{thm:param-segment-length}
\end{theorem}

\begin{theorem}

Assume arbitrary \(t_0, t_1, t_2\) such that \(t_0 \leq t_1 \leq t_2 \leq t_0 + 2\pi\). Then \(\mathbf{b}_{K}^{t_0, t_2}\) is the concatenation of \(\mathbf{b}_{K}^{t_0, t_1}\) and \(\mathbf{b}_{K}^{t_1, t_2}\).

\label{thm:param-concatenation}
\end{theorem}

\begin{theorem}

Assume arbitrary \(t_0 \in \mathbb{R}\) and \(t_1 \in [t_0, t_0 + 2\pi]\). Then the curve area functional of \(\mathbf{b}_K^{t_0, t_1}\) can be represented in two different ways:
\[
\mathcal{I} \left( \mathbf{b}_{K}^{t_0, t_1} \right) = \frac{1}{2} \int_{(t_0, t_1]}p_K(t)\,\sigma_K(dt) = \frac{1}{2} \int_{(t_0, t_1]} v_K^+(t) \times d v_K^+(t)
\]

\label{thm:param-curve-area-functional}
\end{theorem}

\begin{theorem}

For every \(t \in \mathbb{R}\), we have \(|K| = \mathcal{I}\left( \mathbf{b}_K^{t, t + 2\pi} \right)\).

\label{thm:param-positive-area}
\end{theorem}

\begin{proof}
This is a corollary of \Cref{thm:surface-area-measure-area} and \Cref{thm:param-curve-area-functional}.
\end{proof}

We will also show that \(\mathbf{b}_K^{t_0, t_1}\) is one of: a Jordan arc, a Jordan curve, or a single point (\Cref{cor:param-positive-jordan}). We first recall the difference between a Jordan arc and curve (p170 of \autocite{apostolMathematicalAnalysisModern}).

\begin{definition}

A \emph{Jordan curve} is a curve parametrized by continuous \(\mathbf{p} : [a, b] \to \mathbb{R}^2\) such that \(a<b\), \(\mathbf{p}(a) = \mathbf{p}(b)\) and \(\mathbf{p}\) being injective on \([a, b)\).

\label{def:jordan-curve}
\end{definition}

\begin{definition}

A \emph{Jordan arc} is a curve parametrized by continuous and injective \(\mathbf{p} : [a, b] \to \mathbb{R}^2\) such that \(a<b\).

\label{def:jordan-arc}
\end{definition}

In order for \(\partial K\) to be a Jordan curve, \(K\) has to have nonempty interior. For the notion of the orientation of a Jordan curve, we refer to p170 of \autocite{apostolMathematicalAnalysisModern}. The following theorem shows that \(\mathbf{b}_K^{t, t + 2\pi}\) is the unique parametrization of \(\partial K\) as a positively-oriented Jordan curve with \(v_K^+(t)\) as its endpoints.

\begin{theorem}

Assume that \(K\) have nonempty interior. For every \(t \in \mathbb{R}\), the curve \(\mathbf{b}_K^{t, t + 2\pi}\) is a positively oriented arc-length parametrization of the boundary \(\partial K\) as a Jordan curve that starts and ends with the point \(v_K^+(t)\).

\label{thm:param-positive-jordan}
\end{theorem}

Now the following is a corollary of \Cref{thm:param-concatenation} and \Cref{thm:param-positive-jordan}.

\begin{corollary}

Assume that \(K\) have nonempty interior. Assume arbitrary \(t_0 \in \mathbb{R}\) and \(t_1 \in [t_0, t_0 + 2\pi]\). Then \(\mathbf{b}_K^{t_0, t_1}\) is one of: a Jordan arc, a Jordan curve, or a single point.

\label{cor:param-positive-jordan}
\end{corollary}

\subsubsection{Definition of Parametrization}

\begin{definition}

Denote the \emph{perimeter} of \(K\) as \(B_K = \sigma_K\left( S^1 \right)\).

\label{def:convex-body-perimeter}
\end{definition}

Fix an arbitrary convex body \(K\) and the starting angle \(t_0 \in \mathbb{R}\). Our goal to construct an arc-length parametrization \(\mathbf{b}_{K}^{t_0} : [0, B_K] \to \mathbb{R}^2\) of the boundary \(\partial K\) starting with the point \(v_K^+(t_0)\). Take an arbitrary point \(p\) on the boundary \(\partial K\). Let \(s \in [0, B_K]\) be the arc length from \(v_K^+(t_0)\) to \(p\) along \(\partial K\), so that we want \(p = \mathbf{b}_{K}^{t_0}(s)\) in the end. As \(p\) is in \(\partial K\), it is inside the tangent line \(l_K(t)\) for some angle \(t \in (t_0, t_0 + 2\pi]\). Now the arc length \(s \in [0, B_K]\) and the tangent line angle \(t \in (t_0, t_0 + 2\pi]\) are the two different variables attached to \(p \in \partial K\).

Unfortunately, the relation between \(s\) and \(t\) cannot be simply described as a function from one to another. A single value of \(s\) may correspond to multiple values of \(t\) (if \(p\) is a sharp corner of angle \(< \pi\)), and likewise a single value of \(t\) may correspond to multiple values of \(s\) (if \(p\) is on the edge \(e_K(t)\) which is a proper line segment). We need the language of generalized inverse (e.g. \autocite{fortelleStudyGeneralizedInverses}) to describe the relationship between \(s\) and \(t\).

The map \(g_K^{t_0}\) is defined so that it sends \(t\) to the largest possible corresponding \(s\).

\begin{definition}

Define \(g_K^{t_0} :[t_0, t_0 + 2\pi] \to [0, B_K]\) as \(g_{K}^{t_0}(t) = \sigma_K\left( (t_0, t] \right)\).

\label{def:conversion-ts}
\end{definition}

The map \(f_K^{t_0}\) is defined so that sends \(s\) to the smallest possible corresponding \(t\).

\begin{definition}

Define \(f_K^{t_0} : [0, B_K] \to [t_0, t_0 + 2\pi]\) as \(f_K^{t_0}(s) = \min \left\{ t \geq t_0 : \sigma_K((t_0, t]) \geq s \right\}\).

\label{def:conversion-st}
\end{definition}

It is rudimentary to check that \(f_K^{t_0}\) is well-defined. We remark that \(f_K^{t_0}\) is the minimum inverse \(g_K^{t_0\wedge}\) of \(g_K^{t_0}\) as defined in \autocite{fortelleStudyGeneralizedInverses}. Note the following.

\begin{proposition}

The functions \(f_K^{t_0}\) and \(g_{K}^{t_0}\) are monotonically increasing.

\label{pro:conversion-monotone}
\end{proposition}

\begin{proof}
That \(g_K^{t_0}(t)\) is increasing is immediate from \Cref{def:conversion-ts}. For any \(t_1 < t_2\), observe
\[
\left\{ t_1 \geq t_0 : \sigma_K((t_0, t]) \geq s \right\} \subseteq \left\{ t_2 \geq t_0 : \sigma_K((t_0, t]) \geq s \right\}
\]
so by \Cref{def:conversion-st} we have \(f_K^{t_0}(t_1) \leq f_K^{t_0}(t_2)\).
\end{proof}

The following can be checked using \Cref{def:conversion-st}.

\begin{proposition}

We have \(f_K^{t_0}(0) = t_0\) and \(f_K^{t_0}(s) > t_0\) for all \(s > 0\).

\label{pro:conversion-st-nonzero}
\end{proposition}

\begin{proof}
That \(f_K^{t_0}(0) = t_0\) is immediate from \Cref{def:conversion-st}. If \(s > 0\), then any \(t \geq t_0\) satisfying \(\sigma_K((t_0, t]) \geq s\) has to satisfy \(t > t_0\), so we have \(f_K^{t_0}(s) > t_0\).
\end{proof}

We will often write \(f_K^{t_0}\) and \(g_K^{t_0}\) as simply \(f\) and \(g\) in proofs because our \(K\) and \(t_0\) are fixed. With the converstions between \(s\) and \(t\) prepared (\(f\) maps \(s\) to \(t\), and \(g\) maps \(t\) to \(s\)), the path \(\mathbf{b}_{K}^{t_0}\) can be defined by integrating the unit vector \(u_t\) for each \(s\).

\begin{definition}

Define \(\mathbf{b}_{K}^{t_0} : [0, B_K] \to \mathbb{R}^2\) as the absolutely continuous (and thus rectifiable) function with the initial condition \(\mathbf{b}_{K}^{t_0}(0) = v_K^+(t_0)\) and the derivative \(\left(\mathbf{b}_{K}^{t_0}\right)'(s) = v_{f_K^{t_0}(s)}\) almost everywhere. That is:
\[
\mathbf{b}_{K}^{t_0}(s) := v_K^+(t_0) + \int_{s' \in (0, s]} v_{f_K^{t_0}(s')} \, ds'
\]

\label{def:parametrization-formal}
\end{definition}

Note that the function \(f_{K}^{t_0}\) is monotonically increasing, so the integral in \Cref{def:parametrization-formal} is well-defined.

\begin{proposition}

The function \(\mathbf{b}_{K}^{t_0} : [0, B_K] \to \mathbb{R}^2\) is an arc-length parametrization.

\label{pro:parametrization-arc-length}
\end{proposition}

\begin{proof}
Length of an absolutely continuous curve~\(\mathbf{x} : [a, b] \to \mathbb{R}^2\)~is the integral of~\(| | \mathbf{x}'(s) | |\) from \(s=a\) to \(s=b\) \autocite{jones2001lebesgue}. For \(\mathbf{x} = \mathbf{b}_K^{t_0}\), we have \(| | \mathbf{x}'(s) | | = 1\) for almost every \(s\) by \Cref{def:parametrization-formal}, thus completing the proof.
\end{proof}

We define \(\mathbf{b}_K^{t_0, t_1}\) as an initial segment of \(\mathbf{b}_K^{t_0}\) that ends with \(v_K^+(t_1)\).

\begin{definition}

For any \(t_0, t_1 \in \mathbb{R}\) such that \(t_1 \in [t_0, t_0 + 2 \pi]\), define \(\mathbf{b}_{K}^{t_0, t_1}\) as the curve \(\mathbf{b}_{K}^{t_0} (s)\) restricted on the interval \(s \in [0, g_{K}^{t_0}(t_1)]\).

\label{def:boundary-segment-parametrization}
\end{definition}

\subsubsection{Theorems on Parametrization}

We now show that \(\mathbf{b}_K^{t_0}\) does parametrize our boundary \(\partial K\) as intended. We prepare three technical lemmas that handle conversions between \(s\) and \(t\).

\begin{lemma}

The followings hold.

\begin{enumerate}
\def\labelenumi{\arabic{enumi}.}
\tightlist
\item
  For any \(t_1 \in (t_0, t_0 + 2\pi]\), we have \(\left(f_{K}^{t_0}\right)^{-1}([t_0, t_1]) = [0, \sigma_K\left( (t_0, t_1] \right)] = [0, g_{K}^{t_0}(t_1)]\).
\item
  Moreover, the set \(\left( f_{K}^{t_0} \right)^{-1} (\left\{ t_1 \right\})\) is either \([g_{K}^{t_0}(t_1-), g_{K}^{t_0}(t_1)]\) or \((g_{K}^{t_0}(t_1-), g_{K}^{t_0}(t_1)]\).
\end{enumerate}

\label{lem:parametrization-set-calculation}
\end{lemma}

\begin{proof}
Write \(f_K^{t_0}\) and \(g_K^{t_0}\) as simply \(f\) and \(g\). The first statement comes from manipulating the definitions as the following.

\begin{align*}
f^{-1}([t_0, t_1]) & = \left\{ s \in [0, B_K] : \min \left\{ t \geq t_0 : \sigma_K\left( (t_0, t] \right) \geq s \right\} \in [t_0, t_1] \right\} \\
& = \left\{ s \in [0, B_K] :  \sigma\left( (t_0, t_1] \right) \geq s \right\} \\
& = [0, \sigma_K((t_0, t_1])] = [0, g(t_1)]
\end{align*}
Now send \(t \to t_1^-\) in the equality \(f^{-1}([t_0, t]) = [0, g(t)]\) to obtain that \(f^{-1}([t_0, t_1)) = \bigcup_{t < t_1} [0, g(t)]\) is either \([0, g(t_1-))\) or \([0, g(t_1-)]\). Then use \(f^{-1} (\left\{ t_1 \right\}) = f^{-1}([t_0, t_1]) \setminus f^{-1}([t_0, t_1))\) to get the second statement.
\end{proof}

\begin{lemma}

The measure \(\sigma_K\) on \((t_0, t_0 + 2 \pi]\) is the pushforward of the Lebesgue measure on \((0, B_K]\) with respect to the map \(f_{K}^{t_0} : (0, B_K] \to (t_0, t_0 + 2 \pi]\) restricted to \((0, B_K]\).

\label{lem:parametrization-pushforward}
\end{lemma}

\begin{proof}
Write \(f_K^{t_0}\) as \(f\). Observe that \(f\) restricted to \((0, B_K]\) has range in \((t_0, t_0 + 2\pi]\) by \Cref{pro:conversion-st-nonzero}. The first statement of \Cref{lem:parametrization-set-calculation} then shows that the measure \(\sigma_K\) on \((t_0, t_0 + 2 \pi]\) and the pushforward of the Lebesgue measure on \((0, B_K]\) with respect to \(f : (0, B_K] \to (t_0, t_0 + 2 \pi]\) agree on every closed interval \((t_0, t]\) for all \(t \in (t_0, t_0 + 2\pi]\).
\end{proof}

\begin{lemma}

\(\mathbf{b}_{K}^{t_0}(g_{K}^{t_0}(t)) = v_{K}^+(t)\) for all \(t \in [t_0, t_0 + 2\pi]\) and \(\mathbf{b}_{K}^{t_0}(g_{K}^{t_0}(t-)) = v_{K}^-(t)\) for all \(t \in (t_0, t_0 + 2\pi]\). Moreover, for all \(t \in (t_0, t_0 + 2\pi]\) the function \(\mathbf{b}_{K}^{t_0}\) restricted to the interval \([g_{K}^{t_0}(t_1-), g_{K}^{t_0}(t_1)]\) is the arc-length parametrization of the edge \(e_K(t)\) from vertex \(v_K^-(t)\) to \(v_K^+(t)\).

\label{lem:parametrization-vertex}
\end{lemma}

\begin{proof}
Write \(f_K^{t_0}\) and \(g_K^{t_0}\) as simply \(f\) and \(g\). By \Cref{lem:parametrization-pushforward} and \Cref{thm:boundary-measure}, we have the following calculation.

\begin{align*}
\mathbf{b}_{K}^{t_0} (g(t)) & = v_K^+(t_0) + \int_{s' \in (0, g(t)]} v_{f(s')} \, ds' \\
& = v_K^+(t_0) + \int_{s' \in f^{-1}([t_0, t])} v_{f(s')} \, ds' \\
& = v_K^+(t_0) + \int_{t \in(t_0, t]} v_t \, \sigma(dt) = v^+_K(t)
\end{align*}
For the proof of \(\mathbf{b}_{K}^{t_0}(g_{K}^{t_0}(t-)) = v_{K}^-(t)\), send \(t' \to t^-\) in the equation \(\mathbf{b}_{K}^{t_0}(g_{K}^{t_0}(t')) = v_{K}^+(t')\) and use \Cref{thm:limits-converging-to-vertex}. By the second statement of \Cref{lem:parametrization-set-calculation}, the value \(f(s')\) on the interval \(s' \in (g(t-), g(t)]\) is always equal to \(t\). So the derivative of \(\mathbf{b}_K^{t_0}(s')\) restricted to the interval \([g(t-), g(t)]\) is almost everywhere equal to \(v_t\), and \(\mathbf{b}_{K}^{t_0}\) is the arc-length parametrization of the edge \(e_K(t)\) from vertex \(v_K^-(t)\) to \(v_K^+(t)\) on the interval \([g(t-), g(t)]\).
\end{proof}

We now prove the claimed theorems on \(\mathbf{b}_K^{t_0, t_1}\). That \(\mathbf{b}_K^{t_0, t_1}\) is injective will be proved later.

\begin{proof}[Proof of \Cref{thm:param-segment}]
Write \(f_K^{t_0}\) and \(g_K^{t_0}\) as simply \(f\) and \(g\). By \Cref{lem:parametrization-set-calculation}, the interval \([0, g(t_1)]\) is equal to the inverse image \(f^{-1} ([t_0, t_1])\), and so is the disjoint union of the singleton \(f^{-1} (\left\{ t_0 \right\}) = \left\{ 0 \right\}\) and the intervals \(f ^{-1} (\left\{ t \right\})\) whose closure is \([g(t-), g(t)]\) for all \(t \in (t_0, t_1]\). Under the map \(\mathbf{b}_{K}^{t_0}\), the singleton \(\left\{ 0 \right\}\) maps to \(\left\{ v_K^+(t_0) \right\}\) and the set \([g(t-), g(t)]\) maps to \(e_K(t)\) for all \(t \in (t_0, t_1]\) by \Cref{lem:parametrization-vertex}. This proves that the image of the interval \([0, g(t_1)]\) under the map \(\mathbf{b}_{K}^{t_0}\) is the set \(\left\{ v_K^+(t_0) \right\} \cup \bigcup_{t \in (t_0, t_1]} e_K(t)\).
\end{proof}

\begin{proof}[Proof of \Cref{thm:param-segment-length}]
This comes from \Cref{pro:parametrization-arc-length} and that the domain \([0, g_{K}^{t_0}(t_1)]\) of \(\mathbf{b}_K^{t_0, t_1}\) has length \(\sigma_K((t_0, t_1])\).
\end{proof}

\begin{proof}[Proof of \Cref{thm:param-concatenation}]
Write \(f_K^{t_0}\) and \(g_K^{t_0}\) as simply \(f\) and \(g\). The curve \(\mathbf{b}_{K}^{t_0, t_1}\) is an initial part of the curve \(\mathbf{b}_{K}^{t_0, t_2}\). So it remains to show that \(\mathbf{b}_{K}^{t_0}\) restricted to the interval \([g(t_1), g(t_2)]\) is the same as \(\mathbf{b}_{K}^{t_1}\) restricted to \([0, g_K^{t_1}(t_2)]\), with the domain shifted to right by \(g(t_1)\). Observe \(g(t_1) + g_K^{t_1}(t_2) = g(t_2)\) by \Cref{def:conversion-ts} and additivity of \(\sigma_K\). The initial point of the two curves is equal to \(v_K^+(t_1)\) by \Cref{lem:parametrization-vertex}. We show that the derivatives of \(\mathbf{b}_{K}^{t_0}(t + g(t_1))\) and \(\mathbf{b}_{K}^{t_1}(t)\) match for all \(t \in [0, g(t_2) - g(t_1)]\). By \Cref{def:parametrization-formal}, we only need to check \(f(t + g(t_1)) = f_K^{t_1} (t)\). This immediately follows from \Cref{def:conversion-st}.
\end{proof}

\begin{proof}[Proof of \Cref{thm:param-curve-area-functional}]
Write \(f_K^{t_0}\) and \(g_K^{t_0}\) as simply \(f\) and \(g\). Take any \(s \in (0, g(t_1)]\) and let \(t = f(s)\). Observe that by \Cref{pro:conversion-st-nonzero}, we have \(t \in (t_0, t_1]\) and \(s\) is in \(f^{-1}(\left\{ t \right\})\) which is either \((g(t_1 -), g(t_1)]\) or \([g(t_1 -), g(t_1)]\) by \Cref{lem:parametrization-set-calculation}. Then as \(\mathbf{b}_{K}^{t_0} (s) \in e_K(t)\) by \Cref{lem:parametrization-vertex}, we have \(\mathbf{b}_{K}^{t_0} (s) \times v_{t} = p_K(t)\). So we have the following.

\begin{align*}
\mathcal{I} \left( \mathbf{b}_{K}^{t_0, t_1} \right) & = \frac{1}{2} \int_{s \in (0, g(t_1)]} \mathbf{b}_{K}^{t_0} (s) \times v_{f(s)} \, ds \\
& = \frac{1}{2} \int_{s \in f^{-1}((t_0, t_1])} p_K(f(s)) \, ds \\
& = \frac{1}{2} \int_{t \in(t_0, t_1]} p_K(t) \, \sigma(dt)
\end{align*}
The first equality above uses \Cref{def:parametrization-formal}. The second equality above uses \Cref{lem:parametrization-set-calculation} and \(\mathbf{b}_{K}^{t_0} (s) \times v_{t} = p_K(t)\). The last equality above uses \Cref{lem:parametrization-pushforward}. This proves the first equality stated in the theorem. To show the second stated equality, check \(v_K(t) \times dv_K^+(t) = v_K^+(t) \times v_{t} \sigma_K(dt) = p_K(t) \sigma(dt)\) by \Cref{pro:boundary-measure-differential}.
\end{proof}

\subsubsection{Injectivity of Parametrization}

Proof of \Cref{thm:param-positive-jordan} requires a bit of preparation. The boundary \(\partial K\) is the union of all the edges.

\begin{theorem}

Let \(K\) be any convex body. Then the topological boundary \(\partial K\) of \(K\) as a subset of \(\mathbb{R}^2\) is the union of all edges \(\cup_{t \in S^1} e_K(t)\).

\label{thm:boundary-is-union-all-edges}
\end{theorem}

\begin{proof}
Let \(E = \cup_{t \in S^1} e_K(t)\). \(E \subseteq \partial K\) comes from \(E \subseteq K\) and that any point in \(E\) is on some tangent line of \(K\) so its neighborhood contains a point outside \(K\). Now take any point \(p \in \partial K\). As \(K\) is closed we have \(p \in K\). So \(p \cdot u_t \leq p_K(t)\) for any \(t \in S^1\). Assume that the equality does not hold for any \(t \in S^1\). Then by compactness of \(S^1\) and continuity of \(p_K\) there is some \(\epsilon > 0\) such that \(\epsilon \leq p_K(t) - p\cdot u_t\) for any \(t\). This implies that the ball of radius \(\epsilon\) centered at \(p\) is contained in \(K\). This contradicts \(p \in \partial K\). So it should be that there is some \(t \in S^1\) such that \(p \cdot u_t = p_K(t)\). That is, \(p \in e_K(t)\).
\end{proof}

We define the following segment of \(\partial K\) as well.

\begin{definition}

For any \(t_0, t_1 \in \mathbb{R}\) such that \(t_1 \in [t_0, t_0 + 2 \pi]\), define \(\mathbf{b}_{K}^{t_0, t_1-}\) as the curve \(\mathbf{b}_{K}^{t_0} (s)\) restricted on the interval \(s \in [0, g_{K}^{t_0}(t_1-)]\).

\label{def:param-segment-open}
\end{definition}

By \Cref{lem:parametrization-vertex}, the curve \(\mathbf{b}_K^{t_0, t_1-}\) ends with the vertex \(v_K^-(t_1)\). Moreover, we have the following corollary of the same \Cref{lem:parametrization-vertex}.

\begin{corollary}

For any \(t_0, t_1 \in \mathbb{R}\) such that \(t_1 \in [t_0, t_0 + 2 \pi]\), \(\mathbf{b}_K^{t_0, t_1}\) is the concatenation of \(\mathbf{b}_K^{t_0, t_1 - }\) and the arc-length parametrization of \(e_K(t_1)\) from \(v_K^-(t_1)\) to \(v_K^+(t_1)\).

\label{cor:param-segment-open}
\end{corollary}

By \Cref{def:parametrization-formal} we have \(\left(\mathbf{b}_{K}^{t_0}\right)'(s) = u_{f_K^{t_0}(s)}\) for almost every \(s\), and by \Cref{pro:conversion-st-nonzero} and \Cref{lem:parametrization-set-calculation} we have \(t_0 < f_K^{t_0}(s) < t_1\) for every \(0 < s < g_{K}^{t_0}(t_1-)\). Thus we have the following:

\begin{corollary}

Let \(t_0, t_1 \in \mathbb{R}\) be arbitrary such that \(t_1 \in [t_0, t_0 + 2 \pi]\). Then for almost every \(s\), the derivative \(\left( \mathbf{b}_{K}^{t_0, t_1-} \right)'(s)\) is equal to \(u_t\) for some \(t \in (t_0, t_1)\).

\label{cor:param-segment-open-deriv}
\end{corollary}

We use the following lemma to determine the orientation of a Jordan curve.

\begin{lemma}

Let \(p\) and \(q\) be two different points of \(\mathbb{R}^2\). Define the closed half-planes \(H_0\) and \(H_1\) as the closed half-planes separated by the line \(l\) connecting \(p\) and \(q\), so that for any point \(x\) in the interior of \(H_0\) (resp. \(H_1\)) the points \(x, p, q\) are in clockwise (resp. counterclockwise) order. If a Jordan curve \(J\) consists of the join of two arcs \(\Gamma_0\) and \(\Gamma_1\), where \(\Gamma_0\) connects \(p\) to \(q\) inside \(H_0\), and \(\Gamma_1\) connects \(q\) to \(p\) inside \(H_1\), then \(J\) is positively oriented.

\label{lem:orientation}
\end{lemma}

\begin{proof}
(sketch) We first show that it is safe to assume the case where \(J\) only intersects \(l\) at two points \(p\) and \(q\). Observe that \(H_i\) has a deformation retract to some subset \(S_i \subseteq H_i\) with \(S_i \cap l = \left\{ p, q \right\}\) (push the three segments of \(l \setminus \{p, q\}\) towards the interior of \(H_i\)). Using the retracts, we may continuously deform the arcs \(\Gamma_0\) and \(\Gamma_1\) inside \(S_0\) and \(S_1\) respectively without chainging the orientation of \(J\). Now take any point \(r\) inside the segment connecting \(p\) and \(q\). Continuously move a point \(x\) inside \(J\) in the orientation of \(J\) starting with \(x = p\). As \(x\) moves along \(\Gamma_0\) from \(p\) to \(q\) the argument of \(x\) with respect to \(r\) increases by \(\pi\). And as \(x\) moves along \(\Gamma_1\) the argument of \(x\) with respect to \(r\) again increases by \(\pi\). So the total increase in the argument of \(x \in J\) is \(2\pi\) and \(J\) is positively oriented.
\end{proof}

Now we are ready to prove \Cref{thm:param-positive-jordan}.

\begin{proof}[Proof of \Cref{thm:param-positive-jordan}]
That \(\mathbf{b}_K^{t, t + 2\pi}\) is an arc-length parametrization of \(\partial K\) comes from \Cref{thm:param-segment} and \Cref{thm:boundary-is-union-all-edges}.

We now show that \(\mathbf{b}_K^{t, t + 2\pi}\) is a Jordan curve. By \Cref{thm:param-concatenation} the curve \(\mathbf{b}_K^{t, t + 2\pi}\) is the concatenation of two curves \(\mathbf{b}_K^{t, t + \pi}\) and \(\mathbf{b}_K^{t + \pi, t + 2\pi}\) connecting \(p = v_{K}^+(t)\) and \(q = v_K^+(t + \pi)\) and vice versa. As \(K\) has nonempty interior, the width of \(K\) measured in the direction of \(u_t\) is strictly positive, and the point \(p\) is strictly further than the point \(q\) in the direction of \(u_t\).

We first show that the curve \(\mathbf{b}_K^{t, t + \pi}\) is a Jordan arc from \(p\) to \(q\). The curve \(\mathbf{b}_K^{t, t + \pi}\) is the join of the curve \(\mathbf{b}_K^{t, t + \pi-}\) and \(e_{K}(t + \pi)\) by \Cref{cor:param-segment-open}. Also, by \Cref{cor:param-segment-open-deriv}, the derivative of \(\mathbf{b}_K^{t, t + \pi-}(s) \cdot u_t\) with respect to \(s\) is strictly positive for almost every \(s\), so the curve \(\mathbf{b}_K\) is moving strictly in the direction of \(-u_t\). This with the fact that \(e_K(t + \pi)\) is parallel to \(v_t\) shows that the curve \(\mathbf{x}_{K, t, t + \pi}\) is injective and thus a Jordan arc. A similar argument shows that \(\mathbf{b}_K^{t + \pi, t + 2\pi}\) is also a Jordan arc.

Define the closed half-planes \(H_0\) and \(H_1\) as the half-planes divided by the line \(l\) connecting \(p\) and \(q\), so that for any point \(x\) in the interior of \(H_0\) (resp. \(H_1\)) the points \(x, p, q\) are in clockwise (resp. counterclockwise) order. Observe that \(\mathbf{b}_K^{t, t + \pi}\) (resp. \(\mathbf{b}_K^{t + \pi, t + 2\pi}\)) is in \(H_0\) (resp. \(H_1\)) by \Cref{thm:param-segment}. Let \(\mathbf{b}\) be either of the curves \(\mathbf{b}_K^{t, t + \pi}\) or \(\mathbf{b}_K^{t + \pi, t + 2\pi}\). Call the line segment connecting \(p\) and \(q\) as \(pq\). Then \(\mathbf{b}\) is either \(pq\) (in case \(\mathbf{b}\) passes through a point \(r\) strictly on \(pq\)) or a curve connecting \(p\) and \(q\) through the interior of \(H_0\) (or \(H_1\)). In any case, the curves \(\mathbf{b}_K^{t, t + \pi}\) and \(\mathbf{b}_K^{t + \pi, t + 2\pi}\) only overlap at the endpoints \(p\) and \(q\) because \(K\) has nonempty interior, showing that \(\mathbf{b}_K^{t, t + 2\pi}\) is a Jordan curve. That \(\mathbf{b}_K^{t, t + 2\pi}\) is positively oriented is a consequence of \Cref{lem:orientation}.
\end{proof}

\subsubsection{Parametrization on Closed Interval}

We define the closed-interval variant \(\mathbf{b}_K^{t_0-, t_1}\) of the curve \(\mathbf{b}_K^{t_0, t_1}\) as essentially the arc-length parametrization of the curve connecting \(v_K^-(t_0)\) to \(v_K^+(t_1)\) along the boundary \(\partial K\) counterclockwise.

\begin{definition}

For every \(t_0 \in \mathbb{R}\) and \(t_1 \in [t_0, t_0 + 2\pi)\) define \(\mathbf{b}_K^{t_0-, t_1}\) as the concatenation of the arc-length parametrization of the edge \(e_K(t_0)\) from \(v_K^-(t_0)\) to \(v_K^+(t_1)\) and the curve \(\mathbf{b}_K^{t_0, t_1}\).

\label{def:closed-param}
\end{definition}

This follows from \Cref{thm:param-segment}.

\begin{corollary}

Assume arbitrary \(t_0 \in \mathbb{R}\) and \(t_1 \in [t_0, t_0 + 2\pi)\). Then \(\mathbf{b}_K^{t_0-, t_1}\) is an arc-length parametrization of the set \(\cup_{t \in [t_0, t_1]} e_K(t)\) from point \(v_K^-(t_0)\) to \(v_K^+(t_1)\).

\label{cor:closed-param-segment}
\end{corollary}

This follows from \Cref{thm:param-segment-length}.

\begin{corollary}

Assume arbitrary \(t_0 \in \mathbb{R}\) and \(t_1 \in [t_0, t_0 + 2\pi)\). Then the curve \(\mathbf{b}_K^{t_0-, t_1}\) have length \(\sigma_K([t_0, t_1])\).

\label{cor:closed-param-segment-length}
\end{corollary}

This follows from \Cref{thm:param-concatenation}.

\begin{corollary}

Assume arbitrary \(t_0, t_1, t_2\) such that \(t_0 \leq t_1 \leq t_2 < t_0 + 2\pi\). Then \(\mathbf{b}_{K}^{t_0-, t_2}\) is the concatenation of \(\mathbf{b}_{K}^{t_0-, t_1}\) and \(\mathbf{b}_{K}^{t_1, t_2}\).

\label{cor:closed-param-concatenation}
\end{corollary}

This follows from \Cref{thm:param-curve-area-functional} and \Cref{thm:surface-area-singleton}.

\begin{corollary}

Assume arbitrary \(t_0 \in \mathbb{R}\) and \(t_1 \in [t_0, t_0 + 2\pi)\). Then we have:
\[
\mathcal{I} \left( \mathbf{b}_{K}^{t_0-, t_1} \right) = \frac{1}{2} \int_{[t_0, t_1]}p_K(t)\,\sigma_K(dt)
\]

\label{cor:closed-param-curve-area-functional}
\end{corollary}

\begin{theorem}

Assume that \(K\) have nonempty interior. Assume arbitrary \(t_0 \in \mathbb{R}\) and \(t_1 \in [t_0, t_0 + 2\pi)\). Then \(\mathbf{b}_K^{t_0-, t_1}\) is one of: a Jordan arc, a Jordan curve, or a single point.

\label{thm:closed-param-positive-jordan}
\end{theorem}

\begin{proof}
Take an arbitrary \(t_{-1}\) so that \(t_{-1} < t_0 \leq t_1 < t_{-1} + 2\pi\). Then \(\mathbf{b}_K^{t_{-1}, t_1}\) is the concatenation of \(\mathbf{b}_K^{t_{-1}, t_0-}\), \(e_K(t_0)\), and \(\mathbf{b}_K^{t_0, t_1}\) by \Cref{thm:param-concatenation} and \Cref{cor:param-segment-open}. Then by \Cref{def:closed-param} the concatenation of \(e_K(t_0)\), and \(\mathbf{b}_K^{t_0, t_1}\) is \(\mathbf{b}_K^{t_0-, t_1}\). Now by \Cref{cor:param-positive-jordan} and that \(\mathbf{b}_K^{t_0-, t_1}\) is a part of \(\mathbf{b}_K^{t_{-1}, t_1}\), we prove the theorem.
\end{proof}