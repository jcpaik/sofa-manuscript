This section defines the set of \emph{normal angles} of a convex body \(K\).

\begin{definition}

Define the set of \emph{normal angles} \(\mathbf{n}(K)\) as the support of the surface area measure \(\sigma_K\) on \(S^1\).

\label{def:convex-set-support}
\end{definition}

If \(K\) is a convex polygon, \(\mathbf{n}(K)\) is the collection of all angles \(t\) such that each \(u_t\) is a normal vector of a proper edge of \(K\). The notion \(\mathbf{n}(K)\) generalizes this to arbitrary convex body \(K\). For an example, take the semicircle \(K = \left\{ (x, y) : x^2 + y^2 \leq 1, y \geq 0 \right\}\). Then the normal angles of \(K\) is the set \([0, \pi] \cup \{3\pi/2\}\).

We now collect theorems on \(\mathbf{n}(K)\).

\begin{lemma}

Let \((t_1, t_2)\) be any open interval of \(S^1\) of length \(< \pi\). Then for every \(t \in S^1 \setminus (t_1, t_2)\), we have \(v_K(t_1, t_2) \in H_K(t)\).

\label{lem:vertex-in-half-plane}
\end{lemma}

\begin{proof}
Let \(p = v_K(t_1, t_2)\). We can either assume \(t_1 - \pi < t < t_1\) or \(t_2 < t < t_2 + \pi\). In the first case, the points \(v_K(t_1, t)\) and \(p\) are on the line \(l_K(t_1)\) and \(p\) is further than \(v_K(t_1, t)\) in the direction of \(v_{t_1}\). Since \(v_K(t_1, t) \in H_K(t)\) we now should have \(p \in H_K(t)\). In the second case, the points \(v_K(t, t_2)\) and \(p\) are on the line \(l_K(t_2)\) and \(v_K(t, t_2)\) is further than \(p\) in the direction of \(v_{t_2}\). Since \(v_K(t, t_2) \in H_K(t)\) we now have \(p \in H_K(t)\).
\end{proof}

\begin{theorem}

Let \(K\) be a convex body, and let \((t_1, t_2)\) be any open interval of \(S^1\) of length \(< \pi\). The followings are equivalent.

\begin{enumerate}
\def\labelenumi{\arabic{enumi}.}
\tightlist
\item
  \((t_1, t_2)\) is disjoint from \(\mathbf{n}(K)\)
\item
  There is a single point \(p\) so that we have \(v_K^+(t) = v_K^-(t) = p\) for all \(t \in (t_1, t_2)\).
\item
  Every tangent line \(l_K(t)\) passes through a common point \(p\) for \(t \in [t_1, t_2]\).
\item
  \(v_K(t_1, t_2) \in K\)
\end{enumerate}

\label{thm:convex-set-support-disjoint}
\end{theorem}

\begin{proof}
(1 \(\Rightarrow\) 2) Let \(p = v_K^+(t_1)\). Then \(v_K^-(t_2) = p\) as well by \Cref{cor:boundary-measure-open}. We also have \(p = v_K^{\pm}(t)\) for every \(t \in (t_1, t_2)\) by \Cref{thm:boundary-measure} on the interval \((t_1, t]\) and \Cref{cor:boundary-measure-open} on the interval \((t_1, t)\).

(2 \(\Rightarrow\) 1) By \Cref{thm:limits-converging-to-vertex} we also have \(v_K^+(t_1) = v_K^-(t_2) = p\). By \Cref{cor:boundary-measure-open} we have the integral \(\int_{t \in (t_1, t_2)} v_t \, \sigma_K(dt) = v_K^-(t_2) - v_K^+(t_1)\) equal to \(0\). Now \(\sigma_K\) has to be zero on \((t_1, t_2)\), or otherwise the integral taken the dot product with \(-u_{t_1}\) should be nonzero as well.

(2 \(\Rightarrow\) 3) follows from that every edge \(e_K(t) = l_K(t) \cap K\) is the segment connecting \(v_K^-(t)\) to \(v_K^+(t)\).

(3 \(\Rightarrow\) 4) The point \(p\) that every tangent line \(l_K(t)\) of \(t \in [t_1, t_2]\) pass through should be \(l_K(t_1) \cap l_K(t_2) = v_K(t_1, t_2)\). We have \(p \in H_K(t)\) for all \(t \in [t_1, t_2]\). We also have \(p \in H_K(t)\) for all \(t \in S^1 \setminus (t_1, t_2)\) by \Cref{lem:vertex-in-half-plane}. Now \(p \in \bigcap_{t \in S^1} H_K(t) = K\).

(4 \(\Rightarrow\) 2) Let \(p := v_K(t_1, t_2)\) and define the cone \(F = H_K(t_1) \cap H_K(t_2)\) pointed at \(p\). Take any \(t \in (t_1, t_2)\). Then the value of \(z \cdot u_t\) over all \(z \in F\) has a unique maximum at \(z = p\). Because \(p \in K \subseteq F\), the value of \(z \cdot u_t\) over all \(z \in K\) also has a unique maximum at \(z = p\). This means that \(e_K(t) = \left\{ p \right\}\), completing the proof.
\end{proof}

\begin{theorem}

Let \(\Pi\) be any closed subset of \(S^1\) such that \(S^1 \setminus \Pi\) is a disjoint union of open intervals of length \(< \pi\). Then for any convex body \(K\), the followings are equivalent.

\begin{enumerate}
\def\labelenumi{\arabic{enumi}.}
\tightlist
\item
  \(K = \bigcap_{t \in \Pi} H_K(t)\)
\item
  \(\mathbf{n}(K)\) is contained in \(\Pi\).
\end{enumerate}

\label{thm:convex-set-support}
\end{theorem}

\begin{proof}
(1 \(\Rightarrow\) 2) Let \((t_1, t_2)\) be any connected component of \(S^1 \setminus \Pi\). Then the interval has length \(< \pi\) by assumption. Now take any \(t \in (t_1, t_2)\). The vertex \(v_K(t_1, t_2)\) is in \(\bigcap_{t \in \Pi} H_K(t) = K\) by \Cref{lem:vertex-in-half-plane}. So by \Cref{thm:convex-set-support-disjoint}, \((t_1, t_2)\) is disjoint from \(\mathbf{n}(K)\). Since \((t_1, t_2)\) was an arbitrary connected component of \(S^1 \setminus \Pi\), we are done.

(2 \(\Rightarrow\) 1) It suffices to show that \(\bigcap_{u \in \Pi} H_K(u) \subseteq H_K(t)\) for all \(t \in S^1\). Once this is shown, we can take intersection over all \(t \in S^1\) to conclude \(K \subseteq \bigcap_{u \in \Pi} H_K(u) \subseteq K\).

If \(t \in \Pi\), then we obviously have \(\bigcap_{u \in \Pi} H_K(u) \subseteq H_K(t)\) so the proof is done. Now take any \(t \in S^1 \setminus \Pi\) and let \((t_1, t_2)\) be the connected component of \(S^1 \setminus \Pi\) containing \(t\). By condition 4 of \Cref{thm:convex-set-support-disjoint} the half-plane \(H_K(t)\) contains the intersection \(F := H_K(t_1) \cap H_K(t_2)\). Observe \(t_1, t_2 \in \Pi\). So \(\bigcap_{u \in \Pi} H_K(u) \subseteq F \subseteq H_K(t)\) for all \(t \in S^1 \setminus \Pi\), completing the proof.
\end{proof}

The following theorem is known as the Gauss-Minkowski theorem (\autocite{marckert2014compact}; Theorem 8.3.1, p465 of \cite{schneider_2013}). It gives a bijection between any convex body \(K\) and its boundary measure \(\sigma_K\).

\begin{theorem}

(Gauss-Minkowski) For any finite Borel measure \(\sigma\) on \(S^1\) with \(\int_{S^1} v_t \, \sigma (dt) = 0\) there is a unique convex body \(K\) with \(\sigma_K = \sigma\) up to translations of \(K\).

\label{thm:gauss-minkowski}
\end{theorem}

By \Cref{def:convex-set-support}, we immediately get the following restriction of \Cref{thm:gauss-minkowski} in normal angles.

\begin{corollary}

Let \(\Pi\) be any closed subset of \(S^1\). For any finite Borel measure \(\sigma\) on \(\Pi\) such that \(\int_{\Pi} v_t\,\sigma(dt) = 0\), there is a convex body \(K\) with normal angles \(\mathbf{n}(K)\) in \(\Pi\) such that \(\sigma_K|_{\Pi} = \sigma\) (\Cref{def:measure-restriction}), which is unique up to translations of \(K\).

\label{cor:supported-gauss-measure}
\end{corollary}