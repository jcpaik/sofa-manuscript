For any right-continuous, real-valued function \(F\) of bounded variation on domain \(X = \mathbb{R}\) or \(S^1 = \mathbb{R} / 2\pi \mathbb{Z}\), we will rigorously define its \emph{Lebesgue-Stieltjes measure} \(dF\) on \(X\). The Lebesgue-Stieltjes measure \(dF\) on any half-open interval \((a, b]\) calculates the difference \(F(b) - F(a)\) of \(F\) along the boundary.

Most literature constructs \(dF\) for any right-continuous and monotonically increasing \(F : \mathbb{R} \to \mathbb{R}\) \autocite{steinRealAnalysisMeasure2005,hewittRealAbstractAnalysis1965,halmos2013measure}.

\begin{theorem}

(Theorem 3.5. of \autocite{steinRealAnalysisMeasure2005}) For any right-continuous and monotonically increasing function \(F : \mathbb{R} \to \mathbb{R}\), there exists a unique measure \(dF\) on \(\mathbb{R}\) such that \(dF((c, d]) = F(d) - F(c)\) for any half-open interval \((c, d]\) of \(\mathbb{R}\).

\label{thm:lebesgue-stieltjes}
\end{theorem}

We will often appeal to the following lemma for the uniqueness of Lebesgue-Stieltjes measure. See p288 of \autocite{steinRealAnalysisMeasure2005} for the notion of signed measure.

\begin{lemma}

Let \(X\) be the domain, which is one of \(\mathbb{R}\), a closed interval \([a, b]\), or the circle \(S^1\). Let \(\mu_1, \mu_2\) be two measures on \(X\) that may or may not be signed. If \(\mu_1\) and \(\mu_2\) agrees on any half-open interval \((c, d]\) of \(X\), and additionally \(\left\{ a \right\}\) if \(X = [a, b]\). Then we have \(\mu_1 = \mu_2\).

\label{lem:measure-interval-uniqueness}
\end{lemma}

\begin{proof}
If \(\mu_1, \mu_2\) are measures (not signed) on \(X\), then since the Borel \(\sigma\)-algebra of \(X\) is generated by half-open intervals \(I\) (and also \(I = \left\{ a \right\}\) if \(X = [a, b]\)), we can appeal to the Carathéodory extension theorem to conclude \(\mu_1 = \mu_2\). If \(\mu_1, \mu_2\) are signed measures on \(X\), for each \(i=1, 2\) first write \(\mu_i = \mu_i^+ - \mu_i^-\) as the difference of two measures on \(X\) (p288 of \autocite{steinRealAnalysisMeasure2005}). Then because \(\mu_1(I) = \mu_2(I)\) for any half-open intervals \(I\) that generate the Borel \(\sigma\)-algebra of \(X\), the measures \(\lambda_1 = \mu_1^+ + \mu_2^-\) and \(\lambda_2 = \mu_2^+ + \mu_1^-\) agree on \(I\). So \(\lambda_1 = \lambda_2\) and in turn we have \(\mu_1 = \mu_2\).
\end{proof}

Now let \(F : [a, b] \to \mathbb{R}\) be any right-continuous function of bounded variation, not necessarily increasing. We can still define \(dF\) by allowing \(dF\) to be a \emph{signed} measure.

\begin{theorem}

For any right-continuous function \(F : [a, b] \to \mathbb{R}\) of bounded variation, there exists a unique signed measure \(dF\) on \(\mathbb{R}\) such that \(dF(\left\{ a \right\}) = 0\) and \(dF((c, d]) = F(d) - F(c)\) for any half-open subinterval \((c, d]\) of \([a, b]\).

\label{thm:lebesgue-stieltjes-signed}
\end{theorem}

\begin{proof}
Write \(F\) as the difference \(F = F_1 - F_2\) of two monotonically increasing and bounded functions \(F_1, F_2 : [a, b] \to \mathbb{R}\) (Theorem 3.3, p119 of \autocite{steinRealAnalysisMeasure2005}). Take the right limit on \(F(x) = F_1(x) - F_2(x)\) to further assume that \(F_1, F_2\) are right-continuous. Extend the domain of \(F_i\) for \(i=1, 2\) to \(\mathbb{R}\) by letting \(F_i(x) = F_i(a)\) for \(x < a\) and \(F_i(x) = F_i(b)\) for \(x > b\). The measures \(dF_1\) and \(dF_2\) on \(\mathbb{R}\) are well-defined by \Cref{thm:lebesgue-stieltjes} and bounded on \([a, b]\). So the bounded signed measure \(dF := dF_1 - dF_2\) is well-defined and satisfies \(dF((c, d]) = F(d) - F(c)\) for any half-open subinterval \((c, d]\) of \([a, b]\). To check \(dF(\left\{ a \right\}) = 0\), check \(dF_i(\left\{ a \right\}) = F_i(a) - F_i(a-) = 0\). Appeal to \Cref{lem:measure-interval-uniqueness} for the uniqueness of \(dF\).
\end{proof}

We now allow the domain of \(F : S^1 \to \mathbb{R}\) to be \(S^1\). Define \(q : [0, 2\pi] \to S^1\) as the quotient map \(q(t) = t + 2\pi \mathbb{Z}\) identifying the endpoints \(0\) and \(2\pi\). We will say that \(F\) is \emph{of bounded variation} if and only if \(F \circ q : [0, 2\pi] \to \mathbb{R}\) is of bounded variation. It is natural to define \(dF\) using \(d(F \circ q)\) where \(F \circ q : [0, 2\pi] \to \mathbb{R}\). Recall the convention that the interval \((t_1, t_2]\) of \(\mathbb{R}\) is used to denote the corresponding interval of \(S^1\) mapped under \(\mathbb{R} \to S^1\).

\begin{theorem}

For any right-continuous function \(F : S^1 \to \mathbb{R}\) of bounded variation, there exists a unique bounded signed measure \(dF\) on \(S^1\) such that for any half-open interval \((t_1, t_2]\) of \(S^1\) with \(t_1 < t_2 \leq t_1 + 2\pi\), we have \(dF((t_1, t_2]) = F(t_2) - F(t_1)\). Moreover, such \(dF\) is unique, and is the pushforward of \(d(F \circ q)\) under the quotient map \(q : [0, 2\pi] \to S^1\).

\label{thm:lebesgue-stieltjes-circle}
\end{theorem}

\begin{proof}
The function \(F \circ q : [0, 2\pi] \to \mathbb{R}\) is right-continuous and of bounded variation. So its Lebesgue-Stieltjes measure \(d(F \circ q)\) on \([0, 2\pi]\) is well-defined by \Cref{thm:lebesgue-stieltjes-signed}. Now let \(\mu\) be the pushforward of \(d(F \circ q)\) under the map \(q\). Take any interval \(I = (t_1, t_2]\) of \(S^1\) with \(0 \leq t_1 < t_2 \leq 2\pi\). We have
\[
\mu(I) = d(F \circ q)(q^{-1}(I)) = d(F \circ q)((t_1, t_2]) = F(t_1) - F(t_2)
\]
because \(q^{-1}(I)\) is either \((t_1, t_2]\) (if \(t_2 < 2\pi\)) or \(\left\{ 0 \right\} \cup (t_1, t_2]\) (if \(t_2 = 2\pi\)), and \(d(F \circ q)(\left\{ 0 \right\}) = 0\) by \Cref{thm:lebesgue-stieltjes-signed}. Now it suffices to show \(\mu((t_1, t_2]) = F(t_2) - F(t_1)\) for the case \(0 \leq t_1 \leq 2\pi \leq t_2 \leq 4\pi\) where the interval wraps around \(0 = 2\pi\) in \(S^1\). We have
\[
\mu((t_1, t_2]) = \mu((t_1, 2\pi]) + \mu((2\pi, t_2]) = F(2\pi) - F(t_1) + F(t_2) - F(2\pi) = F(t_2) - F(t_1)
\]
thus completing the proof. Appeal to \Cref{lem:measure-interval-uniqueness} for the uniqueness of such \(dF\) on \(S^1\).
\end{proof}

The Lebesgue-Stieltjes measure \(dF\) can be used as a justification of the differential of \(F\). It satisfies the following properties that a differential is expected to satisfy.

\begin{proposition}

The Lebesgue-Stieltjes measure \(dF\) on domain \(\mathbb{R}\), \([a, b]\), or \(S^1\) is linear with respect to \(F\).

\label{pro:lebesgue-stieltjes-linear}
\end{proposition}

\begin{proof}
Let \(F_1, F_2\) be arbitrary right-continuous functions of bounded variation, and \(a, b\) be arbitrary real values. Observe that \(d(aF_1 + bF_2)\) and \(a \cdot dF_1 + b \cdot dF_2\) both agree on any half-open interval \(I\), and use \Cref{lem:measure-interval-uniqueness} to conclude that they are equal.
\end{proof}

\begin{proposition}

Let \(F, G\) be real-valued functions on the domain \(X\) which is either \(S^1\) or a closed interval of \(\mathbb{R}\). Assume that both \(F\) and \(G\) are of bounded variation. Assume that \(F\) is continuous and \(G\) is right-continuous. Then \(d(FG) = G dF + F dG\).

\label{pro:lebesgue-stieltjes-product}
\end{proposition}

\begin{proof}
We first prove the case where \(X\) is a closed interval \(I\). By Proposition 4.5, Chapter 0 of \autocite{revuzContinuousMartingalesBrownian1999}, we have the integration by parts
\[
\int_{(a, b]} G(x)\, dF(x) + \int_{(a, b]} F(x-) \, dG(x) = F(b) G(b) - F(a) G(a).
\]
for any endpoints \(a \leq b\) in \(I\). Now use \(F(x-) = F(x)\) and \Cref{lem:measure-interval-uniqueness} to conclude \(d(FG) = G dF + F dG\). For the case where the domain is \(X = S^1\), use the same argument for \(F \circ q\) and \(G \circ q\) with domain \(X = [0, 2\pi]\), and apply \Cref{thm:lebesgue-stieltjes-circle}.
\end{proof}