Here, we prove properties of the support function \(p_K\) (\Cref{def:support-function}) and the vertices \(v_K^{\pm}(t)\) (\Cref{def:convex-body-vertex}) of a convex body \(K\).

\subsubsection{Continuity and Linearity}

\begin{definition}

For any function \(f\) that maps an arbitrary convex body \(K\) to a value \(f(K)\) in a vector space \(V\), say that \(f\) is \emph{linear} with respect to \(K\) if the followings hold.

\begin{enumerate}
\def\labelenumi{\arabic{enumi}.}
\tightlist
\item
  For any \(a \geq 0\) and a convex body \(K\), we have \(f(aK) = a f(K)\).
\item
  For any \(a, b \geq 0\) and convex bodies \(K_1, K_2\), we have \(f(K_1 + K_2) = f(K_1) + f(K_2)\).
\end{enumerate}

\label{def:convex-body-linear}
\end{definition}

Note that the sum used in \Cref{def:convex-body-linear} is the Minkowski sum of convex bodies. That is, \(aK = \left\{ ap : p \in K \right\}\) and \(K_1 + K_2 = \left\{ p_1 + p_2 : p_1 \in K_1, p_2 : K_2 \right\}\).

\begin{theorem}

For any convex body \(K\), its support function \(p_K\) is Lipschitz.

\label{thm:support-function-lipschitz}
\end{theorem}

\begin{proof}
If \(K\) is a single point \(z \in \mathbb{R}^{2}\), then \(p_K = p_z\) is a sinusoidal function with amplitude \(|z|\) where \(|z|\) denotes the distance of \(z\) from the origin. For a general convex body \(K\), take \(R \geq 0\) so that \(K\) is contained in a closed ball of radius \(R\) centered at zero. Then
\[
p_K(t) = \max_{z \in K} z \cdot u_t = \sup_{z \in K} p_z(t)
\]
and note that each function \(p_z : S^1 \to \mathbb{R}\) is \(R\)-Lipschitz. So the supremum \(p_K\) of \(p_z\) over all \(z\) is also \(R\)-Lipschitz.
\end{proof}

\begin{proposition}

The support function \(p_K\) is linear with respect to \(K\).

\label{pro:support-function-linear}
\end{proposition}

\begin{proof}
First condition of \Cref{def:convex-body-linear} on \(p_K\) follows from a direct argument.
\[
p_{aK}(t) = \max_{z \in aK} z \cdot u_t = \max_{z' \in K} (az') \cdot u_t = a p_K(t)
\]

For arbirary convex bodies \(K_1, K_2\) and a fixed \(t \in S^1\),

\begin{align*}
p_{K_1 + K_2} (t) & = \max_{z \in K_1 + K_2} z \cdot u_t = \max_{z_1 \in K_1, z_2 \in K_2} (z_1 + z_2) \cdot u_t \\
& = \max_{z_1 \in K_1} z_1 \cdot u_t + \max_{z_2 \in K_2} z_2 \cdot u_t = p_{K_1}(t) + p_{K_2}(t)
\end{align*}
so the second condition of \Cref{def:convex-body-linear} is true.
\end{proof}

We will soon show that the vertex \(v_K^{+}(t)\) (resp. \(v_K^-(t)\)) is right-continuous (resp. left-continuous) respect to \(t\) and is linear with respect to \(K\). To do so, we compute the limit of vertices via \Cref{thm:limits-converging-to-vertex}.

\begin{definition}

For every \(t_1, t_2 \in S^1\) such that \(t_2 \neq t_1, t_1 + \pi\), define \(v_K(t_1, t_2)\) as the intersection \(l_K(t_1) \cap l_K(t_2)\).

\label{def:convex-body-tangent-lines-intersection}
\end{definition}

\begin{lemma}

For any fixed \(t_1, t_2 \in S^1\) such that \(t_2 \neq t_1, t_1 + \pi\), the point \(v_K(t_1, t_2)\) is linear with respect to \(K\).

\label{lem:tangent-lines-intersection-linear}
\end{lemma}

\begin{proof}
The point \(p = v_K(t_1, t_2)\) is the unique point such that \(p \cdot u_{t_1} = p_K(t_1)\) and \(p \cdot u_{t_2} = p_K(t_2)\) holds. By solving the linear equations, observe that the coordinates of \(p\) are linear combinations of \(p_K(t_1)\) and \(p_K(t_2)\). By \Cref{pro:support-function-linear} the result follows.
\end{proof}

\begin{theorem}

Let \(K\) be a convex body and \(t\) be an arbitrary angle. We have the following right limits all converging to \(v_K^+(t)\). In particular, the vertex \(v_K^+(t)\) is a right-continuous function on \(t \in S^1\).
\[
\lim_{ t' \to t^+ } v_K^+(t) = \lim_{ t' \to t^+ } v_K^-(u) = \lim_{ t' \to t^+ } v_K(t, t') = v_K^+(t)
\]
Similarly, we have the following left limits.
\[
\lim_{ t' \to t^- } v_K^+(t') = \lim_{ t' \to t^- } v_K^-(t') = \lim_{ t' \to t^- } v_K(t', t) = v_K^-(t)
\]

\label{thm:limits-converging-to-vertex}
\end{theorem}

\begin{proof}
We only compute the right limits. Left limits can be shown using a symmetric argument.

Let \(\epsilon > 0\) be arbitrary. Let \(p = v_K^+(t) + \epsilon v_t\). Then by the definition of \(v_K^+(t)\) the point \(p\) is not in \(K\). As \(\mathbb{R}^2 \setminus K\) is open, any sufficiently small open neighborhood of \(p\) is disjoint from \(K\), so we can take some positive \(\epsilon' < \epsilon\) such that the closed line segment connecting \(p\) and \(q = p - \epsilon' u_t\) is disjoint from \(K\) as well. Define the closed right-angled triangle \(T\) with vertices \(v_K^+(t)\), \(p\), and \(q\). Take the line \(l\) that passes through both \(q\) and \(v_K^+(t)\). Call the two closed half-planes divided by the line \(l\) as \(H_T\) and \(H'\), where \(H_T\) is the half-plane containing \(T\) and \(H'\) is the other one. By definition of \(H'\), the half-plane \(H'\) contains \(v_K^+(t)\) and \(q\) on its boundary and does not contain the point \(q\). And consequently \(H'\) has normal angle \(t' \in (t, t + \pi/2)\) (\Cref{def:half-plane}) because \(p = v_K^+(t) + \epsilon v_t\) and \(q = p - \epsilon' u_t\).

We show that \(K \cap H_T \subseteq T\). Assume by contradiction that there is \(r \in K \cap H_T\) not in \(T\). As \(r \in K\), \(r\) should be in the tangential half-plane \(H_K(t)\). So \(r\) is in the cone \(H_T \cap H_K(t)\) sharing the vertex \(v_K^+(t)\) and two edge \(l_K(t)\), \(l\) with \(T\). Since \(r \not\in T\), the line segment connecting \(r\) and the vertex \(v_K^+(t)\) of \(T\) should cross the edge of \(T\) connecting \(p\) and \(q\) at some point \(s\). As \(r, v_K^+(t) \in K\) we also have \(s \in K\) by convexity. But the line segment connecting \(p\) and \(q\) is disjoint from \(K\) by the definition of \(q\), so we get contradiction. Thus we have \(K \cap H_T \subseteq T\).

Now take arbitrary \(t_0 \in (t, t')\). We show that the edge \(e_K(t_0)\) should lie inside \(T\). It suffices to show that any point \(z\) in \(K\) that attains the maximum value of \(z \cdot u_{t_0}\) is in \(T\). Define the fan \(F := H_K(t) \cap H'\), so that \(F\) is bounded by lines \(l_K(t)\) and \(l\) with the vertex \(v_K^+(t)\). If \(z \in F\), it should be that \(z = v_K^+(t) \in T\), because \(v_K^+(t) \in K\) and \(v_K^+(t) \cdot u_{t_0} > z \cdot u_{t_0}\) for every point \(z\) in \(F\) other than \(z = v_K^+(t)\). If \(z \in K \setminus F\) on the other hand, we have \(K \setminus F = K \setminus H' \subseteq K \cap H_T \subseteq T\) so \(z \in T\). This completes the proof of \(e_K(t_0) \subseteq T\).

Observe that the triangle \(T\) contains \(v_K^+(t)\) and has diameter \(< 2\epsilon\) because the two perpendicular sides of \(T\) containing \(p\) have length \(\leq \epsilon\). So the endpoints \(v_K^+(u)\) and \(v_K^-(u)\) of the edge \(e_K(t_0) \subseteq T\) are distance at most \(2\epsilon\) away from \(v_K^+(t)\). This completes the epsilon-delta argument for \(\lim_{ t' \to t^+ } v_K^+(t') = \lim_{ t' \to t^+ } v_K^-(t') = v_K^+(t)\).

From \(e_K(t_0) \subseteq T\) and that the vertex \(p\) of \(T\) maximizes the value of \(z \cdot u_{t_0}\) over all \(z \in T\), we get that \(p\) is either on \(l_K(t_0)\) or outside the half-plane \(H_K(t_0)\). On the other hand we have \(v_K^+(t) \in H_K(t_0)\). So the line \(l_K(t_0)\) passes through the segment connecting \(p\) and \(v_K^+(t)\), and the intersection \(v_K(t, t_0) = l_K(t) \cap l_K(t_0)\) is inside \(T\). This with that the diameter of \(T\) is less than \(2 \epsilon\) proves \(\lim_{ t' \to t^+ } v_K(t, t') = v_K^+(t)\).
\end{proof}

The vertex \(v_K^+(t)\) is right-continuous by \Cref{thm:limits-converging-to-vertex}.

\begin{corollary}

The vertex \(v_K^+(t)\) is right-continuous with respect to \(t \in S^1\). Likewise, the vertex \(v_K^-(t)\) is left-continuous with respect to \(t \in S^1\).

\label{cor:vertex-right-continuous}
\end{corollary}

From \Cref{lem:tangent-lines-intersection-linear} and \Cref{thm:limits-converging-to-vertex}, we have the linearlity of vertices \(v_K^{\pm}(t)\).

\begin{corollary}

For a fixed \(t \in S^1\), the vertices \(v_K^{\pm}(t)\) are linear with respect to \(K\).

\label{cor:vertex-linear}
\end{corollary}

\subsubsection{Parametrization of Tangent Line}

We calculate \(v_K(t, t')\) as the following.

\begin{lemma}

Let \(t, t' \in S^1\) be arbitrary such that \(t' \neq t, t + \pi\). The following calculation holds.
\[
v_K(t, t') = p_K(t) u_{t} + \left( \frac{p_K(t') - p_K(t) \cos (t' - t)}{\sin (t' - t)} \right)  v_{t}
\]

\label{lem:intersection-tangent-lines}
\end{lemma}

\begin{proof}
Because the point \(p = v_K(t, t') = l_{K}(t) \cap l_K(t')\) is on the line \(l_K(t)\), we have \(p = p_K(t) u_{t} + \beta v_{t}\) for some constant \(\beta \in \mathbb{R}\). We use \(p \cdot u_{t'} = p_K(t')\) to derive the unique value \(\beta\). Write \(t' - t\) as \(\theta\).

\begin{align*}
p_K(t') &= p_K(t) (u_{t} \cdot u_{t'}) + \beta (v_{t} \cdot u_{t'}) \\
&= p_K(t) \cos \theta + \beta \sin \theta
\end{align*}
This gives \(\beta = p_C(t') \csc \theta - p_C(t) \cot \theta\) as claimed and completes the calculation. The value \(\alpha\) is continuous on \((-\pi, \pi) \setminus \left\{ 0 \right\}\) by the formula.
\end{proof}

Using \Cref{lem:intersection-tangent-lines}, we can show that \(v_K(t, t')\) parametrizes the half-lines in \(l_K(t)\) emanating from \(v_K^{\pm}(t)\).

\begin{theorem}

Take any \(t \in S^1\) and assume that the width \(p_K(t + \pi) + p_K(t)\) of \(K\) measured in the direction of \(u_t\) is strictly positive (e.g.~when \(K\) has nonempty interior). Define \(\mathbf{p} : [t, t + \pi) \to \mathbb{R}^2\) as \(\mathbf{p}(t) = v_K^+(t)\) and \(\mathbf{p}(t') = v_K(t, t')\) for all \(t' \in (t, t + \pi)\). Then the followings hold.

\begin{enumerate}
\def\labelenumi{\arabic{enumi}.}
\tightlist
\item
  \(\mathbf{p}\) is absolutely continuous on any closed and bounded subinterval of \([t, t+ \pi)\).
\item
  \(\mathbf{p}(t') = v_K^+(t') + \alpha(t') v_t\) where \(\alpha(t) = 0\) and the function \(\alpha : [t, t + \pi) \to \mathbb{R}\) is monotonically increasing.
\item
  \(\mathbf{p}\) is a parametrization of the half-line emanating from \(v_K^+(t)\) in the direction of \(v_t\).
\end{enumerate}

\label{thm:tangent-line-parametrization}
\end{theorem}

\begin{proof}
The function \(\mathbf{p}\) is continuous at \(t\) because of \Cref{thm:limits-converging-to-vertex}. The function \(\mathbf{p}\) is absolutely continuous on any closed subinterval of \((t, t+ \pi)\) by \Cref{lem:intersection-tangent-lines} and \Cref{thm:support-function-lipschitz}. So the derivative \(\mathbf{p}'(u)\) of \(\mathbf{p}\) on \(u \in (t, t + \pi)\) exists almost everywhere and \(\mathbf{p}(u_2) - \mathbf{p}(u_1) = \int_{u_1}^{u_2} \mathbf{p}'(u) \, du\) for every \(t < u_1 < u_2 < t + \pi\). Take the limit \(u_1 \to t^+\) to obtain \(\mathbf{p}(u_2) - \mathbf{p}(t) = \int_{t}^{u_2} \mathbf{p}'(u)\,du\). So \(\mathbf{p}\) is absolutely continuous on any closed subinterval of \([t, t + \pi)\) including the endpoint \(t\). This verifies (1).

\begin{enumerate}
\def\labelenumi{(\arabic{enumi})}
\tightlist
\item
  comes from the geometric fact that for every \(t < t_1 < t_2 < t + \pi\), the point \(v_K(t, t_1)\) lies in the segment connecting \(v_K^+(t)\) and \(v_K(t, t_2)\).
\end{enumerate}

By taking the limit \(u \to t + \pi^-\) in \Cref{lem:intersection-tangent-lines}, we have \(\alpha(u) \to \infty\) (note that we use the fact that the width \(p_K(t + \pi) + p_K(t)\) is positive). Now (3) follows from (1), (2), and \(\alpha(u) \to \infty\).
\end{proof}

\begin{theorem}

Let \(K\) be any convex body. Let \(t_0 \in \mathbb{R}\) be any angle. On the interval \(t \in [t_0, t_0 + \pi]\), the value \(v_K^+(t) \cdot v_{t_0}\) is monotonically decreasing. On the interval \(t \in [t_0 - \pi, t_0]\), the value \(v_K^+(t) \cdot v_{t_0}\) is monotonically increasing.

\label{thm:vertex-monotone}
\end{theorem}

\begin{proof}
Take two arbitrary values \(t_1 < t_2\) in the interval \([t_0, t_0 + \pi]\). The points \(v_K^+(t_1), l_K(t_1) \cap l_K(t_2), v_K^-(t_2), v_K^+(t_2)\) goes further in the direction of \(-v_{t_0}\) (\Cref{def:further-in-direction}) in the increasing order. This shows that the value \(v_K^+(t) \cdot v_{t_0}\) is monotonically decreasing on the interval \(t \in [t_0, t_0 + \pi]\). A symmetric argument proves the other claim.
\end{proof}

\begin{theorem}

On any bounded interval \(t \in I\) of \(\mathbb{R}\), \(v_K^+(t)\) is of bounded variation.

\label{thm:vertex-bounded-variation}
\end{theorem}

\begin{proof}
The \(x\) and \(y\) coordinates of \(v_K^+(t)\) either monotonically increases or decreases on each of the intervals \([0, \pi/2]\), \([\pi/2, \pi]\), \([\pi, 3\pi/2]\), \([3\pi/2, 2\pi]\) by \Cref{thm:vertex-monotone}. So the coordinates are of bounded variation on each interval.
\end{proof}