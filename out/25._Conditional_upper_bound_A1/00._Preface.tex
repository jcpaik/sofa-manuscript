In this \Cref{sec:conditional-upper-bound-a1}, we prove the generalized \Cref{thm:main-cap} of the main \Cref{thm:main} by establishing a conditional upper bound \(\mathcal{A}_1 : \mathcal{K}_\omega \to \mathbb{R}\) of the sofa area functional \(\mathcal{A} : \mathcal{K}_\omega \to \mathbb{R}\). For any monotone sofa \(S\) satisfying the injectivity condition, we prove \(\mathcal{A}_1(K) \geq \mathcal{A}(K)\). Then we calculate the maximum value \(1 + \omega^2/2\) of \(\mathcal{A}_1\).

\Cref{sec:definition-of-a1} defines the value \(\mathcal{A}_1(K)\) of a cap \(K\) as the area of \(K\) subtracted by the area \(\mathcal{I}(\mathbf{x}_K)\) enclosed by the inner corner \(\mathbf{x}_K\) (\Cref{def:a1}). Then \Cref{thm:a1-upper-bound} proves that \(\mathcal{A}_1\) is indeed an upper bound of \(\mathcal{A}\) for monotone sofas satisfying the injectivity condition.

\Cref{sec:calculus-of-variation} shows that the domain \(\mathcal{K}_\omega\) of \(\mathcal{A}_1\), the space of all caps \(K\) with rotation angle \(\omega\), is convex (\Cref{pro:cap-space-convex-space}). Then \Cref{sec:calculus-of-variation} sets up the calculus of variations on a general concave quadratic functional \(f\) on a convex domain \(\mathcal{K}\). In particular, we introduce the notion of \emph{directional derivative} \(Df(K; -)\) of \(f\) (\Cref{def:convex-space-directional-derivative}), and show that any critical point \(K \in \mathcal{K}\) satisfying \(D f(K; -) \leq 0\) is a global maximum (\Cref{thm:quadratic-variation}). This will be used to do the calculus of variations on \(f = \mathcal{A}_1\) with domain \(\mathcal{K} = \mathcal{K}_\omega\).

\Cref{sec:boundary-measure} establishes that \(\mathcal{A}_1\) is a quadratic functional on its domain \(\mathcal{K}_\omega\). To do so, we introduce the \emph{boundary measure} \(\beta_K\) of a cap \(K\) (\Cref{def:boundary-measure}) that measures the side lengths of the upper boundary \(\delta K\); read the description following \Cref{def:boundary-measure} for an example. The measure \(\beta_K\) will be useful in computing the derivative of \(\mathcal{A}_1\) (\Cref{thm:variation-a1}). We will establish the bijective correspondence between \(K\) and \(\beta_K\) (\Cref{thm:boundary-measure-cap} and \Cref{thm:cap-from-boundary-measure}).

\Cref{sec:concavity-of-a1} establishes the concavity of \(\mathcal{A}_1\) (\Cref{thm:a1-negative-semidefinite}), so that a local optimum of \(\mathcal{A}_1\) is also a global optimum of \(\mathcal{A}_1\). We will define the area \(\mathcal{S}(K)\) of a region swept by segments tangent to cap \(K\) (see \Cref{fig:mamikon-sofa}). We will apply \emph{Mamikon’s theorem}, a theorem in classical geometry, to show that \(\mathcal{S}(K)\) is convex with respect to \(K\). We then show that \(\mathcal{S}(K) + \mathcal{A}_1(K)\) is linear with respect to \(K\), showing the concavity of \(\mathcal{A}_1\).

\Cref{sec:directional-derivative-of-a1} calculates the directional derivative \(D\mathcal{A}_1(K; -)\) of \(\mathcal{A}_1\) (\Cref{thm:variation-a1}) using the boundary measure \(\beta_K\) and the facts on convex bodies in \Cref{sec:convex-bodies}. Finally, \Cref{sec:maximizer-of-a1} computes the maximizer \(K = K_{\omega, 1}\) of \(\mathcal{A}_1(K)\) with maximum value \(1 + \omega^2/2\) by solving for the condition where directional derivative \(D\mathcal{A}_1(K; -)\) is always zero.