We will show that \(\mathcal{A}_1 : \mathcal{K}_\omega \to \mathbb{R}\) is a quadratic functional.

\begin{theorem}

For any \(\omega \in (0, \pi/2]\), the functional \(\mathcal{A}_1 : \mathcal{K}_{\omega} \to \mathbb{R}\) is quadratic.

\label{thm:a1-quadratic}
\end{theorem}

To establish \Cref{thm:a1-quadratic}, we will define the \emph{boundary measure} \(\beta_K\) of \(K \in \mathcal{K}_\omega\) and utilize it. Also, we will establish a correspondence between any cap \(K \in \mathcal{K}_\omega\) and its boundary measure \(\beta_K\) (\Cref{thm:boundary-measure-cap} and \Cref{thm:cap-from-boundary-measure}).

\subsubsection{Convex-linear values of cap}

We observe that a lot of values defined on the cap \(K \in \mathcal{K}_\omega\) is convex-linear with respect to \(K\). A reader interested in the details of proofs can read \Cref{sec:vertex-and-support-function} for the full details.

\begin{theorem}

The following values are convex-linear with respect to \(K \in \mathcal{K}_\omega\).

\begin{itemize}
\tightlist
\item
  Support function \(p_K\)
\item
  Vertices \(A^{\pm}_K(t)\) and \(C^{\pm}_K(t)\) for a fixed \(t \in [0, \omega]\)
\item
  The inner and outer corner \(\mathbf{x}_K(t)\) and \(\mathbf{y}_K(t)\) of the tangent hallway with any angle \(t \in [0, \omega]\)
\item
  The points \(W_K(t)\), \(Z_K(t)\) and the values \(w_K(t)\), \(z_K(t)\) for a fixed \(t \in (0, \omega)\)
\item
  The perpendicular leg lengths \(g^{\pm}_K(t)\) and \(h^{\pm}_K(t)\) for all \(t \in [0, \omega]\)
\end{itemize}

\label{thm:cap-convex-linear}
\end{theorem}

\begin{proof}
Use \Cref{pro:support-function-linear} for \(p_K\), \Cref{cor:vertex-linear} for \(A^{\pm}_K(t)\) and \(C^{\pm}_K(t)\), \Cref{lem:tangent-lines-intersection-linear} for \(\mathbf{y}_K(t), W_K(t)\), and \(Z_K(t)\). Use the equality \(\mathbf{y}_K(t) = \mathbf{x}_K(t) + u_t + v_t\) for \(\mathbf{x}_K(t)\), the equalities in \Cref{def:wedge-side-lengths} for \(w_K(t)\) and \(z_K(t)\), and the equalities in \Cref{def:cap-tangent-arm-length} for \(g^{\pm}_K(t)\) and \(h^{\pm}_K(t)\).
\end{proof}

\Cref{thm:cap-convex-linear} in particular establishes that the curve area functional \(\mathcal{I}(\mathbf{x}_K)\) (\Cref{def:curve-area-functional}) is quadratic with respect to \(K\).

\begin{corollary}

The value \(\mathcal{I}(\mathbf{x}_K)\) of a cap \(K \in \mathcal{K}_\omega\) is quadratic with respect to \(K\).

\label{cor:inner-corner-quadratic}
\end{corollary}

\subsubsection{Boundary Measure}

We now define the \emph{boundary measure} \(\beta_K\) of a cap \(K\) as the restriction of the \emph{surface measure} \(\sigma_K\) of \(K\) (\Cref{def:surface-area-measure}).

\begin{definition}

For any cap \(K \in \mathcal{K}_\omega\) with rotation angle \(\omega\), define the \emph{boundary measure} \(\beta_K\) of \(K\) on the set \(J_\omega\) (\Cref{def:j-cap}) as the surface area measure \(\sigma_K\) of \(K\) restricted to \(J_\omega\).

\label{def:boundary-measure}
\end{definition}

See \Cref{sec:surface-area-measure} for a brief introduction on the surface area measure \(\sigma_K\). The boundary measure \(\beta_K\) of cap \(K\) describes the information of length of the upper boundary \(\delta K\). For example, let \(K = [0, 1]^2 \cup \left\{ (x, y) : x \leq 0, y \geq 0, x^2 + y^2 \leq 1 \right\}\) be a cap with rotation angle \(\pi/2\), which is the union of a unit square and a quarter-circle of radius one. Then the boundary measure \(\beta_K\) is a measure on \(J_{\pi/2} = [0, \pi]\) such that \(\beta_K\left( \left\{ 0 \right\} \right) = \beta_K\left( \left\{ \pi/2 \right\} \right) = 1\), and \(\beta_K\) equal to zero on the interval \((0, \pi/2)\) and the Lebesgue measure \(\beta_K(dt) = dt\) on the interval \((\pi/2, \pi)\). We now collect the properties of \(\beta_K\).

\begin{proposition}

The boundary measure \(\beta_K\) is convex-linear with respect to \(K \in \mathcal{K}_\omega\).

\label{pro:boundary-measure-linear}
\end{proposition}

\begin{proof}
Immediate from \Cref{thm:surface-area-measure-convex-linear}.
\end{proof}

\begin{proposition}

For any cap \(K \in \mathcal{K}_\omega\), we have
\[
|K| = \left< p_K, \beta_K \right>_{J_\omega}.
\]

\label{pro:boundary-measure-area}
\end{proposition}

\begin{proof}
By \Cref{thm:surface-area-measure-area} we have \(|K| = \left< p_K, \sigma_K \right>_{S^1}\). Apply \Cref{thm:convex-set-support} to the second condition of \Cref{def:cap} to obtain that \(\sigma_K\) is supported on the set \(J_{\omega} \cup \left\{ \omega + \pi, 3\pi/2 \right\}\). The first condition of \Cref{def:cap} gives \(p_K(\omega + \pi) = p_K(3\pi/2) = 0\). From these, we have \(|K| = \left< p_K, \sigma_K \right>_{S^1} = \left< p_K, \beta_K \right>_{J_\omega}\).
\end{proof}

Now the quadraticity of \(|K|\) comes from convex-linearity of \(p_K\) (\Cref{pro:support-function-linear}) and \(\beta_K\) (\Cref{pro:boundary-measure-linear}) with respect to \(K\).

\begin{corollary}

The area \(|K|\) of a cap \(K \in \mathcal{K}_{\omega}\) is a quadratic functional on \(\mathcal{K}_\omega\)

\label{cor:area-quadratic-functional}
\end{corollary}

The quadraticity of \(\mathcal{A}_1\) is now obtained.

\begin{proof}[Proof of \Cref{thm:a1-quadratic}]
Immediate consequence of \Cref{cor:inner-corner-quadratic} and \Cref{cor:area-quadratic-functional}.
\end{proof}

Gauss-Minkowski theorem (\Cref{thm:gauss-minkowski}) states that any convex set \(K\), up to translation, corresponds one-to-one to a measure \(\sigma\) on \(S^1\) such that \(\int_{S^1}u_t\,\sigma(dt) = 0\) by taking the surface area measure \(\sigma = \sigma_K\). Using this correspondence, we can always construct a bijection between a cap \(K \in \mathcal{K}_\omega\) and its boundary measure \(\beta = \beta_K\).

\begin{theorem}

For any cap \(K \in \mathcal{K}_\omega\) with rotation angle \(\omega\), its boundary measure \(\beta_K\) satisfies the following equations.
\[
\int_{t \in [0, \omega]} \cos(t) \, \beta_K(dt) = 1 \qquad \int_{t \in [\pi/2, \omega + \pi/2]} \cos\left( \omega + \pi/2 - t \right)  \, \beta_K(dt) = 1
\]

\label{thm:boundary-measure-cap}
\end{theorem}

\begin{proof}
By the second condition of \Cref{def:cap} and \Cref{thm:convex-set-support}, we have \(\mathbf{n}(K) \subseteq J_\omega \cup \left\{ \pi + \omega, 3\pi/2 \right\}\). Now by \Cref{thm:convex-set-support-disjoint}, that the interval \((-\pi/2, 0)\) of \(S^1\) is disjoint from \(\Pi\) implies that the point \(A_K^-(0)\) is on the line \(l_K(3\pi/2)\) which is \(y=0\). Likewise, that the interval \((\omega, \pi/2)\) of \(S^1\) is disjoint from \(\Pi\) implies that the point \(A_K^+(\omega)\) is on the line \(l_K(\pi/2)\) which is \(y=1\). By \Cref{cor:boundary-measure-closed} we have
\[
\int_{t \in [0, \omega]} v_t \, \beta_K(dt) = A^+_K(\omega) - A^-_K(0)
\]
and by taking the dot product with \(v_0\), we have the first equality. The second equality can be proved similarly by measuring the displacement from \(C_K^+(\omega)\) to \(C_K^-(0)\) along the direction \(u_\omega\).
\end{proof}

\begin{theorem}

Take arbitrary \(\omega \in (0, \pi/2]\). Conversely to \Cref{thm:boundary-measure-cap}, let \(\beta\) be a measure on \(J_\omega\) that satisfies the following equations.
\[
\int_{t \in [0, \omega]} \cos(t) \, \beta(dt) = 1 \qquad \int_{t \in [\pi/2, \omega + \pi/2]} \cos\left( \omega + \pi/2 - t \right)  \, \beta(dt) = 1
\]
Then there exists a cap \(K \in \mathcal{K}_\omega\) such that \(\beta_K = \beta\). Such \(K\) is unique if \(\omega < \pi/2\), and unique up to horizontal translation if \(\omega = \pi/2\).

\label{thm:cap-from-boundary-measure}
\end{theorem}

\begin{proof}
We first show that there is a unique extension \(\sigma\) of \(\beta\) on the set \(\Pi = J_\omega \cup \{\pi + \omega, 3\pi/2\}\) such that \(\int_{t \in \Pi} v_t \, \sigma(dt) = 0\). The values of \(\sigma\) are determined on \(J_\omega\), and we need to find the values of \(\sigma(\left\{ \pi + \omega \right\})\) and \(\sigma(\left\{ 3 \pi/2 \right\})\) that satisfies the equation \(\int_{t \in \Pi} v_t \, \sigma(dt) = 0\).

If \(\omega = \pi/2\), then by subtracting the two equations in \Cref{thm:cap-from-boundary-measure} we have \(\int_{t \in [0, \pi]} \cos(t)\,\beta(dt) = 0\). So the equation \(\int_{t \in \Pi} v_t \, \sigma(dt) = 0\) becomes \(\sigma(\left\{ 3\pi/2 \right\}) = \int_{t \in [0, \pi]} \sin (t) \,\beta(dt)\) which immediately gives a unique solution \(\sigma\).

Now assume \(\omega < \pi/2\). Let \(A := \int_{t \in [0, \omega]}\sin(t)\,\beta(dt) \geq 0\), then we have \(\int_{t \in [0, \omega]} v_t \,\beta(dt) = - A u_0 + v_0\) by the first equality of \Cref{thm:cap-from-boundary-measure}. Likewise, if we let \(B := \int_{t \in [\pi/2, \omega + \pi/2]} \sin(\omega + \pi/2 - t)\,\beta(dt) \geq 0\), then we have \(\int_{t \in [\pi/2, \omega + \pi/2]}v_t\,\beta(dt) = B v_\omega - u_\omega\) by the second equality of \Cref{thm:cap-from-boundary-measure}. Now the equation \(\int_{t \in \Pi} v_t \, \sigma(dt) = 0\) we are solving for becomes
\[
(-Au_0 + v_0) + (Bv_\omega - u_\omega) + \sigma\left( \left\{ 3\pi/2 \right\}  \right)  u_0 - \sigma\left( \left\{ \pi + \omega \right\}  \right)  v_\omega = 0
\]
and \(\sigma(\left\{ \pi + \omega \right\}) = B + v_\omega \cdot o_\omega \geq 0\) and \(\sigma(\left\{ 3 \pi/2 \right\}) = A + u_0 \cdot o_\omega \geq 0\) (remark that \(o_\omega\) is in \Cref{def:parallelogram-vertices}) gives the unique solution of \(\sigma\).

We now use \Cref{cor:supported-gauss-measure} on the measure \(\sigma\) extended on the set \(\Pi\). There is a unique convex body \(K\) up to translation so that \(\mathbf{n}(K) \subseteq \Pi\) (see \Cref{def:convex-set-support}) and \(\sigma_K|_{\Pi} = \sigma\). Our goal now is to translate \(K\) so that it is a cap with rotation angle \(\omega\). Since \(\mathbf{n}(K) \subseteq \Pi\), any translate \(K\) satisfy the second condition of cap in \Cref{def:cap}. It remains to prove the first condition of \Cref{def:cap}.

The width of \(K\) along the directions \(u_\omega\) and \(v_0\) are equal to 1 by applying the equations given in \Cref{thm:cap-from-boundary-measure} to \Cref{cor:boundary-measure-width} with angles \(t = \omega, \pi/2\). If \(\omega = \pi/2\), we only need \(\beta_K(\pi/2) = 1\) for \(K\) to satisfy the first condition of \Cref{def:cap}, and such a cap \(K\) is unique up to horizontal translation. If \(\omega < \pi/2\), we need both \(\beta_K(\pi/2) = \beta_K(\omega) = 1\) to satisfy the first condition of \Cref{def:cap}, so such a cap \(K\) exists uniquely among all translates.
\end{proof}