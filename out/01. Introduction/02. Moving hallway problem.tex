The \emph{rotation angle} \(\omega\) of a moving sofa \(S\) is defined as the clockwise angle that \(S\) rotates as it moves from \(L_H\) to \(L_V\) inside \(L\). Gerver’s sofa has the rotation angle \(\omega = \pi/2\). On the other hand, the unit square \([0, 1]^2\) can move inside \(L\) with only translation, so it has the rotation angle \(\omega = 0\). Let \(\omega\) be the rotation angle of a maximum-area moving sofa. Gerver proved that we can assume \(\pi/ 3 \leq \omega \leq \pi/2\) \autocite{gerverMovingSofaCorner1992}. Kallus and Romik improved the lower bound of \(\omega\) by showing that \(\omega \geq \arcsin(84/85) = 81.203\dots^\circ\) \autocite{kallusImprovedUpperBounds2018}. With this, we will only consider moving sofas with rotation angle \(\omega \in (0, \pi/2]\), and it seems reasonable to conjecture that a maximum-area moving sofa has \(\omega = \pi/2\).

\begin{conjecture}

(Angle Hypothesis) There exists a maximum-area moving sofa with a movement of rotation angle \(\omega = \pi/2\).

\label{con:angle}
\end{conjecture}

\begin{remark}

A single moving sofa \(S\) may admit different movements with varying rotation angles \(\omega\). For any moving sofa \(S\) mentioned in this paper, we will always assume a fixed rotation angle \(\omega\) attached to it. So any moving sofa in this paper is technically a tuple of a shape and its fixed rotation angle. In this way, we can talk about \emph{the} rotation angle of a moving sofa.

\label{rem:angle}
\end{remark}

Once we fix the rotation angle \(\omega \in (0, \pi/2]\) of a moving sofa \(S\), we can change the moving sofa problem as the moving \emph{hallway} problem. In \autocite{gerverMovingSofaCorner1992}, Gerver looked at the hallway \(L\) in the perspective of the moving sofa \(S\), so that \(S\) is fixed and the hallway \(L\) moves around the sofa (see \Cref{fig:monotone-sofa}). In this perspective, we claim that the sofa \(S\) is a common subset of the following sets.

\begin{enumerate}
\def\labelenumi{\arabic{enumi}.}
\tightlist
\item
  For every angle \(t \in [0, \omega]\), the rotated hallway \(L_t\), which is the hallway \(L\) rotated counterclockwise by \(t\) and translated so that the outer walls of \(L_t\) are in contact with \(S\).
\item
  The horizontal strip \(H = \mathbb{R} \times [0, 1]\).
\item
  A translation of \(V_\omega\), where \(V_\omega\) is the vertical strip \(V = [0, 1] \times \mathbb{R}\) rotated counterclockwise by \(\omega\) around the origin.
\end{enumerate}

To observe (1) that \(S \subseteq L_t\) for every angle \(t \in [0, \omega]\), note that the sofa \(S\) is rotated initially by a clockwise angle of \(0\) and finally by \(\omega\) during its movement. By the intermediate value theorem, there is a moment where a copy of \(S\) is rotated clockwise by \(t\) inside \(L\). Push the rotated copy of \(S\) in \(L\) towards the positive \(x\) and \(y\) directions, until it makes contact with the outer walls \(x=1\) and \(y=1\) of \(L\). See this configuration in the perspective of \(S\) to conclude \(S \subseteq L_t\). To observe (2) and (3), note that the sofa \(S\) is initialy in \(L_H\) and finally in \(L_H\) rotated clockwise by \(\omega\); see the configurations in perspective of \(S\) ot conclude that \(S\) is in \(H\) and a translation of \(V_\omega\).

\begin{figure}
\centering
\includesvg[width=1\textwidth,height=\textheight]{images/monotone-sofa-combined.svg}
\caption{The movement of a moving sofa \(S\) with rotation angle \(\omega = \pi/2\) in perspective of the hallway (left) and the sofa (right). The monotone sofa \(S\) is equal to the cap \(K\) minus its niche \(N\). The cap \(K\) is a convex body with the outer walls of \(L_t = L_S(t)\) as the tangent lines of cap \(K\) (and sofa \(S\)). The niche \(N\) is the union of all triangular regions bounded from above by the inner walls of \(L_t = L_S(t)\).}
\label{fig:monotone-sofa}
\end{figure}

Without loss of generality, we can translate the moving sofa \(S\) horizontally and assume that \(S\) is contained in \(V_\omega\), not its arbitrary translation.\footnote{Translating the moving sofa \(S\) may invalidate the initial condition \(S \subseteq L_H\). To resolve this, we relax the \Cref{def:sofa} of a moving sofa \(S\) in so that only some translation of \(S\) is required to be movable from \(L_H\) to \(L_V\) inside \(L\).} Then from (1), (2) and (3) above, we can always assume that \(S\) is contained in the intersection
\begin{equation}
\label{eqn:monotone}
\mathcal{M}(S) := H \cap V_\omega \cap \bigcap_{0 \leq t \leq \omega} L_t
\end{equation}
that we will define as the \emph{monotonization} of a moving sofa \(S\). The set \(\mathcal{M}(S)\) is also movable inside \(L\); observe that \(\mathcal{M}(S)\) is in each rotating hallway \(L_t\) of angle \(t \in [0, \omega]\), and see this in the perspective of the fixed hallway \(L\) instead of the moving hallway \(L_t\) and vary \(t\) from \(0\) to \(\omega\). So the monotonization \(\mathcal{M}(S)\) is a moving sofa containing \(S\) (\Cref{thm:monotonization-is-sofa})\footnote{This uses an implicit assumption that the intersection \(\mathcal{M}(S)\) should be connected. Indeed, \autocite{gerverMovingSofaCorner1992} assumes the connectedness of \(\mathcal{M}(S)\) in its proof of Theorem 1 of \autocite{gerverMovingSofaCorner1992} without further justification. We prove this assumption rigorously in \Cref{thm:monotonization-is-connected}.}, and we call such a sofa \(\mathcal{M}(S)\) from monotonization a \emph{monotone sofa} (\Cref{def:monotone-sofa}). Since any moving sofa \(S\) is contained in a larger monotone sofa \(\mathcal{M}(S)\), we only need to consider monotone sofas for the moving sofa problem.