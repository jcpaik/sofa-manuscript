Moving a large couch through a narrow hallway requires a well-planned pivoting. The \emph{moving sofa problem}, first published by Leo Moser in 1966 \autocite{moser1966problem}, is asked in a two-dimensional idealization of such a situation:

\begin{quote}
What is the largest area \(\mu_{\text{max}}\) of a connected shape that can move around the right-angled corner of a hallway with unit width?
\end{quote}

More precisely, define the hallway \(L\) as the union \(L = L_H \cup L_V\) of sets \(L_H = (-\infty, 1] \times [0, 1]\) and \(L_V = [0, 1] \times (-\infty, 1]\) representing the horizontal and vertical side of \(L\) respectively. A \emph{moving sofa} \(S\) may be defined as a connected subset of \(L_H\) that can be moved inside \(L\) by a continuous rigid motion to a subset of \(L_V\). It is known that there exists a moving sofa attaining the maximum area \(\mu_{\text{max}}\) \autocite{gerverMovingSofaCorner1992,croft2012unsolved}, but the precise value of \(\mu_{\text{max}}\) remains unknown despite decades of partial progress \autocite{hammersley1968enfeeblement,gerverMovingSofaCorner1992,romikDifferentialEquationsExact2018,kallusImprovedUpperBounds2018}.

The best bounds currently known on \(\mu_{\max}\) are summarized as
\begin{equation}
\label{eqn:best-bounds}
\mu_G := 2.2195\dots \leq \mu_{\max} \leq 2.37.
\end{equation}
The lower bound \(2.2195\dots \leq \mu_{\max}\) comes from Gerver’s sofa \(S_G\) of area \(\mu_G :=
2.2195\dots\) constructed in 1994 \autocite{gerverMovingSofaCorner1992} (see \Cref{fig:gerver}).
Gerver derived his sofa from local optimality considerations\footnote{Gerver assumed five stages of
the movement of a sofa to derive his sofa \(S_G\) \autocite{gerverMovingSofaCorner1992}. While his
sofa \(S_G\) is locally optimal (Theorem 2 of \autocite{gerverMovingSofaCorner1992}), this does not
eliminate the possibility of a maximum-area sofa with a different kind of movement. Romik’s
simplified derivation of \(S_G\) in \autocite{romikDifferentialEquationsExact2018} also relies on
the same assumption (Equation 24, p324 of \autocite{romikDifferentialEquationsExact2018}). So their
derivations do not constitute a full proof of Gerver’s conjecture \(\mu_{\max} = \mu_G\).} and
conjectured \(\mu_G = \mu_{\max}\) that his sofa attains the maximum area. Approximate solutions
found by computer experiments are consistent with Gerver’s conjecture.\footnote{Wagner used Monte
Carlo simulation to find an approximate solution (Figure 2 of \autocite{wagner1976sofa}) that
resembles Gerver’s sofa in shape. More recent approximate solutions, as found by Gibbs
\autocite{gibbsComputationalStudySofas2014} in 2014 and Batsch \autocite{batschNumericalApproachAnalysing2022} in 2022, deviate in area from Gerver’s sofa by small margins of \(1.7 \times 10^{-7}\) and \(5.7 \times 10^{-9}\) respectively.} On the other hand, the upper bound \(\mu_{\max} \leq 2.37\) was proved by Kallus and Romik using extensive computer assistance \autocite{kallusImprovedUpperBounds2018}.

\begin{figure}
\centering
\includesvg[width=0.7\textwidth,height=\textheight]{images/gerver-sofa.svg}
\caption{Gerver’s sofa \(S_G\). The ticks denote the endpoints of 18 analytic curves and segments constituting the boundary of \(S_G\) (see \autocite{romikDifferentialEquationsExact2018} for details). The lower portion of \(S_G\) is made of two small ‘tails’ (depicted red) and one large ‘core’ (depicted blue).}
\label{fig:gerver}
\end{figure}

We prove that any moving sofa satisfying a certain property, named as the \emph{injectivity condition} (\Cref{def:injectivity}), has an area at most \(1 + \pi^2/8 = 2.2337\dots\) (\Cref{thm:main}). This upper bound, while conditional, is much closer to the lower bound \(2.2195\dots\) of Gerver than the upper bound \(2.37\) of Kallus and Romik. Since Gerver’s sofa \(S_G\) satisfies the injectivity condition, our conditional upper bound is effective on the domain containing the conjectured optimum \(S_G\). The proof does not rely on any computer assistance, and reframes the problem to a convex quadratic optimization problem as a new approach. Then we use a calculus of variation based on the Brunn-Minkowski theory on convex bodies to prove the upper bound \(1 + \pi^2/8\). We also conjecture the \emph{injectivity hypothesis} (\Cref{con:injectivity}) that there exists a maximum-area moving sofa satisfying the injectivity condition. With our result, proving the injectivity hypothesis amounts to improving the upper bound to \(\mu_{\max} \leq 1 + \pi^2/8\).
