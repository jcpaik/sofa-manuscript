\Cref{sec:notations-and-conventions} contains basic definitions that will be used thoroughout this paper. Since \Cref{sec:notations-and-conventions} is comprehensive, the reader is not expected to read everything in \Cref{sec:notations-and-conventions} at one setting. Instead, the reader can start off by reading the definitions on moving sofas and convex bodies, and later refer to the parts of this section as needed.

\Cref{sec:monotone-sofas} proves \Cref{thm:monotonization-is-sofa} that any moving sofa can be enlarged to a monotone sofa, and proves structural \Cref{thm:monotone-sofa-structure} and \Cref{thm:niche-in-cap} on monotone sofas. Using this, \Cref{sec:sofa-area-functional-a} reduces the moving sofa problem to the maximization of the \emph{sofa area functional} \(\mathcal{A} : \mathcal{K}_\omega \to \mathbb{R}\) on the space of caps \(\mathcal{K}_\omega\). \Cref{sec:conditional-upper-bound-a1} proves the main \Cref{thm:main} by establishing the upper bound \(1 + \omega^2/2\) of the sofa area \(\mathcal{A}\) using the upper bound \(\mathcal{A}_1\). Each chapter starts with an overview of the chapter.

\Cref{sec:convex-bodies} proves numerous properties of an arbitrary planar convex body \(K\) that
we will use thoroughly in this paper. So the logical ordering of this paper is
\Cref{sec:notations-and-conventions}, followed by \Cref{sec:convex-bodies}, then the chapters
starting \Cref{sec:monotone-sofas}. A logically inclined reader may read in this ordering to verify
the correctness of all arguments. On the other hand, readers who are interested in the overall idea
may either start by reading the chapters in order and refer to the appendix when needed, or skim the introduction of each section of \Cref{sec:convex-bodies} and then read the rest of the paper.
