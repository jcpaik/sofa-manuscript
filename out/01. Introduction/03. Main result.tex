Assume the maximum-area monotone sofa \(S\) with rotation angle \(\omega \in (0, \pi/2]\). Then we can assume \(S = \mathcal{M}(S)\) in \Cref{eqn:monotone} as taking further monotonization does not increase the area.\footnote{In fact, that \(S\) attains the maximum-area only guarantees that the gain \(\mathcal{M}(S) \setminus S\) of monotonization is of measure zero. We show in \Cref{thm:monotonization-fixpoint} that for any monotone sofa, we have \(\mathcal{M}(S) = S\).} For each angle \(t \in [0, \omega]\), let \(R_t : \mathbb{R}^2 \to \mathbb{R}^2\) be the rotation of \(\mathbb{R}^2\) around the origin, and let \(\mathbf{x}_S(t)\) be the coordinate of the inner corner of the rotated hallway \(L_t\). Then the hallway \(L_t = R_t(L) + \mathbf{x}_S(t)\) rotated counterclockwise by angle \(t \in [0, \pi/2]\) is determined by its inner corner \(\mathbf{x}_S(t)\). So the parametrization \(\mathbf{x}_S : [0, \omega] \to \mathbb{R}^2\) of the inner corner with respect to angle \(t\), defined as the \emph{rotation path} of \(S\) in \autocite{romikDifferentialEquationsExact2018}, determines the monotone sofa
\[
S = H \cap V_\omega \cap \bigcap_{0 \leq t \leq \omega} \left( R_t(L) + \mathbf{x}_S(t) \right) 
\]
completely. In particular, Romik in \autocite{romikDifferentialEquationsExact2018} derived Gerver’s sofa \(S = S_G\) by solving for the local optimality of area of \(S\) as a set of ordinary differential equations on \(\mathbf{x}_S\).{[}\^{}gerver-derivation{]}

The \emph{injectivity condition} of \(S\) essentially states that the inner corner \(\mathbf{x}_S(t)\) is injective with respect to angle \(t\).

\begin{definition}

(Injectivity condition) A monotone sofa \(S\) with rotation angle \(\omega \in (0, \pi/2]\) satisfies the \emph{injectivity condition}, if its rotation path \(\mathbf{x}_S : [0, \omega] \to \mathbb{R}^2\) is injective and never below the bottom line \(y = 0\) of \(H\) nor the bottom line \(x \cos \omega + y \sin \omega = 0\) of \(V_\omega\).

\label{def:injectivity}
\end{definition}

In particular, Gerver’s sofa \(S_G\) is a monotone sofa satisfying the injectivity condition (note that the blue trajectory of \(\mathbf{x}_{S_G}\) in \Cref{fig:gerver} is injective and never below the bottom line \(y=0\)). For any monotone sofa satisfying the injectivity condition, the following upper bound is established.

\begin{theorem}

(Main theorem) The area of any monotone sofa \(S\) with rotation angle \(\omega \in (0, \pi/2]\) satisfying the injectivity condition is at most \(1 + \omega^2/2\).

\label{thm:main}
\end{theorem}

So the upper bound of \Cref{thm:main} is effective on a domain of monotone sofas containing the conjectured optimum \(S_G\). The upper bound \(1 + \omega^2/2\) of \Cref{thm:main} maximizes at \(\omega = \pi/2\) with the value \(1 + \pi^2/8 = 2.2337\dots\), which is much closer to the lower bound \(2.2195\dots\) of Gerver than the currently best upper bound of \(2.37\) of Kallus and Romik (\Cref{eqn:best-bounds}). We conjecture that a monotone sofa of maximum area should satisfy the premise of \Cref{thm:main}.

\begin{conjecture}

(Injectivity hypothesis) There exists a monotone sofa \(S\) of maximum area with rotation angle \(\omega \in (0, \pi/2]\), satisfying the injectivity condition.

\label{con:injectivity}
\end{conjecture}

With \Cref{thm:main}, proving \Cref{con:injectivity} will imply the unconditional upper bound \(\mu_{\max} \leq 1 + \pi^2/2 = 2.2337\dots\).

The main idea for proving \Cref{thm:main} is to overestimate the area of a monotone sofa \(S\). For the sake of explanation, fix the rotation angle \(\omega = \pi/2\). Observe that the lower boundary of Gerver’s sofa \(S_G\) consists of two red ‘tails’ and one blue ‘core’ in \Cref{fig:gerver}. The core is parametrized by the inner corner \(\mathbf{x}_{S_G}(t)\) for the interval \(t \in [\varphi, \pi/2 - \varphi]\) with constant \(\varphi = 0.0392\dots\) \autocite{romikDifferentialEquationsExact2018}, and forms the majority of the lower boundary. The region below the two tails, trimmed out by the inner left and right walls of \(L_t\) respectively, constitutes only about \(0.28 \%\) of the area \(2.2195\dots\) of the whole sofa. Motivated by this, we define the overestimation \(\mathcal{A}_1\) of the area \(\mathcal{A}\) of a monotone sofa \(S\) with rotation angle \(\pi/2\) as: the area of the convex hull \(K\) of \(S\), that we call the \emph{cap} of \(S\), subtracted by the region enclosed by \(\mathbf{x}_S : [0, \omega] \to \mathbb{R}^2\) and the line \(y=0\).

Another key idea for proving \Cref{thm:main} is to consider the area of \(S\) as a function of the cap \(K\). By doing so, we turn the problem to a quadratic optimization problem in convex analysis. In particular, the overestimated area \(\mathcal{A}_1(K)\) as a function of the cap \(K\) turns out to be a \emph{concave} and \emph{quadratic} functional with respect to the cap \(K\). To justify this, we first construct the space \(\mathcal{K}_\omega\) of all possible caps \(K\) with a fixed rotation angle \(\omega\). The space \(\mathcal{K}_\omega\) itself is a convex space equipped with convex combination \(c_\lambda(K_1, K_2) = (1 - \lambda) K_1 + \lambda K_2\) defined by the Minkowski sum of convex bodies. Then the Brunn-Minkowski theory \cite{schneider_2013} is used to establish that \(\mathcal{A}_1\) is a quadratic functional on \(\mathcal{K}_\omega\). We use Mamikon’s theorem \autocite{mnatsakanianAnnularRingsEqual1997}, a theorem in classical geometry, to prove that \(\mathcal{A}_1\) concave on \(\mathcal{K}_\omega\). Finally, we introduce a calculus of variation based on the Brunn-Minkowski theory to find a local optimum \(K_1\) of \(\mathcal{A}_1\). Since \(\mathcal{A}_1\) is concave, the local optimum \(K_1\) is also a global optimum of \(\mathcal{A}_1\), and \Cref{thm:main} is established by computing the maximum value \(\mathcal{A}_1(K_1) = 1 + \omega^2/2\) of \(\mathcal{A}_1\).

For the rotation angle \(\omega = \pi/2\), the maximizer of \(\mathcal{A}_1\) gives an unmovable sofa \(S_1\) of area \(1 + \pi^2/8 = 2.2337\dots\) and width \(\pi\) very close to the area of Gerver’s sofa \(S_G\) (see \Cref{fig:presofa}). The shape of \(S_1\) is very close to \(S_G\), and cutting away the region under the red curves from \(S_1\) gives a valid moving sofa of area approximately \(2.2009\dots\), which is again very close to \(S_G\) in its shape and area. The boundary of \(S_1\) consists of three line segments and three parametric curves. The right side of \(S_1\) is parametrized by the curve \(\gamma : [0, \pi/2] \to \mathbb{R}^2\) with \(\gamma(0) = (1, 1)\) and \(\gamma'(t) = t(\cos t, -\sin t)\), which ends at \(\gamma(\pi/2) = (\pi/2, 0)\). The left side of \(S_1\) is parametrized by the curve symmetric to \(\gamma\) along the \(y\)-axis. The bottom middle part of \(S_1\) is parametrized by the curve \(\mathbf{x}_{S_1} : [0, \pi/2] \to \mathbb{R}^2\) with \(\mathbf{x}_{S_1}(0) = (\pi/2-1, 0)\) and
\[
\mathbf{x}_{S_1}'(t) = -t (\cos t, \sin t) + (\pi/2- t) (-\sin t, \cos t)
\]
which is symmetric along the \(y\)-axis. The endpoints of the three curves are connected by three horizontal line segments of length 1 or 2 to form the boundary of \(S_1\).

\begin{figure}
\centering
\includesvg[width=1\textwidth,height=\textheight]{images/presofa-combined.svg}
\caption{The maximizing shape \(S_1\) of \(\mathcal{A}_1\) of area \(1 + \pi^2/8 = 2.2337\dots\). The regions below two tails (red curves) stick out of the hallway \(L\) during the movement of \(S_1\) in \(L\) (left). The shape \(S_1\) is very similar to Gerver’s sofa \(S_G\) whose boundary is drawn in dotted lines (right).}
\label{fig:presofa}
\end{figure}